


Biosensors based on surface plasmon resonance (SPR) play a central role as
a simple and remarkably responsive label-free method for characterizing and
quantifying biomolecular
interactions~\cite{homola1999surface}~\cite{homola2006surface}.  Among the
most popular of these sensors are those which excite localized surface
plasmon polaritons on thin metal films in prism coupled
configurations~\cite{hoa2007towards}.  These platforms, despite their
ubiquity and commercial success, are host to an amazing depth of useful
phenomena which has yet to be explored in the context of biosensing.  In
addition to the high field enhancement responsible for 

The
sensitivity of SPR is primarily due to the field enhancement by SPPs on the
sensor surface, but the SPPs themselves also possess high spatial
resolution beyond the diffraction limit; a property which traditionally has
not manifest itself as a specific feature of the SPR sensorgram.  

However, in this work it is shown that by considering optical speckle from
singly and multiply scattered SPPs inherent in the SPR signal, an entirely
new set of information can be obtained describing the underlying scattering
microstructure.  It is further demonstrated that one can resolve the motion
and addition of single nanoparticles in an unmodified SPR setup, extending
the breadth of SPR experiments to encompass both bulk sensing and discrete
events.



look at the structure of the re-radiated optical field which contain
additional features such as near field self-interference, speckle, and
optical vortices, and their relations to the scattering microstructure.

Our results suggest several possible avenues for advancing the detection
limits of surface plasmon based biosensors for both single particles and
bulk refractive index measurements.

A photon is a quantized oscillation of an electromagnetic field.  When an
electromagnetic  field is in proximity to an interface such as the surface
of a metal, oscillations of free charge can be induced.  If the field is
evanescent in both directions orthogonal to the surface, the oscillations
become localized and are known as surface plasmons (SPs).  Furthermore, if
conditions exist such that the in-plane momentum and phase of an incident
photon and the surface plasmon match, the coupling produces a hybrid
excitation known as surface plasmon polariton (SPP).  An SPP is trapped on
the interface and propagates until it decays; either re-radiating as a
photon or being absorbed into the metal as heat.

SPPs, like photons, are quantum mechanical objects.  Given system is
momentum-conserving, elastically scattered plasmons will preserve all the
information of their parent field.  

There are few experimental techniques allowing application of force on
biological molecules. Among them, optical or magnetic tweezers and atomic
force microscopes (AFMs) have provided much insight into the mechanics of
molecules such as DNA~\cite{cui2000pulling}~\cite{marko1995stretching},
improved understanding of friction and wear in
proteins~\cite{suda2001origin}~\cite{bormuth2009protein}, and have even
been able to observe stepwise motion of motor
proteins~\cite{asbury2003kinesin}.

However powerful, these methods are not able to be deployed as integrated
devices to probe the mechanical properties of heterogeneous samples in a
wide range of applications; tweezer and AFM experiments rely on highly
trained experimentalists, are not widely applicable as analytical tools,
and are often constrained to the analysis of well prepared homogeneous
samples not amenable to multiplexing.

Among tools suitable for direct mechanical transduction, the quartz crystal
microbalance (QCM) has seen increasing real-world utility as a simple, cost
effective, and highly versatile mechanical biosensing platform. A QCM is
typically a thin disk-shaped piece of strategically cut piezoelectric
quartz with electrodes on either side. When part of an electronic
oscillator circuit, the quartz can form a mechanical resonator which
vibrates at certain fundamental frequencies. Changes in the fundamental
frequencies and their associated bandwidths upon sample adsorption or
desorption are related to the properties of the sample and the strength of
its coupling to the QCM. Since its introduction by
Sauerbrey~\cite{sauerbrey1959verwendung} in 1959 as sub-monolayer thin-film
mass sensors in the gas phase, the understanding of QCM sensors has been
repeatedly enhanced to study phenomena such as viscoelastic films in the
liquid phase~\cite{kanazawa1985frequency}, non-destructive contact
mechanics~\cite{johannsman2007contacts}, and complex topologies of
biopolymers and biomacromolecules~\cite{marx2003quartz}. 

Naturally, QCMs do not come without their own disadvantages. The underlying
mechanical properties of the sample are often not revealed by the stepwise
changes in the QCM sensorgram, an issue complicated by the choice of
theoretical model.  Operation of QCMs in the liquid phase is also
associated with a rather low-Q resonance, limiting their sensitivity and
precluding their use for single molecule detection.  Furthermore, up to now
it has not been possible to integrate the application of force on
biomolecules in QCM measurements.

In light of these issues and the analytical power of force based
techniques, herein is described a novel type of instrument using a QCM as a
direct mechanical transducer for the response of discrete samples
(molecules, particles) placed in a variable force field provided by a
standard commercial centrifuge.  This \textit{centrifugal force quartz
crystal microbalance} (CF-QCM) concept is concerned with direct
introduction of pico to nanoscale forces in the liquid phase for analyzing
the mechanical properties of biomaterials.

\subsection{Organization}
This chapter begins with a small historical perspective and proceeds
with a derivation of the relevant mathematics which will allow the
prediction of the response of a QCM for samples under centrifugal force.
Subsequently, \Chapter{ch:qcmexperimental} outlines the technical details of the
CF-QCM; hardware, electronics, and data acquisition.  Drawing on the theory
presented, the main results of this work are presented in
\Chapter{ch:qcmloadsituations} for different types of samples.  A
finite element simulation was also employed to compliment the
theoretical model.  The simulation is presented in
\Chapter{ch:qcmsimulation} along with predictions of CF-QCM behavior for
samples which have not yet been subjected to experiment.

This work is roughly organized in the following way.
\Chapter{ch:existence} is what you are currently reading.  Here is laid out
theory, mathematical details regarding the conditions under which SPPs may
be excited, and physical properties.  In \Chapter{ch:experimental} we
describe all details regarding the physical experiment: construction,
protocols, , and data analysis.  These first two chapters form the
mathematical and physical basis for the remaining text.

Discussion of our studies and new results begins with \Chapter{ch:bulkri}.
Here, the bulk refractive index sensing properties of the cone are
described.  Closely related, \Chapter{ch:interference} discusses a newly
discovered interference phenomena in the cone and the possible utility in
the context of SPR refractive index sensing.

Perhaps one of the most interesting features of the cone is speckle, the
subject of \Chapter{ch:speckle}.  Here we compare the properties of cone
speckle with those of classic speckle fields in the context of
correlations, refractive index perturbations, and multiple scattering
effects.

Finally, in \Chapter{ch:scatteringmicro} we look at the influence of the
cone speckle on the scattering microstructure itself.

