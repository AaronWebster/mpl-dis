The commercialization of biosensing platforms is driven by application.
Toward this goal, some of the more interesting physics of said platforms
are often de-emphasized in the quest for a quantitative and repeatable
assay.  As 

As the engineering of the platform is refined, something, there can be a
rift in the knowledge between the people who are interested in new physical
phenomena, the physics of the system, and those who are mainly concerned
with refining the device.  No place is that more true that with two
biotransduction devices.  Two of the most popular biotransduction devices
are optical and mechanical.

Surface plasmon polaritons (SPPs) are excited by evanescent wave in
incident light of photons which interact with free electrons in a metal and
generate collective oscillation propagating along the metal/dielectric
interface [1,2]. Consequently, they possess the character of not only
ultra-high resolution beyond the diffraction limit, but substantially
enhancement of the evanescent field intensity. 

The commercial success of biosensors based on surface plasmon resonance
seems to have brought about a knowledge gap between the biosensing
community and their more theoretical predecessors from whom the field owes
its genesis. This is to say that the scope of SPR biosensing experiments is
disproportionately narrower than the breadth of phenomena discovered since
Richie. 
