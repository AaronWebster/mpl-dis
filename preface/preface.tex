\subsection*{Organization}

This work is organized into two distinct parts: \Part{part:spr}, which
contains the investigations regarding interference and scattering in surface
plasmon resonance, and \Part{part:qcm} which contains investigations regarding
the centrifugal force quartz crystal microbalance.

\Part{part:spr} begins with \Chapter{ch:sprintro}, an introduction to the
historical perspective motivating the work.  \Chapter{ch:existence} proceeds
with a short mathematical derivation of the existence of \glspl{spp} and the
conditions under which they may be excited.  The derivation furthers the
motivation for specific choices of materials and experimental configurations
in \Chapter{ch:experimental}.  Accordingly, \Chapter{ch:experimental}
describes the physical experiment: mechanical construction, protocols, and
data analysis.  \Chapter{ch:existence} and \Chapter{ch:experimental} form the
mathematical and physical basis for the remaining part of this work.

Discussion and analysis of new results for \Part{part:spr} are the subject of
the remaining three chapters.  \Chapter{ch:interference} describes a newly
discovered interference phenomena and its utility in the context of
conventional \gls{spr} bulk refractive index sensing measurements.

Following, \Chapter{ch:speckle} describes one of the most significant
properties of the experiment: cone speckle.  A formalism for describing cone
speckle is proposed, relating its properties to classical optical speckle
fields using statistical properties, correlations, refractive index
perturbations, and single and multiple scattering phenomena.

Finally, \Chapter{ch:scatteringmicro} describes investigations regarding the
influence of cone speckle on the underlying scattering microstructure.  It is
demonstrated that in an unmodified conventional \gls{spr} setup, it is possible to
resolve the motion and presence of single nanoparticles, extending the breadth
of \gls{spr} experiments to encompass both bulk sensing and discrete events.

Moving from optical to mechanical biosensors, \Part{part:qcm} describes
experiments with a new type of instrument, the centrifugal force quartz
crystal microbalance (CF-QCM), as a tool designed to investigate the
mechanical properties of discrete and bulk samples interacting with a QCM
under centrifugal force.  \Part{part:qcm} begins with
\Chapter{ch:qcmfoundations}, motivating the instrument and providing
a historical context for its invention.  \Chapter{ch:qcmfoundations} also
describes the relevant physical model used allowing one to predict its
response under different samples and load situations.

As in \Chapter{ch:experimental} of \Part{part:spr},
\Chapter{ch:qcmexperimental} of \Part{part:qcm} details all experimental
aspects of the instrument: its construction, operation, and data
acquisition circuitry.  Inherent additions to the \gls{qcm} sensorgram,
environmental effects and noise, are also considered.

The experimental results of \Part{part:qcm} are presented in
\Chapter{ch:qcmloadsituations}, describing the behavior of the CF-QCM for a
wide range of samples: bulk liquid, discrete microparticles, and
viscoelastic layers of DNA\@.  The different samples are investigated both
free and, in the case of microparticles, attached to the surface with
DNA\@.

Finally, predictions regarding CF-QCM behavior for samples which have not yet
been subject to experiment are contained in \Chapter{ch:qcmsimulation},
facilitated by a finite element simulation.  Complimenting the experiment, the
finite element simulation is used to predict the response of discrete
biomollecules: cells, agarose microparticles, and lysozyme microcrystals.
These bioparticles and molecules are representative of the enormous range of
viscoelastic material properties found in nature and highlight the potential
uses of the CF-QCM sensing concept.
