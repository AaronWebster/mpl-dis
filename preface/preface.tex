The first part of this chapter has yet to be written.  I will talk about
biosensors in general, different types, get in to the state of the art, and
then launch into what is wrong with them and what can be done about it
i.e.\ what I have done here. Then the structure of the thesis will be
outlined.

This chapter will tie together Parts I and II.\@

%SPR stuff is so general, different types of problems

%Biosensors based on surface plasmon resonance (SPR) play a central role as
%a simple and remarkably responsive label-free method for characterizing and
%quantifying biomolecular
%interactions~\cite{homola1999surface}~\cite{homola2006surface}.  Among the
%most popular of these sensors are those which excite localized surface
%plasmon polaritons on thin metal films in prism coupled
%configurations~\cite{hoa2007towards}.  These platforms, despite their
%ubiquity and commercial success, are host to an amazing depth of useful
%phenomena which has yet to be explored in the context of biosensing.  In
%addition to the high field enhancement responsible for 
%
%The
%sensitivity of SPR is primarily due to the field enhancement by SPPs on the
%sensor surface, but the SPPs themselves also possess high spatial
%resolution beyond the diffraction limit; a property which traditionally has
%not manifest itself as a specific feature of the SPR sensorgram.  
%
%However, in this work it is shown that by considering optical speckle from
%singly and multiply scattered SPPs inherent in the SPR signal, an entirely
%new set of information can be obtained describing the underlying scattering
%microstructure.  It is further demonstrated that one can resolve the motion
%and addition of single nanoparticles in an unmodified SPR setup, extending
%the breadth of SPR experiments to encompass both bulk sensing and discrete
%events.
%
%
%
%look at the structure of the re-radiated optical field which contain
%additional features such as near field self-interference, speckle, and
%optical vortices, and their relations to the scattering microstructure.
%
%Our results suggest several possible avenues for advancing the detection
%limits of surface plasmon based biosensors for both single particles and
%bulk refractive index measurements.
%
%A photon is a quantized oscillation of an electromagnetic field.  When an
%electromagnetic  field is in proximity to an interface such as the surface
%of a metal, oscillations of free charge can be induced.  If the field is
%evanescent in both directions orthogonal to the surface, the oscillations
%become localized and are known as surface plasmons (SPs).  Furthermore, if
%conditions exist such that the in-plane momentum and phase of an incident
%photon and the surface plasmon match, the coupling produces a hybrid
%excitation known as surface plasmon polariton (SPP).  An SPP is trapped on
%the interface and propagates until it decays; either re-radiating as a
%photon or being absorbed into the metal as heat.
%
%SPPs, like photons, are quantum mechanical objects.  Given system is
%momentum-conserving, elastically scattered plasmons will preserve all the
%information of their parent field.  
%
%There are few experimental techniques allowing application of force on
%biological molecules. Among them, optical or magnetic tweezers and atomic
%force microscopes (AFMs) have provided much insight into the mechanics of
%molecules such as DNA~\cite{cui2000pulling}~\cite{marko1995stretching},
%improved understanding of friction and wear in
%proteins~\cite{suda2001origin}~\cite{bormuth2009protein}, and have even
%been able to observe stepwise motion of motor
%proteins~\cite{asbury2003kinesin}.
%
%However powerful, these methods are not able to be deployed as integrated
%devices to probe the mechanical properties of heterogeneous samples in a
%wide range of applications; tweezer and AFM experiments rely on highly
%trained experimentalists, are not widely applicable as analytical tools,
%and are often constrained to the analysis of well prepared homogeneous
%samples not amenable to multiplexing.
%
%Among tools suitable for direct mechanical transduction, the quartz crystal
%microbalance (QCM) has seen increasing real-world utility as a simple, cost
%effective, and highly versatile mechanical biosensing platform. A QCM is
%typically a thin disk-shaped piece of strategically cut piezoelectric
%quartz with electrodes on either side. When part of an electronic
%oscillator circuit, the quartz can form a mechanical resonator which
%vibrates at certain fundamental frequencies. Changes in the fundamental
%frequencies and their associated bandwidths upon sample adsorption or
%desorption are related to the properties of the sample and the strength of
%its coupling to the QCM. Since its introduction by
%Sauerbrey~\cite{sauerbrey1959verwendung} in 1959 as sub-monolayer thin-film
%mass sensors in the gas phase, the understanding of QCM sensors has been
%repeatedly enhanced to study phenomena such as viscoelastic films in the
%liquid phase~\cite{kanazawa1985frequency}, non-destructive contact
%mechanics~\cite{johannsman2007contacts}, and complex topologies of
%biopolymers and biomacromolecules~\cite{marx2003quartz}. 
%
%Naturally, QCMs do not come without their own disadvantages. The underlying
%mechanical properties of the sample are often not revealed by the stepwise
%changes in the QCM sensorgram, an issue complicated by the choice of
%theoretical model.  Operation of QCMs in the liquid phase is also
%associated with a rather low-Q resonance, limiting their sensitivity and
%precluding their use for single molecule detection.  Furthermore, up to now
%it has not been possible to integrate the application of force on
%biomolecules in QCM measurements.
%
%In light of these issues and the analytical power of force based
%techniques, herein is described a novel type of instrument using a QCM as a
%direct mechanical transducer for the response of discrete samples
%(molecules, particles) placed in a variable force field provided by a
%standard commercial centrifuge.  This \textit{centrifugal force quartz
%crystal microbalance} (CF-QCM) concept is concerned with direct
%introduction of pico to nanoscale forces in the liquid phase for analyzing
%the mechanical properties of biomaterials.

\subsection*{Organization}

The present work is organized into two distinct parts: \Part{part:spr},
which contains the investigations regarding interference and scattering in
surface plasmon resonance, and \Part{part:qcm} which contains
investigations regarding the centrifugal force quartz crystal microbalance.

\Part{part:spr} begins with \Chapter{ch:sprintro}, an introduction
historical perspective motivating the work.  Following,
\Chapter{ch:existence} proceeds with a short mathematical derivation of the
existence of SPPs and the conditions under which they may be excited.  The
derivation further lays out the motivation for specific choices of
materials and experimental configurations in \Chapter{ch:experimental}.
Accordingly, \Chapter{ch:experimental} describes the physical experiment:
mechanical construction, protocols, and data analysis.
\Chapter{ch:existence} and \Chapter{ch:experimental} form the mathematical
and physical basis for the remaining part.

Discussion and analysis of new results for \Part{part:spr} are the subject
of the remaining three chapters.  Beginning with \Chapter{ch:bulkri}, the
bulk refractive index sensing properties of the cone are described.
Closely related, \Chapter{ch:interference} presents a newly discovered
interference phenomena in the cone and its possible utility in the context
of conventional SPR bulk refractive index sensing measurements.

Following, \Chapter{ch:speckle} describes one of the most fascinating
properties of the experiment: cone speckle.  A formalism for describing
cone speckle is proposed, relating its properties to classical optical
speckle fields using statistical properties, correlations, refractive index
perturbations, and single and multiple scattering phenomena.

Finally, \Chapter{ch:scatteringmicro} describes investigations regarding
the influence of cone speckle on the underlying scattering microstructure.
It is demonstrated that in an unmodified conventional SPR setup, it is
possible to resolve the motion and presence of single nanoparticles,
extending the breadth of SPR experiments to encompass both bulk sensing and
discrete events.  

Moving from optical to mechanical biosensors, \Part{part:qcm} describes
experiments with a new type of instrument, the centrifugal force quartz
crystal microbalance (CF-QCM), as a tool to investigate the mechanical
properties of discrete and bulk samples interacting with a QCM under
centrifugal force.  \Part{part:qcm} begins with
\Chapter{ch:qcmfoundations}, motivating the instrument and providing a
historical context for its invention.  Following,
\Chapter{ch:qcmfoundations} additionally describes the relevant physical
model used allowing one to predict its response under different samples and
load situations.

As in \Chapter{ch:experimental} of \Part{part:spr},
\Chapter{ch:qcmexperimental} of \Part{part:qcm} details all experimental
aspects of the instrument: its construction, operation, and data
acquisition circuitry.  Inherent additions to the QCM sensorgram, 
environmental effects and noise, are also considered.

The experimental results of \Part{part:qcm} are presented in
\Chapter{ch:qcmloadsituations}, describing the behavior of the CF-QCM for a
wide range of samples: bulk liquid, discrete microparticles, and
viscoelastic layers of DNA\@.  The different samples are investigated both
free and, in the case of microparticles, attached to the surface with
DNA\@.

Finally, predictions regarding CF-QCM behavior for samples which have not
yet been subject to experiment are contained in \Chapter{ch:qcmsimulation},
facilitated by a finite element simulation.  Complimenting the experiment,
the finite element simulation is used to predict the response of discrete
biomollecules: cells, agarose microparticles, and lysozyme microcrystals.
These three samples are representative the enormous range of viscoelastic
material properties found in nature, and highlight the potential uses of
the CF-QCM sensing concept.
