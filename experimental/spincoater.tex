During our experimental work, we did not have convienent access to a
standard programmable spin coater.  This is of course a pre-requisite to
producing uniform cytop coatings and thus long range surface plasmons.
Being a rather expensive piece of equipment, we opted to build one
ourselves using the brushless DC motor from a repurposed hard disk drive.
Spin coaters built this way have been reported
before~\cite{bianchi2006spin}, but we have made enough improvements on the
design to justify its inclusion here.  Most significantly, the entire
electronic drive and control are put together from off the shelf
components.

A block diagram of our home-built spin coater is shown in \Figure{fig:x}.
The cover from a hard disk drive is removed and all of the internal
components except for the motor are removed.  A solid aluminum plate is
affixed to the top of the motor using the existing screw taps.  The motor
is then connected directly to a programmable electronic speed controller
(ESC).  The ESC itself is a standard component found in most land and air
based radio controlled drones.  Once configured, the ESC accepts a pulse
width modulated (PWM) input with pulse widths varying from \num{1000} (off)
to \SI{2000}{\micro\second} (full speed).  The PWM signal for the ESC is
generated by an Arduino Uno platform, based on the Atmel ATMega128
\SI{8}{bit} microcontroller.  The Arduino can be programmed to do spins on
its own, but we found it more convienent to send the desired spin speeds in
real time to the Arduino via its USB serial interface.  Finally, connected
to the motor spindle is a small flange and an optical interrupter switch
which serves to monitor the spin speed.  

We have experimented with using the optical interrupter as feedback for a
proportional-integral-derivative (PID) loop in hopes that this would be a
superior way of controlling the motor, but this idea was abandoned in favor
of a proportional-only feedback loop.  The spin speed of the motor was also
calibrated offline and a linear relationship with the PWM output determined
such that it could function equally well without feedback.  

The stability of the spin coater and the range of speeds it offers is shown
in \Table{tbl:spincoatererror}.  Perhaps the most obvious downside of the
ESC is that it does not operate very well in the low RPM limit.
Fortunately, this is not an issue for us.

brake on
timing  high
low voltage

\begin{table}
 \begin{tabular}{ll}
  \toprule
  angular speed (RPM) & error \\
  \midrule
  1000 & x \\
  2000 & x \\
  3000 & x \\
  4000 & x \\
  5000 & x \\
  6000 & x \\
  7000 & x \\
  \bottomrule
 \end{tabular}
 \label{tbl:spincoatererror}
\end{table}