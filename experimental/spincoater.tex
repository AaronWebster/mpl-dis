During our experimental work, we did not have convienent access to a
standard programmable spin coater.  This is of course a pre-requisite to
producing uniform cytop coatings and thus long range surface plasmons.
Being a rather expensive piece of equipment, we opted to build one
ourselves using a repurposed spinning platter type hard disk drive.
Spin coaters built this way have been reported
before~\cite{bianchi2006spin}, but we have made enough improvements on the
design to justify its inclusion here.  Most significantly, the entire
electronic drive and control are put together from off the shelf
components.

The cover from all internal components except for the motor are first
removed.  A solid aluminum plate is affixed to the top of the motor using
the existing screw taps.  The three leads of the hard disk's three phase
motor are connected directly to a programmable electronic speed controller
(ESC).  The ESC itself is a standard component found in most land and air
based radio controlled drones.  Once configured, the ESC accepts a pulse
width modulated (PWM) input with pulse widths varying from
\SI{1000}{\micro\second} (off) to \SI{2000}{\micro\second} (full speed).
The PWM signal for the ESC is generated by an Arduino Uno platform, based
on the Atmel ATMega128 \SI{8}{\bit} microcontroller.  The Arduino is
programmed for several preset spin sequences which are initialized by
pressing a button.  Finally, connected to the motor spindle is a small
flange and an optical interrupter switch which serves to monitor the spin
speed.  

We have experimented with using the optical interrupter as feedback for a
proportional-integral-derivative (PID) loop in hopes that this would be a
superior way of controlling the motor, but this idea was abandoned in favor
of a proportional-only feedback loop.  The spin speed of the motor was also
calibrated offline and a linear relationship with the PWM output determined
such that it could function equally well without feedback.  

The stability of the spin coater and the range of speeds it offers is shown
in \Table{tbl:spincoatererror}.  Perhaps the most obvious downside of the
ESC is that it does not operate very well in the low RPM limit.
%brake on
%timing  high
%low voltage
\begin{table}
\sisetup{round-mode=places,round-precision=3,fixed-exponent=0,scientific-notation=fixed}
 \centering
 \begin{tabular}{SS}
  \toprule
	{angular speed (RPM)} & {standard deviation} \\
  \midrule
1269.8894 & 1.14048935075416\\
2107.5322 & 1.86546956391079\\
2812.9204 & 79.3212956700501\\
3471.1698 & 0.879734398693702\\
4075.6306 & 1.71437548574471\\
4598.6858 & 0.877575719158775\\
5152.03 & 0.454653938816874\\
5683.387  & 0.891548183367202\\
6176.233  & 0.733084759973966\\
6691.978  & 1.61663545800572\\
  \bottomrule
 \end{tabular}
 \caption{Standard deviation of the spin speed at certain angular speeds
 for the hard disk centrifuge.}
 \label{tbl:spincoatererror}
\end{table}

\begin{figure}
 \centering
 \import{includes/}{setpgfinc}
 \import{experimental/figures/}{cytopthicknessfig}
 \caption{Measured Cytop layer thickness verses spin speed.  Error bars
 denote the accuracy of the spin coating process.}
 \label{fig:cytopfit}
\end{figure}
