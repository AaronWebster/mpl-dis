During the course of the present experimental work, a standard programmable
spin coater was not available.  Being a rather expensive piece of equipment, a
ersatz spin coater was made using a spinning platter type hard disk drive.
Spin coaters built in this fashion have been reported
before~\cite{bianchi2006spin}, but herein is described a much simpler and
economic implementation using off-the-shelf components.

The cover from all internal components except for the motor are first removed.
A solid aluminum plate is affixed to the top of the motor using the existing
screw taps.  The three leads of the hard disk's three phase motor are
connected directly to a programmable electronic speed controller (ESC).  The
ESC itself is a standard component found in most land and air based radio
controlled drones.  Once configured, the ESC accepts a pulse width modulated
(PWM) input with pulse widths varying from \SI{1000}{\micro\second} (off) to
\SI{2000}{\micro\second} (full speed).  The PWM signal for the ESC is
generated by an Arduino Uno platform, based on the Atmel ATMega128
\SI{8}{\bit} microcontroller.  The Arduino is programmed for several preset
spin sequences which are initialized by pressing a button.  Finally, a small
flange connected to the motor spindle is used in conjunctino with an optical
interrupter switch to monitor the spin speed.


The spin speed of the motor was recorded as a function of PWM signal and
the relationship fit to a second order polynomial.  The fit parameters were
then used to set the PWM output given a desired spin speed without
feedback.

The stability of the spin coater and the range of speeds it offers is shown
in \Table{tbl:spincoatererror}.  Perhaps the most obvious downside of the
ESC is that it does not operate very well in the low RPM limit,
particularly at around \SI{2800}{RPM}.
%brake on
%timing  high
%low voltage
\begin{table}[h]
  \sisetup{round-mode=places,round-precision=3,fixed-exponent=0,scientific-notation=fixed}
  \centering
  \begin{tabular}{SS}
    \toprule
    {spin speed (RPM)} & {standard deviation} \\
    \midrule
    1269.8894          & 1.14048935075416     \\
    2107.5322          & 1.86546956391079     \\
    2812.9204          & 79.3212956700501     \\
    3471.1698          & 0.879734398693702    \\
    4075.6306          & 1.71437548574471     \\
    4598.6858          & 0.877575719158775    \\
    5152.03            & 0.454653938816874    \\
    5683.387           & 0.891548183367202    \\
    6176.233           & 0.733084759973966    \\
    6691.978           & 1.61663545800572     \\
    \bottomrule
  \end{tabular}
  \caption{Standard deviation of the spin speed at nominal angular speeds for the hard disk spin coater.}
  \label{tbl:spincoatererror}
\end{table}

The reproducability of Cytop layers obtained in this way was found to be on
average approximately \SI{70}{\nano\meter} across the entire spin speed range.
The is greater than the fit error of the Fresnel reflectivity, which was
always found to be on the order of or less than \SI{1}{\nano\meter}.

% 5000 4.6406e-08
% 4000 7.6441e-08
% 3000 6.4829e-08
% 7000 9.4872e-08
%\begin{table}[h]
%\centering
%\begin{tabular}{SSS}
%\toprule
%{spin speed (RPM)} & {average thickness (\si{\meter})} & {error (\si{\meter})} \\
%\midrule
%3000  & 1.54e-06 & 2.429e-09\\
%4000  & 1.18e-06 & 6.000e-10\\
%5000  & 1.05e-06 & 5.540e-10\\
%6000  & 9.81e-07 & 8.349e-10\\
%\bottomrule
%\end{tabular}
%\caption{Accuracy of Cytop layers produced using the spin coating method.}
%\label{tbl:spincoataccuracy}
%\end{table}

%\begin{figure}
% \centering
% \import{includes/}{setpgfinc}
% \import{experimental/figures/}{cytopthicknessfig}
% \caption{Measured Cytop layer thickness versus spin speed.  Error bars
% denote the accuracy of the spin coating process.}
% \label{fig:cytopfit}
%\end{figure}
