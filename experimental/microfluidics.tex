Microfluidic flow cells made of polydimethylsiloxane (PDMS) were used to
introduce analytes into the sensing volume.  PDMS is a bio-compatible
organosilicon compound obtained as a two part elastomer kit\footnote{Dow
Corning Sylgard\textregistered 184}.  The two part liquid is combined in a
10:1 ratio by mass in a small plastic mixing cup.  The correct proportions
are determined by an analytical balance to a precision of \SI{0.01}{\gram}.
They are then combined and mixed using a magnetic stir paddle for
\SI{5}{\minute}.  The mixture is poured on to a polished silicon wafer in a
shallow glass dish and evacuated in a vacuum bell jar until all dissolved
gas has been removed.  The PDMS is then baked for \SI{20}{\minute} at
\SI{120}{\celsius}.  Once removed, the cured PDMS is cut free from the mold
using a scalpel and transferred to a clean glass slide where it is stored
in a dust free environment until use.

The PDMS is cast into \SI{1}{\milli\meter} thick layers which are cut into
\SI{25x25}{\milli\meter} squares.  Horizontally embedded and centered in the
PDMS layer was a length of \SI{0.6}{\milli\meter} outside diameter
silicone tubing.  A circular biopsy punch was used to extract a
\SI{6}{\milli\meter} hole from the center, which cut the tubing in its center.
The hypotenuse of the prism with its complete layer substructure was then
placed over the remaining side of the central hole; the surface attraction
between the PDMS and the prism substrate was sufficient to prevent leakage.  A
siphon was set up between the input and output channels in the PDMS and the
experiment run with a flow rate of \SI{1.85}{\micro\liter\per\second}.  The
microfluidic flow cell as integrated into the experiment is shown schematically
in \Figure{fig:microfluidiccell}.

As a cursory note, when handling PDMS or anything PDMS comes into contact
with, a clean pair of nitrile gloves should be used.  It was observed that
surfaces which had been touched with powder-free latex gloves could no
longer be directly bonded.  The present anecdote seems to be prevalent in
the literature, but no description of the underlying mechanism is known.

\begin{figure}[ht]
\centering
\import{experimental/figures/}{microfluidics.pdf_tex}
\caption{Schematic of the microfluidic cell as integrated in the experiment.}
\label{fig:microfluidiccell}
\end{figure}
