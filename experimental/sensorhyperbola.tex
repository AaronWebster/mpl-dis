The relative two dimensional spatial coordinates in the plane of the
imaging sensor are known given the size and number of its pixels.  The
absolute three dimensional coordinates with respect to the scattering spot
are more difficult to assess.  The way in which the experiment is
constructed makes it difficult to accurately measure the distance between
the imaging sensor and the focal spot.  However, with knowledge of the
angle in which the sensor is tilted, this distance can be extracted
\textit{a posteriori} from measuring the shape of the light falling on the
sensor.
\begin{figure}[ht]
\centering
\includegraphics[keepaspectratio,scale=1.25]{figures/hyperbolageoa.pdf}
\caption{Geometry for determining the sensor distance $d$.}
\label{fig:propgeo}
\end{figure}

Consider the experimental geometry.  Light from scattered SPPs eminates as
a cone and is imaged by a planar sensor.  This is by definition a conic
section, with SPP light as the cone and the sensor as the cutting
plane.  Following, the shape of the light on the sensor will be either a
circle, an ellipse, a parabola, or a hyperbola.  In the experimental setup
we can easily set things up such that sensor is fixed
orthogonal to the SPP cone.  The geometrical situation is depicted in
\Figure{fig:propgeo}.  SPPs scatter into the far field from point $O$ at an
angle $\thetasp$.  The detector, which has a finite transverse length, is at $B$ and is
orthogonal to $\overline{OB}$ at an angle $\phi=\pi/2-\thetasp$ with respect to
the normal vector from the prism surface $\overline{OA}$.  Because
$\phi<\thetasp$ and $\thetasp>\pi/2$, lines $\overline{BA}$ and $\overline{OC}$ never
intersect. The resulting shape is a hyperbola.  We wish to determine, given
the parameters of the hyperbolic section, the distance between the focal
spot and the sensor, the line $\overline{OB}$ with length $d$.  

We denote the coordinates of the sensor as $(\xi,y)$.  Here we have the
following relationships
\begin{align}
x &= \xi \cos \thetasp \label{eqn:conic01a} \\
r &= \left(d \sec \thetasp + \xi \sin \thetasp\right) \tan\thetasp \label{eqn:conic01b}
\end{align}
and
\begin{equation}
x^2+y^2=r^2
\label{eqn:conic01c}
\end{equation}
Note $d \sec\thetasp$ is the length of $\overline{OA}$.  These two sets of
equations can be combined to obtain
\begin{equation}
\xi^2 \left(\cos ^2\thetasp-\sin ^2\thetasp \tan
^2\thetasp\right)+2 \xi d \tan ^3\thetasp+y^2=d^2 \tan
^2\thetasp \sec ^2\thetasp
\end{equation}
Which, after a bit of algebra, can be solved for $\xi$, taking
the positive root
\begin{equation}
\xi(y) = \frac{4 \sqrt{2} \sqrt{-2 y^2 \cos 2 \thetasp  \cos ^6\thetasp +d^2
\cos ^6\thetasp -d^2 \cos 2 \thetasp  \cos^6\thetasp}-2 d \sin 2 \thetasp
+d \sin 4 \thetasp }{2 \left(2 \cos 2 \thetasp +\cos 4 \thetasp +1\right)}
\label{eqn:conic02}
\end{equation}

To find the offset, we solve the above minimum and find $\xi(y=0) = 4 d
\sec\thetasp$, subtracting this from \Equation{eqn:conic02}.  The remaining
equation is then solved for $d$.
\begin{align}
d =& \Biggl(-4 \sqrt{2} \Bigl(\delta^2 \sin ^2\thetasp \cos
^4\thetasp+\delta^2 \sin ^2\thetasp \cos \thetasp \cos ^4\thetasp-4 y^2 \nonumber \\
   & \sin ^2\thetasp \cos ^5\thetasp-6 y^2 \sin ^2\thetasp
     \cos ^4\thetasp-16 y^2 \sin ^2\thetasp \cos \thetasp \nonumber \\
  & \cos ^4\thetasp+4 y^2 \sin ^2\thetasp \cos \thetasp
    \cos ^4\thetasp-8 y^2 \sin ^2\thetasp \cos \thetasp \cos
    ^4\thetasp\Bigr)^{1/2} \nonumber \\
&-2 \delta\sin \thetasp-4 \delta \sin 3 \thetasp+\delta  \sin \thetasp-4 \delta \sin \thetasp\Biggr) \nonumber \\
&\cdot\Bigl(4 \left(2 \cos \thetasp+3 \cos \thetasp-3 \cos \thetasp+\cos 5 \thetasp-2 \cos \thetasp-1\right)\Bigr)^{-1}
\end{align}
where $\delta$ is the distance the cone goes up at the coordinate $y$.  
In this way, given the dimensions of the pixels on the imaging sensor, $d$
can be determined.  Once this is known, the absolute coordinates $(x,y,z$ of the
sensor are given by 
\begin{equation}
	\begin{pmatrix} 
					x\\ 
					y\\ 
					z
	\end{pmatrix} 
	=
	\begin{pmatrix} 
					x^\prime \cos\thetasp + d \sin\thetasp\\ 
					y^\prime \\
					d \cos\thetasp - x^\prime \sin\thetasp\\ 
	\end{pmatrix} 
\label{eqn:sensorcoordinates}
\end{equation}
where $x^\prime$ and $y^\prime$ are the local coordinates of the sensor
with $(0,0)$ as its origin.  
