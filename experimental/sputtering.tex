The metal films were deposited directly on substrates by means of a plasma
sputtering process.  It is through this process that the surface roughness
of the metal film was also controlled.  Our base process consists of
top-down sputtering in Argon plasma at a pressure of \SI{1}{\milli\torr}
and a flow rate of \SI{12}{STP}.  The deposition rates were
\SI{6.60}{\angstrom\per\second} for gold and
\SI{5.88}{\angstrom\per\second} for silver at 150 and \SI{66}{\watt} DC,
respectively.  During sputtering, the samples were rotated at \SI{50}{RPM}.
Adhesion of the metal films to both glass and flouropolymer substrates was
found to be very poor, but we did not attempt to address this issue.
Instances of film-substrate separation were rare enough (4-5 times amongst
hundreds of experiments) that it did not warrant the effort given our
careful attention to stresses on the film.  Typical surface qualities are
shown in \Figure{fig:sputter} for both gold and silver.  
\begin{figure}
\centering
\ersatzfigure
\caption{SEM images of typical films produced with the plasma sputtering
technique using the base set of parameters described in the text.}
\label{fig:semsputter}
\end{figure}

The substrate material consisted of \SI{12}{\milli\meter} diameter BK7
glass slides with nominal suface roughness of XXX.  They were cleaned in
the following way.  A \SI{2}{\percent} solution of Hellmanex II was first
prepared in a pyrex beaker with a volume of \SI{25}{\milli\liter} and
heated to a temperature of \SI{60}{\celsius} on a hot plate.  The BK7
slides are placed in this solution on the hot plate and washed for
\SI{300}{\second}, swirling gently by hand occasionally.  The container
with the solution and slides was then sonicated for \SI{300}{\second}.
After sonication, the slides are removed from the solution and washed
liberally with distilled water and dried under a nitrogen stream.  Cleaned
glass slides were stored individually on top of a small
\SI{1}{\milli\meter\cubed} piece of PDMS plasma bonded to a microscope
slide, and batches kept in a dark box until use.  Slides sputtered with
metal films were stored in the same manner, but always used within one week
of preparation.
