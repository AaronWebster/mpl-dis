% figure of experimental setup
% annotated picture of experimental setup

The paragon setup for all experiments is depicted schematically in
\Figure{fig:sppexperimentalsetup}.  Light from a \SI{50}{\milli\watt}
\SI{660}{\nano\meter} diode laser with a bandwidth of \SI{30}{\giga\hertz}
is first coupled into a single mode optical fiber via an integrated
air-spaced doublet fiber collimnator.  The fiber guides the laser light to
an optical breadboard mounted on an inverted microscope.  All primary
functions of the experiment take place here.  At the breadboard, the
Gaussian beam from the single mode fiber is collimnated by an output
coupler to a beam waist of $w_0=\SI{5}{\milli\meter}$ and proceeds through
a polarizing beamsplitter, passing $p$ polarized light.  The light is
focused by a 10X microscope objective onto the hypotenuse of a
hemispherical prism of diameter $d=\SI{10}{\milli\meter}$.  The
beamsplitter and microscope objective are mounted on rotation stage which
itself is mounted on a micrometer stage.  The objective itself is mounted
on a zoom housing enabling linear travel along the optical axis.  When the
rotation stage is at the surface plasmon resonance angle, the micrometer
$z$ axis is fixed with a diffraction limited spot at the center of the
prism's hypotenuse.  The size of the spot then be modified via the zoom
housing, and its transverse location on the surface by the $x$ and $y$ lead
screws on the micrometer stage.  

The hemispherical prism is mounted on a planar structure supporting either
normal or long range SPPs.  Reverse of this structure is a microfluidic
flow cell where samples can be introduced.  The specularly reflected and
scattered light containing information from SPP interference and scattering
is ultimately captured at desired locations with a series of imaging
sensors.

Both the microfluidics and the substructure supporting the prism are thin and
optically transparent.  In this way, the sensing surface may be imaged with
an objective from the inverted microscope.  The breadboard containing the
experiment was fixed to the optical table with its own micrometer stage,
permitting translation of the entire experiment in the plane of the optical
table.

We experimented with several different focussing strategies in the
refinement of the experiment, and found that a microscope objective is the
simplest choice.  Because microscope objectives fulfill the sine condition,
focussing to the central point on the hypotenuse of a hemispherical prism
should (ideally) introduce no aberrations.

\subsection{Imaging Sensor}
As stated, light from the experiment was directly recorded with imaging
sensors at select locations around the ring.  Because no optics (besides
the hemispherical prism itself) were used in this capture, it is important
that the sensors be as close to the prism as possible, to capture the
widest amount of feature area possible.  To facilitate this, the sensors
were removed from their housings and the naked PCB used as the mount.
% picture here of the modified imaging sensor

Two types of imaging sensors were used in the experiment: an IDS USB 2.0
8-bit CMOS camera with a resolution of 1280x1024 pixels and a PixelGrey
IEEE-1394 (Firewire) 16-bit CCD  with a resolution of 1280x1024 pixels.
Both cameras are capable of what is known as a ``limited area of interest
(AOI)''.  This feature allows the cameras to capture a sub-region of their
available pixel region \textit{before} the information traverses the serial
bus to the host computer.  The IDS datasheet specifies that in AOI mode,
with sufficient lighting conditions framerates of up to \SI{250}{fps} are
possible.  We have found that this framerate was a limitation imposed due
to bad design of both the stock acquisition software and the USB serial
bus, and with simple modifications this can be extended to over \SI{1400}{fps}.

The modifications were carried out in the following way.  On the software
side, the raw sensor data is read directly from camera memory and written
to a single binary file on the disk of the host computer.  This carries
with it two speed advantages: first, it suppresses all calls to image or
video libraries by removing the need to transcode the pixel data.  Second,
saving to a single file is much faster than saving to multiple files
because it elliminates (seeking and updating of metadata).

Second, each camera is assigned to its own USB bus.  This allows the USB
polling frequency to be set at maxiumun and it is not interupted by other
devices or whatever.
%(480 megabits per second)/((1280*128*8 bits) per frame)
