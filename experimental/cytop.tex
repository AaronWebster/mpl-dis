A fluoropolymer, poly(1,1,2,4,4,5,5,6,7,7-decafluoro-3-oxa-1,6-heptadiene),
more commonly referred to by its commercial name Cytop, was employed to
make multilayer substrates supporting long-range surface plasmons.

At an operating wavelength of \SI{660}{\nano\meter} and at
\SI{20}{\celsius}, the refractive index of Cytop is very close to water ---
1.341 for Cytop~\cite{mikevs2005synthesis} versus 1.331 for distilled
water~\cite{andreasson1971measurement}.  The similar refractive index
allows a near-symmetric refractive index profile required for long-range
surface plasmons as described in \Section{sec:fresnelexist}.

Cytop was purchased from AGC Chemicals as item number CTX-809A.  The last
letter ``A'' in the item number designates it is intended for use on amorphous
(glass) applications, and the subsequent number designates the concentration.
For example, CTX-809A is \SI{9}{\percent} cytop and \SI{91}{\percent} solvent,
and likewise CTX-803A is \SI{3}{\percent} cytop and \SI{97}{\percent} solvent.
Atachi manufactures another item with an ``M'' designation designed for
metals, but in either case the multilayer structure would have one side in
contact with a non-optimal material.  Similar to the metal layers described in
\Section{sec:expsputtering}, the adhesion of the cytop to both metals and
glass was poor, as such careful handling combined with a low pressure
differentials in the microfluidic stage was employed to prevent separation.

Thin films were made following a spin-coating procedure.  A small drop of
CTX-809A was placed directly on a clean \SI{12}{\milli\meter} round glass
coverslip and the coverslip spun at \SI{3000}{RPM} for \SI{30}{\second}.
Immediately after spinning, the coverslip was removed and placed on a hot
plate at \SI{50}{\celsius} for \SI{30}{\minute}.  The temperature of the hot
plate was then increased to \SI{200}{\celsius} and the Cytop allowed to cure
for \SI{12}{\hour} to assure all solvents had evaporated.  After cooling, the
metal films were sputtered directly on the Cytop layer without any further
surface modification.

The resultant film thicknesses were determined with a spectrometer by
measuring its broadband reflectance at $\theta=\SI{45}{\degree}$ and using
a nonlinear least squares routine~\cite{more1977LevAlgImpThe_2} to
fit the Fresnel intensity reflection coefficient for a single-layer
film $R$~\cite{steck2006classical}
\begin{equation}
				R = \frac{n_1^2{(n_i-n_s)}^2 \cos^2\delta + {(n_i n_s - n_1^2)}^2\sin^2\delta}
				{n_1^2{(n_i+n_s)}^2 \cos^2\delta + {(n_i n_s + n_1^2)}^2\sin^2\delta}
          \label{eqn:fitfresnel}
\end{equation}
where
\begin{equation}
 \delta = \frac{2\pi}{\lambda_0} n_i d_i \cos \theta
\end{equation}
and the Cytop layer has a refractive index $n_i$ and thickness $d_i$, the
substrate a refractive index $n_s$, and the system in the presence of
refractive index $n_1$ (air).  \Equation{eqn:fitfresnel} assumes the
reflection coefficient for TE polarization, of which \Equation{eqn:fitfresnel}
has only one free parameter: $d_i$.  The thickness $d_i$ was solved for using
a linear least squares fitting algorithm.  An example of the raw spectrometer
data and its best fit is shown in \Figure{fig:fresnelfit}.

\begin{figure}
 \centering
 \import{includes/}{setpgfinc}
 \import{experimental/figures/}{fresnelfit}
 \caption{Reflectivity of a thin Cytop layer on BK7 at
 $\theta=\SI{45}{\degree}$.  Shown is the experimental data and the
	best-fit with \Equation{eqn:fitfresnel}.  Calculated layer thickness was \SI{1150}{\nano\meter}.}
\label{fig:fresnelfit}
\end{figure}

The surface roughness of the Cytop films was not measured, but the experiments
suggest the surface roughness must be very small.  The intensity of light
scattered into the cone is too low to be observed when exciting SPPs in an
unmodified cytop-gold-water setup --- additional surface features must be
added before the cone becomes visible.  In comparison to metal films sputtered
directly on BK7 glass substrates with a typical surface roughness of
\SI{0.3}{\nano\meter} within
\SI{1}{\micro\meter\squared}~\cite{cheang2011study}~\cite{chiu2011optimizing},
the cone is always found to be visible.
