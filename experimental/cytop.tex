For the fabrication of multilayer substrates supporting long range surface
plasmons, we used a flouropolymer,
poly(1,1,2,4,4,5,5,6,7,7-decafluoro-3-oxa-1,6-heptadiene), better known as
Cytop.  At our operating wavelength of \SI{660}{\nano\meter} and at
\SI{20}{\celsius}, its refractive index is very close to water --
1.341 for Cytop~\cite{mikevs2005synthesis} versus 1.331 for distilled
water.  This index matching property allows us to construct a symmetric for
long range surface plasmons.

Cytop was purchased from AGC Chemicals as item number CTX-809A.  The last
letter ``A'' in the item number designates it is intended for use on
``amorphous'' (i.e. glass) applications, and the number immediately
following that designates the concentration.  For example, CTX-809A is
\SI{9}{\percent} cytop and \SI{91}{\percent} solvent, and likewise CTX-803A
is \SI{3}{\percent} cytop and \SI{97}{\percent} solvent.  Its adhesion to
both metals and glass was very poor, but again not enough to justify taking
action to remedy it.

The thin films were made according to the following spin-coating procedure.
A small drop of CTX-809A was placed directly on a clean
\SI{12}{\milli\meter} round glass slide and the slide spun at
\SI{3000}{RPM} for \SI{30}{\second}.  Immediately the slide was removed and
placed on a hot plate at \SI{50}{\celsius} for \SI{30}{\minute}.  The
temperature of the hot plate was then increased to \SI{200}{\celsius} and
the Cytop allowed to cure at this temperature for \SI{12}{\hour} to remove
any remaining solvent.  After cooling, the metal films were sputtered
directly without any further surface modification.

The resultant film thicknesses were determined with a spectrometer by
measuring its broadband reflectance at $\theta=\SI{45}{\degree}$ and
fitting with the Fresnel reflectivity $R$
\begin{equation}
 R = \frac{n_1^2(n_i-n_s)^2 \cos^2\delta + (n_i n_s - n_1^2)^2\sin^2\delta}
          {n_1^2(n_i+n_s)^2 \cos^2\delta + (n_i n_s + n_1^2)^2\sin^2\delta}
          \label{eqn:fitfresnel}
\end{equation}
where
\begin{equation}
 \delta = \frac{2\pi}{\lambda_0} n_i d_i \cos \theta
\end{equation}
and the Cytop layer has a refractive index $n_i$ and thickness $d_i$, the
substrate a refrative index $n_s$, and the whole thing is in the presence
of refractive index $n_1$ (air).  Note that stated this way,
\Equation{eqn:fitfresnel} has only one free parameter: $d_i$.  An example
of the raw spectrometer data and and its best fit is shown in
\Figure{fig:fresnelfit}.

\begin{figure}
 \centering
 \import{includes/}{setpgfinc}
 \import{experimental/figures/}{fresnelfit}
 \caption{Reflectivity of a thin Cytop layer on BK7 at
 $\theta=\SI{45}{\degree}$.  Shown is the experimental data and the
 best-fit theory.  Calculated layer thickness for this measurement was
\SI{1150}{\nano\meter}.}
 \label{fig:fresnelfit}
\end{figure}
