For the fabrication of multilayer substrates supporting long range surface
plasmons, we used a flouropolymer,
poly(1,1,2,4,4,5,5,6,7,7-decafluoro-3-oxa-1,6-heptadiene), better known by
its brand name Cytop.  At our operating wavelength of \SI{660}{\nano\meter}
and at \SI{20}{\celsius}, its refractive index is very close to water --
1.341 for Cytop~\cite{mikevs2005synthesis} versus 1.331 for water.  This
index matching property allows us to construct a symmetric for long range
surface plasmons.

Cytop was purchased from AGC Chemicals as item number CTX-809A.  The last
letter ``A'' in the item number designates it is intended for use on
``amorphous'' (i.e. glass) applications, and the number immediately
following that designates the concentration.  For example, CTX-809A is
\SI{9}{\percent} cytop and \SI{91}{\percent} solvent, and likewise CTX-803A
is \SI{3}{\percent} cytop and \SI{97}{\percent} solvent.

\begin{equation}
 R = \frac{n_1^2(n_i-n_s)^2 \cos^2\delta + (n_i n_s - n_1^2)^2\sin^2\delta}
          {n_1^2(n_i+n_s)^2 \cos^2\delta + (n_i n_s + n_1^2)^2\sin^2\delta}
\end{equation}
where
\begin{equation}
\delta = 2\pi/\lambda n_j d_j cos \theta
\end{equation}
