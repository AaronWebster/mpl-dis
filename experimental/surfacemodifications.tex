The width of the SPR resonance and the structure of speckle in the cone are
fundamentally related to the underlying scattering microsctucture.  In this
respect it is important to note that for a perfectly smooth film, SPPs are
non-radiative and do not couple with the exciting
field~\cite{johansson1990theory}~\cite{otto1968excitation} and none of the
aforementioned physical manifestations of surface plasmon resonance (e.g.\ the
notch and the cone) can be observed.  Surface roughness of some degree is
therefore required.  Fortunately, creation of perfectly smooth films is rather
difficult to achieve, and some degree of roughness will almost always be
present.  In the present case where films are fabricated using plasma
sputtering, surface roughness is unavoidable.  Notwithstanding, several
procedures were investigated to introduce increased amounts of surface
roughness, motivated by the desire to better understand the nature of the
scattering proceses.

\subsection{Measurement of Surface Roughness}
To quantify the amount of surface roughness, polarization measurements were
conducted to determine the so-called ``rest light
intensity''~\cite{horstmann1977multiple} of the system.  The rest light
intensity is determined by measuring the angular dependence of the $p$- to
$s$-polarization components of light emitted from the components of light
which is emitted from the \textit{back} of the
prism~\cite{kretschmann1972decay}~\cite{hornauer1976light}.  For surface scattering processes, the
angular dependent ratio is given by~\cite{kretschmann1972thesis}
\begin{equation}
\frac{I_p}{I_s} = \frac{\tan^2\theta}{1+(\tan^2\theta)/|\epsilon|}
\label{eqn:ipssurface}
\end{equation}
and for volume scattering by
\begin{equation}
\frac{I^\prime_p}{I^\prime_s} = \frac{1}{|\epsilon|^2} I_p I_s
\label{eqn:ipsvolume}
\end{equation}
where $I_p$ and $I_s$ are the intensities of $p$ and $s$ polarized light,
respectively.

The incident light and the excited SPPs are strictly $p$-polarized;
polarization mixing cannot arise from a first order scattering process.
In this way, the degree of polarization mixing measures the degree of
higher-order (multiple) scattering.

The volume scattering components in \Equation{eqn:ipsvolume} are not taken
into account, as its value is too low to be determined with the present
apparatus, and furthermore it is expected not to be a contributing
factor~\cite{kretschmann1972decay}.  The increased rest light at
$\theta=\SI{0}{\degree}$ fits the model well, suggestive of increased
surface roughness.
\begin{figure}[ht]
 \centering
 \import{includes/}{setpgfinc}
 \import{experimental/figures/}{spfig}
 \caption{Rest light measurements, the ratio of polarization intensties
									$I_p/I_s$,  comparing an unmodified ``smooth'' silver film
									to a ``rough'' film sputtered using heat pretreatment
									technique.  For comparison, \Equation{eqn:ipssurface} shown
									as an example of theoretical perfect smoothness.}
\label{fig:spratio}
\end{figure}

\subsubsection{Heat Pretreatment}
During sputtering, heat applied to the substrate can be used to enhance
surface roughness.  Following \name{Horstmann}~\cite{horstmann1977multiple},
the hypotenuse of a \SI{25}{\milli\meter} hemispherical BK7 glass prism was
used as a substrate, heated in an oven to \SI{300}{\celsius} immediately prior
to entering the vacuum chamber for sputtering.  After sputtering, the prism
was allowed to cool to room temperature.

In \Figure{fig:spratio} is shown rest light intensity measurements conducted
for an unmodified silver film, ``measured smooth'', and a film roughened with
heat pretreatment, ``measured rough'', along with the theoretical prediction
of \Equation{eqn:ipssurface} for a film with no multiple scattering film,
`theoretical perfect smoothness''.  The p- and s-polarization components were
measured separately with a polarizing beamsplitter and to two cameras whose
pixel intensities were integrated to obtain an intensity value.  These ersatz
photodiodes were positioned on a rotation stage on the bottom halfspace of the
hemisphere and measurements taken in \SI{5}{\degree} increments.

The film for which heat pretreatment was applied, ``measured rough'', in
\Figure{fig:spratio}, has a much higher $I_p/I_s$ ratio than the film without,
``measured smooth''.  Comparing these values with those of \name{Horstmann}
and the first order theory of
\name{Kroger}~\cite{kroger1970scattering}~\cite{horstmann1977multiple} gives
an estimated RMS roughnesses of $\sim\SI{7}{\angstrom}$ for the unmodified
film and $\sim\SI{40}{\angstrom}$ for the heat pretreatment.

Enhancing the surface roughness through heat pretreatment was also observed to
have a deleterious effect on the quality of light on the cone, precluding
measurements of optical speckle presented in \Chapter{ch:speckle}.  As such,
this method was not used except as presented in \Figure{fig:spratio}.  A
detailed explanation is given in \Section{sec:sprspecklesize}.

\section{Nanoparticle Adsorption}
The most fruitful technique for surface modification was based on adsorption
of spherical gold nanoparticles onto a gold substrate.  The presence of a
nanoparticle is conceptually similar to the presence of a tip in SPOM
experiments (\Section{sec:coneexist}); both function as in-plane scatterers of
SPPs.  Using nanoparticles this way was observed to carry the additional
advantage of preserving the structure of optical speckle in the cone
(\Section{sec:sprspecklesize}).

Citrate-capped spherical gold nanoparticles (AuNPs) with diameters in the
range of \SIrange{20}{100}{\nano\meter} were used for the modification.  These
specific particles were chosen for two reasons: first, gold has a significant
amount of established bio-relevant chemistry protocols, and second, the
localized surface plasmon resonance of these particles is not coincident with
the laser frequency.  A UV-Vis spectrum for the particles, shown in
\Figure{fig:aunpspectrum}, confirms the second assumption.  Though it would be
perhaps interesting at some point to look at the effects of matched
resonances, it was desired to narrow the range of physical principles in play
in order to simplify interpretation of results during the investigation.

\begin{figure}[ht]
\centering
\import{includes/}{setpgfinc}
\import{experimental/figures/}{aunpspectrum}
\caption{UV-Vis absorbance spectrum of 25, 50, 75, and \SI{100}{\nano\meter} spherical gold
nanoparticles used in nanoparticle adsorption experiments.}
\label{fig:aunpspectrum}
\end{figure}

