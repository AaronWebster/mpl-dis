Several methods were developed to control surface roughness.  The first
method was to heat the substrate to \SI{200}{\celsius} immediately prior to
sputtering.  After sputtering, the substrates were allowed to cool slowly
at room temperature before use.  This method was able to increase the
measured surface roughness by nearly two fold (see
\Chapter{ch:scatteringmicro} for analysis).

The second method used nanoparticles to modify the surface, for which
\SI{57}{\nano\meter} citrate capped spherical gold nanoparticles were
obtained.  These specific particles were chosen because
\begin{inparaenum}{(a)}
\item gold has a significant amount of established biorelevant chemistry
				protocols, and
\item the localized surface plasmon resonance of these particles is not
				coincident with the SPP resonance.
\end{inparaenum}
A UV-Vis spectrum for these particles, shown in \Figure{fig:aunpspectrum},
confirms the second assumption.  Though it would be perhaps interesting at
some point to look at the effects of matched resonances, we wished to
narrow the range of physical principles in play in order to simplify
interpretation of results during our investigation.
\begin{figure}[ht]
 \centering
 \import{includes/}{setpgfinc}
 \import{experimental/figures/}{aunpspectrum}
 \caption{UV-Vis absorbance spectrum of \SI{57}{\nano\meter} spherical gold
 nanoparticles.}
 \label{fig:aunpspectrum}
\end{figure}

To quantify the influence of surface roughness, we conducted polarization
measurements of the so called ``rest light
intensity''~\cite{horstmann1977multiple}.  Essentially this involves
looking at the angular dependence of the ratios between the polarization
components of light which is emitted from the \textit{back} of the
prism~\cite{kretschmann1972decay}.
For surface scattering process, the angular dependent ratio is said to be
\begin{equation}
\frac{I_p}{I_s} = \frac{\tan^2\theta}{1+(\tan^2\theta)/|\epsilon|}
\label{eqn:ipssurface}
\end{equation}
and for volume scattering
\begin{equation}
\frac{I^\prime_p}{I^\prime_s} = \frac{1}{|\epsilon|^2} I_p I_s
\label{eqn:ipsvolume}
\end{equation}
The supporting literature for Equations~\ref{eqn:ipssurface} and
\ref{eqn:ipsvolume} is lacking.  Apparently it is to be found in a 1972
dissertation \name{Kretschmann} made during his Ph.D. work in Hamburg, but
at the time we have not actually seen this document.  In any event, the
polarization mixing cannot be described as a first order process and thus
is though to arise from multiple scattering of SPPs.

We have measured this ratio for films which we believe to be smooth, those
which we have not made modifications to, and ones roughened with heat
pretreatment.  This was carried out with the normal setup but with the
addition of a polarizing beamsplitter connected to two cameras (acting as
photodiodes) which is positioned on a rotation stage on the bottom
halfspace of the hemisphere.  Our results are shown in \Figure{fig:spratio}.
The theoretically perfect smooth value of \Equation{eqn:ipssurface} is also
plotted for comparison.  We do not plot volume scattering, as this value is
too low to be determined with our apparatus, and furthermore it is expected
not to be a contributing factor~\cite{kretschmann1972decay}.  The
significantly increased rest light at $\theta=\SI{0}{\degree}$ fits well
with this model, suggestive of increased surface roughness.
\begin{figure}[ht]
 \centering
 \import{includes/}{setpgfinc}
 \import{experimental/figures/}{spfig}
 \caption{Ratio of polarization intensties $I_p/I_s$ scattering microsctructure}
 \label{fig:spratio}
\end{figure}

We will refer to these measurements and rest light intensity more when we
discuss the scattering microstructure in \Chapter{ch:scatteringmicro}.
