Several methods were developed to control surface roughness.  The first
method was to heat the substrate to \SI{200}{\celsius} immediately prior to
sputtering.  After sputtering, the substrates were allowed to cool slowly
at room temperature before use.  This method was able to increase the
measured surface roughness by nearly two fold (see
\Chapter{ch:scatteringmicro} for analysis).

The second method used nanoparticles to modify the surface, for which
\SI{57}{\nano\meter} citrate capped spherical gold nanoparticles were
obtained.  These specific particles were chosen because
\begin{inparaenum}{(a)}
\item gold has a significant amount of established biorelevant chemistry
				protocols, and
\item the localized surface plasmon resonance of these particles is not
				coincident with the SPP resonance.
\end{inparaenum}
A UV-Vis spectrum for these particles, shown in \Figure{fig:aunpspectrum},
confirms the second assumption.  Though it would be perhaps interesting at
some point to look at the effects of matched resonances, we wished to
narrow the range of physical principles in play in order to simplify
interpretation of results during our investigation.
\begin{figure}[ht]
 \centering
 \import{includes/}{setpgfinc}
 \import{experimental/figures/}{aunpspectrum}
 \caption{UV-Vis absorbance spectrum of \SI{57}{\nano\meter} spherical gold
 nanoparticles.}
 \label{fig:aunpspectrum}
\end{figure}
