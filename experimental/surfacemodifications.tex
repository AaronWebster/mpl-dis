The width of the SPR resonance, the speckle in the cone, and the type of
SPP scattering are all intimately related to the scattering microsctucture.
It is therefore essential that we have control, or at least consistency,
regarding the metal surface.  To this end we have investigated several
methods for influencing the surface properties.  

\subsection{Heat}
The idea of substrate heating as a mechanism for increasing surface
roughness came to use though an anedcote by
\name{Horstmann}~\cite{horstmann1977multiple}.  The experimental procedure
consists of heating the glass substrate to \SI{200}{\celsius} immediately
prior to sputtering.  After sputtering, the substrates are allowed to cool
slowly at room temperature before use.  We did not have the ability to heat
our substrate \textit{in situ}, however because the primary mechanism for
heat dissipation in a vacuum (with a properly insulated sample) is radiant,
pre-heating the substrate before evacuating gave us ample time to carry out
the sputtering procedure at the appropriate tempterature.  In consideration
of this method we find that it was able to increase the measured surface
roughness by nearly two fold See \Chapter{ch:scatteringmicro} for analysis.

\subsection{Shadow Mask Lithography}
A second method was explored in collaboration with
\name{Huang}~\cite{huang2014speckle} which we refer to as \textit{shadow
mask lithography}.  This method involves the deposition of large,
micron-sized particles on the surface before sputtering, and then removing
them after the process is complete.  The result is a random arrangement of
holes, which act as scatterers.  To understand why this is, note the
Fresnel reflection coefficent for a single interface at normal incidence in
terms of the refractive indicies
\begin{equation}
r^s = \frac{n_1 - n_2}{n_1+n_2}
\end{equation}
and the reflectance $R^p=|r^p|^2$.  In other words, the Fresnel
reflectivity is
roughly proportional to the difference in refractive indicies, and when
considering the reflectance the phase shift is discarded and the results
are the same for both $n_1>n_2$ and $n_2>n_1$.

We carried out this method as follows.  Free \SI{1}{\micro\meter} (mean
diameter \SI{0.9}{\micro\meter}) silica particles in water at a density of
\SI{2}{\gram\per\centi\meter\cubed} were centrifuged and their suspending
solution decanted and replaced with dry acetone.  This was done three times
to minimize the amount of residual water in the final suspension.
Additional acetone was then added to obtain a final solution concentration
of \SI{8e5}{particle\per\micro\liter}.  

Glass substrates were cleaned in piranha solution (three parts
\SI{96}{\percent} \ce{H2SO4} to one part \SI{30}{\percent} \ce{H2O2}) for
\SI{10}{\minute} prior to use.  This process makes the surfaces
hydrophillic, and combined with the lowered surface tension
(\SI{23.4}{\milli\newton\per\meter} for acetone compared with
\SI{72.8}{\milli\newton\per\meter} for water at \SI{20}{\celsius}), a
droplet placed on the surface will spread evenly.  We let this evaporate on
its own and the try surfaces are sputtered with metal.  Afterwards, we
found that sonicating the substrate for \SI{10}{\minute} in dry acetone was
sufficient to overcome the Van der Waals forces between the particles and
substrate, washing them away and leaving holes in their stead.  For these
particles of mean diameter \SI{0.9}{\micro\meter} the resultant average
hole size was \SI{600}{\nano\meter}, circular.  As with the heat method,
the effect on surface roughness is discussed in
\Chapter{ch:scatteringmicro}.

\subsection{Nanoparticles}
The surface modification method which ultimately proved most fruitful
involved nanoparticle deposition.  Addition of nanoparticles is
conceptually similar to shadow mask lithography; in both cases
discontinuities in the systems permittivity cause reflections in the
propagating waves.  For this we chose citrate-capped spherical gold
nanoparticles (AuNPs) with diameters in the range of
\SIrange{20}{100}{\nano\meter}.  These specific particles were chosen for
two reasons: gold has a significant amount of established biorelevant
chemistry protocols, and the localized surface plasmon resonance of these
particles is not coincident with the laser frequency.  A UV-Vis spectrum for
these particles, shown in \Figure{fig:aunpspectrum}, confirms the second
assumption.  Though it would be perhaps interesting at some point to look
at the effects of matched resonances, we wished to narrow the range of
physical principles in play in order to simplify interpretation of results
during our investigation.
\begin{figure}[ht]
 \centering
 \import{includes/}{setpgfinc}
 \import{experimental/figures/}{aunpspectrum}
 \caption{UV-Vis absorbance spectrum of \SI{57}{\nano\meter} spherical gold
 nanoparticles.}
 \label{fig:aunpspectrum}
\end{figure}

\subsection{Measurement of Surface Roughness}
To quantify the amount of surface roughness, we conducted polarization
measurements of the so called ``rest light
intensity''~\cite{horstmann1977multiple}.  Essentially this involves
looking at the angular dependence of the ratios between the polarization
components of light which is emitted from the \textit{back} of the
prism~\cite{kretschmann1972decay}.
For surface scattering process, the angular dependent ratio is said to be
\begin{equation}
\frac{I_p}{I_s} = \frac{\tan^2\theta}{1+(\tan^2\theta)/|\epsilon|}
\label{eqn:ipssurface}
\end{equation}
and for volume scattering
\begin{equation}
\frac{I^\prime_p}{I^\prime_s} = \frac{1}{|\epsilon|^2} I_p I_s
\label{eqn:ipsvolume}
\end{equation}
Where $I_p$ and $I_s$ are the intensities of $p$ and $s$ polarized light,
respectively.

The supporting literature for Equations~\ref{eqn:ipssurface} and
\ref{eqn:ipsvolume} is lacking.  Apparently it is to be found in a 1972
dissertation \name{Kretschmann} made during his doctoral work in Hamburg, but
at the time we have not actually seen this document.  In any event, the
polarization mixing cannot arise from a first order process and thus
the mixing is supported by the idea of SPP multiple scattering.

We have measured this ratio for films which we believe to be smooth, those
which we have not made modifications to, and ones roughened with heat
pretreatment.  This was carried out with the normal setup but with the
addition of a polarizing beamsplitter connected to two cameras whose pixel
intensities we integrated to obtain an intensity value.  These \textit{de
facto} photodiodes are positioned on a rotation stage on the bottom
halfspace of the hemisphere.  Our results are shown in \Figure{fig:spratio}.
The theoretically perfect smooth value of \Equation{eqn:ipssurface} is also
plotted for comparison.  We do not plot volume scattering, as this value is
too low to be determined with our apparatus, and furthermore it is expected
not to be a contributing factor~\cite{kretschmann1972decay}.  The
significantly increased rest light at $\theta=\SI{0}{\degree}$ fits well
with this model, suggestive of increased surface roughness.
\begin{figure}[ht]
 \centering
 \import{includes/}{setpgfinc}
 \import{experimental/figures/}{spfig}
 \caption{Ratio of polarization intensties $I_p/I_s$ for different
 surface roughnesses.}
 \label{fig:spratio}
\end{figure}

We will refer to these measurements and rest light intensity again when we
discuss the scattering microstructure in \Chapter{ch:scatteringmicro}.
