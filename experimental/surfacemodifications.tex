The width of the SPR resonance and the characteristics of speckle in the
cone are fundamentally related to the underlying scattering microsctucture.
In this respect it is important to note that in the limit of a perfectly
flat and smooth film, surface plasmons are non-radiative and do not couple
with the exciting
field~\cite{johansson1990theory}~\cite{otto1968excitation}, and therefore
none of the aforementioned physical manifestations of surface plasmon
resonance (e.g. the notch and the cone) can be observed.  Surface roughness
of some degree is required.  Fortunately, fabrication of perfectly smooth
films is difficult to achieve and some degree of roughness will always be
present using the sputtering procedure described in the previous section.

Though fabrication of a perfectly smooth film was not possible with the
present apparatus, several procedures were investigated to introduce
increased amounts of surface roughness on the metal film, to be discussed
after the following subsection on experimental determination of surface
roughness.

\subsection{Measurement of Surface Roughness}
To quantify the amount of surface roughness, polarization measurements were
conducted to determine the so-called ``rest light
intensity''~\cite{horstmann1977multiple} of the system.  The rest light
intensity is determined by measuring the angular dependence of the p- to
s-polarization components of light emitted from the components of light
which is emitted from the \textit{back} of the
prism~\cite{kretschmann1972decay}.  For surface scattering processes, the
angular dependent ratio is given by~\cite{kretschmann1972thesis}
\begin{equation}
\frac{I_p}{I_s} = \frac{\tan^2\theta}{1+(\tan^2\theta)/|\epsilon|}
\label{eqn:ipssurface}
\end{equation}
and for volume scattering by
\begin{equation}
\frac{I^\prime_p}{I^\prime_s} = \frac{1}{|\epsilon|^2} I_p I_s
\label{eqn:ipsvolume}
\end{equation}
where $I_p$ and $I_s$ are the intensities of $p$ and $s$ polarized light,
respectively.

The incident light and the excited SPPs are strictly p-polarized;
polarization mixing cannot arise from a first order scattering process.
In this way, the degree of polarization mixing measures the degree of
higher-order (multiple) scattering.

Rest light intensity measurements were conducted for an unmodified silver
film, a film roughened with substrate heating, and a film subjected to
shadow mask lithography, shown in \Figure{fig:spratio} along with the
theoretical prediction for a film with no multiple scattering film (denoted
``perfectly smooth'').  The p- and s-polarization components were measured
separately with a polarizing beamsplitter and to two cameras whose pixel
intensities were integrated to obtain an intensity value.  These ersatz
photodiodes were positioned on a rotation stage on the bottom halfspace of
the hemisphere and measurements taken in \SI{5}{\degree} increments.

The volume scattering components in \Equation{eqn:ipsvolume} are not taken
into account, as its value is too low to be determined with the present
apparatus, and furthermore it is expected not to be a contributing
factor~\cite{kretschmann1972decay}.  The increased rest light at
$\theta=\SI{0}{\degree}$ fits the model well, suggestive of increased
surface roughness.
\begin{figure}[ht]
 \centering
 \import{includes/}{setpgfinc}
 \import{experimental/figures/}{spfig}
 \caption{Rest light measurements, the ratio of polarization intensties
									$I_p/I_s$,  for different surface roughness enhancing procedures.}
\label{fig:spratio}
\end{figure}

\subsection{Heat}
During sputtering, heating applied to the substrate can be used to enhance
surface roughness.  Following
\name{Horstmann}~\cite{horstmann1977multiple}, the experimental procedure
consisted of heating the glass substrate to \SI{200}{\celsius} immediately
prior to sputtering.  After sputtering, the substrates are allowed to cool
slowly at room temperature before use.  Though the sputtering apparatus
employed did not have the ability to heat the substrate \textit{in situ},
surface roughness could be reliably enhanced by heating the substrate
immediately prior to evacuation.  From rest light intensity measurements it
was found that this method increased the measured surface roughness by a
factor of two.

\subsection{Shadow Mask Lithography}
A second method was explored in collaboration with
\name{Huang}~\cite{huang2014speckle} which is herein referd to as
\textit{shadow mask lithography}.  Shadow mask lithography was carried out
by depositing large, micron-sized particles on the surface before
sputtering, and then removing them after the process completed.  The
resulting random arrangement of holes acted as scatterers, increasing
surface roughness.  Note that the Fresnel reflection coefficient is roughly
proportional to the difference in refractive indices $n_1$ and $n_2$ for a
single-layer interface.  When considering the magnitude squared reflectance
the phase shift is discarded, resulting in the same reflectance for both
$n_1>n_2$ and $n_2>n_1$.

The procedure for shadow mask lithography was carried out as follows.  Free
silica particles in water, mean diameter \SI{0.9}{\micro\meter} and density
\SI{2}{\gram\per\centi\meter\cubed}, were centrifuged and their suspending
solution decanted and replaced with dry acetone.  The centrifuging and
decanting step was then repeated three times to minimize the amount of
residual water in the final suspension.  Additional acetone was then added
to obtain a final solution concentration of
\SI{8e5}{particle\per\micro\liter}.  

Glass substrates were cleaned in piranha solution (three parts
\SI{96}{\percent} \ce{H2SO4} to one part \SI{30}{\percent} \ce{H2O2}) for
\SI{10}{\minute} prior to use.  Cleaning glass with piranha makes the
surface hydrophilic, and combined with the lowered surface tension
(\SI{23.4}{\milli\newton\per\meter} for acetone compared with
\SI{72.8}{\milli\newton\per\meter} for water at \SI{20}{\celsius}), a
droplet placed on the surface will spread evenly.  The droplet was allowed
to evaporate in air and the sputtering process was then carried out as
usual.  Afterwards, sonicattion of the substrate for \SI{10}{\minute} in
dry acetone was sufficient to remove the adsorbed particles and reveal the
underlying holes.  For particles of mean diameter \SI{0.9}{\micro\meter}
the resultant average hole size was \SI{600}{\nano\meter}, circular.  As
with the heat method, the effect on surface roughness is further discussed
in \Chapter{ch:scatteringmicro}.

\subsection{Nanoparticles}
The third surface modification method which ultimately proved most fruitful
involved nanoparticle deposition.  Addition of nanoparticles is
conceptually similar to shadow mask lithography; in both cases
discontinuities in the systems permittivity cause reflections in the
propagating waves.  Citrate-capped spherical gold nanoparticles (AuNPs)
with diameters in the range of \SIrange{20}{100}{\nano\meter} were used for
the modification.  These specific particles were chosen for two reasons:
first, gold has a significant amount of established bio-relevant chemistry
protocols, and second, the localized surface plasmon resonance of these
particles is not coincident with the laser frequency.  A UV-Vis spectrum
for the particles, shown in \Figure{fig:aunpspectrum}, confirms the second
assumption.  Though it would be perhaps interesting at some point to look
at the effects of matched resonances, it was desired to narrow the range of
physical principles in play in order to simplify interpretation of results
during the investigation.

\begin{figure}[ht]
 \centering
 \import{includes/}{setpgfinc}
 \import{experimental/figures/}{aunpspectrum}
 \caption{UV-Vis absorbance spectrum of \SI{57}{\nano\meter} spherical gold
 nanoparticles.}
\label{fig:aunpspectrum}
\end{figure}

