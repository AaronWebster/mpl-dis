Thus far we have not found a way to use this phenomena to enhance
measurements of bulk refractive index sensitivity in \gls{spr}\@.  The responses in
the near field track well as the field propagates into the far field.
There is a possibility that the additional structure in the spatial
oscillations may help fitting routines, but we have not investigated this
rigorously.
%What you should do here is the following: use MRF's code to make near field
%x-z plots of system A and system B.  Then take the difference between the
%two.  If there is any better way that will let you know.
%It is also possible to make some plots using the original dango wiggles
%code.
