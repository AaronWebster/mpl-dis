There are several perspectives of SPR phenomena which can be insightful in
explaining why exactly the spatial oscillations are one-sided.  The first
is an argument from causality.  Assume a complex function $\chi(\omega) =
\chi'(\omega) + \mi \chi''(\omega)$ whose real and imaginary parts are
related by Kramers-Kronig relations
\begin{align}
\chi(\omega)=\mi \hf{\chi(\omega)}
\end{align}
with 
\begin{align}
\chi'(\omega) &= \hf{\chi''(\omega)}\\
\chi''(\omega) &= -\hf{\chi'(\omega)}
\end{align}
where $\hf{\chi(\omega)}$ is the Hilbert transform of $\chi(\omega)$.
The Fourier transform of $\chi(\omega)$ is
\begin{align}
\chi(\omega) &= \chi'(\omega) + \mi \chi''(\omega)\\
\ff{\chi(\omega)} &= \ff{\chi'(\omega) + \mi \chi''(\omega)}\\
&= \ff{\chi'(\omega)} + \ff{\mi \chi''(\omega)}\\
&= \ff{\chi'(\omega)} + \ff{\mi \hf{\chi'(\omega)}} \\
&= \ff{\chi'(\omega)} + \sgn(\omega) \ff{\chi'(\omega)} \\
\end{align}
Or succinctly,
\begin{align}
\ff{\hf{\chi(\omega)}} = (-\mi \sgn(\omega)) \ff{\chi(\omega)}
\end{align}
In other words, the Fourier transform of any function which satisfies
Kramers-Kronig relations is ``one-sided'' as a necessary
condition of causality.  This seems to be true for the Fresnel
reflectivity as well as the complex permittivity.

The second perspective is couched in Fourier optics.  Here, because
SPR occurs at the focus of a Gaussian beam, it can be seen as a sort of
spatial filter which modifies the local $k$-vectors to produce
the resulting far field optical pattern.  If the SPR resonance condition is
sharp, the Fourier integral (\Equation{eqn:fourier123}) is truncated and
the discontinuity acts as a low pass filter for light.  The one-sided
oscillations are then essentially a manifestation of Gibb's phenomena.
This seems to be well supported, because if the SPR resonance is broadened,
say in the case of a BK7-Au-\ce{H2O} system, the interference is greatly
attenuated.
%
%Another possible interpretation is that the near and
%far field patterns in conically scattered light are qualatatively identical
%to the simple impulse-response of a forced, damped harmonic oscillator.
%This is of course equivalent to what the Lorentz-Drude model says about the
%material properties.  The far field conically scattered light likewise
%appears to be equivalent to the energy absorbed in the metal film which can
%be detected as heat.

\subsection{Spatial Evolution}
The Fresnel reflectivity predicts the far field angular distribution when
illuminated by light at a specific angle, consisting of a single
$k$-vector.  Now we will look at the evolution of light from this system
as it propagates to the far field when excited by an incident Gaussian beam
$g(x)$ containing a continuum of $k$-vectors.  We begin with treatment of
light in the specular direction.  In the context of Fourier
optics, the field on the surface in $k$-space is described by the Fourier
decomposed incident Gaussian beam $\tilde{g}(k_x)$ multiplied by the
Fresnel reflectivity
\begin{align}
\tilde{E}_\text{spec}(k_x)=\tilde{g}(k_x)\,\tilde{r}_\text{123}(k_x)
\end{align}
where
\begin{align}
\tilde{g}(k_x) = \intinfty g(x)\, \me^{\mi k_x x} \md x
\end{align}
and $g(x)$ represents a Gaussian beam in spatial dimensions.  We omit any 
specific definition for $g(x)$ because its exact form
is not important to our analysis.

The complete optical field in both $x$ and $z$ can be obtained by computing
the Fourier transform of $\tilde{E}_\text{spec}(k_x)$ multiplied 
by the free space transfer function $\me^{\mi k_{z,1} z}$
\begin{align}
E_\text{spec}(x,z) &= \intinfty \tilde{E}_\text{spec}\, \me^{\mi k_{z,1} z}\, \me^{\mi k_x x} \md k_x\\
 &= \intinfty \tilde{g}(k_x)\, \tilde{r}_{123}(k_x)\, \me^{\mi \sqrt{k_0^2 \epsilon_1 - k_x^2}z}\, \me^{\mi k_x x} \md k_x
\label{eqn:fourier123}
\end{align}
Likewise, conically scattered light may be found using the same treatment
\begin{align}
E_\text{cone}(x,z) = \intinfty \tilde{g}(k_x)\, \tilde{r}_{321}(k_x)\,\me^{\mi k_{z,1} z}\, \me^{\mi k_x x} \md k_x
\label{eqn:fourier321}
\end{align}
Though this integral seems to have no analytic solution, its evaluation is
nonetheless straightforward on a computer.
The evolution of scattered and specularly directed light based on
\Equation{eqn:fourier123} and \Equation{eqn:fourier321} is shown in
\Figure{fig:fresnelpropagate} as a function of distance from the 2-3
interface.  As can be seen, as the field propagates it diffracts,
displaying a one-sided oscillatory structure. 

\subsubsection{Evanescent Solutions}
In the context of diffraction integrals, the free space transfer function
$\exp(\mi k_{z,1} z)$ is often concerned with evanescent and propagating
terms, giving rise to the diffraction limit.  Here,
$k_{z,1}=\sqrt{k_0^2 \epsilon_1 - k_x^2}$ is imaginary for $k_0^2
\epsilon_1 < k_x^2$.  It is interesting to note that this never occurs.
Ignoring the condition for total internal reflection, 
$k_x = k_0 \sqrt{\epsilon_1} \sin \theta$ and we can set up an inequality
\begin{align}
k_0^2 \epsilon_1 &< k_x^2\\
k_0^2 \epsilon_1 &< \left(k_0 \sqrt{\epsilon_1} \sin \theta\right)^2\\
k_0^2 \epsilon_1 &< k_0^2 \epsilon_1 \sin^2 \theta\\
1 &< \sin^2 \theta
\end{align}
which is false, suggesting that if all light is collected, no information is
lost during propagation from the near to far field.