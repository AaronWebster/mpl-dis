\begin{figure}
	\begin{center}
		\begin{tikzpicture}
			\begin{axis}[
				width=10cm,
				height=4cm,
				xmin=0, xmax=1.5,
				ymin=0, ymax=1.5,
				xlabel={},
				ylabel={},
				axis lines=left,
				xtick={0,0.5,1.0,1.4142,1.5},
				xticklabels={0,0.5,1.0,$\sqrt{2}$,1.5},
				]
				% Square distribution (red) - two parts
				\addplot[red, domain=0:1, samples=100] {x^2*(4*x - pi - 2*x)};
				\addplot[red, domain=1:1.4142, samples=100] {4*sqrt(x^2-1)*(x^2 - 2 - pi + 4*rad(atan(sqrt(x^2-1)))*x)*2*x};
				% Disk distribution (blue)
				\addplot[blue, domain=0:1, samples=100] {4*x/(pi*0.5^2)*rad(acos(x/(2*0.5)))*2*x^2/(pi*0.125) - sqrt(1-x^2)*4/(0.5^2)};
			\end{axis}
		\end{tikzpicture}
	\end{center}
	\caption{Probability distribution function for the square and disk line
		picking problems.}
	\label{fig:linepickingpdf}
\end{figure}
The way we choose scattering trajectories in the simulation leads to an
interesting statistical distribution for the number of scatterers visited.
Again, there are two criteria which can end a trajectory
\begin{itemize}
	\item The same scatter is chosen twice.
	\item The path length reaches a hard limit.
\end{itemize}
The first exit criteria was chosen ostensibly to provide for the strong
presence of single scattering off the tip.  Consider a rectangular area
containing $N$ scatterers.  The probability at each scatterer of choosing
pairwise sequential the same scatterer is $1/N$.  The extension of this to
the probability of the path ending on the $n$th scattering site, $P(n)$, is
modeled by the familiar differential equation for exponential decay
\begin{align}
	\frac{d P(n)}{dn} = -\frac{1}{N}P(n)
\end{align}
which, when solved and the initial condition $P(0)=1/N$ is applied, can be
expressed as
\begin{align}
	P(n)=\frac{1}{N}e^{-n/N}
\end{align}
%seperation of variables results in
%\begin{align}
%\frac{d P(n)}{P(n)}=-\frac{1}{N}dn
%\end{align}
%integrating yields
%\begin{align}
%\log P(n) = -\frac{n}{N} + C
%\end{align}
%and therefore
%\begin{align}
%P(N)= e^C e^{-n/N}
%\end{align}
%The initial condition represented by $e^C$ is found by noting $P(0)=1/N$,
%therefore the final form of this equation is given by
This is the lifetime of the plasmon trajectory considering its only exit
possibility is due to choosing the same scatterer twice.

The second exit criteria comes when our path length has been exhausted.
Since the scatterers are randomly distributed throughout the area, it is
useful to know the mean free path for such a system. The proper physical
model for path length distributions in a scattering medium is given by the
Beer-Lambert Law, which describes the exponential attenuation of intensity
as light propagates through a medium with scatterers.

For a plasmon propagating through a medium with scatterer density $\rho$
and scattering cross-section $\sigma$, the probability that the plasmon
travels a distance $l$ before scattering follows an exponential distribution:
\begin{align}
	P(l) = \frac{1}{\Lambda} e^{-l/\Lambda}
\end{align}
where $\Lambda = \frac{1}{\rho \sigma}$ is the mean free path, which
depends on both the scatterer density $\rho$ and the scattering
cross-section $\sigma$. This exponential decay naturally emerges from
considering that the probability of scattering in an infinitesimal distance
$dl$ is proportional to $\rho \sigma \, dl$, leading to the differential
equation $dP/dl = -P/\Lambda$ whose solution is the exponential form above.

The simulation must physically enforce this exponential decay model rather
than relying on geometric fixed limits. This ensures that as the scattering
density is increased, the plasmon visits more scatterers and the mean free
path decreases appropriately, consistent with physical scattering theory.
The hard path length cutoff used in the current implementation should be
understood as an artificial boundary condition rather than a fundamental
physical constraint.
