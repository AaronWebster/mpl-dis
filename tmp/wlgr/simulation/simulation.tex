A parallel Monte Carlo simulation was written to model plasmon scattering.
It assumes the geometry shown in Figure \ref{fig:plasmongeo} consisting of
\begin{itemize}
\item An elliptical {\it illuminated region}, representing the incident evanescent
field used to excite plasmons on the surface.  This ellipse has 
radii $r_a=\SI{5}{\micro\meter}$ and $r_b=\SI{3}{\micro\meter}$.  
\item A {\it tip} - a single movable scatterer representing the STM tip.
\item The {\it scan area}, a square area wherein the tip rasters, typically
$128\times128$ points in a maximal area of $2.27\times\SI{2.27}{\micro\meter}$.  
\item {\it Scatterers}, fixed point scatterers (usually \num{20})
distributed randomly in a square of $8.8281\times\SI{8.8281}{\micro\meter}$
where the plasmon can visit, and
\item A {\it trajectory} or {\it path} meaning the ordered sequence of scatterers a
plasmon visits before exiting the system.  The trajectory has a typical maximum
path length of \SI{18}{\micro\meter}.
\end{itemize}
\begin{figure}
\centering
\begin{pspicture}(-4.41405,-4.41405)(4.41405,4.41405)
% tip
\psdot[dotstyle=triangle,linecolor=orange](0.2,-0.3)
% scan area
\psframe[linestyle=dashed,linecolor=green,dash=0.1](-1.35,-1.35)(1.35,1.35)
\psclip{\psframe[linestyle=none](-4.41405,-4.41405)(4.41405,4.41405)}
% illuminated area
\psellipse[linestyle=dashed,linecolor=red,dash=0.1 0.05](0,0)(5,3)
\endpsclip
\psaxes[labels=none,axesstyle=frame,tickstyle=inner,ticksize=0
4pt,Ox=-5,Oy=-5]{-}(-4.41405,-4.41405)(4.41405,4.41405)
\psset{dotstyle=Basterisk,linecolor=blue}
% random scatterers
\psdot[](1.08675765054645,-1.15339223391766)
\psdot[](-1.65945473768577,3.30126294690662)
\psdot[](3.08452901035693, 3.75801196082581)
\psdot[](-0.298028242005936, -0.947682947635213)
\psdot[](-1.13254778504099,-2.86291168735803)
\psdot[](-2.42224615064296,3.94409083629355)
\psdot[](2.84552547917312, -3.06371460124055)
\psdot[](-3.65079702863291,-0.125947393237618)
\psdot[](0.816143940611938,2.47247755178442)
\psdot[](3.66121582314356, -1.29098349555369)
\psdot[](2.41348287067284, 2.00794065198278)
\psdot[](3.73678615525102, 3.90756858304069)
\psdot[](1.64323248166686, 3.07351951851559)
\psdot[](1.79314213547156, -1.92551152476177)
\psdot[](-3.77510373532563,-2.39690980082588)
% line discribing a path
\psset{linestyle=dotted,dotsep=2pt,linecolor=gray}
\psline[](0.2,-0.3)(-0.298028242005936, -0.947682947635213)
\psline[](-0.298028242005936, -0.947682947635213)(-1.13254778504099,-2.86291168735803)
\psline[](-1.13254778504099,-2.86291168735803)(-3.65079702863291,-0.125947393237618)
\psline[](-3.65079702863291,-0.125947393237618)(3.73678615525102, 3.90756858304069)
% bunch of labels in a parbox
\rput[l](-4,-3.85){\parbox{5cm}{
 \Rnode{A}{} \hskip 0.5cm \Rnode{B}{  illuminated region} \ncline[linestyle=dashed,linecolor=red,dash=0.1 0.05]{A}{B} \\
 \Rnode{A}{} \hskip 0.5cm \Rnode{B}{  scan area} \ncline[linestyle=dashed,linecolor=green,dash=0.1]{A}{B} \\
}}
\rput[l](0,-3.85){\parbox{5cm}{
 \rput[c](0.25cm,0.7ex){\dotnode[dotstyle=Basterisk,linecolor=blue]{A}} \hskip 0.5cm \Rnode{B}{  scatterers} \\
 \rput[c](0.25cm,0.7ex){\dotnode[dotstyle=triangle,linecolor=orange]{A}} \hskip 0.5cm \Rnode{B}{  tip} \\
}}
\rput[l](2.5,-3.85){\parbox{5cm}{
 \Rnode{A}{} \hskip 0.5cm \Rnode{B}{  path} \ncline[linestyle=dotted,dotsep=2pt,linecolor=gray]{A}{B} \\
}}
\end{pspicture}
\psset{linecolor=black}
\caption{Definition of terms: geometry for the Monte Carlo scattering
simulation}
\label{fig:plasmongeo}
\end{figure}

The plasmon is assumed to be a time independent monochromatic plane wave
of the form $E(\mathbf{r})=e^{ik_\mathrm{sp}\mathbf{r}}$, where
$k_\mathrm{sp}$ is the plasmon wavevector and $\mathbf{r}$ is some
spatial variable.  The plasmon trajectories are modeled as follows:
First, a random scatterer with index $n$ and coordinates $(x_n,y_n)$ is
chosen within the illuminated region as the first scatterer a plasmon
visits.  Since the illuminated region always encloses the scan area, we are
guaranteed at least one scatterer (the tip) to choose from.  The plasmon
initializes with local phase $\varphi_\mathrm{l,n}=k_\mathrm{sp}x_n$
representing the local phase advance from the (arbitrary) point where the
plasmon is excited by the (linear in either $x$ or $y$) evanescent field to
the first scatterer.  Next, scatterers are randomly and sequentially chosen
from amongst all possible scatterers including the tip and the total path
length accumulated as the plasmon visits each scatterer.  The trajectory
ends when
\begin{itemize}
\item The same scatterer is visited twice in a row, or
\item The total accumulated path length exceeds a pre-defined limit
(\SI{18}{\micro\meter}).  
\end{itemize}
This process is repeated many times for each tip position in the scan area.
Usually \numrange{250000}{1000000} trajectories will be used for each point
of the $128\times128$ grid.  During this process the path length, and
therefore the total phase from multiple scattering at $K$ sites
\begin{align}
\varphi_\mathrm{ms,n}=\sum_{k=0}^{K-1} \sqrt{(x_{k+1}-x_k)^2+(y_{k+1}-y_k)^2}
\end{align}
is, along with the final scatterer, saved.

The ultimate output from the Monte Carlo simulation consists of {\it
topology maps}, colloquially dubbed {\it weirdospace}, having the same pixel
dimensions as the scan area.  Each weirdospace image represents, for each
pixel, the far field phase and intensity of a particular point at a certain
angle along the plasmon ring for the respective tip position in the scan
area.  The weirdospace image for each angle $\phi$ along the ring is
calculated by adding a final phase $\varphi_\mathrm{ff,n}$ equal to the
propagation from the final scatterer to the far field along plasmon
scattering angle $\theta_\mathrm{sp}\approx \SI{44}{\degree}$.  This is 
\begin{align}
\varphi_\mathrm{ff,n} = k_0 \sin
\theta_\mathrm{sp}\left(x_n\cos\phi+y_n\sin\phi\right)
\end{align}
Where $k_\mathrm{sp}=k_0\sin \theta_\mathrm{sp}$.  Each path therefore
represents a wave of the form
\begin{align}
E_n=e^{i(\varphi_\mathrm{l}+\varphi_\mathrm{ms}+\varphi_\mathrm{ff})}
\end{align}
And the total far field intensity is the linear superposition of these
waves
\begin{align}
E_\mathrm{total}(\phi) = 
\left|\sum_{n=0}^{N} E_n\right|^2
\end{align}

In total, the program writes the following data
\begin{itemize}
\item Weirdospace images in a 64 bit floating point HDF5 files.  The images
are names {\tt XXXXX-scatter.h5}, where {\tt XXXXX} represents the angle
in degrees of that particular frame along the ring.  Both the phase and
intensity are stored in each file.
\item {\tt ring.dat}, a tab separated values file containing the angle,
mean intensity, and variance of intensity for each weirdospace image.  This
file is no longer created by the main program, but computed from the output
by a supporting script {\tt mkring.sh}.
\item {\tt parameters}, a file containing program variables, scatterer
locations, and statistics.  The parameters are as shown in Table
\ref{tbl:parameters}.
\begin{table}
\label{tbl:parameters}
\begin{center}
\begin{tabular}{ll}
\toprule
parameter & description \\
\midrule
{\tt SX}, {\tt SY} & integer size of scan area\\
{\tt ELLIPSE\_A} & first radius of elliptical illuminated area\\
{\tt ELLIPSE\_B} & second radius of elliptical illuminated area\\
{\tt LAMBDA} & wavelength of incident light, in microns\\
{\tt PATHLEN} & maximum path length for a plasmon\\
{\tt ENTIREAREA} & square boundary for random scatterer generation\\
{\tt SCANAREA} & dimensions of the scan area in microns \\
{\tt EVENTS} & events per point in the scan area\\
{\tt NSCAT} & number of randomly distributed scatterers\\
\bottomrule
\end{tabular}
\end{center}
\caption{Explanation of variables used in the {\tt parameters} file.}
\end{table}
At the end there is a section called {\tt STATS}.  The first column of this
section represents the number of scatterers visited, the second the
percentage of all trajectories including exactly this many scatterers, and
the third column the percentage of those trajectories included the tip in
some way.  The first row is the sum of all the following data.
\end{itemize}

The weirdospace images are colored in such a way as to show both phase and
intensity information at the same time.  This is done by mapping the
information on to a circular coordinate system as shown in Figure
\ref{fig:hsv}, where the intensity is mapped to the magnitude of the radial
component $r$ and phase to the angle $\theta$.  
\begin{figure}
\begin{center}
\psset{unit=1cm}
% made this picture with the gimp... has HSV effect built right in
\begin{overpic}[width=5.0cm,height=5.0cm]{simulation/hsvradial.eps}
\begin{pspicture}(-2.5,-2.5)(2.5,2.5)
\pnode(0,0){C} % center
\pnode(! 90 sin 2.5 mul 90 cos 2.5 mul){A} % edge 1
\pnode(! 55 sin 2.5 mul 55 cos 2.5 mul){B} % edge 2
\psset{linecolor=white}
\ncline{->}{C}{B}\taput{{\white$r$}} % line to edge 2
\ncline{->}{C}{A} % line to edge 1
\pstMarkAngle[]{A}{C}{B}{{\white$\theta$}} % mark the angle
\end{pspicture}
\end{overpic}
\end{center}
\caption{Weirdospace images are colored by using the above model based on a
circular coordinate system.  Intensity is mapped to the radial component
$r$ and phase the angle $\theta$.}
\label{fig:hsv}
\end{figure}
This coding scheme is identical to that which can be produced in HSV color
space with the saturation always being set to unity.
