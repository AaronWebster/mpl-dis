\begin{figure}
\begin{center}
\psset{xAxisLabel=,yAxisLabel=}
\begin{psgraph}(0,0)(1.5,1.5){10cm}{4cm}
\psplot[linecolor=red]{0}{1}{x dup mul 4 x mul sub PI add 2 x mul mul}
\psplot[linecolor=red]{1}{1.4142}{4 x dup mul 1 sub sqrt mul x dup mul 2
add PI sub sub 4 x dup mul 1 sub sqrt ATAN mul sub 2 x mul mul}
\psplot[linecolor=blue]{0}{1}{4 x mul PI 0.5 dup mul mul div x 2 0.5 mul
div ACOS mul 2 x dup mul mul PI 0.125 mul div 1 x dup mul sub 4 0.5 dup mul
mul div sqrt mul sub}
\psxTick(1.4142){\sqrt{2}} % max distance for a square
\end{psgraph}
\end{center}
\caption{Probability distribution function for the square and disk line
picking problems.}
\label{fig:linepickingpdf}
\end{figure}
The way we choose scattering trajectories in the simulation leads to an
interesting statistical distribution for the number of scatterers visited.
Again, there are two criteria which can end a trajectory
\begin{itemize}
\item The same scatter is chosen twice.
\item The path length reaches a hard limit.
\end{itemize}
The first exit criteria was chosen ostensibly to provide for the strong
presence of single scattering off the tip.  Consider a rectangular area
containing $N$ scatterers.  The probability at each scatterer of choosing
pairwise sequential the same scatterer is $1/N$.  The extension of this to
the probability of the path ending on the $n$th scattering site, $P(n)$, is
modeled by the familiar differential equation for exponential decay
\begin{align}
\frac{d P(n)}{dn} = -\frac{1}{N}P(n)
\end{align}
which, when solved and the initial condition $P(0)=1/N$ is applied, can be
expressed as
\begin{align}
P(n)=\frac{1}{N}e^{-n/N}
\end{align}
%seperation of variables results in
%\begin{align}
%\frac{d P(n)}{P(n)}=-\frac{1}{N}dn
%\end{align}
%integrating yields
%\begin{align}
%\log P(n) = -\frac{n}{N} + C
%\end{align}
%and therefore
%\begin{align}
%P(N)= e^C e^{-n/N}
%\end{align}
%The initial condition represented by $e^C$ is found by noting $P(0)=1/N$,
%therefore the final form of this equation is given by
This is the lifetime of the plasmon trajectory considering its only exit
possibility is due to choosing the same scatterer twice.

The second exit criteria comes when our path length has been exhausted.
Since the scatterers are randomly distributed throughout the area, it is
useful to know the mean free path for such a system.  Consider a unit
square.  The average distance between any two points is given by the box
integral
\begin{align}
\int_0^1 \int_0^1 \int_0^1 \int_0^1 \sqrt{(x_1-y_1)+(x_2-y_2)}\; dx_1 dx_2 dy_1 dy_2
\end{align}
This is the {\it square line picking} problem.  Analytic evaluation of the
above integral is complicated, but yields for its unit dimensions an
average distance of
\begin{align}
\overline{r} = \frac{\sqrt{2}+2+5\log\left(1+\sqrt{2}\right)}{15}
\end{align}
and the probability distribution
\begin{align}
P_\mathrm{s}(l) = \left\{
\begin{array}{l l}
2l\left(l^2 -4l + \pi\right) & \quad  0\leq l \leq 1\\
2l\left[4\sqrt{l^2-1}-\left(l^2+2-\pi\right)-4\arctan\left({\sqrt{l^2-1}}\right)\right]
& \quad 0 \leq l \leq \sqrt{2}\\
\end{array}
\right.
\end{align}
If we instead suppose a unit disk instead of a unit square, the mean
distance can be found to be
\begin{align}
\overline{r}= \frac{128}{45 \pi}
\end{align}
and probability distribution for a disk of radius $R$ is
\begin{align}
P_\mathrm{d}(l)=\frac{4l}{\pi R^2} \arccos\left(\frac{l}{2R}\right) - \frac{2
l^2}{\pi R^3} \sqrt{1-\frac{l^2}{4 R^2}}
\end{align}
It is assumed that the superposition of the two probabilities should
produce the statistical results found in \ref{fig:scatterhist}, but as yet
I have not found a way to do so.

These line picking problems are useful because they characterize how many
scatterers on average a plasmon will visit until exiting in the Monte Carlo
simulation.  They also reveal a limitation: as the scattering density is
increased, a plasmon as simulated does not visit more scatterers, nor does
its mean free path decrease.  Rather, the result of increasing the number
of scatterers is only to suppress single scattering off the tip.  This may
or may not be physical. 
