For a small number of scatterers~\cite{jakeman1984speckle} $N$ of intensity
$a$, $\rho_I(I)$ becomes~\cite{goodman2007speckle}
\begin{equation}
p_I(I) = 2\pi^2 \int_0^\infty\! \rho J_0\!\left(\frac{2\pi a
\rho}{\sqrt{N}}\right)^N J_0\!\left(2\pi\sqrt{I}\rho\right)\md\rho,\quad \rho_I^2=(1-1/N)a^4.
\label{eqn:smallscatteqn}
\end{equation}

\Equation{eqn:smallscatteqn} is shown in \Figure{fig:lowscatthist} for
$N=2,\ldots,5$ along with a histogram for a single simulated realization of a
2D speckle pattern generated using \Equation{eqn:phasorsum}.  That the
simulated speckle pattern is in two dimensions instead of three is not
significant; in each case the notion of a random phasor sum hold.  No ensemble
averaging was used, though is typically required to produce reasonable
statistics~\cite{goodman2007speckle}.  

From \Figure{fig:lowscatthist} it is apparent that, for speckle described by a
random phasor sum with a low number of scatterers,
\Equation{eqn:smallscatteqn} converges to \Equation{eqn:exppdf} rapidly as $N$
increases.  At $N=6$, the sum squared error as compared with $N\to\infty$ is
only \num{4.75e-5}.  In terms of physical SPP scattering, random phasor sum
statistics can be used to a good approximation to describe the resulting
distribution, even down to a fairly low number of scatterers.
