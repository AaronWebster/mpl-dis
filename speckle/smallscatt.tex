For a small number of scatterers~\cite{jakeman1984speckle} $N$ of intensity
$a$, $\rho_I(I)$ becomes~\cite{goodman2007speckle}
\begin{equation}
p_I(I) = 2\pi^2 \int_0^\infty\! \rho J_0\!\left(\frac{2\pi a
\rho}{\sqrt{N}}\right)^N J_0\!\left(2\pi\sqrt{I}\rho\right)\md\rho,\quad \rho_I^2=(1-1/N)a^4.
\label{eqn:smallscatteqn}
\end{equation}
This is shown in \Figure{fig:lowscatthist} for $N=2,\ldots,5$ and a single
realization of a speckle pattern was generated in 2D using
\Equation{eqn:phasorsum} and its histogram plotted against the theoretical
prediction of \Equation{eqn:smallscatteqn}.  No ensemble averaging was used,
though this is typically required to produce reasonable
statistics~\cite{goodman2007speckle}.  \Figure{fig:lowscatthist} shows an
important point about random phasor sums with a low number of scatterers.
\Equation{eqn:smallscatteqn} converges on \Equation{eqn:exppdf} fairly rapidly
as $N$ increases.  At $N=6$, the sum squared error as compared with
$N\to\infty$ is only \num{4.75e-5}.
