For a small number of scatterers $N$ of intensity $a$, $\rho_I(I)$
becomes~\cite{goodman2007speckle}
\begin{equation}
p_I(I) = 2\pi^2 \int_0^\infty\! \rho J_0\!\left(\frac{2\pi a
\rho}{\sqrt{N}}\right)^N J_0\!\left(2\pi\sqrt{I}\rho\right)\md\rho,\quad \rho_I^2=(1-1/N)a^4.
\label{eqn:smallscatteqn}
\end{equation}
This is shown in \Figure{fig:lowscatthist} for
$N=2,\ldots,5$.  In this figure, a single realization of a speckle pattern was generated in 2D
using \Equation{eqn:phasorsum} and its histogram plotted against the
theoretical prediction of \Equation{eqn:smallscatteqn}.  No ensemble
averaging was used, though this is typically required to produce reasonable 
statistics~\cite{goodman2007speckle}.
\Figure{fig:lowscatthist} shows an important point about random phasor sums
with a low number of scatterers.  \Equation{eqn:smallscatteqn} converges on
\Equation{eqn:exppdf} fairly rapidly as $N$ increases.  Even at $N=6$ the
difference is fairly negligible, with a sum squared error \num{4.75e-5}.
The implication here is that the statistics of systems with even a small
number of scatterers follow that of a random phasor sum.
