The next edge case is that of a strong single scatterer, shown in 
\Figure{fig:strongsinglefig}.  The intensity PDF in this case is modified
from a Rayleigh distribution to a Rician~\cite{goodman2007speckle} 
\begin{equation}
p_I(I) = \frac{1}{\bar{I}_n} \exp\left(-\frac{I}{\bar{I}_n} - r\right) 2 I_0
\sqrt{\frac{I}{\bar{I}_n} r}
\label{ricedpf}
\end{equation}
where $r$ is the relative intensity of the single scatterer.  Modifying the
intensity of a single scatterer is equivalent to a random phasor sum plus a
constant background.  It is therefore impossible to experimentally
distinguish between a speckle fields with a marginally strong single
scatterer and one whose background has not been completely subtracted.  
As described in \Chapter{ch:experimental}, we have been mindful of this
and have attempted to keep background noise to a minimum, though some is
unaviodable.
