The next edge case is that of a strong single scatterer, shown in 
\Figure{fig:strongsinglefig}.  The intensity PDF is modified
from a Rayleigh distribution to a Rician~\cite{goodman2007speckle} 
\begin{equation}
p_I(I) = \frac{1}{\bar{I}_n} \exp\left(-\frac{I}{\bar{I}_n} - r\right) 2 I_0
\sqrt{\frac{I}{\bar{I}_n} r},
\label{ricedpf}
\end{equation}
where $r$ is the relative scattering strength of the dominant scatterer.
Modifying the intensity of a single scatterer is equivalent to a random phasor
sum plus a constant background.  Unfortunately, in an experiment, failure to
completely account for background light using dark frames results in speckle
statistics indistinguishable from the presence of a marginally strong single
scatterer.  In experimentally acquiring images of speckle, dark frames have
been used in an attempt to keep background noise to a minimum, though some is
unavoidable.
