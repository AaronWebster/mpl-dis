The next edge case is that of a strong single scatterer, which is shown in 
Figure{fig:strongsinglefig}.  The intensity PDF in this case is modified
from a Rayleigh distribution to a Rician 
\begin{equation}
p_I(I) = \frac{1}{\bar{I}_n} \exp\left(-\frac{I}{\bar{I}_n} - r\right) 2 I_0
\sqrt{\frac{I}{\bar{I}_n} r}
\label{ricedpf}
\end{equation}
where $r$ is the relative intensity of the single scatterer.  Modifying the
intensity of a single scatterer is equivalent to a random phasor sum plus a
constant background.  It is thus difficult to distinguish in an experiment
between a speckle fields with strong single scatterer and those whose
background have not been adequately subtracted.
\begin{figure}[ht]
\centering
\import{speckle/figures/lowscatthist/}{spk_hist_intensity_1}
\import{speckle/figures/lowscatthist/}{spk_hist_intensity_2}
\import{speckle/figures/lowscatthist/}{spk_hist_intensity_4}
\import{speckle/figures/lowscatthist/}{spk_hist_intensity_6}
\caption{Simulated speckle fields and PDF of the resulting intensities for
the case of one scatter $r$ times brighter than the rest.}
\label{fig:strongsinglefig}
\end{figure}
