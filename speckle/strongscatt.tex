The next edge case is that of a strong single scatterer, shown in
\Figure{fig:strongsinglefig}.  The intensity PDF is modified
from a Rayleigh distribution to a Rician~\cite{goodman2007speckle}
\begin{equation}
p_I(I) = \frac{1}{\bar{I}_n} \exp\left(-\frac{I}{\bar{I}_n} - r\right) 2 I_0
\sqrt{\frac{I}{\bar{I}_n} r},
\label{eqn:ricepdf}
\end{equation}
where $r$ is the relative scattering strength of the dominant scatterer
compared to the mean intensity, $r=I_0/\bar{I}_n$ where $I_0$ is the intensity
of the single scatterer.  Modifying the intensity of a single scatterer is
equivalent to a random phasor sum plus a constant background.  Unfortunately,
in an experiment, failure to completely account for background light using
dark frames results in speckle statistics indistinguishable from the presence
of a marginally strong single scatterer.  In experimentally acquiring images
of speckle, dark frames have been used in an attempt to keep background noise
to a minimum, though some is unavoidable.
\begin{figure}[ht]
\centering
\import{includes/}{setpgfinc}
\import{speckle/figures/hitspot/}{exp_strongscatt_both}
\caption{Experimental verification of the speckle intensity PDF for a strong
scatterer (left) versues traditional Rayleigh speckle (right).  Surface images
are seen below their respective histograms.}
\label{fig:strongscattexperiment}
\end{figure}

In \Figure{fig:strongscattexperiment}, experimental speckle intensity PDFs
along with surface images are shown for a typical experiment compared with the
case of a ``strong single scatterer'' --- the incident beam focused directly
on a surface feature which strongly scatters light.  The strongly scattering
surface feature is readily apparent in its corresponding surface image.
\SI{57}{\nano\meter} citrate-capped gold nanoparticles were used in this
experiment as scatterers.  It is unknown whether the surface feature causing
light to scatter strongly is a nanoparticle or nanoparticle aggregation, or
simply a surface defect.  Plotted along with the speckle intensity PDFs is
\Equation{eqn:ricepdf} for $r=0$, typical Rayleigh scattering, and $r=10$,
which fits the observed distribution.
