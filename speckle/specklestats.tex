We begin with the assumption that speckle is due to a random phasor sum,
following the treatment of \name{Goodman}~\cite{goodman2007speckle}.
Specifically, we assume the field at each detector position $\mathbf{E}(\mathbf{r})$
is due to the coherent linear superposition of $N$ random phasors
\begin{equation}
\mathbf{E}(\mathbf{r}) = \frac{1}{\sqrt{N}} \sum_{n=1}^{N} a_n \me^{\mi \phi_n}
\label{eqn:phasorsum}
\end{equation}
where $a_n$ and $\phi_n$ is the amplitude and phase of the $n$th phasor.
Furthermore, $\phi$ is randomly distributed on the interval $(-\pi,\pi)$.
In reality, these conditions may be loosened by the physics of the actual
experiment, which we will describe in turn.

By the central limit theorem, as $N\to\infty$, the resultant probability
distribution $p_\mathbf{E}(\mathbf{E})$ of $\mathbf{E}(\mathbf{r})$ is Rayleigh distributed and the
corresponding probability distribution $p_I(I)$ of the intensity
$I(\mathbf{r})=|\mathbf{E}(\mathbf{r})|^2$ is an exponential
\begin{equation}
p_I(I) = \frac{1}{2\sigma_I^2}\exp\left(-\frac{I}{2\sigma_I^2}\right)
\label{eqn:propexp}
\end{equation}

Because the $q$th moment of $p_I(I)$ is given by 
\begin{align}
\bar{I}^q&=(2\sigma_I^2)^q q!\\
         &=\bar{I}^q q!
\end{align}
the standard deviation of is equal to the mean,
$\sigma_I=\bar{I}$.  \Equation{eqn:propexp} can then be rewritten
\begin{equation}
p_I(I) = \frac{1}{\bar{I}}\exp\left(-\frac{I}{\bar{I}}\right)
\label{eqn:exppdf}
\end{equation}
We define here the notion of the \textit{contrast} of speckle, $C$, as the
ratio of the standard deviation to the mean of the intensity
\begin{equation}
C=\frac{\sigma_I}{\bar{I}}
\label{eqn:specklecontrast}
\end{equation}
The exponential PDF and unity contrast account for the discincitive
appearance of speckle.

Before looking at the PDF of the intensity of the experimental cone
speckle, we will study two edge cases to prepare us for some of the
observed behavior.  These cases are for
\begin{inparaenum}[(a)]
\item a small number of scatters, and 
\item a strong single scatterer.
\end{inparaenum}

\subsection{Small Number of Scatterers}
For a small number of scatterers $N$ of intensity $a$, $\rho_I(I)$ becomes
\begin{equation}
p_I(I) = 2\pi^2 \int_0^\infty\! \rho J_0\!\left(\frac{2\pi a
\rho}{\sqrt{N}}\right)^N J_0\!\left(2\pi\sqrt{I}\rho\right)\md\rho
\label{eqn:smallscatteqn}
\end{equation}
with $\rho_I^2=(1-1/N)a^4$.  This is demonstrated in \Figure{fig:lowscatthist} for
$N=2,\ldots,5$.  In this figure, a single speckle pattern was generated in 2D
using \Equation{eqn:phasorsum} and its histogram plotted against the
theoretical prediction of \Equation{eqn:smallscatteqn}.  No ensemble
averaging was used, though this is typically necessary.
\Figure{fig:lowscatthist} shows an important point about random phasor sums
with a low number of scatterers.  \Equation{eqn:smallscatteqn} converges on
\Equation{eqn:exppdf} fairly rapidly as $N$ increases.  Even at $N=6$ the
difference is fairly negligble, with a sum squared error \num{4.75e-5}.
This means we can treat systems with even a small number of scatterers
using the statistics of random phasor sums.
\begin{figure}[ht]
\centering
\import{speckle/figures/lowscatthist/}{spk_hist_2}
\import{speckle/figures/lowscatthist/}{spk_hist_3}
\import{speckle/figures/lowscatthist/}{spk_hist_4}
\import{speckle/figures/lowscatthist/}{spk_hist_5}
\caption{Simulated speckle fields and PDF of the resulting intensities for
the case of a small number of scatterers.}
\label{fig:lowscatthist}
\end{figure}

\subsection{Strong Single Scatterer}
The next edge case is that of a strong single scatterer.
\begin{figure}[ht]
\centering
\import{speckle/figures/lowscatthist/}{spk_hist_intensity_1}
\import{speckle/figures/lowscatthist/}{spk_hist_intensity_2}
\import{speckle/figures/lowscatthist/}{spk_hist_intensity_4}
\import{speckle/figures/lowscatthist/}{spk_hist_intensity_6}
\caption{intensity guy}
\label{fig:lowscatthist}
\end{figure}

\subsection{PDFs of the Cone Speckle}
\begin{figure}[ht]
\centering
\import{speckle/figures/conepdf/}{spk_cone_pdf}
\caption{cone pdf}
\label{fig:conepdf}
\end{figure}

std/mean 1.0637

