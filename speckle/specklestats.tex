We begin with the assumption that speckle is due to a random phasor sum,
following the treatment of \name{Goodman}~\cite{goodman2007speckle}.
Specifically, we assume the field at each detector position $\mathbf{E}(\mathbf{r})$
is due to the coherent linear superposition of $N$ random phasors
\begin{equation}
\mathbf{E}(\mathbf{r}) = \frac{1}{\sqrt{N}} \sum_{n=1}^{N} a_n \me^{\mi \phi_n}
\label{eqn:phasorsum}
\end{equation}
where $a_n$ and $\phi_n$ is the amplitude and phase of the $n$th phasor.
Furthermore, $\phi$ is randomly distributed on the interval $(-\pi,\pi)$.
In reality, these conditions may be loosened by the physics of the actual
experiment, which we will describe in turn.

By the central limit theorem, as $N\to\infty$, the resultant probability
distribution $p_\mathbf{E}(\mathbf{E})$ of $\mathbf{E}(\mathbf{r})$ is Rayleigh distributed and the
corresponding probability distribution $p_I(I)$ of the intensity
$I(\mathbf{r})=|\mathbf{E}(\mathbf{r})|^2$ is an exponential
\begin{equation}
p_I(I) = \frac{1}{2\sigma_I^2}\exp\left(-\frac{I}{2\sigma_I^2}\right)
\label{eqn:propexp}
\end{equation}

Because the $q$th moment of $p_I(I)$ is given by 
\begin{align}
\bar{I}^q&=(2\sigma_I^2)^q q!\\
         &=\bar{I}^q q!
\end{align}
the standard deviation of is equal to the mean,
$\sigma_I=\bar{I}$.  \Equation{eqn:propexp} can then be rewritten
\begin{equation}
p_I(I) = \frac{1}{\bar{I}}\exp\left(-\frac{I}{\bar{I}}\right)
\label{eqn:exppdf}
\end{equation}
We define here the notion of the \textit{contrast} of speckle, $C$, as the
ratio of the standard deviation to the mean of the intensity
\begin{equation}
C=\frac{\sigma_I}{\bar{I}}
\label{eqn:specklecontrast}
\end{equation}
The exponential PDF and unity contrast account for the discincitive
appearance of speckle.

Before looking at the PDF of the intensity of the experimental cone
speckle, we will study two edge cases to prepare us for some of the
observed behavior.  These cases are for
\begin{inparaenum}[(a)]
\item a small number of scatters, and 
\item a strong single scatterer.
\end{inparaenum}
