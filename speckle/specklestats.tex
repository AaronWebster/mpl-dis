Consider an SPP excited by a focused beam in a prism-coupled setup.  Without
loss of generality, the center of the focused beam is defined as the origin
of a three-dimensional Cartesian coordinate system $(x,y,z)$.  If the beam is
incident in the $x$-$z$ plane, the SPP has an initial phase of $\exp(\mi \ksp
x)$.  In the present analysis, the out of plane components of the Gaussian
beam are neglected.

As mentioned in \Section{sec:coneexist}, surface roughness or other
inhomogeneities can elastically scatter an SPP during propagation.  SPP
scattering on the surface changes the direction, but not the magnitude, of the
in-plane $k$-vector.  Upon re-radiation as a photon, the phase to a point
$(x^\prime,y^\prime,z^\prime)$ in the far field is
\begin{equation}
\exp\!\left(\mi k_0 \sqrt{{(x-x^\prime)}^2 + {(y-y^\prime)}^2 + {(z-z^\prime)}^2}\right).
\end{equation}
In the present coordinate system, $z=0$ is the plane of scattering and the
contribution to the field by a single scattering event is
\begin{equation}
\mathbf{E}\big(x^\prime,y^\prime,z^\prime\big)=\exp\!\big( \mi \ksp x \big)
\exp\!\left(\mi k_0 \sqrt{{(x-x^\prime)}^2 + {(y-y^\prime)}^2 + {(z-z^\prime)}^2}\right).
\label{eqn:sppsinglescattering}
\end{equation}
\Equation{eqn:sppsinglescattering} assumes single scattering: an
SPP is scattered only once during its propagation.
%We will relax this
%assumption in \Section{sec:multiplescattering} when we consider multiple
%scattering.

It is further assumed that the $(x,y,z=0)$ coordinates from which an SPP is
re-radiated as a photon are randomly distributed following the treatment of
\name{Goodman}~\cite{goodman2007speckle} for random phasor sums.  To wit, the
field at each detector position $\mathbf{E}(\mathbf{r})$ is due to the
coherent linear superposition of $N$ random phasors
\begin{equation}
\mathbf{E}(\mathbf{r}) = \frac{1}{\sqrt{N}} \sum_{n=1}^{N} a_n \me^{\mi \varphi_n},
\label{eqn:phasorsum}
\end{equation}
where $a_n$ and $\varphi_n$ is the amplitude and phase of the $n$th phasor
and $\varphi$ is randomly distributed on the interval $(-\pi,\pi)$.  For
SPP scattering, $\varphi$ is the phase of \Equation{eqn:sppsinglescattering}.
In reality these conditions may be loosened by the physics of the actual
experiment, which we describe shortly.

By the central limit theorem, as $N\to\infty$, the resultant probability
distribution function (PDF) $p_\mathbf{E}(\mathbf{E})$ of $\mathbf{E}(\mathbf{r})$ is Rayleigh distributed and the
corresponding probability distribution function $p_I(I)$ of the intensity
$I(\mathbf{r})=|\mathbf{E}(\mathbf{r})|^2$ is an exponential
\begin{equation}
p_I(I) = \frac{1}{2\sigma_I^2}\exp\left(-\frac{I}{2\sigma_I^2}\right).
\label{eqn:propexp}
\end{equation}

Given that the $q$th moment of $p_I(I)$ is
\begin{align}
\bar{I}^q&={(2\sigma_I^2)}^q q!\\
         &=\bar{I}^q q!,
\end{align}
the standard deviation is equal to the mean,
$\sigma_I=\bar{I}$.  \Equation{eqn:propexp} can then be rewritten as
\begin{equation}
p_I(I) = \frac{1}{\bar{I}}\exp\left(-\frac{I}{\bar{I}}\right).
\label{eqn:exppdf}
\end{equation}

As a convienent measure, the \textit{contrast} of speckle, $C$, is defined as
the ratio of the standard deviation to the mean of the intensity
\begin{equation}
C=\frac{\sigma_I}{\bar{I}}\quad\text{(speckle contrast)}.
\label{eqn:specklecontrast}
\end{equation}
The exponential PDF and unity contrast are often mentioned as the reason
for the discincitive ``speckled'' appearance of speckle~\cite{goodman2007speckle}.
