Consider a surface plasmon polariton excited by a focussed beam in a prism
coupled setup.  Without loss of generality, we define the center of the
focussed beam as the origin of a three dimensional Cartesian coordinate
system.  If the beam is incident in the $x$-$z$ plane, the SPP will have an
initial phase upon creation of $\exp(\mi \ksp x)$.  For simplicity, we
neglect any potential out of plane components a Gaussian beam may posess. 

As mentioned in \Section{sec:coneexist}, surface roughness or other
inhomogenities can elastically scatter an SPP during propagation.  This
acts to change the direction, but not the magnitude, of the in-plane
$k$-vector.  Upon re-radiation as a photon, the phase to a point
$(x^\prime,y^\prime,z^\prime)$ in the far field is 
\begin{equation}
\exp\!\left(\mi k_0 \sqrt{ (x-x^\prime)^2 + (y-y^\prime)^2 + (z-z^\prime)^2}\right).
\end{equation}

Because $z=0$ is the plane of scattering, we can rewrite the contribution
to the field by a single SPP scattering event as
\begin{equation}
\mathbf{E}\big(x^\prime,y^\prime,z^\prime\big)=\exp\!\big( \mi \ksp x \big)
\exp\!\left(\mi k_0 \sqrt{ (x-x^\prime)^2 + (y-y^\prime)^2 + (z-z^\prime)^2}\right).
\label{eqn:sppsinglescattering}
\end{equation}
In \Equation{eqn:sppsinglescattering} we have assumed single scattering; an
SPP is scattered only once during propagation.  We will relax this
assumption in \Section{sec:multiplescattering} when we consider multiple
scattering.

We assume that the $(x,y,z=0)$ coordinates from which an SPP is re-radiated
as a photon are randomly distributed and follow the treatment of
\name{Goodman}~\cite{goodman2007speckle}, describing the speckle as a
random phasor sum.  To wit, we assume the field at each detector position
$\mathbf{E}(\mathbf{r})$ is due to the coherent linear superposition of $N$
random phasors
\begin{equation}
\mathbf{E}(\mathbf{r}) = \frac{1}{\sqrt{N}} \sum_{n=1}^{N} a_n \me^{\mi \varphi_n},
\label{eqn:phasorsum}
\end{equation}
where $a_n$ and $\varphi_n$ is the amplitude and phase of the $n$th phasor 
and $\varphi$ is randomly distributed on the interval $(-\pi,\pi)$.  For SPP
SPP scattering, $\varphi$ is the phase of \Equation{eqn:sppsinglescattering}.
In reality these conditions may be loosened by the physics of the actual
experiment, which we describe shortly.

By the central limit theorem, as $N\to\infty$, the resultant probability
distribution function (PDF) $p_\mathbf{E}(\mathbf{E})$ of $\mathbf{E}(\mathbf{r})$ is Rayleigh distributed and the
corresponding probability distribution function $p_I(I)$ of the intensity
$I(\mathbf{r})=|\mathbf{E}(\mathbf{r})|^2$ is an exponential
\begin{equation}
p_I(I) = \frac{1}{2\sigma_I^2}\exp\left(-\frac{I}{2\sigma_I^2}\right).
\label{eqn:propexp}
\end{equation}

Given that $q$th moment of $p_I(I)$ is 
\begin{align}
\bar{I}^q&=(2\sigma_I^2)^q q!\\
         &=\bar{I}^q q!,
\end{align}
the standard deviation of is equal to the mean,
$\sigma_I=\bar{I}$.  \Equation{eqn:propexp} can then be rewritten as
\begin{equation}
p_I(I) = \frac{1}{\bar{I}}\exp\left(-\frac{I}{\bar{I}}\right).
\label{eqn:exppdf}
\end{equation}

We define here the \textit{contrast} of speckle, $C$, as the
ratio of the standard deviation to the mean of the intensity
\begin{equation}
C=\frac{\sigma_I}{\bar{I}}\quad\text{(speckle contrast)}.
\label{eqn:specklecontrast}
\end{equation}
The exponential PDF and unity contrast are often mentioned as the reason
for the discincitive appearance of speckle~\cite{goodman2007speckle}.

As our ultimate goal in this section is to understand the statistics of cone
speckle in the context of the mathematics of a random phasor sum, it
is pertenant us to look at a couple of edge cases: situations for which
there are only small number of scatterers in the phasor sum, and ones in
which the intensity of a single scatterer dominate amongst many.
