The complex two-dimensional field on the sensor is given by
$\mathbf{E}(\mathbf{r}) = \mathbf{E}(x,y)$.
There are six fundamental parameters describing a vortex, which in turn
completely define the underlying field.  In two dimensions, these are
$\mathbf{r}$ and the partial derivative $\nabla \mathbf{E}(\mathbf{r})$.
These in turn can be transformed into six ``morphological parameters'': its
coordinates, rotation, skewness, amplitude, anisotropy, and a derived
charge, signifying whether the vortex phase is clockwise or anti-clockwise
\begingroup
\newcommand{\rx}{\re(\nabla \mathbf{E})_x}
\newcommand{\ry}{\re(\nabla \mathbf{E})_y}
\newcommand{\ix}{\im(\nabla \mathbf{E})_x}
\newcommand{\iy}{\im(\nabla \mathbf{E})_y}
\begin{align}
\rho   &= \arctan\!\left(\frac{\ry}{\rx}\right) \quad & \text{(rotation)}\\
\sigma &= -\arctan\!\left(\frac{\ix}{\iy}\right)-\rho \quad & \text{(skewness)}\\
a      &= \frac{\rx}{\cos\rho} = \frac{\ry}{\sin\rho} \quad & \text{(amplitude)}\\
\alpha &= \frac{-\ix}{\bigl(\alpha \sin\left(\rho+\sigma\right)\bigr)}
        = \frac{-\iy}{\bigl(\alpha \cos\left(\rho+\sigma\right)\bigr)} \quad & \text{(anisotropy)} \\
q      &= \sgn\!\left( \left|
          \begin{pmatrix}
          \rx & \ry \\
          \ix & \iy
          \end{pmatrix}
          \right|\right) \quad & \text{(charge)}
\end{align}
\endgroup

In terms of these parameters, the field is then
\begin{equation}
\mathbf{E}(x,y) = a \Bigl( x \cos\rho + y \sin \rho 
+ \mi \alpha \bigr(y \cos(\rho + \sigma) + x \sin(\rho+\sigma)\bigr)\Bigr)
\end{equation}

note that \rho must be \pm \pi, \sigma \pm \pi/2

% FIGURE H
