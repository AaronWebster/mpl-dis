The intensity PDF of the cone is shown in \Figure{fig:conepdf} for a single
speckle realization at an azimuthal angle of $\phi=\SI{90}{\degree}$.  From a
theoretical point of view, the choice of $\phi=\SI{90}{\degree}$ is arbitrary;
the speckle intensity about the cone is
homogeneous~\cite{schumann2009surface}.  However, it is experimentally
convienent to observe at this location due to its non-coincidence with the
focussing optics.  Again, it is usually required to ensemble average over many
different realizations, however useful statistics emerge for a single
realization when averaging over many speckles, the same strategy employed in
multispeckle diffusing wave spectroscopy~\cite{zakharov2006multispeckle}.  The
PDF was taken for pixel values along the transverse (azimuthal) speckle
coordinate and ensemble averaged over all avaliable pixel ranges.

The observed PDF is well described by the Rayleigh distribution of
\Equation{eqn:exppdf}.  There is a small deviation from Rayleigh statistics to
Rician likely attributed to an imperfect background subtraction.  The contrast
is near unity, with a value of $C=1.0637$.  Averaging over many speckle
realizations in addition to the one shown in \Figure{fig:conepdf} gives the
same result of a near unity contrast.
\begin{figure}[ht]
\centering
\import{speckle/figures/conepdf/}{spk_cone_pdf}
\caption{Intensity PDF for light in the cone.  Top: histogram.  Bottom:
unwrapped raw data on the sensor.}
\label{fig:conepdf}
\end{figure}
