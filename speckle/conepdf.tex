The intensity PDF of the cone is shown in \Figure{fig:conepdf} for a single
speckle realization at an azimuthal angle of $\phi=\SI{90}{\degree}$.
Again, it is usually required to ensemble average over many different
realizations, but as shown useful statistics emerge for a single
realization when averaging over many speckles; the same strategy employed
in multispeckle diffusing wave spectroscopy~\cite{zakharov2006multispeckle}.

The observed PDF is well described by the Rayleigh distribution of
\Equation{eqn:exppdf}.  There is a small constant background, but this is
likely attributed to experimental noise or an imperfect background
subtraction in image processing.  The contrast is near unity, with a value
of $C=1.0637$.  Averaging over many speckle realizations we obtain the same
result of a near unity contrast.  
\begin{figure}[ht]
\centering
\import{speckle/figures/conepdf/}{spk_cone_pdf}
\caption{cone pdf}
\label{fig:conepdf}
\end{figure}

<<<<<<< HEAD
The experimental intensity PDF and contrast fit very well with the
predictions of found using random phasor sums.  
=======
Averaging over many speckle realizations we obtain the same result of a
near unity contrast.  The intensity PDF of cone speckle is therefore very
well described by the statistics of a random phasor sum.
>>>>>>> 5f08d3f09e4ac496045f1752eb7c3247c58410c9
