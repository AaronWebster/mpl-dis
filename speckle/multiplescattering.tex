Thus far we have described speckle mathematically as a sum of random,
statistically independent, phasors.  This is in essence what is known as
\textit{single scattering}.  However, a much more interesting and
phenomenologically complex situation arises when we allow higher order
scattering processes to take over.  This is, unsurprisingly, known as
\textit{multiple scattering}, a theme which will arise repeatedly in
our investigations of surface plasmon random scattering.  

The transition from the single to the multiple scattering occurs upon
fulfillment of the Ioffe-Regel criterion, that is that the transport mean free path
$l^*$ is much larger than the spatial frequency of the optical wave, or
\begin{equation}
k l^* \approx 1
\end{equation}
where $k=\omega/c$ as usual.  We define the transport mean path as the
characteristic distance over which the incident wave is scattered out of
its incoming direction~\cite{berkovits1994correlations}.  This is the
optical analog of electron transport in condensed matter physics.  There,
one uses the Fermi wavenumber $k_F$ and takes $l^*$ to be the mean free
path.  In this sense, materials for which $k_F l^* \gg 1$ are conductors, and materials
for which $k_F l^* \ll 1$ are insulators.  
We further define the multiple scattering regime in terms of the sample
length $L$, such that if $l^* \ll L$ and $1/(k l^*) \ll 1$, the system is in
the multiple scattering regime.  


\Figure{fig:spratio}

enhanced sensitivity of multiple scattering
average change in B. Simons and B. Altshuler, Phys. Rev. Lett. 70, 4063

intensity fluctuations, not scatterer displacements

High sensitivity of multiple-scattering speckle pat-
terns to scatterer motion gave rise to a new technique
for studying the scatterer dynamics in disordered, tur-
bid media, the so-called “diffusing-wave spectroscopy”
(DWS).19,26– 29 The latter is now widely applied in
concentrated colloidal suspensions,19,26–30 foams,31–35
emulsions,36 – 38 granular39 – 41 and biological42–44 media.

