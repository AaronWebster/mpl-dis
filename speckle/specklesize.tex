The ``size'' of a speckle is another important figure of merit.
Following \name{Goodman}~\cite{goodman1975statistical} and
\name{Dainty}~\cite{dainty1975laser}, we define the size of a
speckle in terms of the area of the normalized
autocovariance function of the speckle intensity pattern, $c_I(\Delta x,
\Delta y)$.  The normalized autocovariance function is defined in terms of
the autocorrelation $\Gamma_I(\Delta x, \Delta y)$ by 
\begin{equation}
c_I(\Delta x, \Delta y) = \frac{\Gamma_I(\Delta x, \Delta y) - \bar{I}^2}{\bar{I}^2},
\end{equation}
and the autocorrelation is defined as
\begin{equation}
\Gamma_I(\Delta x, \Delta y) = \intinfty\intinfty I(x,y) \tilde{I}(x+\Delta x,y+\Delta y) \md x \md y.
\end{equation}
For computation on real images, the integral is replaced by a sum.  Taking
advantage of the Wiener–Khinchin theorem, the autocorrelation is simply the
Fourier transform of the power spectral density, vis.
\begin{equation}
\Gamma_I(\Delta x, \Delta y) = \ff{|I(x,y)|^2}(\Delta x, \Delta y)
\end{equation}
and thus  $c_I(\Delta x, \Delta y)$ can be efficiently calculated on a
computer\footnote{For example in MATLAB this is implimented with
\texttt{xcov(I,'coeff')}}.

With these definitions out of the way, the area of $c_I(\Delta x, \Delta
y)$ is simply its integral over $\Delta x$ and $\Delta y$
\begin{align}
\mathscr{A}_c &= \intinfty\intinfty c_I(\Delta x, \Delta y) \md \Delta x \md \Delta y\\
&=\frac{(\lambda z)^2}{A}
\label{eqn:spotsize}
\end{align}

Conceptually, \Equation{eqn:spotsize} can be thought of as follows: the
difference between a ``bright'' spot and a ``dark'' spot in a speckle field
occurs when the scattering angle changes such that the optical path length
difference from the scattering center to the far field changes by
$\lambda$; here the speckle intensity is uncorrelated.

We have measured the speckle size in the experiment and found that, in
apparent contradiction with \Equation{eqn:spotsize}, the speckle size does
\textit{not} change with spot size for long range surface plasmons.  This
is shown in \Figure{fig:spotsize} for three different characteristic spot
sizes.  Here we have made a relative measurement of the scattering spot
size by fitting a Gaussian to its transverse dimension, and reporting the
full width at half maximum of that measurement as a measurement of the
actual spot size.
\begin{figure}[ht]
 \centering
 \import{speckle/figures/}{spotsizefig}
 %\includegraphics[keepaspectratio]{speckle/figures/spotsize/test.pdf}
 \caption{Speckle size versus spot size.  Inset images show an example of
 the raw spot size data from which the plot was computed.  Inset images are
 \SI{2.26x2.26}{\milli\meter} with major ticks at
 \SI{500}{\micro\meter} intervals.}
 \label{fig:spotsize}
\end{figure}


