Wiener–Khinchin theorem
measure of speckle size
speckle size independent
plots, show spot scattering size
predicted speckle size

The "size" of the speckles is a function of the wavelength of the light,
the size of the laser beam which illuminates the first surface, and the
distance between this surface and the surface where the speckle pattern is
formed. This is the case because when the angle of scattering changes such
that the relative path difference between light scattered from the centre
of the illuminated area compared with light scattered from the edge of the
illuminated changes by λ, the intensity becomes uncorrelated. Dainty [1]
derives an expression for the mean speckle size as λz/L where L is the
width of the illuminated area and z is the distance between the object and
the location of the speckle pattern.

3. Speckle size
To estimate the speckle size, we calculated the normalized autocovariance
function of the
intensity speckle pattern obtained in the observation plane ( x, y ) . This
function corresponds
to the normalized autocorrelation function of the intensity; it has a zero
base and its width
provides a reasonable measurement of the “average width” of a speckle [1].
If I ( x 1 , y 1 ) and
I ( x2 , y2 ) are the intensities of two points in the observation plane
 ( x, y ) , the intensity
 autocorrelation function is defined by Eq. (2) [1]:
 RI (∆x, ∆y ) = I ( x1 , y1 ) I ( x2 , y 2 )
  (2)
	where ∆x = x 1 − x2 and ∆ y = y 1 − y2 .
	y2 = 0 , x 1 = x and y 1 = y , we can write:
	corresponds to a spatial average. If x2 = 0 ,

