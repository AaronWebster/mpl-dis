\begin{abstract}
  The commercialization of biosensing is driven by application.  Toward this
  goal, many interesting physical phenomena inherent in biosensing platforms
  are de-emphasized in favor of a quantitative and repeatable assay.  Such
  phenomena are often treated as a source of noise to be overcome, and, as the
  engineering of a platform is perfected towards a specific biodetection
  goal, the principles of these phenomena become obscured to those who employ
  it solely as a tool in pursuit of other matters.

  This is a ubiquitous pattern in science, perhaps the best examples of which
  are two most ubiquitous biotransducers: the optical \gls{spr}, and the
  mechanical piezoelectric \gls{qcm}.  These two devices, the subject of
  the present work, both possess novel modes of operation which extend
  their versatility to other bio-relevant sensing applications.

  In the optical regime \glspl{spp} excited evanescently
  by light are by far the most popular label-free affinity biotransducer for
  monitoring bulk refractive index changes.  The sensitivity of \gls{spr} is primarily
  due to the field enhancement by \glspl{spp} on the sensor surface, however \glspl{spp}
  themselves also possess high spatial resolution beyond the diffraction limit;
  a property typically absent as specific feature of the \gls{spr} sensorgram.  In
  this work, it is shown that by considering optical speckle from single and
  multiply scattered \glspl{spp} inherent in the \gls{spr} signal itself, an entirely new set
  of information can be obtained descriptive of the underlying scattering
  microstructure.  Furthermore, it is demonstrated that the motion and addition
  of single nanoparticles can be resolved in an unmodified \gls{spr} setup, whereby
  the breadth of \gls{spr} experiments may be extended to encompass both bulk sensing
  and discrete events on the nanoscale.

  Moving into the mechanical regime, the sensitivity, low cost, ease of use, and
  integrability have made the piezoelectric quartz crystal microbalance an ideal
  biotransducer for real time monitoring properties such as viscoelasticity, as
  well as an affinity sensor for mass adsorption.  Naturally, these desirable
  features do not come without disadvantages: the underlying mechanical
  properties of the sample are often not revealed by the relative and stepwise
  changes in the \gls{qcm} sensorgram, an issue complicated by choice of theoretical
  model.  Here it is shown that application of controlled centrifugal forces in
  a \gls{qcm} assay has a profound utility in revealing the underlying biomechanical
  properties of a sample.  This \textit{centrifugal force quartz crystal
    microbalance} concept works by modifying the \gls{qcm}-sample coupling mechanism.
  Centrifugal force is demonstrated to be useful in not only to enhancing the
  sensitivity of a traditional \gls{qcm} measurement, but also in obtaining the
  sample's complex biomechanical properties repeatedly \textit{in situ} and in
  real time.
\end{abstract}
