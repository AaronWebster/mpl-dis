\begin{abstract}
Biosensors based on surface plasmon resonance play a central roll as a
simple and remarkably responsive label-free method for characterizing and
quantifying biomolecular interactions.  Amongst the most popular of these
sensors are those which excite localized surface plasmon polaritons on thin
metal films in prism coupled configurations.  These platforms, despite
their ubiquity and commercial success, are host to an amazing depth of
useful phenomena which has yet to be explored in the context of biosensing.
In this work we detail investigations of single and multiply scattered
surface plasmon polaritons from such a system.  Specifically, we look at
the structure of the re-radiated optical field which contain additional
features such as near field self-interference, speckle, and optical
vorticies, and their relations to the scattering microstructure.  Our
results are suggestive of several possible avenue for advancing the
ultimate detection limits of surface plasmon based biosensors for both
single particles and bulk refractive index measurements.
\end{abstract}
