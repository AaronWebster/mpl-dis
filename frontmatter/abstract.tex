\begin{abstract}
The commercialization of biosensing platforms is driven by application.  To
this end, some of the more interesting physics often is de-emphasized in the
quest for a quantitative and repeatable assay.  In this work we take a step
back and investigate novel applications of two such platforms: the optical
surface plasmon resonance biosensor and the mechanical quartz crystal
microbalance.  For the surface plasmon resonance biosensor, we detail
investigations of single and multiply scattered surface plasmon polaritons
from such a system.  Specifically, we look at the structure of the
re-radiated optical field which contain additional features such as near
field self-interference, speckle, and optical vortices, and their
relations to changes in the scattering microstructure.  For the quartz crystal
microbalance, we describe a new centrifugal force-based biosensing
technique.  This technique allows for repeated, non-destructive
interrogation of a samples mechanical and viscoelastic properties to be
obtained \textit{in situ} and in real time.
\end{abstract}
