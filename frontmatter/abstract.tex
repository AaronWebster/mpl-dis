\begin{abstract}
The commercialization of biosensing is driven by application.  Toward this
goal, many of the interesting phenomena inherent in biosensing platforms
are de-emphasized in favor of a quantitative and repeatable assay.  Such
phenomena are often treated as a source of noise to be overcome, and as the
engineering of a platform is perfected towards a specific biodetection
goal, the principles of these phenomena become obscured to those who employ
it solely as a tool to advance other studies.

This is a ubiquitous pattern in science, and perhaps the best examples of
which are the two most ubiquitous and popular biotransducers: the optical
surface plasmon resonance (SPR), and the mechanical piezoelectric quartz
crystal microbalance (QCM).  These two devices are the subject of the
present work, wherein it is shown that they both possess new modes of
operation extending their versatility to other types of samples relevant to
the biosensing community. 

In the optical regime, surface plasmon polaritons (SPPs) excited
evanescently by light are by far the most popular label-free affinity
biotransducer for monitoring bulk refractive index changes.  The
sensitivity of SPR is primarily due to the field enhancement by SPPs on the
sensor surface, but the SPPs themselves also possess high spatial
resolution beyond the diffraction limit; a property which traditionally has
not manifest itself as a specific feature of the SPR sensorgram.  However,
in this work it is shown that by considering optical speckle from single
and multiply scattered SPPs inherent in the SPR signal, an entirely new set
of information can be obtained describing the underlying scattering
microstructure.  It is further demonstrated that one can resolve the motion
and addition of single nanoparticles in an unmodified SPR setup, extending
the breadth of SPR experiments to encompass both bulk sensing and discrete
events.

In the mechanical regime, the sensitivity, low cost, ease of use, and
integrability have made the piezoelectric quartz crystal microbalance an
ideal biotransducer for real time monitoring properties such as
viscoelasticity, as well as an affinity sensor for monitoring mass
adsorption.  Naturally, these desirable features do not come without
disadvantages: the underlying mechanical properties of the sample are often
not revealed by the relative and stepwise changes in the QCM sensorgram, an
issue complicated by choice of theoretical model.  In this work, it is
shown that application of controlled centrifugal forces in a QCM assay has
a profound utility in revealing the underlying biomechanical properties of
a sample.  This centrifugal force quartz crystal microbalance concept works
by modifying the QCM-sample coupling mechanism.  It is demonstrated that
centrifugal force can be used not only to enhance the sensitivity of a
traditional QCM measurement, but also to obtain the sample's complex
biomechanical properties in situ and in real time.  
\end{abstract}
