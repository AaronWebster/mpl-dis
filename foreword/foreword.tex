Note: probably should brush this up at the end when all the topics are said
and done.

This dissertation is all about what happens when surface plasmon polaritons
interfere and scatter.

\section{What is New Here}
In many large works it is sometimes difficult to discern what is actually
new from rehashings of things others have done.  Here I intended to clarify
this situation.  To this accord, the following is a list of explicitly new
results which I present:

\begin{description}
\item[{Interference of Scattered SPPs}]
For SPPs excited in the Kretschmann configuration with a focussed beam
whose propagation distance is larger than focal spot, a particular one-sided
interference pattern, is observed in the specular direction.  This was
thought to be an interference effect between the specularly reflected beam
and a re-radiated plasmon component, but we have discovered a further
insight in interpreting it as a simple consequence of causality.
\item[{Bulk Refractive Index Sensitivity}] 
Using Fourier analysis, we have studied the sensitivity of SPR in an intensity
interrogation configuration as a function of propagation distance.  This
provides a useful measure of the sensitivity of the one-sided interference
pattern found in the specular and conically directed beams.
\item[{Refractive Index Sensing With Conical Speckle}]
If the surface is rough, the conically directed light contains speckle.  We
have analyzed the speckle patterns as a function of changes in bulk
refractive index.
\item[{SPP Nanoparticle Scattering}]
Using the speckle in the cone, we observe the changes in intensity and
correlation functions associated with the addition or motion of a single
nanoparticle.
After writing, update the list with all the new stuff.
\end{description}

\section{Organization}
This work is roughly organized in the following way.  \Chapter{ch:existence}
is what you are currently reading.  \Chapter{ch:existence} lays out the
mathematical details regarding the conditions under which SPPs may be
excited and their corresponding physical properties.  Following,
information regarding the experiment: protocols, construction, and data
analysis, are described in detail in \Chapter{ch:experimental}.  These two
chapters form the mathematical (\Chapter{ch:existence}) and physical
(\Chapter{ch:experimental}) basis for the remaining text.

Analysis of the system begins with \Chapter{ch:bulkri}.  Here, the bulk
refractive index sensing properties of the cone are described.  Closely
related, \Chapter{ch:interference} discusses a newly discovered
interference phenomena in the cone and the possible utility in the context
of SPR refractive index sensing.

Perhaps one of the most interesting features of the cone is speckle, the
subject of \Chapter{ch:speckle}.  Here we compare the properties of cone
speckle with those of classic speckle fields in the context of
correlations, refractive index pertrubations, and multiple scattering
effects.

Finally, in \Chapter{ch:scatteringmicro} we look at the influence of the cone
speckle on the scattering microstructure itself.
