\usepackage{t1enc}
\usepackage[T1]{fontenc}
\usepackage{ulem}
\usepackage{import}
\usepackage{amsmath}
\usepackage{amssymb}
\usepackage{paralist}
\usepackage{booktabs}
\usepackage[version=3]{mhchem}
\usepackage{wrapfig}
\usepackage{pifont}
\usepackage{mathrsfs}
\usepackage{tikz}
\usepackage{xcolor}
\usepackage{color}
\usepackage{enumitem}
\usepackage{datetime}
\usepackage{fullpage}

%\hypersetup{
% colorlinks=false,
% linkbordercolor={1 1 1},
% citebordercolor={1 1 1},
% urlbordercolor={1 1 1} 
%}

% New definition of square root:
% it renames \sqrt as \oldsqrt
% This definition puts a little vertical guy at the end so it's more
% obvious where the square root actually ends.
\let\oldsqrt\sqrt
% it defines the new \sqrt in terms of the old one
\def\sqrt{\mathpalette\DHLhksqrt}
\def\DHLhksqrt#1#2{%
\setbox0=\hbox{$#1\oldsqrt{#2\,}$}\dimen0=\ht0
\advance\dimen0-0.2\ht0
\setbox2=\hbox{\vrule height\ht0 depth -\dimen0}%
{\box0\lower0.4pt\box2}}

% integrals with infinity bounds
\newcommand{\intinfty}{\int_{-\infty}^{\infty}}

% consistent formatting of object labels
\newcommand{\Figure}[1]{Figure \ref{#1}}
\newcommand{\Equation}[1]{Equation \ref{#1}}
\newcommand{\Table}[1]{Table \ref{#1}}
\newcommand{\Section}[1]{Section \ref{#1}}
\newcommand{\Chapter}[1]{Chapter \ref{#1}}
\newcommand{\Appendix}[1]{Appendix \ref{#1}}

% missing mathematical operators
\DeclareMathOperator{\sinc}{sinc}
\DeclareMathOperator{\sech}{sech}
\DeclareMathOperator{\sgn}{sgn}
\DeclareMathOperator{\erf}{erf}
\DeclareMathOperator{\inverf}{inverf}
\DeclareMathOperator{\arcsinh}{arcsinh}
\DeclareMathOperator{\arccosh}{arccosh}
\DeclareMathOperator{\arctanh}{arctanh}
%\DeclareMathOperator{\Re}{Re}
%\DeclareMathOperator{\Im}{Im}

% use roman type for natural base e and sqrt(-1)
\newcommand{\me}{{\mathrm{e}}}
\newcommand{\mi}{{\mathrm{i}}}

% roman type for the derivative, plus a space
\newcommand{\md}{\,\mathrm{d}}

% fourier transform and the reverse
\newcommand{\ff}[1]{{\mathscr{F}^{+}\bigl(#1\bigr)}}
\newcommand{\fr}[1]{{\mathscr{F}^{-}\bigl(#1\bigr)}}

% hilbert transform and the reverse
\newcommand{\hf}[1]{{\mathscr{H}^{+}\bigl(#1\bigr)}}
\newcommand{\hr}[1]{{\mathscr{H}^{-}\bigl(#1\bigr)}}

% custom lengths for figures

% width for side by side figures
\newlength{\twoupwidth}
\setlength{\twoupwidth}{7.5cm}

% width and height for default single figure
\newlength{\oneupwidth}
\setlength{\oneupwidth}{0.90\textwidth}
\newlength{\oneupheight}
\setlength{\oneupheight}{0.55623059\textwidth}

\usepackage{pgfplots}
\pgfplotsset{compat=newest}
\usepgfplotslibrary{units}
\usetikzlibrary{pgfplots.units}
\usetikzlibrary{calc}

% custom colors for 2D plots - these are the same ones mathematica uses by
% default
\definecolor{colora}{RGB}{63,61,153}
\definecolor{colorb}{RGB}{153,61,113}
\definecolor{colorc}{RGB}{153,139,61}
\definecolor{colord}{RGB}{61,153,86}
\definecolor{colore}{RGB}{61,90,153}
\definecolor{colorf}{RGB}{153,61,144}
\definecolor{colorg}{RGB}{153,109,61}
\definecolor{colorh}{RGB}{67,153,61}
\definecolor{colori}{RGB}{61,121,153}
\definecolor{colorj}{RGB}{132,61,153}

\pgfplotsset{
 /pgfplots/colormap={jet}{rgb255(0cm)=(0,0,128) rgb255(1cm)=(0,0,255)
 rgb255(3cm)=(0,255,255) rgb255(5cm)=(255,255,0) rgb255(7cm)=(255,0,0)
 rgb255(8cm)=(128,0,0)}
}
\usepgfplotslibrary{units}
\usetikzlibrary{pgfplots.units} 

\usepackage{siunitx}
\DeclareSIUnit\molar{\mole\per\cubic\deci\metre}
\DeclareSIUnit\Molar{\textsc{M}}

\sisetup{ 
% load-configurations=abbreviations
% round-mode = places,
}%

