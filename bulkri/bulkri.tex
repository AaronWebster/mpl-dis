The sensitivity of surface plasmon resonance to bulk refractive index
pertrubations is perhaps its most popular and widely utilized property.
This is motivated by the system's quite remarkable response, which has been
reported at up to \SI{1e-8}{RIU} (refractive index units) in some
applications~\cite{sprreview}, shot noise limited.  Bulk refractive index
sensing with SPR is well understood, so in this chapter we focus on these
types of measurements in the cone without consideration of speckle, a topic
reserved for \Chapter{ch:speckle}.  

In a prism coupled setup, there are three main methods used to
monitor changes in the SPR resonance condition.  All three of these
interrogation methods are equivalent in terms of their real-world
resolution and detection limit.~\cite{pines1952collective}
\begin{description}
	\item [{Angular}] The change in the angle of the SPR minimum as a function
					of refractive index is monitored.  In the angular modulation method,
					The prism may be scanned through a range of angles or the angular
					spectrum may be obtained simultaneously using a focussed beam and
					sensor array.
 \item [{Wavelength}] The prism is fixed in place and its reflectivity at
  a single angle is monitored as a function of wavelength.
	\item [{Intensity}] For a fixed wavelength and incident angle, the
		intensity of the signal at a certain angular or spatial point is
		monitored as the experiment progresses.
\end{description}

In this chapter we will restrict ourselves to analysis in the angular and
intensity interrogation methods.  We do not possess the means to carry out
wavelength interrogation, though this is funamentally equivalent to
sweeping the SPR angle. 

\section{Angular Interrogation}

To study the sensitivity in the angular interrogation method, first we
simulate angular response with ($\delta n = 0$) and without ($\delta n \ne
0$) a refractive index pertrubation.  The choice here of the pertrubation
$\delta n = 0.01$ in Figures~\ref{sensangularasp} and \ref{sensangularasp}
was done for illustrative purposes.  The minimum of the angular response
was determined by the zero of the function found using nonlinear
minimization~\cite{fminbnd} and optimizing that function as $\delta n \to
0$.  In this way the maximum theoretical sensitivity for angular
sensitivity, $\delta \theta/\delta n$ is determined.  This theoretical
sensitivity should not be confused with acheivable sensitivity as we do not
take into account real world instruments and noise.  Also note that we are
not concerned at the moment with the \textit{optimization} of sensitivity.
Rather we focus on the difference between angular response to refractive
index pertrubations in the cone and the notch.  

Simulated responses for symmetric (SSP) and asymmetric (ASP) configurations
in the notch and the cone are shown in \Figure{fig:sensangularasp} and
\ref{fig:sensangularssp} for $\delta n = 0.01$.  The maximum obtainable
results for these conigurations are tabulated in \Table{tbl:angularsens}.
\begin{figure}[ht]
 \centering
 \import{includes/}{setpgfinc}
	\import{bulkri/figures/}{sens_angular_asp}
 \caption{Angular sensitivity of the notch and the cone for the Kretschmann
									three layer (ASP) system.  $\lambda_0=\SI{660}{\nano\meter}$, $n_1 =
									\num{1.5142}$, $n_2=\num{0.2843 + 3.3825i}$, and
									$n_3=1.3310 + \delta n$.
									The thickness of the metal layer is \SI{45}{\nano\meter}.}
 \label{fig:sensangularasp}
\end{figure}
\begin{figure}[ht]
 \centering
 \import{includes/}{setpgfinc}
	\import{bulkri/figures/}{sens_angular_ssp}
 \caption{Angular sensitivity of the notch and the cone for the Kretschmann
									four layer (SSP) system.  $\lambda_0=\SI{660}{\nano\meter}$, $n_1 =
									\num{1.5142}$, $n_2=1.3489$, $n_3=\num{0.2843 +
									3.3825i}$, and $n_4=1.3310+\delta n$.
									The thickness of the metal layer is \SI{16.97}{\nano\meter}.}
 \label{fig:sensangularssp}
\end{figure}
\begin{table}
\centering
\begin{tabular}{lSS}
\toprule
{configuration} & {notch [$\mathrm{deg}/\mathrm{RIU}]$} & {cone [$\mathrm{deg}/\mathrm{RIU}$]} \\
\midrule
ASP & 1.452312e+02 & 1.453671e+02 \\
SSP & 3.943097e+01 & 3.938117e+01 \\
\bottomrule
\end{tabular}
\caption{Theoretical maximum angular sensitivity, $\delta \theta/\delta n$,
in degrees per refractive index unit, for the configurations in Figures
\ref{fig:sensangularasp} and \ref{fig:sensangularssp}.}
\label{tbl:angularsens}
\end{table}

\Table{tbl:angularsens} tells us all we need to know about using this
interrogation method: there is no readily apparent sensitivity benefit to
looking at the cone versues looking at the notch.  Though the angle of the
SPR minima (notch) and maxima (cone) are at slightly different angles --
approximately \SI{0.006}{\degree} for the SSP setup -- their responses
track each other very well.

\section{Intensity Interrogation}
In contrast to angular interrogation, intensity interrogation shows
significant differences between

aspnotch = 8.0708
aspcone = 4.7795
sspnotch = 0.2349
sspcone = 0.2325

\begin{table}
\centering
\begin{tabular}{lSS}
\toprule
{configuration} & {notch} & {cone} \\
\midrule
ASP & 32.1657 & 42.3053 \\
SSP & 203.5365 & 251.9765 \\
\bottomrule
\end{tabular}
\caption{Theoretical maximum intensity sensitivity.}
\label{tbl:intensitysens}
\end{table}



