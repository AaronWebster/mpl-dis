\textbf{ASP SHOULD BE CONVENTIONAL SP}
The sensitivity of surface plasmon resonance to bulk refractive index
pertrubations is perhaps its most popular and widely utilized property.
This is motivated by the system's quite remarkable response, which has been
reported at up to \SI{1e-8}{RIU} (refractive index units) in some
applications~\cite{sprreview}, shot noise limited.  Bulk refractive index
sensing with SPR is well understood, so in this chapter we focus on these
types of measurements in the cone without consideration of speckle, a topic
reserved for \Chapter{ch:speckle}.  

In a prism coupled setup, there are three main methods used to
monitor changes in the SPR resonance condition.  All three of these
interrogation methods are equivalent in terms of their real-world
resolution and detection limit.~\cite{pines1952collective}
\begin{description}
	\item [{Angular}] The change in the angle of the SPR minimum as a function
					of refractive index is monitored.  In the angular modulation method,
					The prism may be scanned through a range of angles or the angular
					spectrum may be obtained simultaneously using a focussed beam and
					sensor array.
 \item [{Wavelength}] The prism is fixed in place and its reflectivity at
  a single angle is monitored as a function of wavelength.
	\item [{Intensity}] For a fixed wavelength and incident angle, the
		intensity of the signal at a certain angular or spatial point is
		monitored as the experiment progresses.
\end{description}

In this chapter we will restrict ourselves to analysis in the angular and
intensity interrogation methods.  We do not possess the means to carry out
wavelength interrogation, though this is funamentally equivalent to
sweeping the SPR angle. 

\section{Angular Interrogation}

To study the sensitivity in the angular interrogation method, first we
simulate angular response with ($\Delta n = 0$) and without ($\Delta n \ne
0$) a refractive index pertrubation of the sensing layer.  The choice here
of the pertrubation $\Delta n = 0.01$ in Figures~\ref{fig:sensangularasp}
and \ref{fig:sensangularasp} is somewhat arbitrary and only serves to
illustrate the change of the resonance line.  The minimum of the angular
response was determined by the zero of the function found using nonlinear
minimization~\cite{brent1973algorithms} and optimizing that function as
$\Delta n \to 0$.  In this way the maximum theoretical sensitivity for
angular sensitivity, $\Delta \theta/\Delta n$ is determined.  This
theoretical sensitivity should not be confused with real world sensitivity:
we do not take into sources of noise in the experiment and detector.  We
are also not concerned at the moment with the \textit{optimization} of
sensitivity.  Rather we focus on the difference between angular response to
refractive index pertrubations in the cone and the notch.  

Simulated responses for symmetric (SSP) and conventional (SP)
configurations in the notch and the cone are shown in
\Figure{fig:sensangularasp} for $\Delta n = 0.01$.  The maximum obtainable
results for these conigurations are tabulated in \Table{tbl:angularsens}.
\begin{figure}[ht]
 \centering
 \import{includes/}{setpgfinc}
	\import{bulkri/figures/}{sens_angular_asp}
	\import{bulkri/figures/}{sens_angular_ssp}
 \caption{(top) Angular sensitivity of the notch and the cone for the Kretschmann
									three layer (SP) system.  $\lambda_0=\SI{660}{\nano\meter}$, $n_1 =
									\num{1.5142}$, $n_2=\num{0.2843 + 3.3825i}$, and
									$n_3=1.3310 + \Delta n$.  The thickness of the metal layer is
									\SI{45}{\nano\meter}. (bottom) 
	Angular sensitivity of the notch and the cone for the Kretschmann
									four layer (SSP) system.  $\lambda_0=\SI{660}{\nano\meter}$, $n_1 =
									\num{1.5142}$, $n_2=1.3489$, $n_3=\num{0.2843 +
									3.3825i}$, and $n_4=1.3310+\Delta n$.
									The thickness of the metal layer is \SI{16.97}{\nano\meter}.  }
 \label{fig:sensangularasp}
\end{figure}

\Table{tbl:angularsens} tells us all we need to know about using this
interrogation method: there is no readily apparent sensitivity benefit to
the cone versues the notch in angular interrogation.  Though the angle of the
SPR minima (notch) and maxima (cone) are at slightly different angles, 
approximately \SI{0.006}{\degree} for SPs, their responses
track each other very well.  
\begin{table}[ht]
\centering
\sisetup{round-mode=places,round-precision=3,fixed-exponent=0,scientific-notation=fixed}
\begin{tabular}{lSS}
\toprule
{configuration} & {notch [$\mathrm{deg}/\mathrm{RIU}]$} & {cone [$\mathrm{deg}/\mathrm{RIU}$]} \\
\midrule
SP & 1.452312e+02 & 1.453671e+02 \\
SSP & 3.943097e+01 & 3.938117e+01 \\
\bottomrule
\end{tabular}
\caption{Theoretical maximum angular sensitivity, $\Delta \theta/\Delta n$,
in degrees per refractive index unit, for the configurations in 
\Figure{fig:sensangularasp}.}
\label{tbl:angularsens}
\end{table}

\section{Intensity Interrogation}
In contrast to angular interrogation, intensity interrogation shows
significant differences between the notch and the cone.  This is due to
both the sharper resonance and narrower linewidth.  For the systems under
discussion, we calculate the angular width of the notch to be
$\theta_{1/2}=\SI{8.0708}{\degree}$ for SPs, with the cone being nearly
twice as narrow at $\theta_{1/2}=\SI{4.7795}{\degree}$.  SSPs show the same
trend, but the discrepancy is not nearly as large with
$\theta_{1/2}=\SI{0.2349}{\degree}$ for the notch and
$\theta_{1/2}=\SI{0.2325}{\degree}$ for the cone.  The theoretical maximum
sensitivities for both cases are shown in \Table{tbl:intensitysens}.  This
was calculated using the same nonlinear minimization method as in angular
case, but instead we look for the angular location where the difference in Fresnel
coefficents is greatest and take $\Delta n \to 0$.  Specifically, for an
$N$ layer system with refractive indicies $(n_1,n_2, \ldots,n_N)$ we
calculate for the notch
\begin{equation}
								\lim_{\Delta n \to 0}\frac{\left||r^p(n_N=n_N)|^2 - |r^p(n_N=n_N + \Delta n)|^2\right|_\mathrm{max}}{\Delta n}
\label{eqn:fresnelsenspertrubation}
\end{equation}
For the cone we use the same approach but instead of the Fresnel
reflectivity we use \Equation{eqn:conefield}, $|t^p_+|^2|t^p_-|^2$.  In
both cases the signal normalized to $[0,1]$ within the same angular range
for comparison.

The results of this calculation are summarized in
\Table{tbl:intensitysens}.  We observe that in both cases, $\Delta I/\Delta
n$ is higher for the cone as compared with the notch, though the increase
is only marginal.  In terms of SPR based biosensing this data suggests an
avenue for sensitivity enhancement.  Whether this avenue is warranted in
commercial systems remains to be seen.
\begin{table}[ht]
\centering
\sisetup{round-mode=places,round-precision=3,fixed-exponent=0,scientific-notation=fixed}
\begin{tabular}{lSS}
\toprule
{configuration} & {notch [1/RIU]} & {cone [1/RIU]} \\
\midrule
SP & 32.1657 & 42.3053 \\
SSP & 203.5365 & 251.9765 \\
\bottomrule
\end{tabular}
\caption{Theoretical maximum intensity sensitivity, $\Delta I/\Delta n$,
								for the configurations in \Figure{fig:angularsens}. }
\label{tbl:intensitysens}
\end{table}

\section{Experimental Observations}
Here's the deal: your experimental data for bulk RI sucks.  You need to
take two more datasets, or perhaps just one using LRSPPs.  Make sure to
record the distance between the sensor and the scattering spot to make
angular calculations easier.
