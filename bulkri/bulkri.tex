The sensitivity of surface plasmon resonance to bulk refractive index
pertrubations is perhaps its most popular and widely utilized property.
This is motivated by their quite remarkable response --- up to
\SI{1e-8}{RIU} (refractive index units) in some
applications~\cite{sprreview}.  Bulk refractive index sensing with SPR is
well understood, so in this chapter we focus on these types of measurements
in the cone without consideration of speckle ,a topic reserved for
\Chapter{ch:speckle}.  

In a prism coupled setup, there are three main methods used to
monitor changes in the SPR resonance condition.  All three of these
interrogation methods are equivalent in terms of their real-world
resolution and detection limit.~\cite{pines1952collective}
\begin{description}
	\item [{Angular}] The change in the angle of the SPR minimum as a function
					of refractive index is monitored.  In the angular modulation method,
					The prism may be scanned through a range of angles or the angular
					spectrum may be obtained simultaneously using a focussed beam and
					sensor array.
 \item [{Wavelength}] The prism is fixed in place and its reflectivity at
  a single angle is monitored as a function of wavelength.
	\item [{Intensity}] For a fixed wavelength and incident angle, the
		intensity of the signal at a certain angular or spatial point is
		monitored as the experiment progresses.
\end{description}

In this chapter we will restrict ourselves to analysis in the angular and
intensity interrogation methods.  We do not possess the means to carry out
wavelength interrogation, though this is funamentally equivalent to
sweeping the SPR angle. 

\section{Angular Interrogation}
The maximum theoretical sensitivity for angular sensitivity, $\delta
\theta/\delta n$ can be found by solving the relevant dispersion relations.
For the three layer system this is \Equation{eqn:dispersion1}.  For other
systems this needs to be done numerically.  Here we do not concentrate on
optimizing the sensitivity in a prism coupled setup, rather we focus on the
difference between angular response to refractive index pertrubations in
the cone and the notch.  Simulated responses for symmetric (SSP) and
asymmetric configurations in the notch and the cone are found in
\Figure{fig:sensangularasp} and \ref{fig:sensangularssp}.  The results are
tabulated in \Table{tbl:angularsens}.
\begin{figure}[ht]
 \centering
 \import{includes/}{setpgfinc}
	\import{bulkri/figures/}{sens_angular_asp}
 \caption{Angular sensitivity of the notch and the cone for the Kretschmann
									three layer (ASP) system.  $\lambda_0=\SI{660}{\nano\meter}$, $n_1 =
									\num{1.5142}$, $n_2=\num{0.2843 + 3.3825i}$, and $n_3=1.3310$.
									The thickness of the metal layer is \SI{45}{\nano\meter}.}
 \label{fig:sensangularasp}
\end{figure}
\begin{figure}[ht]
 \centering
 \import{includes/}{setpgfinc}
	\import{bulkri/figures/}{sens_angular_ssp}
 \caption{Angular sensitivity of the notch and the cone for the Kretschmann
									four layer (SSP) system.  $\lambda_0=\SI{660}{\nano\meter}$, $n_1 =
									\num{1.5142}$, $n_2=1.3489$, $n_3=\num{0.2843 + 3.3825i}$, and $n_4=1.3310$.
									The thickness of the metal layer is \SI{16.97}{\nano\meter}.}
 \label{fig:sensangularssp}
\end{figure}
\begin{table}
\centering
\begin{tabular}{lSS}
\toprule
{configuration} & {notch} & {cone} \\
\midrule
ASP & 1.452312e+02 & 1.453671e+02 \\
SSP & 3.943097e+01 & 3.938117e+01 \\
\bottomrule
\end{tabular}
\caption{Theoretical maximum angular sensitivity calculations for the
								configurations in Figures \ref{fig:sensangularasp} and
								\ref{fig:sensangularssp}.}
\label{tbl:angularsens}
\end{table}

\section{Intensity Interrogation}
\begin{table}
\centering
\begin{tabular}{lSS}
\toprule
{configuration} & {notch} & {cone} \\
\midrule
ASP & 2.721749e+01 & 3.529935e+01 \\
SSP & 9.295699e+01 & 9.666660e+01 \\
\bottomrule
\end{tabular}
\caption{Theoretical maximum intensity sensitivity.}
\label{tbl:intensitysens}
\end{table}
%>> sens_angular_ASP
%cone angular:    1.453671e+02
%notch angular:   1.452312e+02
%cone intensity:  3.529935e+01
%notch intensity: 2.721749e+01

%>> sens_angular_SSP
%cone angular:    3.938117e+01
%notch angular:   3.943097e+01
%cone intensity:  9.666660e+01
%notch intensity: 9.295699e+01

\section{Maximum Theoretical Sensitivity}
Before beginning, it is approprate to 

We will now derive the sensitivity of the three layer Kretschmann
configuration to bulk index changes. This case will be used as a test
case for further analysis of the propagation dependent sensitivity.

The bulk sensitivity in an intensity modulation setup is defined as
the ratio of the change of normalized signal intensity $\delta I=I_{n_{0}}-I_{n_{1}}$
to the change in bulk refractive index $\delta n=n_{0}-n_{1}$ for
two refractive indicies $n_{0}$ and $n_{1}$. We restrict ourselves
to small changes in the refractive index $n$, $\delta n/n_{0}\ll1$.
The intensity sensitivity is defined as
\begin{equation}
S_{\mathrm{I}}=\frac{\delta I}{\delta n}
\end{equation}
Without loss of generality, the sensor intensity $I$ may be a function
of any appropriate coordinate system (e.g. $I(x,y)$). In such a coordinate
system we always take $S_{\mathrm{I}}$ at the location of maximum
shift in that coordinate system.

In the Lorentzian approximation{[}{]} using Drude model coefficents
$\epsilon=\epsilon'+\mi\epsilon''$, the maximum sensitivity for bulk
index sensing is approximated as
\begin{equation}
(S_{\text{I}})_{\text{max}}=\frac{4\sqrt{3}}{9}\frac{\epsilon_{\mathrm{m}}'}{\epsilon_{\mathrm{m}}''n^{3}}
\end{equation}
Where $\epsilon_{\mathrm{m}}$ is the complex Drude-model permittivity
in the metal layer. At a operating wavelength of \SI{660}{\nano\meter},
for Au on N-BK7, $(S_{\mathrm{I}})_{\text{max}}=151.66$. This is
only an approximation, though it is close to the value obtained through
Lorentz-Drude coefficents used in combination with the transfer matrix
method, $(S_{\mathrm{I}})_{\text{max}}=153.04$.
