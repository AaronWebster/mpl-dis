The sensitivity of surface plasmon based biosensors to bulk refractive
index changes is of primary importance in extracting bio-relevant
properties. In a prism coupled setup, there are three main modulation
methods used to monitor changes in the SPR resonance.

\begin{description}
 \item [{Angular}] The prism is rotated and the signal reflectivity
  monitored
  as a function of angle. The prism may be scanned through a range of
  angles or the angular spectrum may be obtained simultaneously using
  a focussed beam and sensor array.
 \item [{Wavelength}] The prism is fixed in place and its reflectivity at
  a single angle is monitored as a function of wavelength.
 \item [{Intensity}] The intensity of the signal at a certain angular or
  spatial point is monitored as the experiment progresses.
\end{description}
All three of these interrogation methods are equivalent in terms of
their real-world resolution and detection limit. In this section we
will look at the sensitivity of intensity interrogation to different
(something about parameters)


\section{Three Layer Kretschmann, Far Field}
We will now derive the sensitivity of the three layer Kretschmann
configuration to bulk index changes. This case will be used as a test
case for further analysis of the propagation dependent sensitivity.

The bulk sensitivity in an intensity modulation setup is defined as
the ratio of the change of normalized signal intensity $\delta I=I_{n_{0}}-I_{n_{1}}$
to the change in bulk refractive index $\delta n=n_{0}-n_{1}$ for
two refractive indicies $n_{0}$ and $n_{1}$. We restrict ourselves
to small changes in the refractive index $n$, $\delta n/n_{0}\ll1$.
The intensity sensitivity is defined as
\begin{equation}
S_{\mathrm{I}}=\frac{\delta I}{\delta n}
\end{equation}
Without loss of generality, the sensor intensity $I$ may be a function
of any appropriate coordinate system (e.g. $I(x,y)$). In such a coordinate
system we always take $S_{\mathrm{I}}$ at the location of maximum
shift in that coordinate system.

In the Lorentzian approximation{[}{]} using Drude model coefficents
$\epsilon=\epsilon'+\mi\epsilon''$, the maximum sensitivity for bulk
index sensing is approximated as
\begin{equation}
(S_{\text{I}})_{\text{max}}=\frac{4\sqrt{3}}{9}\frac{\epsilon_{\mathrm{m}}'}{\epsilon_{\mathrm{m}}''n^{3}}
\end{equation}
Where $\epsilon_{\mathrm{m}}$ is the complex Drude-model permittivity
in the metal layer. At a operating wavelength of \SI{660}{\nano\meter},
for Au on N-BK7, $(S_{\mathrm{I}})_{\text{max}}=151.66$. This is
only an approximation, though it is close to the value obtained through
Lorentz-Drude coefficents used in combination with the transfer matrix
method, $(S_{\mathrm{I}})_{\text{max}}=153.04$.