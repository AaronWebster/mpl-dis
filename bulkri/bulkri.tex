The sensitivity of surface plasmon resonance to bulk refractive index
pertrubations is perhaps its most popular and widely utilized property.
(Why is SPR so sensitive, a little bit about field enhancements and so
forth).  In a biosensing context, molecules binding to the metal layer will
change the local refractive index a surface plasmon ``sees'', and its
resonance condition will change.

In a prism coupled setup, there are three main methods used to
monitor changes in the SPR resonance condition.  All three of these
interrogation methods are equivalent in terms of their real-world
resolution and detection limit.~\cite{pines1952collective}
\begin{description}
	\item [{Angular}] The change in the angle of the SPR minimum as a function
					of refractive index is monitored.  In the angular modulation method,
					The prism may be scanned through a range of angles or the angular
					spectrum may be obtained simultaneously using a focussed beam and
					sensor array.
 \item [{Wavelength}] The prism is fixed in place and its reflectivity at
  a single angle is monitored as a function of wavelength.
	\item [{Intensity}] For a fixed wavelength and incident angle, the
		intensity of the signal at a certain angular or spatial point is
		monitored as the experiment progresses.
\end{description}

In this chapter we will study the bulk refractive index of the cone as
compared to the specular notch using both angular and intensity
interrogation methods.  We do not possess the means to carry out wavelength
interrogation, though this is funamentally equivalent to sweeping the SPR
angle. 

\section{ASPs}

\begin{figure}[ht]
 \centering
 \import{includes/}{setpgfinc}
	\import{bulkri/figures/}{sens_angular_asp}
 \caption{Angular sensitivity of the notch and the cone for the Kretschmann
									three layer system.  $\lambda_0=\SI{660}{\nano\meter}$, $n_1 =
									\num{1.5142}$, $n_2=\num{0.2843 + 3.3825i}$, and $n_3=1.3310$.
									The thickness of the metal layer is \SI{45}{\nano\meter}.}
 \label{fig:sensangularasp}
\end{figure}

\begin{figure}[ht]
 \centering
 \import{includes/}{setpgfinc}
	\import{bulkri/figures/}{sens_angular_ssp}
	\caption{dingus}
 \label{fig:sensangularssp}
\end{figure}

\begin{table}
\centering
\begin{tabular}{lll}
\toprule
location & {angular ($d\theta/dn$, deg)} & {intensity ($dI/dn$)} \\
\midrule
cone & \num{1.365122e+02} & \num{2.421350e+01}\\
notch & \num{1.452312e+02} & \num{2.721749e+01}\\
\bottomrule
\end{tabular}
\caption{Sensitivity calculations.}
\label{tab:3layersens}
\end{table}

%>> sens_angular_ASP
%cone angular:    1.453671e+02
%notch angular:   1.452312e+02
%cone intensity:  3.529935e+01
%notch intensity: 2.721749e+01

%>> sens_angular_SSP
%cone angular:    3.938117e+01
%notch angular:   3.943097e+01
%cone intensity:  9.666660e+01
%notch intensity: 9.295699e+01

\section{Maximum Theoretical Sensitivity}
Before beginning, it is approprate to 

We will now derive the sensitivity of the three layer Kretschmann
configuration to bulk index changes. This case will be used as a test
case for further analysis of the propagation dependent sensitivity.

The bulk sensitivity in an intensity modulation setup is defined as
the ratio of the change of normalized signal intensity $\delta I=I_{n_{0}}-I_{n_{1}}$
to the change in bulk refractive index $\delta n=n_{0}-n_{1}$ for
two refractive indicies $n_{0}$ and $n_{1}$. We restrict ourselves
to small changes in the refractive index $n$, $\delta n/n_{0}\ll1$.
The intensity sensitivity is defined as
\begin{equation}
S_{\mathrm{I}}=\frac{\delta I}{\delta n}
\end{equation}
Without loss of generality, the sensor intensity $I$ may be a function
of any appropriate coordinate system (e.g. $I(x,y)$). In such a coordinate
system we always take $S_{\mathrm{I}}$ at the location of maximum
shift in that coordinate system.

In the Lorentzian approximation{[}{]} using Drude model coefficents
$\epsilon=\epsilon'+\mi\epsilon''$, the maximum sensitivity for bulk
index sensing is approximated as
\begin{equation}
(S_{\text{I}})_{\text{max}}=\frac{4\sqrt{3}}{9}\frac{\epsilon_{\mathrm{m}}'}{\epsilon_{\mathrm{m}}''n^{3}}
\end{equation}
Where $\epsilon_{\mathrm{m}}$ is the complex Drude-model permittivity
in the metal layer. At a operating wavelength of \SI{660}{\nano\meter},
for Au on N-BK7, $(S_{\mathrm{I}})_{\text{max}}=151.66$. This is
only an approximation, though it is close to the value obtained through
Lorentz-Drude coefficents used in combination with the transfer matrix
method, $(S_{\mathrm{I}})_{\text{max}}=153.04$.
