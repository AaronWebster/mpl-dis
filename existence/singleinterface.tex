%\todo{the ansatz here is that we're looking for bound solutions k1 must be
%positive and k2 negative because we are looking for bounds solutions}
\begin{figure}[ht]
  \centering
  \import{existence/figures/}{singlelayerinterfacegeo}
  \caption{Geometry and coordinate system for a single interface which can
    support \glspl{spp}.  Domain $D_1$ has permittivity $\epsilon_1$, typically a
    dielectric, and domain $D_2$ a permittivity $\epsilon_2$, typically a
    conductive metal.}
  \label{fig:singleinterfacegeo}
\end{figure}

From the plane wave ansatz to Maxwell's equations, the
derivation for the existence of surface plasmon polaritons proceeds
by imposing boundary conditions consistent with a single
interface.  Since \gls{spp} modes are not supported with TE polarization, it is
convenient to restrict the problem to TM polarization in two dimensions.
Consider the geometry in \Figure{fig:singleinterfacegeo}.  If
a TM wave is incident at an angle on the $D_1$-$D_2$ interface,
$\mathbf{k}=(k_x,0,k_z)$,
$\mathbf{r}\cdot\mathbf{k}=k_x x + k_z z$ and $\mathbf{E}_0 = (E_x, 0,
  E_z)$. The electric field is then
\begin{align}
  \mathbf{E} ( \mathbf{r}, t ) & = \mathbf{E}_0\, \me^{\mi (\mathbf{k}
    \cdot \mathbf{r} - \omega t )}                                     \\
  \mathbf{E}(x,z,t)            & =\begin{pmatrix}
    E_x \\ 0\\ E_z
  \end{pmatrix}
  \, \me^{\mi(k_{x,i}x+k_{z,i}z-\omega t)}.
  \label{eqn:planewavexz}
\end{align}
Similarly, for the magnetic field,
\begin{align}
  \mathbf{H} ( \mathbf{r}, t ) & = \mathbf{H}_0\, \me^{\mi (\mathbf{k}
    \cdot \mathbf{r} - \omega t )}                                     \\
  \mathbf{H}(x,z,t)            & =\begin{pmatrix}
    0 \\ H_y\\ 0
  \end{pmatrix}
  \, \me^{\mi(k_{x,i}x+k_{z,i}z-\omega t)}.
\end{align}
At the interface between $D_1$ and $D_2$ there are two values for the permittivity,
$\epsilon_1$ and $\epsilon_2$.  Consequently there are two sets of plane
wave solutions,
\begin{align}
  \left.\begin{aligned}
    \mathbf{H}_1(x,z,t) & =
    \begin{pmatrix}
      0       \\
      H_{y,1} \\
      0
    \end{pmatrix} \me^{\mi(k_{x,1}x+k_{z,1}z-\omega t)} \\
    \mathbf{E}_1(x,z,t) & =
    \begin{pmatrix}
      E_{x,1} \\
      0       \\
      E_{z,1} \\
    \end{pmatrix} \me^{\mi(k_{x,1}x+k_{z,1}z-\omega t)}
  \end{aligned}
  \right\} & \quad D_1\label{eqn:planewavedielectric} \\
  \left.\begin{aligned}
    \mathbf{H}_2(x,z,t) & =
    \begin{pmatrix}
      0       \\
      H_{y,2} \\
      0
    \end{pmatrix}
    \me^{\mi(k_{x,2}x+k_{z,2}z-\omega t)} \\
    \mathbf{E}_2(x,z,t) & =
    \begin{pmatrix}
      E_{x,2} \\
      0       \\
      E_{z,2} \\
    \end{pmatrix}
    \me^{\mi(k_{x,2}x+k_{z,2}z-\omega t)}
  \end{aligned}
  \right\} & \quad D_2,
  \label{eqn:planewavemetal}
\end{align}
where the subscript designates which material the wave equation refers to,
e.g.\ $\mathbf{E}_1$ is the electric field in $D_1$, $\mathbf{H}_2$
the magnetic field in $D_2$, etc. Continuity at the boundary
requires
\begin{align}
  E_{x,2}            & =E_{x,1}             \\
  H_{y,2}            & =H_{y,1}             \\
  \epsilon_2 E_{z,2} & =\epsilon_1 E_{z,1}.
\end{align}
Applying Ampere's law (\Equation{eqn:ampereslaw}) to the field on
each boundary and relating the vector components gives
%each boundary gives
%\begin{align}
%\nabla \times \mathbf{H}_i &= \epsilon_i \frac{\partial \mathbf{E}_i}{\partial t}\\
%\begin{pmatrix}
%\frac{\partial H_{z,i}}{\partial y} - \frac{\partial H_{y,i}}{\partial z}\\
%\frac{\partial H_{x,i}}{\partial y} - \frac{\partial H_{z,i}}{\partial z}\\
%\frac{\partial H_{y,i}}{\partial y} - \frac{\partial H_{x,i}}{\partial z}
%\end{pmatrix}
%&= \begin{pmatrix}
%-\mi k_{z,i} H_{y,i}\\
%0\\
%\mi k_{x,i} H_{y,i}
%\end{pmatrix}
%\\
%&= \begin{pmatrix}
%-\mi \omega \epsilon_i E_{x,i}\\
%0\\
%-\mi \omega \epsilon_i E_{z,i}
%\end{pmatrix}.
%\label{eqn:vectordisp}
%\end{align}
%The vector components of \Equation{eqn:vectordisp} are therefore related,
\begin{align}
  -\mi k_{z,i} H_{y,i} & = -\mi \omega \epsilon_i E_{x,i} \\
  k_{z,1} H_{y,1}      & = \omega \epsilon_1 E_{x,1}      \\
  k_{z,2} H_{y,2}      & = \omega \epsilon_2 E_{x,2}.
  \label{eqn:spderivsteptwo}
\end{align}
%\todo{not the components, either e or h, since Ez is perpendicular to the interface}
Since $E_{x,i}$ and $H_{y,i}$ are continuous,
by substitution of $E_{x,i}$, \Equation{eqn:spderivsteptwo} becomes
\begin{align}
  \frac{k_{z,1}}{\epsilon_1}H_{y,1} & =\frac{k_{z,2}}{\epsilon_2}H_{y,2} \\
  %\Aboxed{
  \frac{k_{z,1}}{\epsilon_1}        & =\frac{k_{z,2}}{\epsilon_2}.
  %}
  \label{eqn:sprcondition}
\end{align}

\Equation{eqn:sprcondition} is the surface plasmon resonance condition for
a single interface.  Note an important aspect of
\Equation{eqn:sprcondition}:  due to the way the geometry was defined
(\Figure{fig:singleinterfacegeo}), $k_{z,1}$ must be positive and $k_{z,2}$
negative to avoid unphysical diverging solutions.  For
\Equation{eqn:sprcondition} to hold, this implies the real part of
$\epsilon_1$ and $\epsilon_2$ are opposite in sign.  For natural materials,
\Equation{eqn:sprcondition} is fulfilled if $\epsilon_1$ is a dielectric
and $\epsilon_2$ is a metal.
