Conditions for the existence of SPPs can be predicted from the plane wave
solutions to Maxwell's equations.  We begin with a derivation of such
waves, starting with the differential form of Maxwell's equations.
\begin{align}
\nabla \cdot \mathbf{D} &= 0 & \text{Gauss's law} \label{eqn:gausslaw}\\
\nabla \cdot \mathbf{B} &= 0 & \text{Gauss's law for magnetism} \label{eqn:gausslawmagnetism}\\
\nabla \times \mathbf{E} &= -\frac{\partial \mathbf{B}} {\partial t}
& \text{Faraday's law of induction} \label{eqn:faradayslaw} \\
\nabla \times \mathbf{H} &= \frac{\partial \mathbf{D}} {\partial
t}  & \text{Ampere's law with Maxwell's correction}
\label{eqn:ampereslaw}
\end{align}
Where $\mathbf{E}$ is the electric field in,
$\mathbf{B}$ is the magnetic field, $\mathbf{D}$ is the electric
displacement field, and $\mathbf{H}$ is
the so-called auxiliary magnetic field.

In formulating Maxwell's equations it is assumed all electromagnetic
propagation takes place in a {\it simple dielectric}, that is
\begin{enumerate}
\item The polarization density $\mathbf{P}$ is linear with the electric
field $\mathbf{E}$.  
\begin{align}
\mathbf{P}=\epsilon_0\chi_e\mathbf{E}
\end{align}
where $\chi_e$ is the electric susceptibility.  From this and the definition
of $\mathbf{D}$
\begin{align}
\mathbf{D}=\epsilon_0\mathbf{E}+\mathbf{P}
\label{eqn:dfield}
\end{align}
Where $\epsilon_0$ is the permittivity of free space, one can simplify the
$\mathbf{D}$-field as 
\begin{align}
\mathbf{D}&=\epsilon_0\mathbf{E}+\mathbf{P}\\
&=\epsilon_0 \mathbf{E}+\epsilon_0 \chi_e \mathbf{E}\\
&=\epsilon_0(1+\chi_e)\mathbf{E}\\
&=\epsilon\mathbf{E}
\end{align}
Here $\epsilon$ has been defined to be the permittivity of the
dielectric.  The convention in this work is usually that variables with a
subscript $0$ can always be assumed to be that variable {\it in vacuo},
while their material dependent counterparts have no subscript.
\item The magnetization $\mathbf{M}$, related to the auxiliary magnetic
field by is linear in $\mathbf{H}$
\begin{align}
\mathbf{M}=\mu_0\chi_m\mathbf{H}
\end{align}
where $\chi_m$ is the volume magnetic susceptibility.  This allows the
magnetic field to be rewritten according to its definition
\begin{align}
\mathbf{B}&=\mu_0\left(\mathbf{H}+\mathbf{M}\right)\\
&=\mu_0\left(1+\chi_m\right)\mathbf{H}\\
&=\mu \mathbf{H}
\end{align}
where $\mu_0$ is the vacuum permeability of free space.  In practice, most
dielectrics have nearly zero magnetic response, so $\mathbf{M}=0$ and
$\mu_0=\mu$.  Nevertheless for the sake of generality it is treated as a
material dependent parameter.
\item The material is homogeneous.
\item The material is isotropic.
\end{enumerate}
Note that with the enumerated requirements and the form of Ampere's law we
have assumed a region with no charges and no currents.  Given that a SPP is
a charge density oscillation, this is not strictly rigorous.  It does,
however, simplify the derivation.  The presence of free charges is
accounted for {a posteriori} by using a complex dielectric $\epsilon =
\epsilon' + \mi \epsilon''$.  With regards to surface roughness, these
assumptions are also acceptable as we assume such roughness to be likewise
homogeneous and isotropic.

Taking these assumptions into account, plane wave solutions for the
electric field can be derived. Beginning with Faraday's law
\begin{align}
\nabla \times \mathbf{E} &= -\frac{\partial \mathbf{B}} {\partial t}
\end{align}
and taking the curl of both sides
\begin{align}
\nabla \times
\left(\nabla\times\mathbf{E}\right)&=\nabla\times\left(-\frac{\partial\mathbf{B}}{\partial
t}\right)\\
&=-\frac{\partial}{\partial t}\left(\nabla \times \mathbf{B}\right)
\end{align}
The left side can be expanded and simplified using the vector identity
\begin{align}
\nabla \times \left( \nabla \times \mathbf{A} \right) = \nabla \left(
\nabla \cdot \mathbf{A} \right) - \nabla^2 \mathbf{A}
\end{align}
in combination with Gauss's law, $\nabla \cdot \mathbf{E}=0$.
The right hand side is modified using Ampere's law with $\mathbf{B} =\mu
\mathbf{H}$.
\begin{align}
\nabla\left(\nabla\cdot\mathbf{E}\right)-\nabla^2\mathbf{E}
&=-\mu\frac{\partial}{\partial t}\left(\nabla \times \mathbf{H}\right)\\
-\nabla^2\mathbf{E}&=-\mu\frac{\partial^2 \mathbf{D}}{\partial t^2}\\
-\nabla^2\mathbf{E}&=-\mu\epsilon\frac{\partial^2 \mathbf{E}}{\partial t^2}\\
\left(\nabla^2-\mu\epsilon\frac{\partial^2}{\partial t^2}\right)\mathbf{E}&=0
\label{eqn:ewe}
\end{align}
This is the electromagnetic wave equation in terms of the electric field.
Choosing to solve this equation by separation of variables the following
plane wave solutions can be obtained
\begin{align}
\Aboxed{
 \mathbf{E} ( \mathbf{r}, t ) &= \mathbf{E}_0\, \me^{\mi (\mathbf{k}
 \cdot \mathbf{r} - \omega t )}
}
\label{eqn:planewaves}
\end{align}
where $k=\omega/c=\omega\sqrt{\epsilon\mu}$.  In this notation,
$\mathbf{k}$ is the material specific vectorial wavenumber, $\mathbf{r}$ is the
spatial position, $\omega$ is angular frequency, and $t$ is
the dimension of time.
The initial value is chosen with the vectorial constant $\mathbf{E}_0$.
The magnetic field follows the same form
\begin{align}
 \mathbf{H} ( \mathbf{r}, t ) &= \mathbf{H}_0\, \me^{\mi (\mathbf{k}
 \cdot \mathbf{r} - \omega t )}
\end{align}
but it is orthogonal $\mathbf{E}$ by 
\begin{align}
\mathbf{E} \times \mathbf{H} = 0
\end{align}
and the two are mutually orthogonal with the direction of propagation
$\mathbf{k}$.
