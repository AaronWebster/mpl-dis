The existence of SPPs and the conditions under which they may be excited
can be fully predicted from Maxwell's equations.  The differential form of
Maxwell's equations~\cite{maier2007plasmonics}~\cite{benson2009elements}
including external sources are
\begin{align}
\nabla \cdot \mathbf{D} &= \rho & \text{Gauss's law} \label{eqn:gausslaw}\\
\nabla \cdot \mathbf{B} &= 0 & \text{Gauss's law for magnetism} \label{eqn:gausslawmagnetism}\\
\nabla \times \mathbf{E} &= -\frac{\partial \mathbf{B}} {\partial t}
& \text{Faraday's law of induction} \label{eqn:faradayslaw} \\
\nabla \times \mathbf{H} &= \frac{\partial \mathbf{D}} {\partial
t} + \mathbf{J}  & \text{Ampere's law with Maxwell's correction},
\label{eqn:ampereslaw}
\end{align}
where $\mathbf{E}$ is the electric field, $\mathbf{B}$ is the magnetic
field, $\mathbf{D}$ is the electric displacement field with an external
charge density $\rho$, and $\mathbf{H}$ is the auxiliary magnetic field
with current density $\mathbf{J}$.

In formulating Maxwell's equations it is assumed all electromagnetic
propagation takes place in a {\it simple dielectric}, that is
\begin{enumerate}
\item The polarization density $\mathbf{P}$ is linear with the electric
field $\mathbf{E}$,
\begin{align}
\mathbf{P}=\epsilon_0\chi_e\mathbf{E}.
\label{eqn:pdensity}
\end{align}
with $\chi_e$ defined as the electric susceptibility.  From
\Equation{eqn:pdensity} and the definition
of $\mathbf{D}$,
\begin{align}
\mathbf{D}=\epsilon_0\mathbf{E}+\mathbf{P},
\label{eqn:dfield}
\end{align}
with $\epsilon_0$ is the permittivity of free space, one can simplify the
$\mathbf{D}$-field as 
\begin{align}
\mathbf{D}&=\epsilon_0\mathbf{E}+\mathbf{P}\\
%&=\epsilon_0 \mathbf{E}+\epsilon_0 \chi_e \mathbf{E}\\
&=\epsilon_0(1+\chi_e)\mathbf{E}\\
&=\epsilon\mathbf{E}.
\label{eqn:permittivitydangle}
\end{align}
In \Equation{eqn:permittivitydangle}, $\epsilon$ is defined as the
permittivity of the dielectric.  The convention is taken 
that variables with a subscript $0$ can always be assumed to be 
{\it in vacuo}, while material dependent counterparts have
no subscript.  
\item The magnetization $\mathbf{M}$, related to the
auxiliary magnetic field by 
\begin{align}
\mathbf{M}=\mu_0\chi_m\mathbf{H},
\label{eqn:auxfield}
\end{align}
where $\chi_m$ is the volume magnetic susceptibility.
\Equation{eqn:auxfield} allows the
magnetic field to be rewritten as
\begin{align}
\mathbf{B}&=\mu_0\left(\mathbf{H}+\mathbf{M}\right)\\
&=\mu_0\left(1+\chi_m\right)\mathbf{H}\\
&=\mu \mathbf{H},
\end{align}
where $\mu_0$ is the vacuum permeability of free space.  In practice, most
dielectrics have nearly zero magnetic response, so $\mathbf{M}=0$ and
$\mu_0=\mu$ is a valid assumption.  Nevertheless, for the sake of generality it is treated as a
material dependent parameter.
\item The material is homogeneous and isotropic.
\item Charge is conserved.  $\mathbf{J} = \sigma \mathbf{E}$, where
 $\sigma$ is the conductivity and
\begin{align}
 \frac{\partial \rho}{\partial t} 
  &= -\nabla \cdot \mathbf{J}\\
  &= -\sigma \left(\nabla \cdot \mathbf{E} \right) \\
  &= -\frac{\sigma}{\epsilon} \rho.
\end{align}
\end{enumerate}
