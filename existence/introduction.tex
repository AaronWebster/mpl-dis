Biosensors based on surface plasmon resonance (SPR) play a central role as
a simple and remarkably responsive label-free method for characterizing and
quantifying biomolecular
interactions~\cite{homola1999surface}~\cite{homola2006surface}.  Among the
most popular of these sensors are those which excite localized surface
plasmon polaritons on thin metal films in prism coupled
configurations~\cite{hoa2007towards}.  These platforms, despite their
ubiquity and commercial success, are host to an amazing depth of useful
phenomena which has yet to be explored in the context of biosensing.  In
addition to the high field enhancement responsible for 

The
sensitivity of SPR is primarily due to the field enhancement by SPPs on the
sensor surface, but the SPPs themselves also possess high spatial
resolution beyond the diffraction limit; a property which traditionally has
not manifest itself as a specific feature of the SPR sensorgram.  

However, in this work it is shown that by considering optical speckle from
singly and multiply scattered SPPs inherent in the SPR signal, an entirely
new set of information can be obtained describing the underlying scattering
microstructure.  It is further demonstrated that one can resolve the motion
and addition of single nanoparticles in an unmodified SPR setup, extending
the breadth of SPR experiments to encompass both bulk sensing and discrete
events.



look at the structure of the re-radiated optical field which contain
additional features such as near field self-interference, speckle, and
optical vortices, and their relations to the scattering microstructure.

Our results suggest several possible avenues for advancing the detection
limits of surface plasmon based biosensors for both single particles and
bulk refractive index measurements.

A photon is a quantized oscillation of an electromagnetic field.  When an
electromagnetic  field is in proximity to an interface such as the surface
of a metal, oscillations of free charge can be induced.  If the field is
evanescent in both directions orthogonal to the surface, the oscillations
become localized and are known as surface plasmons (SPs).  Furthermore, if
conditions exist such that the in-plane momentum and phase of an incident
photon and the surface plasmon match, the coupling produces a hybrid
excitation known as surface plasmon polariton (SPP).  An SPP is trapped on
the interface and propagates until it decays; either re-radiating as a
photon or being absorbed into the metal as heat.

SPPs, like photons, are quantum mechanical objects.  Given system is
momentum-conserving, elastically scattered plasmons will preserve all the
information of their parent field.  

In the literature one will find a sense of interchangeability regarding
the terms ``plasmon'', ``surface plasmon'', ``localized surface plasmon'',
and ``surface plasmon polariton''.  To be clear, in this work we are always
referring to surface plasmon \textit{polaritons}: a surface plasmon which is
strongly coupled to the exciting electromagnetic field.  SPPs are a type of
quasiparticle, as they are essentially a collective excitation of free
electrons.  The fundamental point here is that strong coupling preserves
quantum mechanical information.

%\subsection{Organization}
%This work is roughly organized in the following way.
%\Chapter{ch:existence} is what you are currently reading.  Here is laid out
%theory, mathematical details regarding the conditions under which SPPs may
%be excited, and physical properties.  In \Chapter{ch:experimental} we
%describe all details regarding the physical experiment: construction,
%protocols, , and data analysis.  These first two chapters form the
%mathematical and physical basis for the remaining text.
%
%Discussion of our studies and new results begins with \Chapter{ch:bulkri}.
%Here, the bulk refractive index sensing properties of the cone are
%described.  Closely related, \Chapter{ch:interference} discusses a newly
%discovered interference phenomena in the cone and the possible utility in
%the context of SPR refractive index sensing.
%
%Perhaps one of the most interesting features of the cone is speckle, the
%subject of \Chapter{ch:speckle}.  Here we compare the properties of cone
%speckle with those of classic speckle fields in the context of
%correlations, refractive index perturbations, and multiple scattering
%effects.
%
%Finally, in \Chapter{ch:scatteringmicro} we look at the influence of the
%cone speckle on the scattering microstructure itself.
