Biosensors based on \gls{spr} play a central role as a simple and remarkably
responsive label-free method for characterizing and quantifying biomolecular
interactions~\cite{homola1999surface}~\cite{homola2006surface}.  Among the most
popular of these sensor platforms are those which use the excitation of
\glspl{spp} on thin metal films in prism-coupled
configurations~\cite{hoa2007towards}.  Such platforms, despite their ubiquity
and commercial success, are host to an amazing depth of useful phenomena not
yet explored in the context of biosensing.  In addition to the high field
enhancement responsible for their high sensitivity in bulk refractive index
measurements, the \glspl{spp} themselves possess high spatial resolution beyond the
diffraction limit; a property traditionally absent as a specific feature of an
\gls{spr} sensorgram.

The high spatial resolution of \glspl{spp} are perhaps best exemplified by an
instrument known as a scanning plasmon optical
microscope~\cite{kim1995scanning}~\cite{kim1996scanning} (\gls{spom}).  In a \gls{spom},
surface plasmons excited on a metal surface are scattered into a cone by
a sharp tip, typically in the same class as those used for scanning tunneling
microscopes (STMs) with single-atom sharpness~\cite{binnig2000scanning}.  By
raster scanning the tip over a surface of interest where spps are excited,
a sub-micron image of the scattering microstructure may be reconstructed from
the tip-dependent intensity of the conically scattered light.  Specific to the
\gls{spom} configuration, intensity of the conically scattered light has been shown
to be specifically sensitive to defects or irregularities on the
surface~\cite{kim1996scanning}.

The present work is insipred by the \gls{spom} concept, importing it into the heart
of \gls{spr} biosensing.  Instead of the tip used in the \gls{spom} setup,
nanoparticles introduced into a fluidic chamber in a typical \gls{spr} biosensor
are used in conjunction with optical speckle in the conically scattered light
signal to obtain information regarding the underlying scattering
microstructure.  Specifically, it is demonstrated that by monitoring speckle in
the cone, the motion and addition of single nanoparticles can be detected,
extending the breadth of \gls{spr} experiments to encompass both bulk sensing
and discrete events.  In addition, the present work investigates the optical
structure of the conically scattered light, a field which contains additional
bio-relevant features such as near field self-interference, speckle, and
optical vortices.  This completely novel sensing modality suggests several
possible avenues for advancing the detection limits of surface plasmon based
biosensors for both single particles and bulk refractive index measurements.
