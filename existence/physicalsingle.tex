Physical properties of SPP propagation are derived from their complex
spatial vectors $k_x$ and $k_z$.
Since the SPP is confined to the plane of the metal, the surface plasmon
wavelength $\lambda_\text{sp}$ is defined in terms of the real part of
$k_x$
\begin{align}
\lambda_\text{sp} &= \frac{\lambda_0}{2 \pi} k_x\\
& = \Re\left(\sqrt{
  \frac {\epsilon_1+\epsilon_2}
   {\epsilon_1 \epsilon_2 \lambda_0^2}
}\right)
\end{align}
when $k_x$ is imaginary, \Equation{eqn:planewavexz} is evanescent in
$x$ and the SPP decays with a characteristic $1/\me$ length
$\Im(1/(2k_x)$.  Similarly, when $k_{z,i}$ is imaginary, the
$1/\me$ evanescent decay into either the metal or vacuum is given by
$\Im(1/k_{z,i})$.  A table of these properties for many different SPP
excitation configurations is found in \Appendix{ref:physicalproperties}.
