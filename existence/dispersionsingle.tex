\Equation{eqn:sprcondition} can be solved analytically to find
$k_x(\omega)$, the dispersion relation for surface plasmon polaritions.
First, in terms of its vector components, the following holds in general
for all electromagnetic waves
\begin{align}
\mathbf{k}^2=\epsilon_i \left(\frac{\omega}{c}\right)^2=k_x^2 + k_{z,i}^2\\
\epsilon_i k_0^2=k_x^2 + k_{z,i}^2
\label{eqn:dispersion1}
\end{align}
Substitution of \Equation{eqn:sprcondition} with the relation 
$k_{x,1}=k_{x,2}$ into \Equation{eqn:dispersion1} allows 
$k_x$ and $k_{z,i}$ to be rewritten in the form of a dispersion relation
\begin{align}
k_x &= k_0\sqrt{\frac{\epsilon_1 \epsilon_2}{\epsilon_1+\epsilon_2}} 
= \frac{\omega}{c}\sqrt{\frac{\epsilon_1 \epsilon_2}{\epsilon_1+\epsilon_2}}\\
k_{z,i} &= k_0\frac{\epsilon_i}{\sqrt{\epsilon_1+\epsilon_2}}
= \frac{\omega}{c}\frac{\epsilon_i}{\sqrt{\epsilon_1+\epsilon_2}}
\end{align}

\begin{figure}[ht]
 \centering
\import{existence/figures/}{dispersionfig}
\label{fig:dispersionrelation}
\caption{
Dispersion relation for photons and plasmons for a single interface
consisting of a dielectric $\epsilon_1$ (fused silica) and a metal
$\epsilon_2$ (Ag using the Lorentz-Drude model). Conditions
under which SPPs may occur at the intersection between the photon 
line and corresponding SPP line.
}
\end{figure}
Where $\epsilon_i$ is a function of $\omega$, e.g.
$\epsilon_i\to\epsilon_i(\omega)$.  This relation, shown in
\Figure{fig:dispersionrelation}, is useful because it describes the
condition under which the momentum of SPPs are matched for a given
$\omega$.  For a photon in a dielectric $\omega = c k /\sqrt{\epsilon_i}$
for $i=1,2$.  This is known as the ``light line''.  The SPP on the other
hand lies below this and asymptotically approaches
$\omega_p/\sqrt{1+\epsilon_i}$ as $k_x\to\infty$; the two never cross.
However, if light is incident from a dielectric $\epsilon_1$ at an angle
$\theta$, the slope of $\omega(k_x)$ is modified to $c k \sin
\theta/\sqrt{\epsilon_1}$ and the momentum of light can be matched to
excite SPPs.  This is the principle exploited by Kretschmann
\cite{kretschmann1968} to excite SPPs with prisms, known as the Kretschmann
Atenuated Total Reflection (ATR) configuration: the same setup we will use
for SPP excitation.

There are several regions of interest in \Figure{fig:dispersionrelation},
depending on the relative value of $\epsilon_1$ and $\epsilon_2$.  We first
assume that $\epsilon_1$ is a dielectric with $\epsilon_1\in\mathbb{R}$ and
$\epsilon_1 > 0$ (true for most if not all glass).  In this case, for
$\Re(\epsilon_2)>0$, both $k_x$ and $k_z$ are real.  These are known as
``radiative modes'' -- an SPP mode is supported normal to the interface but
it is not bound there.  For $-\epsilon_1<\Re(\epsilon_2)<0$, $k_z$ is real
and $k_x$ imaginary.  The SPP is not confined to the interface and decays
evanescently there; this is known as a ``quasi-bound mode''  However, for
$\Re(\epsilon_2)<-\epsilon_1$, $k_x$ is real and $k_z$ is imaginary.  This
indicates the SPP is localized at the interface in $z$ while having a
propagating solution in $x$, and is exactly the condition we wish to match.
