\Equation{eqn:sprcondition} can be solved analytically to find the SPP
dispersion relation $k_x(\omega)$.
First, in terms of its vector components, the following holds in general
for electromagnetic waves with $k_y=0$,
\begin{align}
\mathbf{k}^2=\epsilon_i \left(\frac{\omega}{c}\right)^2=k_x^2 + k_{z,i}^2\\
\epsilon_i k_0^2=k_x^2 + k_{z,i}^2,
\label{eqn:dispersion1}
\end{align}
where $i$ is the medium index.
Substitution of \Equation{eqn:sprcondition} with the relation 
$k_{x,1}=k_{x,2}$ into \Equation{eqn:dispersion1} allows 
$k_x$ and $k_{z,i}$ to be rewritten as
\begin{align}
k_x &= k_0\sqrt{\frac{\epsilon_1 \epsilon_2}{\epsilon_1+\epsilon_2}} 
= \frac{\omega}{c}\sqrt{\frac{\epsilon_1
\epsilon_2}{\epsilon_1+\epsilon_2}} \label{eqn:kayexx}\\
k_{z,i} &= k_0\frac{\epsilon_i}{\sqrt{\epsilon_1+\epsilon_2}}
= \frac{\omega}{c}\frac{\epsilon_i}{\sqrt{\epsilon_1+\epsilon_2}}.
\label{eqn:dandangus}
\end{align}
where $\epsilon_i$ is always assumed to be a function of $\omega$,
$\epsilon_i\equiv\epsilon_i(\omega)$.  The dispersion relation,
\Equation{eqn:dispersion1}, plotted in \Figure{fig:dispersionrelation}, is
useful because it describes the condition under which the momentum of the
incident light is matched to an SPP at a particular frequency.  For a
photon in a dielectric, the ``light line'' is deined as $\omega = c k
/\sqrt{\epsilon_i}$ for $i=1,2$.  The SPP dispersion relation,
\Equation{eqn:dispersion1}, always lies below the light line,
asymptotically $\omega_p/\sqrt{1+\epsilon_i}$ as $k_x\to\infty$.
However, if light is incident from a dielectric $\epsilon_1$ at an angle
$\theta$, the slope of $\omega(k_x)$ is modified to $c k \sin
\theta/\sqrt{\epsilon_1}$ and the momentum of light can be matched to
excite SPPs.  The momentum matching property of light incident from a
dielectric at an angle was the principle exploited by Kretschmann
\cite{kretschmann1968} to excite SPPs with prisms, known as the Kretschmann
Atenuated Total Reflection (ATR) configuration.  There exist a multitude of
different excitation strategies beyond ATR, however in this work only ATR
will be considered.  The restriction to ATR is guided primarily by ease of
fabrication with facilities on hand (a sputtering machine and a metal
target, see \Chapter{ch:experimental}).

There are several regions of interest in \Figure{fig:dispersionrelation},
depending on the relative value of $\epsilon_1$ and $\epsilon_2$.  It is
first assumed that $\epsilon_1$ is a dielectric with
$\epsilon_1\in\mathbb{R}$ and that $\epsilon_1 > 0$, an assumption true for
most if not all glasss.  In this case, for $\Re(\epsilon_2)>0$, both $k_x$
and $k_z$ are real and the mode is radiative: an SPP mode is supported
normal to the interface but it is not bound.  For
$-\epsilon_1<\Re(\epsilon_2)<0$, $k_z$ is real and $k_x$ imaginary,
resulting in a quasi-bound mode where the SPP is not confined to the
interface and decays evanescently.  However, for
$\Re(\epsilon_2)<-\epsilon_1$, $k_x$ is real and $k_z$ is imaginary,
resulting in a bound mode.  In the bound mode, the SPP is localized at the
interface in $z$ while having a propagating solution in $x$, and is exactly
the condition which must be matched to excite SPPs.
\begin{figure}[ht]
 \centering
\import{existence/figures/}{dispersionfig}
\caption{ Dispersion relation for photons and plasmons for a single
interface consisting of a dielectric $\epsilon_1$, fused silica, and a
metal $\epsilon_2$, gold using the Lorentz-Drude model, also in
\Figure{fig:permittivityau}. Conditions under which SPPs are excited are at the
intersection between the photon line in a dielectric and corresponding SPP line.  }
\label{fig:dispersionrelation}
\end{figure}
