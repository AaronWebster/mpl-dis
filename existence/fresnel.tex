Having shown the existence of SPPs under the momentum-matching conditions
of Equations~\ref{eqn:sprcondition} and~\ref{eqn:lrsppdispersion}, it is
now relevant to predict how SPP excitation would manifest in a physical
experiment.  To do this, the Fresnel equations are used to calculate the
coupling conditions and properties of the excited field and the profiles of
the reflected, transmitted, and scattered optical fields.  
%We begin this section with our strategy for efficient computation of the
%Fresnel relations, and follow with a discussion of its predictions for both
%the near and far optical fields.  For implementation of this method on a
%computer, see the note on the Abeles matrix formalism in
%\Chapter{ch:abeles}.

\subsection{Prism Coupled Excitation}
According to the dispersion relation, \Equation{eqn:dispersion1}, SPPs are
excited on a metal-dielectric interface if an optical field is incident at
an angle $ck/\sqrt{\epsilon} \to ck\sin\theta/\sqrt{\epsilon}$.
Furthermore, the following conditions must also be true:
\begin{enumerate}
\item SPPs are excited with evanescent waves.  The
incident light must be at an angle greater than the critical angle, i.e.
$\theta>\arcsin\left(n_1/n_2\right)$.
\item The metal layer must
\begin{enumerate}
\item be thin enough to be at least partially optically transparent at the excitation wavelength
\item have small enough damping, set by its permittivity, such that SPPs can propagate
\end{enumerate}
\end{enumerate}

\begin{figure}[ht]
 \centering
 \begin{subfigure}[b]{0.4\textwidth}
  \centering
  \import{existence/figures/}{singlelayerinterfacegeo-kretschmann}
  \caption{}
 \end{subfigure}
 \begin{subfigure}[b]{0.4\textwidth}
  \import{existence/figures/}{multilayerinterfacegeo-kretschmann}
  \caption{}
 \end{subfigure}
\caption{Prism coupled excitation of SPPs in the Kretschmann configuration,
(a), a three layer interface system, and (b), a four layer interface system. }
\label{fig:prismcoupledsetups}
\end{figure}

The above conditions are satisfied when using the Kretschmann ATR
configuration to excite SPPs.  In the Kretschmann ATR configuration, light
incident at an angle from a dielectric prism is totally reflected from a
surface with the aforementioned multilayer structure.  The incident angle
is tuned such that the momentum of evanescent light on the surface matches
that of an SPP, i.e.\ the intersection of the SPP and light line
for a dielectric in \Figure{fig:dispersionrelation}.  

The Kretschmann ATR configuration is perhaps the most popular amongst many
different strategies capable of exciting SPPs.  Apart from ATR, the next
most commonly encountered strategy matches the SPP momentum using
diffractive gratings~\cite{homola1999surface}, though the grating coupler's
refractive index sensitivity is not as high as
ATR~\cite{homola1999senscomparison}.

Consider the single interface system, first shown in
\Figure{fig:singleinterfacegeo}.  The single interface system is modified
into a two interface, three layer system by adding another layer, $D_3$, as
shown in \Figure{fig:prismcoupledsetups}(a).  In the three layer system,
light is incident from the bulk media $D_1$ where $\epsilon_1>\epsilon_3$,
both satisfying $\epsilon^\prime_1>\epsilon^\prime_3>1$ as per the first
constraint.  $D_2$ is a metal.  Since the real part of the metal
permittivity, $\epsilon_2^\prime$, is a measure of its ability to store
energy, and the imaginary part, $\epsilon_2^{\prime\prime}$, a measure of
its dissipation, a metal is desired for $D_2$ with a large negative real
part and a small imaginary part.  
\begin{table}[ht]
\centering
\sisetup{round-mode=places,round-precision=3}
\begin{tabular}{lSS}
\toprule
metal & {$\epsilon^\prime$} & {$\epsilon^{\prime\prime}$} \\
\midrule
Ag & -16.0648691498177 & 1.18528458742412\\
Al & -55.0747552231730 & 22.2259823038134\\
Au & -11.3606990407702 & 1.92302171486855\\
Cu & -14.1858191543517 & 2.33420338076507\\
Cr & -6.04524862130375 & 31.2186141440352\\
Ni & -10.1637074922586 & 15.8784172933422\\
W  & +5.66969276029247 & 21.5082218271489\\
Ti & -5.85926367167276 & 13.6799220553247\\
Be & +1.61777122030528 & 22.0352487178570\\
Pd & -15.1527887328324 & 16.2452542701191\\
Pt & -12.7783923033094 & 20.6732739687699\\
\bottomrule
\end{tabular}
\caption{Complex permittivity for select metals at
$\lambda=\SI{660}{\nano\meter}$ calculated using the Lorentz-Drude
model.  Model parameters are from Refs.~\cite{ung2007interference} and
\cite{rakic1998optical}.}
\label{tbl:epsmetal600}
\end{table}

\Table{tbl:epsmetal600} shows the complex permittivity for several metals
at $\lambda=\SI{660}{\nano\meter}$, from which it is straightforward to
select an appropriate metal: silver, gold, or copper, with silver having
the smallest value for $\epsilon^{\prime\prime}$.  Since
$\epsilon^{\prime\prime}$ is related to energy lost in the medium, a small
value suggests longer SPP propagation lengths (indeed, silver is more
electrically conductive than copper which is more electrically conductive
than gold).  From these three gold is chosen and the parameter space
restricted to this metal.  Among the metals listed, gold has the second
lowest loss, does not oxidize, and most importantly has the most
compatibility with existing biochemical protols.  Furthermore, the Fresnel
equations predict for a three layer interface system, $D_2$ gold, the
thickness must be below \SI{100}{\nano\meter} to maintain transparency.
The multilayer system is modified as in \Figure{fig:prismcoupledsetups}(b),
with the prism being a supporting layer $D_4$ with permittivity
$\epsilon_4$, such that $\epsilon^\prime_4>\epsilon^\prime_{1,3}>1$ where
$\epsilon_1=\epsilon_3$ (in practice, $\epsilon_1\approx\epsilon_3$).

For a fixed incident wavelength of $\lambda_0=\SI{660}{\nano\meter}$, layer
$D_1$ $\epsilon_1=1.5142$ (BK7), layer $D_2$
$\epsilon_2=\num{-11.361+1.923i}$ (gold), layer $D_3$ $\epsilon_3=1.33$
(water~\cite{andreasson1971measurement}), and $D_4$ $\epsilon_4=1.33$,
simultaneous optimization of the incident angle $\theta$ and thicknesses of
$D_2$ and optionally $D_4$ was carried out using a nonlinear
minimization~\cite{brent1973algorithms} of the Fresnel reflectivity.  A
minimum in the Fresnel reflectivity is commensurate with optimal coupling
of light into SPPs.  For the three layer interface system in
\Figure{fig:prismcoupledsetups}(a), optimal coupling occurs for a $D_2$
thickness of approximately \SI{50}{\nano\meter}.  For the four layer
interface system in \Figure{fig:prismcoupledsetups}(b), optimal coupling
occurs for a $D_2$ thickness of \SI{16}{\nano\meter} and a $D_1$ thickness
of \SI{1150}{\nano\meter}.

The Fresnel reflectivity for $p$-polarization in the three and four layer
interface systems of \Figure{fig:prismcoupledsetups} are shown graphically
in in \Figure{fig:fresnelangle} as a function of both excitation wavelength
and coupling angle.  In particular, note the broad range of conditions for
which SPP excitation is possible (though not necessarily optimal),
corresponding to regions to the right of the critical angle for total
internal reflection with a minimum in $|r_p|^2$.

\begin{figure}[ht]
\centering
\import{existence/figures/}{fresneltogether}
\caption{Fresnel reflection coefficient for $p$ polarization, $|r^p|^2$, a
function of angle and excitation wavelength for conventional (right) and
long range (left) SPP structures in a Kretschmann ATR configuration.
Coupling of light into SPPs is possible in regions of low reflectivity to the right of the
critical angle for total internal reflection (dashed line).}
\label{fig:fresnelangle}
\end{figure}

\subsection{Near and Far Fields}\label{sec:fresnelnearfar}
The Fresnel equations have used to predict the conditions for
optimal SPP excitation, manifest as a minimum in the Fresnel reflectivity
for light incident at a particular angle and wavelength for a given layer
geometry.  In addition to the coupling conditions, the Fresnel equations
can also be used to predict the complete near and far optical fields,
taking into account effects from SPPs.

Evaluation of the near and far fields is carried out by taking advantage of
the shift theorem using the standard Fourier transform recipe,
\begin{equation}
\mathbf{E}(\mathbf{r}) = \intinfty \tilde{\mathbf{E}}(\mathbf{k})
\,\me^{\mi \mathbf{k} \cdot \mathbf{r}}
\,\me^{\mi \mathbf{k} \cdot \mathbf{z}^\prime} \md \mathbf{k},
\end{equation}
where $\tilde{\mathbf{E}}(\mathbf{k})$ is the $k$-space electric field at
the surface ($z^\prime=0$) given by Fresnel equations, taking into account
the distribution of the exciting field, and $\exp(\mi \mathbf{k} \cdot
\mathbf{z}^\prime)$ is the free-space transfer function along the
propagation direction $\mathbf{z}^\prime$ which includes the evanescent
wave due to the domain of integration.  The mathematical details of the
computation are further elaborated in \Section{sec:interferencetheory} and
\Section{sec:mrfsim}.  Even though the Fresnel equations provides 
an angular spectrum, the far field is not simply given by multiplying
$\tilde{\mathbf{E}}(\mathbf{k})$ by the angular spectrum of the exciting
field; a small lateral shift of the beam is introduced in certain
cases~\cite{chuang1986lateral}.

The Fourier transform recipe was used to generate plots of the incident,
reflected, and near field intensities, shown in
\Figure{fig:fresnelnearfieldspp} for the three layer structure, designated
``conventional SPPs'', and in \Figure{fig:fresnelnearfieldlrspp} for the
four layer structure, designated ``long range SPPs'' (LRSPPs).  The use of
this terminology is motivated by the respective propagation lengths,
though physically the SPPs are identical.
Figures~\ref{fig:fresnelnearfieldspp} and~\ref{fig:fresnelnearfieldlrspp}
were computed using an excitation wavelength of \SI{660}{\nano\meter}
and a Gaussian incident beam of numerical aperture $\mathrm{NA}=0.2$. 

As evidenced from the coupling conditions predicted by the Fresnel
equations, the most prominent optical feature of SPP excitation is a
minimum, or ``notch'' in the specularly reflected light at the surface
plasmon resonance angle $\thetasp$.  The angular location of the minimum is
highly sensitive to the refractive index in which the SPP propagates; the
mechanism most commonly exploited for biosensing.  In prism coupled setups,
the notch is always be found close to the critical angle for total internal
reflection, with sharper resonances being closer than broader ones.
Regarding the notch, owing to their greater attenuation, excitation of
conventional SPPs displays a relatively broader notch compared with long
range SPPs.

In the near field $x$-$y$ plots, both \Figure{fig:fresnelnearfieldspp} and
\Figure{fig:fresnelnearfieldlrspp} show a diffraction limited focal spot at
the origin, as a small \SI{3x5}{\micro\meter} ellipse with SPPs propagating
along the metal interface at $z=0$ in the positive $x$ direction.  The
multilayer structure (LRSPPs) exhibits a propagation length nearly 10 times
that of the single layer (conventional SPPs).  Such increased propagation
lengths aren't necessarily connected to the system's inherent bulk
refractive index sensitivity, though in practice physical instrumentation
provides a better signal with sharper optical resonances compared with
broad ones.  
%Questions of bulk refractive index sensitivity are further expounded in
%\Chapter

\begin{figure}[ht]
\centering
\import{existence/figures/}{dis-specular-spp.tex}
\caption{Near and far field intensities for excitation of conventional SPPs
				in a three layer system. $\lambda_0=\SI{660}{\nano\meter}$, $n_1 =
				\num{1.5142}$, $n_2=\num{0.2843 + 3.3825i}$, and $n_3=1.3310$.  The thickness of the metal layer is
				\SI{45}{\nano\meter}.}
\label{fig:fresnelnearfieldspp}
\end{figure}

\begin{figure}[ht]
\centering
\import{existence/figures/}{dis-specular-lrspp.tex}
\caption{Near and far field intensities for excitation long range SPPs in a
				symmetric layer system $\lambda_0=\SI{660}{\nano\meter}$, $n_1 =
				\num{1.5142}$, $n_2=1.3489$, $n_3=\num{0.2843 + 3.3825i}$, and
				$n_4=1.3310$.  The thickness of the metal layer is
				\SI{16.97}{\nano\meter}.}
\label{fig:fresnelnearfieldlrspp}
\end{figure}

