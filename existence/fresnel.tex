In the proceeding sections we have predicted the existence of SPPs and the
conditions under which they can be excited, but what is the physical
manifestation of SPP excitation?  To answer this question, we turn to the
Fresnel relations.  The fresnel reflection and transmission coefficents for
$s$ and $p$ polarization and a single ($i$-$j$) interface are
\begin{align}
r^s &= \frac{k_{z,i}-k_{z,j}}{k_{z,i}+k_{z,j}}\\
r^p &= \frac{\epsilon_i k_{z,j} - \epsilon_j k_{z,k}}{\epsilon_i k_{z,j} + \epsilon_j k_{z,k}} \\
t^s &= \frac{2 k_{z,i}}{k_{z,i}+k_{z,j}}\\
t^p &= \frac{2 \epsilon_i \epsilon_j k_{z,i}}{\epsilon_i k_{z,j} + \epsilon_j k_{z,i}}\\
\end{align}

For multilayer systems, these equations can be applied recursively.  For
our purposes we use the much more computationally efficent 
Abeles matrix formalism.  Here, the $n$th system matrix for $s$ and $p$
polarization, $C^s_n$ and $C^p_n$, are
\begin{align}
C^s_n = \left(\begin{array}{cc}
\me^{-\mi k_{z,n} d_n} & r^s_{n,n+1} \me^{-\mi k_{z,n} d_n} \\
r^s_{n,n+1} \me^{\mi k_{z,n} d_n} & \me^{\mi k_{z,n} d_n} 
\end{array}\right)\\
C^p_n = \left(\begin{array}{cc}
\me^{-\mi k_{z,n} d_n} & r^p_{n,n+1} \me^{-\mi k_{z,n} d_n} \\
r^p_{n,n+1} \me^{\mi k_{z,n} d_n} & \me^{\mi k_{z,n} d_n} 
\end{array}\right)
\end{align}
where $n$ is the index of the layer in question.  A matrix is constructed
for each layer of the system and a resultant matrix found
\begin{align}
M^s=\prod_n C^s_n\\
M^p=\prod_n C^p_n
\end{align}

From this matrix the system Fresnel coefficents can be found.
\begin{align}
r^p = \frac{M^p_{21}}{M^p_{11}}\\
r^s = \frac{M^s_{21}}{M^s_{11}}\\
t^p = \frac{1}{M^p_{11}}\\
t^s = \frac{1}{M^s_{11}}\\
\end{align}

Note that factor of two in the exponential: the SPP accumulates phase
through the metal film twice (once upon excitation, another upon decay).
In this equation, $k_{z,i}$ can be equivalently expressed either as a
function of incident angle $\theta$ or $k_x$
\begin{align}
k_{z,i} &= k_0 \sqrt{\epsilon_i - \epsilon_1 \sin \theta}\\
&= \sqrt{k_0^2\epsilon_i - k_x^2}
\end{align}
%
%\begin{figure}[ht]
% \centering
%\import{existence/figures/}{fresnellrsppfig}
%\label{fig:fresnellrsppfig}
%\caption{Fresnel reflection.}
%\end{figure}
%
%\begin{figure}[ht]
% \centering
%\import{existence/figures/}{fresnelsppfig}
%\label{fig:fresnellrsppfig}
%\caption{Fresnel reflection.}
%\end{figure}

\begin{figure}[ht]
 \centering
 \pgfplotsset{
 minor tick num=3,
 footnotesize,
 trim axis right,
 max space between ticks=30pt,
}
\tikzset{baseline}
\begin{tabular}{rr}
\import{existence/figures/}{dis-specular-sppplot-a}&\import{existence/figures/}{dis-specular-sppplot-c}\\
\import{existence/figures/}{dis-specular-sppplot-b}&\import{existence/figures/}{dis-specular-sppplot-d}
\end{tabular}
\begin{tabular}{rr}
\import{existence/figures/}{dis-specular-lrsppplot-a}&\import{existence/figures/}{dis-specular-lrsppplot-c}\\
\import{existence/figures/}{dis-specular-lrsppplot-b}&\import{existence/figures/}{dis-specular-lrsppplot-d}
\end{tabular}

\label{fig:fresnellrsppfig}
\caption{Fresnel reflection.}
\end{figure}

