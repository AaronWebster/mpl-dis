We have shown that SPPs can exist under certain momentum-matching
conditions layed out in Equations \ref{eqn:sprcondition} and
\ref{eqn:lrsppdispersion}.  We move now to predictions of what SPP
excitation actually looks like in an experiment.  To this accord, we will
use the Fresnel equations to calculate the coupling conditions and
properties of the excited field and the profiles of the reflected,
transmitted, and scattered optical fields.   We begin this section with our
strategy for efficient computation of the Fresnel relations, and follow
with a discussion of its predictions for both the near and far optical
fields. 

(say here that you use the ables matrix stuff)

\subsection{Prism Coupled Excitation}
From the information in the dispersion relation, SPP excitation at a
metal-dielectric interface is possible if light is incident at an angle such
that  $ck/\sqrt{\epsilon} \to ck\sin\theta/\sqrt{\epsilon}$.  Furthermore, we observe the following
conditions must also be true:
\begin{enumerate}
\item SPP excitation will be accomplished with evanescent waves.  The
incident light must be at an angle greater than the critical angle, i.e.
$\theta>\arcsin\left(n_1/n_2\right)$.
\item The metal layer must
\begin{enumerate}
\item be thin enough to be at least partially optically transparent at the excitation wavelength
\item have small enough damping, set by its permittivity, so that SPPs can propagate
\end{enumerate}
\end{enumerate}

\begin{figure}[ht]
 \centering
 \begin{subfigure}[b]{0.4\textwidth}
  \centering
  \import{existence/figures/}{singlelayerinterfacegeo-kretschmann}
  %\caption{}
 \end{subfigure}
 \begin{subfigure}[b]{0.4\textwidth}
  \centering
  \import{existence/figures/}{multilayerinterfacegeo-kretschmann}
  %\caption{Kretschmann SP.}
 \end{subfigure}
\caption{Prism coupled excitation of SPPs in the Kretschmann configuration.  (a) three layer
system.  (b) four layer system. }
\label{fig:prismcoupledsetups}
\end{figure}
Let us discuss these first in terms of the single interface system, as
first shown in \Figure{fig:prismcoupledsetups}.  We modify this system with
another layer, $\epsilon_3$, as shown in
\Figure{fig:prismcoupledsetups}(a).  In this system, light is incident from
$\epsilon_1$ and $\epsilon_1>\epsilon_3$, with both satisfying
$\epsilon^\prime_1>\epsilon^\prime_3>1$ as per the first constraint.
$\epsilon_2$ is a metal.  Since the real part of the metals permittivity
$\epsilon^\prime$ is a measure of its ability to store energy, and the
imaginary part $\epsilon^{\prime\prime}$ a measure of its dissipation, we
are looking for a metal with a large (negative) real part and a small
imaginary part.  \Table{tbl:epsmetal600} shows the complex permittivity for
several metals at $\lambda=\SI{660}{\nano\meter}$.

\begin{table}[ht]
\centering
\sisetup{round-mode=places,round-precision=3}
\begin{tabular}{lSS}
\toprule
metal & {$\epsilon^\prime$} & {$\epsilon^{\prime\prime}$} \\
\midrule
Ag & -16.0648691498177 & 1.18528458742412\\
Al & -55.0747552231730 & 22.2259823038134\\
Au & -11.3606990407702 & 1.92302171486855\\
Cu & -14.1858191543517 & 2.33420338076507\\
Cr & -6.04524862130375 & 31.2186141440352\\
Ni & -10.1637074922586 & 15.8784172933422\\
W  & +5.66969276029247 & 21.5082218271489\\
Ti & -5.85926367167276 & 13.6799220553247\\
Be & +1.61777122030528 & 22.0352487178570\\
Pd & -15.1527887328324 & 16.2452542701191\\
Pt & -12.7783923033094 & 20.6732739687699\\
\bottomrule
\end{tabular}
\caption{Complex permittivity for select metals at
$\lambda=\SI{660}{\nano\meter}$ calculated using the Lorentz-Drude model.}
\label{tbl:epsmetal600}
\end{table}

From \Table{tbl:epsmetal600} it is easy to filter out an appropriate
choice: silver, gold, or copper, with silver having the lowest loss.  At
this point we choose gold and restrict our parameter space to this metal.
It has the second lowest loss, does not oxidize, and most importantly has
the most compatability with existing biochemistry.  Furthermore, we find
via the Fresnel equations for such a three layer system that the thickness
of the metal layer must be below \SI{100}{\nano\meter}.  The multilayer
system is also modified as in \Figure{fig:prismcoupledsetups}(b), with the
prism being a supporting layer $\epsilon_0$ such that
$\epsilon^\prime_0>\epsilon^\prime_{1,3}>1$.  We also assume that
$\epsilon_1=\epsilon_3$ and that $\epsilon_2$ is gold.

Simultaneous optimization of all of these parameters, including the
incident angle is a task best suited for a computer.  For $\epsilon_2$ gold
and $\epsilon_3$ water, the three layer system predicts optimum coupling
around \SI{50}{\nano\meter} metal thickness, and \SI{16}{\nano\meter} for
the four layer system.  This is shown graphically in
\Figure{fig:fresnelangle} for different excitation wavelengths and coupling
angles.  In particular, note the broad range of conditions for which SPP
excitation is possible (though not necessairly optimal).

\begin{figure}[ht]
\centering
\import{existence/figures/}{fresneltogether}
\caption{Fresnel reflection coefficent as a function of angle and
excitation wavelength for conventioal (right) and long range (left) SPP structures in a Kretschmann ATR 
configuration.}
\label{fig:fresnelangle}
\end{figure}

\subsection{Near and Far Fields} \label{sec:fresnelnearfar}
We now use the Fresnel equations to predict the near and far optical
fields, including effects from SPP propagation.  Computation of the far
field is straightforward, as the Fresnel equations already provide an
angular spectrum.  The near field is found by taking advantage of the shift
theorem using the standard Fourier transform recipe
\begin{equation}
\mathbf{E}(\mathbf{r}) = \intinfty \tilde{\mathbf{E}}(\mathbf{k})
\,\me^{\mi \mathbf{k} \cdot \mathbf{z}^\prime}
\,\me^{\mi \mathbf{k} \cdot \mathbf{r}}\md \mathbf{k}
\end{equation}
where $\tilde{\mathbf{E}}(\mathbf{k})$ is the $k$-space electric field
found using the Fresnel equations, taking into account the distribution of
the incident field, and $\exp(\mi \mathbf{k} \cdot
\mathbf{z}^\prime)$ is the free-space transfer function along the 
propagation direction $\mathbf{z}^\prime$.  The mathematical details of this procedure
will be elaborated in \Section{sec:interferencetheory}, and the
computational details in \Section{sec:mrfsim}.

Plots of the incident, reflected, and near field intensities are shown in
\Figure{fig:fresnelnearfieldspp} for the three layer structure, which we
will call ``conventional SPPs'', and in \Figure{fig:fresnelnearfieldlrspp}
for the four layer structure, which we will call ``long range SPPs''
(LRSPPs).  This terminology will shortly become clear.  Both were excited
at \SI{660}{\nano\meter} with a Gaussian beam of numerical aperture
$\mathrm{NA}=0.2$.  The important features of these plots are as follows:
\begin{itemize}
\item As might be infered from \Figure{fig:fresnelangle},
the most prominent feature of SPP excitation is a minimum, or ``notch'' in
the specularly reflected light at an angle $\thetasp$.  The angular
location of this minimum is highly sensitive to the refractive index in
which the SPP propagates; this is the mechanism most commonly exploited for
biosensing.  The notch can always be found close to the critical
angle for total internal reflection, with sharper resonances being closer
than broader ones (see \Appendix{ch:reference}).
\item Owing to their greater attenuation, excitation of conventional SPPs
displays a relatively broader notch compared with long range SPPs.
\item In the near field $x$-$y$ plots, both show the diffraction limited
focal spot at the origin as a small \SI{3x5}{\micro\meter} ellipse, with
SPPs propagating along the metal interface in the positive $x$ direction.
It can be seen that the LRSPP propagation distance is nearly 10 times that
of conventional SPPs.  This isn't necessarily a comment on the system's
refractive index sensitivity, though we will use this property to explore
single and multiple scattering effects in \Chapter{ch:scatteringmicro}.
\end{itemize}

\begin{figure}[ht]
\centering
\import{existence/figures/}{dis-specular-spp.tex}
\caption{Near and far field intensities for excitation of conventional SPPs
				in a three layer system. $\lambda_0=\SI{660}{\nano\meter}$, $n_1 =
				\num{1.5142}$, $n_2=\num{0.2843 + 3.3825i}$, and $n_3=1.3310$.  The thickness of the metal layer is
				\SI{45}{\nano\meter}.}
\label{fig:fresnelnearfieldspp}
\end{figure}

\begin{figure}[ht]
\centering
\import{existence/figures/}{dis-specular-lrspp.tex}
\caption{Near and far field intensities for excitation long range SPPs in a
				symmetric layer system $\lambda_0=\SI{660}{\nano\meter}$, $n_1 =
				\num{1.5142}$, $n_2=1.3489$, $n_3=\num{0.2843 + 3.3825i}$, and
				$n_4=1.3310$.  The thickness of the metal layer is
				\SI{16.97}{\nano\meter}.}
\label{fig:fresnelnearfieldlrspp}
\end{figure}

