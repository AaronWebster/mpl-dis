We move now to using the wave equation to predict some of the the
observable phenomena of SPP exciation.  To this accord, we will use the
Fresnel equations to calculate the coupling conditions and properties of
the excited field.   We begin this section with our strategy for efficient
computation of the Fresnel relations, and follow with a discussion of its
predictions for both the near and far optical fields. 

\subsection{Matrix Formalism}
The Fresnel reflection and transmission coefficents for $s$ and
$p$ polarization and a single ($i$-$j$) interface are given by
\begin{align}
r^s &= \frac{k_{z,i}-k_{z,j}}{k_{z,i}+k_{z,j}}\\
r^p &= \frac{\epsilon_i k_{z,j} - \epsilon_j k_{z,k}}{\epsilon_i k_{z,j} + \epsilon_j k_{z,k}} \\
t^s &= \frac{2 k_{z,i}}{k_{z,i}+k_{z,j}}\\
t^p &= \frac{2 \epsilon_i \epsilon_j k_{z,i}}{\epsilon_i k_{z,j} + \epsilon_j k_{z,i}}\\
\end{align}
where the superscript denotes the polarization.
Note that $k_{z,i}$ can be equivalently expressed either as a
function of incident angle $\theta$ or $k_x$
\begin{align}
 k_{z,i} &= k_0 \sqrt{\epsilon_i - \epsilon_{i-1} \sin \theta}\\
&= \sqrt{k_0^2\epsilon_i - k_x^2}
\end{align}
For multilayer systems, these equations can be applied recursively. In 
this work we use the much more computationally efficent 
Abeles matrix formalism.  Here, the $n$th system matrix for $s$ and $p$
polarization, $C^s_n$ and $C^p_n$, are
\begin{align}
C^s_n = \left(\begin{array}{cc}
\me^{-\mi k_{z,n} d_n} & r^s_{n,n+1} \me^{-\mi k_{z,n} d_n} \\
r^s_{n,n+1} \me^{\mi k_{z,n} d_n} & \me^{\mi k_{z,n} d_n} 
\end{array}\right)\\
C^p_n = \left(\begin{array}{cc}
\me^{-\mi k_{z,n} d_n} & r^p_{n,n+1} \me^{-\mi k_{z,n} d_n} \\
r^p_{n,n+1} \me^{\mi k_{z,n} d_n} & \me^{\mi k_{z,n} d_n} 
\end{array}\right)
\end{align}
where $n$ and $n+1$ are the layer indicies for a single interface.  A
matrix is constructed for each layer of the system and a resultant matrix
found as the product of these matricies.
\begin{align}
M^s=\prod_n C^s_n\\
M^p=\prod_n C^p_n
\end{align}

From this matrix the system Fresnel coefficents are
\begin{align}
r^p = \frac{M^p_{21}}{M^p_{11}}\\
r^s = \frac{M^s_{21}}{M^s_{11}}\\
t^p = \frac{1}{M^p_{11}}\\
t^s = \frac{1}{M^s_{11}}\\
\end{align}

\subsection{Prism Coupled Excitation}
From the information in the dispersion relation, SPP excitation at a
metal-dielectric interface is possible if we are incident at an angle such
that the light line moves from $ck/\sqrt{\epsilon}$ to
$ck\sin\theta/\sqrt{\epsilon}$.  Furthermore, we observe the following
conditions must also hold
\begin{enumerate}
\item SPP excitation will be accomplished with evanescent waves.  The
incident light must be at an angle greater than the critical angle, e.g.
$\theta>\arcsin\left(n_1/n_2\right)$.
\item The metal layer must
\begin{enumerate}
\item be thin enough to be at least partially optically transparent at the excitation wavelength
\item have small enough damping, set by its permittivity, so that SPPs can propagate
\end{enumerate}
\end{enumerate}

\begin{figure}[ht]
 \centering
 \begin{subfigure}[b]{0.4\textwidth}
  \centering
  \import{existence/figures/}{singlelayerinterfacegeo-kretschmann}
  \caption{Kretschmann SSP.}
 \end{subfigure}
 \begin{subfigure}[b]{0.4\textwidth}
  \centering
  \import{existence/figures/}{multilayerinterfacegeo-kretschmann}
  \caption{Kretschmann ASP.}
 \end{subfigure}
\caption{Prism coupled excitation of SPPs.  (a) Kretschmann three layer
system, supporting asymmetric SPPs (ASP).  (b) Kretschmann
four layer system supporting symmetric SPPs (SSP). }
\label{fig:prismcoupledsetups}
\end{figure}
Let us discuss these first in terms of the single interface system, as
first shown in \Figure{fig:prismcoupledsetups}.  We modify this system with
another layer, $\epsilon_3$, as shown in
\Figure{fig:prismcoupledsetups}(a).  In this system, light is incident from
$\epsilon_1$ and $\epsilon_1>\epsilon_3$, with both satisfying
$\epsilon^\prime_1>\epsilon^\prime_3>1$ as per the first constraint.
$\epsilon_2$ is a metal.  Since the real part of the metals permittivity
$\epsilon^\prime$ is a measure of its ability to store energy, and the
imaginary part $\epsilon^{\prime\prime}$ a measure of its dissipation, we
are looking for a metal with a large (negative) real part and a small
imaginary part.  \Table{tbl:epsmetal600} shows the complex permittivity for
several metals at $\lambda=\SI{660}{\nano\meter}$.

\begin{table}[ht]
\centering
\sisetup{round-mode=places,round-precision=3}
\begin{tabular}{lSS}
\toprule
metal & {$\epsilon^\prime$} & {$\epsilon^{\prime\prime}$} \\
\midrule
Ag & -16.0648691498177 & 1.18528458742412\\
Al & -55.0747552231730 & 22.2259823038134\\
Au & -11.3606990407702 & 1.92302171486855\\
Cu & -14.1858191543517 & 2.33420338076507\\
Cr & -6.04524862130375 & 31.2186141440352\\
Ni & -10.1637074922586 & 15.8784172933422\\
W  & +5.66969276029247 & 21.5082218271489\\
Ti & -5.85926367167276 & 13.6799220553247\\
Be & +1.61777122030528 & 22.0352487178570\\
Pd & -15.1527887328324 & 16.2452542701191\\
Pt & -12.7783923033094 & 20.6732739687699\\
\bottomrule
\end{tabular}
\caption{Complex permittivity for select metals at
$\lambda=\SI{660}{\nano\meter}$ calculated using the Lorentz-Drude model.}
\label{tbl:epsmetal600}
\end{table}

From \Table{tbl:epsmetal600} it is easy to filter out an appropriate
choice: silver, gold, or copper, with silver having the lowest loss.  At
this point we choose gold and restrict our parameter space to this metal.
It has the second lowest loss, does not oxidize, and most importantly has
the most compatability with existing biochemistry.  Furthermore, we find
via the Fresnel equations for such a three layer system that the thickness
of the metal layer must be below \SI{100}{\nano\meter}.  The multilayer
system is also modified as in \Figure{fig:prismcoupledsetups}(b), with the
prism being a supporting layer $\epsilon_0$ such that
$\epsilon^\prime_0>\epsilon^\prime_{1,3}>1$.  We also assume that
$\epsilon_1=\epsilon_3$ and that $\epsilon_2$ is gold.

Simultaneous optimization of all of these parameters is a task best suited
for a computer.  For the three layer system something \SI{50}{\nano\meter},
 for the four layer system someting \SI{16}{\nano\meter}

something \Figure{fig:fresnelangle} shows you get excitation for many
wavelengths as a function of angle.

\begin{figure}[ht]
\centering
\import{existence/figures/}{fresneltogether}
\caption{Fresnel reflection coefficent as a function of angle and
excitation wavelength for the two SPP structures in a Kretschmann ATR 
configuration.}
\label{fig:fresnelangle}
\end{figure}

\subsection{Near Field}
\begin{figure}[ht]
 \centering
 \pgfplotsset{
 minor tick num=3,
 footnotesize,
 trim axis right,
 max space between ticks=30pt,
}
\tikzset{baseline}
\begin{tabular}{rr}
\import{existence/figures/}{dis-specular-sppplot-a}&\import{existence/figures/}{dis-specular-sppplot-c}\\
\import{existence/figures/}{dis-specular-sppplot-b}&\import{existence/figures/}{dis-specular-sppplot-d}
\end{tabular}
\begin{tabular}{rr}
\import{existence/figures/}{dis-specular-lrsppplot-a}&\import{existence/figures/}{dis-specular-lrsppplot-c}\\
\import{existence/figures/}{dis-specular-lrsppplot-b}&\import{existence/figures/}{dis-specular-lrsppplot-d}
\end{tabular}
\label{fig:fresnellrsppfig}
\caption{Fresnel reflection.}
\end{figure}

