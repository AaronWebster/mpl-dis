The theoretical groundwork for the existence of \glspl{spp} was first
introduced by \name{Richie} in his seminal 1957 paper \textit{Plasma Losses by
  Fast Electrons in Thin Films} \cite{ritchie1957plasma}.  Like any scientific
work, Richie's was incremental and has its roots in earlier theoretical
proposals by \name{Pines} and
\name{Bohm}~\cite{bohm1951collective}~\cite{pines1952collective}.  Ultimately
the work of \name{Richie}, \name{Pine}, and \name{Bohm} would explain a bulk
\gls{spp} phenomena known as ``Wood's anomaly'' --- a sharp and spectrally
narrow energy loss observed in diffraction gratings by \name{Wood} in
1902~\cite{wood1902remarkable}~\cite{rayleigh1907remarkable}.

Optical excitation of surface plasmons was made accessible through
pioneering work in the late 1960's by
\name{Kretschmann}~\cite{kretschmann1968},
\name{Raether}~\cite{raether1988springer} and
\name{Otto}~\cite{otto1968excitation}.  These \gls{spp} experiments used the
principle of \gls{atr} in a prism to excite surface
plasmons evanescently.  A great deal of research on the topic of surface
plasmons took place in the subsequent decade, including an improved
theoretical understanding based on the Fresnel
relations~\cite{chen1976excitation} and investigation of conically scattered
light in the presence of surface roughness~\cite{simon1976directional}.  A
concise historical overview of prism-coupled \gls{spp} studies can be found
in~\cite{raether1997surface}.

The introduction of \gls{spr} as a biosensing platform began in the early
1980's with work by \name{Liedberg}, \name{Nylander} and
\name{Lundstrom}~\cite{liedberg1983surface} who described the extraordinary
sensitivity of \gls{spr} to perturbations of the refractive
index of a medium on one side of the metal film.  The subsequent
commercialization of \gls{spr} biosensors has largely been influenced by these
authors and their spinoff company Pharmacia Biosensor AB (now
Biacore)~\cite{liedberg1995biosensing}.

The commercial success of \gls{spr} biosensors brought about a knowledge gap between
the biosensing community and their more theoretical predecessors from whom the
field owes its genesis; i.e\@. the scope of \gls{spr} biosensing experiments is
disproportionately narrower than the breadth of phenomena discovered since
\name{Richie}.  For example in 2005 and 2007, two
papers~\cite{andaloro2005optical}~\cite{simon2007observation} based on
theoretical work by \name{Chuang}~\cite{chuang1986lateral} and \name{Chen}
\cite{chen1976excitation} reported a curious interference pattern occurring in
the specularly reflected light for certain systems (among them,
Kretschmann-Raether type) illuminated with a focused beam.  The interference
pattern was also independently reported a year later by \name{Schumann} and
\name{Gregory}~\cite{schumann2008near}.  Observation of the interference
required nothing more than the addition of a lens pair to an otherwise
ubiquitous optical setup, but it somehow escaped attention during earlier
research.

The thrust of this work is primarily inspired by experiments done at the
University of Oregon in the laboratory of \name{Gregory}, summarized in a 2009
thesis \textit{Surface Plasmon Random Scattering and Related
  Phenomena}~\cite{schumann2009surface} by \name{Schumann}.  \name{Schumann}
describes a series of experiments using a scanning apertureless near-field
probe in a Kretschmann-Raether type configuration~\cite{kim1995scanning}.  The
probe, composed of a sharp tungsten tip, was shown to elastically scatter \glspl{spp}
in a way analogous to surface roughness, but in this case its location and
interaction could be precisely determined.  By controlling the location of
a single scatterer among a fixed background, \name{Schumann} showed the
possibility of determining the entire surface scattering microstructure from
the scattered light alone.  This work demonstrates that analogous experiments
can be carried out in the context of biosensing using gold nanoparticles as
scatterers.

The final important historical development was on correlations in elastic
multiple scattering through disordered media.  Most notable is a seminal paper
by \name{Feng}, \name{Lee}, \name{Stone}, and
\name{Douglas}~\cite{feng1988correlations}.  The authors showed that the
seemingly random fluctuations in the output channel contain correlations which
track the behavior of the input channel.  In the optical regime such
fluctuations are known as speckle.  Statistical properties have been explored
in depth by
\name{Goodman}~\cite{goodman2007speckle}~\cite{goodman1975statistical}.
Correlations in optical speckle patterns were explored in depth using
theoretical methods by \name{Berkovits} and
\name{Feng}~\cite{berkovits1994correlations}, establishing that the multiple
scattering regime is extraordinarily sensitive to both the
position~\cite{berkovits1990theory} and motion~\cite{berkovits1991sensitivity}
of even a single scatterer.  The motion of scattering locations in a multiple
scattering system has important implications in many diverse fields of study:
diffusive wave spectroscopy~\cite{pine1988diffusing}, tracking and
identification of targets with radio waves (the ``cruise missile''
problem~\cite{atkins1991neural}), fluctuations in signal power in cellular
telephone networks~\cite{abdi2001estimation}, and very recently in detecting
stress fractures in aggregates such as concrete~\cite{larose2010locating}.
