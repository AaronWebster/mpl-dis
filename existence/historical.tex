The theoretical groundwork for the existence of surface plasmon polaritons
(SPPs) was first introduced by \name{Richie} in his seminal 1957 paper
\textit{Plasma Losses by Fast Electrons in Thin Films}
\cite{ritchie1957plasma}.  Like any scientific work, Richie's was
incremental and has its roots in earlier theoretical proposals by
\name{Pines} and
\name{Bohm}~\cite{bohm1951collective}~\cite{pines1952collective}.
Ultimately the work of \name{Richie}, \name{Pine}, and \name{Bohm} would
explain a bulk SPP phenomena known as ``Wood's anomaly'' --- a sharp and
spectrally narrow energy loss observed in diffraction gratings by
\name{Wood} in
1902~\cite{wood1902remarkable}~\cite{rayleigh1907remarkable}.

Optical excitation of surface plasmons was made accessible through
pioneering work in the late 1960's by
\name{Kretschmann}~\cite{kretschmann1968},
\name{Raether}~\cite{raether1988springer} and
\name{Otto}~\cite{otto1968excitation}.  These SPP experiments used the
principle of attenuated total reflection (ATR) in a prism to excite surface
plasmons evanescently.  A great deal of research on the topic of surface
plasmons took place in the subsequent decade, including an improved
theoretical understanding based on the Fresnel
relations~\cite{chen1976excitation} and descriptions of conically scattered
light in the presence of surface roughness~\cite{simon1976directional}.  A
concise historical overview of prism coupled SPP studies can be found
in~\cite{raether1997surface}.

The introduction of surface plasmon resonance (SPR) as a biosensing
platform began in the early 1980's with work by \name{Liedberg},
\name{Nylander} and \name{Lundstrom}~\cite{liedberg1983surface} who
described the extraordinary sensitivity of the surface plasmon resonance
condition to perturbations in the refractive index of the medium on one
side of the metal film.  The subsequent commercialization of SPR biosensors
has largely been influenced by these authors and Pharmacia Biosensor AB
(now Biacore)~\cite{liedberg1995biosensing}.

The commercial success of SPR biosensors brought about a knowledge gap
between the biosensing community and their more theoretical predecessors
from whom the field owes its genesis; the scope of SPR biosensing
experiments is disproportionately narrower than the breadth of phenomena
discovered since \name{Richie}.  As an example particular to this work, in
2005 and 2007, two
papers~\cite{andaloro2005optical}~\cite{simon2007observation} based on
theoretical work by \name{Chuang}~\cite{chuang1986lateral} and \name{Chen}
\cite{chen1976excitation} reported a curious interference pattern occurring
in the specularly reflected light for certain systems (among them,
Kretschmann-Raether type) illuminated with focused Gaussian beams.  The
interference pattern was also independently reported a year later in
\cite{schumann2008near}.  Interestingly, observation required nothing more
than the addition of a lens pair to an otherwise ubiquitous optical setup,
but it somehow escaped attention during earlier research.  

The thrust of the present work is primarily inspired by experiments done at
the University of Oregon in the laboratory of \name{Gregory}, summarized in
a 2009 thesis \textit{Surface Plasmon Random Scattering and Related
Phenomena}~\cite{schumann2009surface} by \name{Schumann}.  \name{Schumann}
describes plasmon experiments with scanning apertureless near-field probes
in a Kretschmann-Raether type configuration~\cite{kim1995scanning}.  The
probe, composed of a sharp tungsten tip, is able to elastically scatter
SPPs in a way analogous to surface roughness, but its location precisely
controlled.  Controlling the location of a single scatterer among a fixed
background was shown to be revealing of the entire scattering
microstructure.  In this work it is shown that analogous experiments can be
carried out in the context of biosensing, using metallic nanoparticles as
scatterers.

The final important historical development was on correlations in elastic
multiple scattering through disordered media.  Most notable is a seminal
paper by \name{Feng}, \name{Lee}, \name{Stone}, and
\name{Douglas}~\cite{feng1988correlations}.  Here it was established that
the random fluctuations in the output channel contain correlations which
track the behavior of the input channel.  In the optical regime, such
fluctuations are known as speckle, whose statistical properties have been
explored in depth by
\name{Goodman}~\cite{goodman2007speckle}~\cite{goodman1975statistical}.
Correlations in optical speckle patterns were theoretically explored in
depth by \name{Berkovits} and \name{Feng}~\cite{berkovits1994correlations},
establishing that the multiple scattering regime is extraordinarily
sensitive to both the position~\cite{berkovits1990theory} and
motion~\cite{berkovits1991sensitivity} of even a single scatterer.  The
motion of scattering locations in a multiple scattering system has
important implications in many diverse fields of study: diffusive wave
spectroscopy~\cite{pine1988diffusing}, tracking and identification of
targets with radio waves (the ``cruise missile''
problem~\cite{atkins1991neural}), fluctuations in signal power in cellular
telephone networks~\cite{abdi2001estimation}, and very recently in
detecting stress fractures in aggregates such as
concrete~\cite{larose2010locating}.  
