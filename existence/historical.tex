The theoretical groundwork for the existence of surface plasmon polaritons
(SPPs) was first introduced by \name{Richie} in his seminal 1957 paper
\textit{Plasma Losses by Fast Electrons in Thin Films}
\cite{ritchie1957plasma}.  Like any scientific work, Richie's was
incremental and has its roots in earlier theoretical proposals by
\name{Pines} and
\name{Bohm}~\cite{bohm1951collective}~\cite{pines1952collective}.
Ultimately this research functioned to explain the phenomena of sharp and
spectrally narrow energy losses observed in diffraction gratings by
\name{Wood} in 1902, known as ``Wood's anomaly''.

Optical excitation of surface plasmons was made accessible through
pioneering work in the late 1960's by
\name{Kretschmann}~\cite{kretschmann1968},
\name{Raether}~\cite{raether1965springer} and
\name{Otto}~\cite{otto1968excitation}.  These experiments used the
principle of attenuated total reflection (ATR) to excite surface plasmons
evanescently, using a prism to match their resonance condition.  A great
deal of understanding on the topic of surface plasmons took place in the
subsequent decade, including an improved theoretical understanding based on
Fresnel relations \cite{chen1976excitation} and descriptions of of
conically scattered light in the presence of surface
roughness~\cite{simon1976directional}.  A concise overview of this research
can be found in~\cite{raether1997surface}.

The introduction of SPR as a biosensing platform began in the early 1980's
with work by \name{Liedberg}, \name{Nylander} and
\name{Lundstrom}~\cite{liedberg1983surface} who described the
extraordinary sensitivity of the surface plasmon resonance condition to
perturbations in the refractive index of the medium on one side of the
metal film.  The subsequent commercialization of SPR biosensors has largely
been influenced by these authors and Pharmacia Biosensor AB (now
Biacore)~\cite{liedberg1995biosensing}.

The commercial success of biosensors based on surface plasmon resonance
brought about a knowledge gap between the biosensing community and their
more theoretical predecessors from whom the field owes its genesis; the
scope of SPR biosensing experiments is disproportionately narrower than the
breadth of phenomena discovered since \name{Richie}.  As an example particular to
this dissertation, in 2005 and 2007, two
papers~\cite{andaloro2005optical}~\cite{simon2007observation} based on
theoretical work by \name{Chuang}~\cite{chuang1986lateral} and
\name{Chen} \cite{chen1976excitation} reported a curious interference
pattern occurring in the specularly reflected light for certain (among
them, Kretschmann-Raether type) systems illuminated with focused Gaussian
beams.  This was also independently reported a year later in
\cite{schumann2008near}.  Interestingly, observation of this interference
required nothing more than the addition of a lens pair to an otherwise 
ubiquitous optical setup, but it somehow escaped attention during earlier
research.  

The thrust of this work is primarily inspired by experiments done at the
University of Oregon in the labratory of \name{Stephen Gregory}, summarized
in a 2009 thesis \textit{Surface Plasmon Random Scattering and Related
Phenomena} \cite{schumann2009surface} by \name{R\@.P.\@~Schumann}.  Here
the authors describe plasmon experiments with scanning apertureless
near-field probes in a Kretschmann-Raether type configuration.  This probe,
a sharp tungsten tip, is able to elastically scatter SPPs in a way
analogous to surface roughness, but in this case its location and
interaction of a single scatterer can be precisely controlled.  We will
show that similar experiments can be carried out in the context of
biosensing, using metallic nanoparticles as scatterers.

The final important historical development we cross is that of correlations
in elastic multiple scattering through disordered media.  Most notable is
the seminal paper by \name{Feng}, \name{Lee}, \name{Stone}, and
\name{Douglas}~\cite{feng1988correlations}.  Here it was established that
the seemingly random fluctuations in the output channel through a
disordered system are not random at all, but contain correlations which
track the behavior of the input channel.  In the optical regime these
fluctuations are known as speckle, whose statistal properties have been
explored in depth by
\name{Goodman}~\cite{goodman2007speckle}~\cite{goodman1975statistical}.
The important implications speckle correlations were perhaps best
summarized by \name{Berkovits} and
\name{Feng}~\cite{berkovits1994correlations}.  These works established that
the multiple scattering regime is extraordinary sensitive to both the
position~\cite{berkovits1990theory} and
motion~\cite{berkovits1991sensitivity} of even a single scatterer.  This
problem has important implications in many diverse fields of study:
diffusive wave spectroscopy~\cite{pine1988diffusing}, tracking and
identification of targets with radio waves (the ``cruise missile''
problem~\cite{atkins1991neural}), fluctuations in signal power in
cellular telephone networks~\cite{abdi2001estimation}, and
very recently in detecting stress fractures in aggregates such as
concrete~\cite{larose2010locating}.  

%Our historical perspective ends here.  Scattering of surface plasmons
%affords one a unique system with which to study these effects.  By changing
%the amount of surface roughness, 
