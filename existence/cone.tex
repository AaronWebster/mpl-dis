The dark band or notch in the specular direction is perhaps the most
classic signature of SPR in the Kretchmann ATR setup.  Fundamentally this
is due to the interference between the specularly reflected beam and the
re-radiated SPP field which is antiphase at $\theta_\mathrm{SP}$.  There
is, however, an additional important signature which can be observed in the
same half-space.  During propagation, roughness or other surface
inhomogenities can elastically modify the in-plane momentum of an SPP.
This is known as ``directional scattering'', the signature of which is a
annular cone of light at $\theta_\mathrm{SP}$ along the azimuthal
coordinate $\phi$.

Physically, an SPP first is transmitted from the prism and permeates the
metal film.  We identifiy this by the Fresnel forward transmission
coefficent, $t^p_+$.  Upon re-radiation as a photon, the SPP will take this
path in reverse, which we identify by the Fresnel reverse transmission,
$t^p_-$.  The field in the cone $E_\mathrm{cone}$ is proportional to the
product of these two.
\begin{equation}
|E_\mathrm{cone}|^2 \propto	|t^p_+|^2 |t^p_-|^2
\end{equation}

In this aspect the Fresnel equations do not 

In the subsequent discussions, we will use the term ``cone'' when refering
to light scattered into the cone, and ``notch'' when refering to light in
the specular direction.

\begin{figure}[ht]
 \centering
 \import{includes/}{setpgfinc}
 \import{existence/figures/}{cone_vs_notch}
 \caption{Comparison of the angular profiles of the cone and notch.  The
	cone has been inverted and normalized to the same range as the notch for
	comparison.}
 \label{fig:conevsnotch}
\end{figure}


% cone related to roughness spectrum
% cone very dim
% dis_specular_cone.m, plot of the cone
% some diagram showing where it looks like
