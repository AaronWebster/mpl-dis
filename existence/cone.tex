The dark band or notch in the specular direction is perhaps the most
classic signature of SPR in the Kretchmann ATR setup.  Fundamentally this
is due to the interference between the specularly reflected beam and the
re-radiated SPP field which is antiphase at $\theta_\mathrm{SP}$.  There
is, however, an additional important signature which can be observed in the
same half-space.  During propagation, surface roughness or defects in the
metal surface can change the in-plane momentum of an SPP such that for
initial and resultant $k$-vectors $\mathbf{k}_a$ and $\mathbf{k}_b$,
$\mathbf{k}_a = (k_x,0,k_z) \to \mathbf{k}_b = (k_x,k_y,k_z)$ and
$\mathbf{k}_a\cdot\mathbf{k}_a = \mathbf{k}_b\cdot\mathbf{k}_b$.  The
plasmon is elastically scattered in the plane of propagation.  This is also
known as ``directional scattering'', the signature of which is a annular
cone of light at $\theta_\mathrm{SP}$.

% fresnel transmission
% cone related to roughness spectrum
% cone very dim
% dis_specular_cone.m, plot of the cone
% some diagram showing where it looks like
% fresnel r321 or t123
