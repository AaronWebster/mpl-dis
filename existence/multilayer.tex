\begin{figure}[ht]
 \centering
 \import{existence/figures/}{multilayerinterfacegeo}
 \caption{Geometry and coordinate system for the multilayer system. }
 \label{fig:multilayergeo}
\end{figure}
Having predicted the existence of SPPs on a single interface, we now
consider a multilayer geometry, shown in \Figure{fig:multilayergeo}.  This
has the potential to excite a different variety of SPP, as we shall soon
see.  For
layers $\epsilon_1$ and $\epsilon_3$, the equations are the same the
single layer system
\begin{align}
\left.\begin{aligned}
\mathbf{H}_1(x,z,t) &=
\begin{pmatrix}
0\\
H_{y,1}\\
0
\end{pmatrix} \me^{\mi(k_{x,1}x+k_{z,1}z-\omega t)}\\
\mathbf{E}_1(x,z,t) &=
\begin{pmatrix}
E_{x,1}\\
0\\
E_{z,1}\\
\end{pmatrix} \me^{\mi(k_{x,1}x+k_{z,1}z-\omega t)}
\end{aligned}
\right\}& \quad \epsilon_1\label{eqn:lrsppplanewavedielectric}\\
\left.\begin{aligned}
\mathbf{H}_3(x,z,t) &=
\begin{pmatrix}
0\\
H_{y,3}\\
0
\end{pmatrix}
\me^{\mi(k_{x,3}x+k_{z,3}z-\omega t)}\\
\mathbf{E}_3(x,z,t) &=
\begin{pmatrix}
E_{x,3}\\
0\\
E_{z,3}\\
\end{pmatrix}
\me^{\mi(k_{x,3}x+k_{z,3}z-\omega t)}
\end{aligned} 
\right\}&\quad \epsilon_3
\label{eqn:lrsppplanewavemetal}
\end{align}

If $a$ is sufficently small, this type of setup may be though of as a waveguide for SPPs.
In the ``core'' region, $\epsilon_2$ and $-a<z<a$, there are two interfaces at $\pm a$
\begin{align}
\mathbf{H}_2(x,z,t) &=
\begin{pmatrix}
0\\
H^+_{y,2}\\
0
\end{pmatrix} \me^{\mi(k_{x,2}x+k_{z,2}z-\omega t)}
+
\begin{pmatrix}
0\\
H^-_{y,2}\\
0
\end{pmatrix} \me^{\mi(k_{x,2}x+k_{z,2}z-\omega t)} \\
\mathbf{E}_2(x,z,t) &=
\begin{pmatrix}
\mi E^+_{x,2}\\
0\\
E^+_{z,2}\\
\end{pmatrix} \me^{\mi(k_{x,2}x+k_{z,2}z-\omega t)}
+
\begin{pmatrix}
\mi E^-_{x,2}\\
0\\
E^-_{z,2}\\
\end{pmatrix} \me^{\mi(k_{x,2}x+k_{z,2}z-\omega t)}
\end{align}

Taking this system of equations and applying continuity of $H_y$ and $E_x$
at the boundaries, with a bit of algebra the following dispersion relation
is obtained
\begin{equation}
\me^{-4 \mi k_{x,1} a} = 
\frac{k_{x,1}/\epsilon_1 + k_{x,2}/\epsilon_2}
     {k_{x,1}/\epsilon_1 - k_{x,2}/\epsilon_2}
\frac{k_{x,1}/\epsilon_1 + k_{x,3}/\epsilon_3}
     {k_{x,1}/\epsilon_1 - k_{x,3}/\epsilon_3}
\label{eqn:lrsppdispersion}
\end{equation}

\Equation{eqn:lrsppdispersion} is transcendental, but we can still look at
the dispersion relation graphically by plotting the difference between the
left and right hand side of the equation.

\begin{figure}[ht]
 \centering
\import{existence/figures/}{lrsppdispersionfig}
\label{fig:lrsppdispersionrelation}
\caption{Disperion relation for multilayer system, shown as the difference
 between the right hand side (rhs) and left hand side (lhs) of
\Equation{eqn:lrsppdispersion}.  Possible solutions are found within the
red areas.}
\end{figure}

\subsection{Photonic Crystal Structures}
The systems in Figures \ref{fig:singleinterfacegeo} and
\ref{fig:multilayergeo} are by far the most popular geometries for SPP
excitation, at least as far as biosensing is concerned.  There
is, however, a very intriguing geometry based on a multilayer stack with
which one can obtain SPP propagation distances in the \textit{millimeter}
range.  Currently, most of the work on these structures are being carried
out by \name{Konopsky}~\cite{konopsky2006long}~\cite{konopsky2009long}.
The working principle is akin to a Bragg mirror: a one dimensional stack of
layers with varying refractive index is built up on the surface of a prism,
ultimately depositing a metal as the final layer.  The multilayer stack
works to minimize the tangential component of the electric field in the
metal, reducing SPP atenuation.  This has several advantages to the
geometries mentioned: specifically, it is possible to use lossy metals with
more interesting surface chemistries (Pd and Pt), and it becomes possible
to study long range surface plasmons with near field probes, which the
liquid phase would otherwise make difficult.
