\begin{figure}[ht]
\centering
\import{existence/figures/}{multilayerinterfacegeo}
\caption{Geometry and coordinate system for the multilayer system. }
\label{fig:multilayergeo}
\end{figure}

The single metal-dielectric interface is the principle under which most
labratory and commercial \gls{spr} biosensors operate.  Apart from this, certain
multilayer interface geometries, such as the one shown in
\Figure{fig:multilayergeo}, have attractive features relevant to the
present work.  In particular, it will be shown that multilayer interface
geometries support \gls{spp} propagation lengths more than an order of magnitude
greater than the single-layer interface.  For example, symmetric multilayer
interface with \SI{10}{\nano\meter} gold layers at \SI{632.8}{\nano\meter}
see ehancements of $10$-$60$, with propagation lengths up to
\SI{250}{\micro\meter}~\cite{kuwamura1983experimental}~\cite{craig1983experimental}.

For layers $D_1$ and $D_3$, the equations are the same as for the single-layer system
\begin{align}
\left.\begin{aligned}
\mathbf{H}_1(x,z,t) &=
\begin{pmatrix}
0\\
H_{y,1}\\
0
\end{pmatrix} \me^{\mi(k_{x,1}x+k_{z,1}z-\omega t)}\\
\mathbf{E}_1(x,z,t) &=
\begin{pmatrix}
E_{x,1}\\
0\\
E_{z,1}\\
\end{pmatrix} \me^{\mi(k_{x,1}x+k_{z,1}z-\omega t)}
\end{aligned}
\right\}& \quad D_1\label{eqn:lrsppplanewavedielectric}\\
\left.\begin{aligned}
\mathbf{H}_3(x,z,t) &=
\begin{pmatrix}
0\\
H_{y,3}\\
0
\end{pmatrix}
\me^{\mi(k_{x,3}x+k_{z,3}z-\omega t)}\\
\mathbf{E}_3(x,z,t) &=
\begin{pmatrix}
E_{x,3}\\
0\\
E_{z,3}\\
\end{pmatrix}
\me^{\mi(k_{x,3}x+k_{z,3}z-\omega t)}
\end{aligned}
\right\}&\quad D_3.
\label{eqn:lrsppplanewavemetal}
\end{align}

If the half-thickness of $D_2$, $a$, is sufficiently small, $D_2$ will act
as a waveguide for \glspl{spp}.  In $D_2$ there are two interfaces, the
$D_2$-$D_1$ interface at $+a$ and the $D_2$-$D_3$ interface at $-a$,
leading to
\begin{align}
\mathbf{H}_2(x,z,t) &=
\begin{pmatrix}
0\\
H^+_{y,2}\\
0
\end{pmatrix} \me^{\mi(k_{x,2}x+k_{z,2}z-\omega t)}
+
\begin{pmatrix}
0\\
H^-_{y,2}\\
0
\end{pmatrix} \me^{\mi(k_{x,2}x-k_{z,2}z-\omega t)} \label{eqn:multihhx} \\
\mathbf{E}_2(x,z,t) &=
\begin{pmatrix}
\mi E^+_{x,2}\\
0\\
E^+_{z,2}\\
\end{pmatrix} \me^{\mi(k_{x,2}x+k_{z,2}z-\omega t)}
+
\begin{pmatrix}
\mi E^-_{x,2}\\
0\\
E^-_{z,2}\\
\end{pmatrix} \me^{\mi(k_{x,2}x-k_{z,2}z-\omega t)}.
\label{eqn:multieex}
\end{align}

Taking Equations~\ref{eqn:multihhx} and~\ref{eqn:multieex} and applying
continuity of $H_y$ and $E_x$ at the boundaries, the dispersion relation is
expressed as
\begin{equation}
\me^{-4 \mi k_{x,1} a} =
\frac{k_{x,1}/\epsilon_1 + k_{x,2}/\epsilon_2}
     {k_{x,1}/\epsilon_1 - k_{x,2}/\epsilon_2}
\frac{k_{x,1}/\epsilon_1 + k_{x,3}/\epsilon_3}
     {k_{x,1}/\epsilon_1 - k_{x,3}/\epsilon_3}.
\label{eqn:lrsppdispersion}
\end{equation}

\Equation{eqn:lrsppdispersion} is transcendental, but it can still be
visualized by plotting the difference between the left-hand side (\textsc{lhs}) and
right-hand side (\textsc{rhs}) of the equation,
$\log\left(\Re\left(|\mathrm{\textsc{rhs}}-\mathrm{\textsc{lhs}}|\right)\right)$.  The result
is plotted in \Figure{fig:lrsppdispersionrelation}, with the negative
values representing regions where \gls{spp} modes are supported.

\begin{figure}[ht]
 \centering
\import{existence/figures/}{lrsppdispersionfig}
\caption{Dispersion relation for the multilayer interface system, shown as the difference
				between the right-hand side (\textsc{rhs}) and left-hand side (\textsc{lhs}) of
\Equation{eqn:lrsppdispersion}.  \gls{spp} excitation occurs in the red areas.
This specific system is also used in the experiment.  Plot is for a three-layer interface system with $\epsilon_1=1.331$, $\epsilon_2$ Au using the
frequency-dependent Lorentz-Drude model, $\epsilon_3=1.332$, and the
thickness of $D_2$ is \SI{15}{\nano\meter}.  }
\label{fig:lrsppdispersionrelation}
\end{figure}

\subsection{Photonic Crystal Structures}
The layer systems in Figures~\ref{fig:singleinterfacegeo}
and~\ref{fig:multilayergeo} are the most popular layer
configurations for \gls{spr} biosensing.  There is however a very intriguing
geometry based on a multilayer stack with which one can obtain \gls{spp}
propagation lengths in the \textit{millimeter} range.  A typical three-layer system in the Kretschmann \gls{atr} configuration will exhibit \gls{spp}
propagation lengths on the order of \SI{8}{\micro\meter} for gold.  The
symmetric configuration has been experimentally reported to increase the
propagation length by a factor of
$10$-$60$~\cite{kuwamura1983experimental}~\cite{craig1983experimental}, and
the photonic crystal structure has been reported to further increase the
propagation length by a factor of 100~\cite{konopsky2009long}.

Currently, much of the work on photonic crystal structures is being carried
out by \name{Konopsky}~\cite{konopsky2006long}~\cite{konopsky2009long}.
The working principle is akin to a Bragg mirror; a one-dimensional stack of
layers with varying refractive index is built-up on the surface of a prism,
ultimately depositing a metal as the final layer.  The multilayer stack
works to minimize the tangential component of the electric field in the
metal, reducing \gls{spp} attenuation.  Reduction of \gls{spp} attenuation has several
advantages to \gls{atr} configurations; specifically, it is possible to use lossy metals with
more interesting surface chemistries (Pd and Pt), and it becomes possible
to study \glspl{lrspp} with near field probes, which the
liquid phase would otherwise make difficult.
