
(Drude metals) The following figures plot the frequency dependent complex permittivity
$\epsilon = \epsilon' + \mi \epsilon''$ and refractive index $n = n' + \mi
n''$ for silver, aluminum, gold, copper, chromium, nickel, tungsten,
titanium, beryllium, palladium, and platinum using either the Drude
(\Equation{eqn:drude}),
Lorentz-Drude (\Equation{eqn:lorentzdrude}), or Brendel-Bormann
(\Equation{eqn:brendelbormann}  models in the range
\SIrange{200}{2000}{\nano\meter}.  The material parameters and mathematical
formalism detailed in \cite{rakik1998}.  These tables are generated
programmatically.
Three different models for the complex permittivity are tabulated.  The
first two are the Drude and Lorentz-Drude (LD) models.
\begin{align}
\epsilon_{D}&=\epsilon_D\\
\epsilon_{LD}&=\epsilon_D+\epsilon_L
\end{align}
where $\epsilon_D$ is contribution from the Drude model, representing
free electron effects
\begin{align}
\epsilon_D=1-\frac{\sqrt{f_0} \omega_p^{\prime 2}}{\omega(\omega -
\mi\Gamma_0^\prime)}
\label{eqn:drude}
\end{align}
and $\epsilon_L$ is the Lorentz contribution, representing the bound
electron effects
\begin{align}
\epsilon_L =\sum_{j=0}^{k}\frac{f_j\omega_p^{\prime 2}}{\omega_j^{\prime
2}-\omega^2+\mi\omega\Gamma_j^\prime}
\label{eqn:lorentzdrude}
\end{align}
The third is the Brendel-Bormann model which is based instead on an
infinite superposition of oscillators
\begin{align}
\epsilon_\text{BB} = \frac{1}{\sqrt{2 \pi} \sigma_n} \intinfty
\exp\left(-\frac{(x-\omega'_n)}{2 \sigma_n^2}\right)
\frac{f_j \omega_p^2}{(x^2-\omega^2)+\mi \omega \Gamma'_n} \md x
\label{eqn:brendelbormann}
\end{align}
