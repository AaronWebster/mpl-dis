In our derivation of the plane wave solutions to Maxwell's equations, we
have preformed a common trick regarding the permittivity.
Specifically, we have treated frequency dependent permittivity as complex
\begin{align}
 \epsilon(\omega) &= \epsilon^\prime + \mi \frac{\sigma}{\omega}\\
                   &= \epsilon^\prime + \mi\epsilon^{\prime\prime}
\end{align}
This allows the use of plane wave solutions which are essentially identical
to the solutions obtained for Maxwell's equations in the absence of
external fields: $\nabla \cdot \mathbf{D} = 0$ and $\mathbf{J} = 0$.  This
is justified if we can make an adiabatic approximation for the exciting
field viz. $\sigma\gg1$ and $\rho\approx 0$. 
\todo{Come back to this later.
Invoke argument from Alfred Leitner's lecture. $\nabla \cdot \mathrm{D} =
0$}
Physically, $\epsilon$ represents
both the polarizability in Equations \ref{eqn:pdensity} and
\ref{eqn:dfield} and the phase delay $\delta$ between $\mathbf{D}$ and
$\mathbf{E}$
\begin{equation}
 \epsilon(\omega) = \frac{\mathbf{D}}{\mathbf{E}} \,|\epsilon| \,\me^{\mi \delta}
\end{equation}
This way of treating the permittivity is also completely consistent with a complex
refractive index $\tilde{n} = n(1+\mi \kappa) = \sqrt{\epsilon}$.

There are several models~\cite{rakic1998optical} with which one
can obtain the frequency dependent complex permittivity for metals.
Theoretically, the most general expression simply describes
$\epsilon(\omega)$ as a superposition of Lorentzian resonances plus a DC
term
\begin{equation}
\epsilon(\omega)= \epsilon_\infty+\sum_n \frac{\sigma_n \omega_n^2} {\omega_n^2-\omega^2-{\mathrm{i}}\omega\Gamma_n}
\label{eqn:meepdispersion}
\end{equation}
where $\epsilon_\infty$ is the instentaneous (DC) dielectric response,
$\sigma_\mathrm{D}$ is the electric conductivity, $\omega_n$ and $\Gamma_n$ are
material constants, and $\sigma_n$ is a function of position specifying the
strength of the $n$th resonance.  \Equation{eqn:meepdispersion} is
dimensionless ($c=\hbar=1$), and is usually encountered when working with
numerical simulations of electromagnetic fields~\cite{oskooi2010meep}.
In most other areas, the Lorentz-Drude model is used.  Formally, the Lorentz-Drude model
\begin{align}
\epsilon_\mathrm{D}(\omega)&=\epsilon_\mathrm{D}(\omega)\\
\epsilon_\mathrm{LD}(\omega)&=\epsilon_\mathrm{D}(\omega)+\epsilon_\mathrm{L}(\omega)
\end{align}
where $\epsilon_\mathrm{D}$ is contribution from the Drude model, representing
free electron effects
\begin{align}
\epsilon_\mathrm{D}(\omega)=1-\frac{\sqrt{f_0} \omega_p^{\prime 2}}{\omega(\omega -
\mi\Gamma_0^\prime)}
\label{eqn:drude}
\end{align}
and $\epsilon_L$ is the Lorentz contribution, representing the bound
electron effects
\begin{align}
\epsilon_L(\omega) =\sum_{j=0}^{k}\frac{f_j\omega_p^{\prime 2}}{\omega_j^{\prime
2}-\omega^2+\mi\omega\Gamma_j^\prime}
\label{eqn:lorentzdrude}
\end{align}
Here, $\omega'_p$ is the plasma frequency of an electron gas.
The third and perhaps more accurate is the Brendel-Bormann model, is based instead on an
\textit{infinite} superposition of oscillators
\begin{align}
\epsilon_\text{BB}(\omega) = \frac{1}{\sqrt{2 \pi} \sigma_n} \intinfty
\exp\left(-\frac{(x-\omega'_n)}{2 \sigma_n^2}\right)
\frac{f_j \omega_p^2}{(x^2-\omega^2)+\mi \omega \Gamma'_n} \md x
\label{eqn:brendelbormann}
\end{align}

Converting between \Equation{eqn:meepdispersion} and
Equations~\ref{eqn:drude}, \ref{eqn:lorentzdrude}, and
\ref{eqn:brendelbormann} is accomplished through the subsitutions $\omega_1
= \num{1e-20}$, $\sigma_n = {f_n \omega_p^{\prime 2}}/{\omega_n^2}$,
$\epsilon_\infty=1$ and $\Gamma_n^\prime$.

\begin{figure}[ht]
\centering
\import{includes/}{setpgfinc}
\import{existence/figures/}{permittivityau}
\caption{Comparison of the complex permittivity for gold calculated using the Drude
and the Lorentz Drude model.}
\label{fig:permittivityau}
\end{figure}

To get a feel for what a complex permittivity looks like, in
\Figure{fig:permittivityau} $\epsilon_\mathrm{LD}(\omega)$ and
$\epsilon_\mathrm{D}(\omega)$ are shown for gold, a metal which we will use
extensively in this work.  There are several important points to this plot.
First, note that the Lorentz-Drude model is much better at predicting high
frequency behavior.  Second, the real part is negative and the
imaginary part positive.  This is necessary; $\epsilon(\omega)$ is
\textit{casual} and as such obeys the Kramers-Kronig relations
\begin{align}
\epsilon^\prime(\omega)=\mi \hf{\epsilon^{\prime\prime}(\omega)}
\end{align}
where $\hf{\epsilon(\omega)}$ is the Hilbert transform of
$\epsilon(\omega)$.  In practice, 
the choice of model ultimately depends on how well it fits observation for
a particular experiment.  Here, we will use the Lorentz-Drude model almost
exclusively.
