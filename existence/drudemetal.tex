For metals, there are several models~\cite{rakik1998optical} with which one
can obtain the frequency dependent complex permittivity $\epsilon(\omega)$.
Theoretically, the most general expression simply describes
$\epsilon(\omega)$ as a superposition of (Lorentzian) resonances plus a DC
term
\begin{equation}
\epsilon(\omega)= \epsilon_\infty+\sum_n \frac{\sigma_n \omega_n^2} {\omega_n^2-\omega^2-{\mathrm{i}}\omega\Gamma_n}
\label{eqn:meepdispersion}
\end{equation}
where $\epsilon_\infty$ is the instentaneous (DC) dielectric response,
$\sigma_D$ is the electric conductivity, $\omega_n$ and $\Gamma_n$ are
material constants, and $\sigma_n$ is a function of position specifying the
strength of the $n$th resonance.  \Equation{eqn:meepdispersion} is
dimensionless ($c=\hbar=1$), and is usually encountered when working with
numerical simulations of electromagnetic fields~\cite{oskooi2010meep}.

In most other areas, the Lorentz-Drude model is used.  The Lorentz-Drude model is 
\begin{align}
\epsilon_{D}(\omega)&=\epsilon_D(\omega)\\
\epsilon_{LD}(\omega)&=\epsilon_D(\omega)+\epsilon_L(\omega)
\end{align}
where $\epsilon_D$ is contribution from the Drude model, representing
free electron effects
\begin{align}
\epsilon_D(\omega)=1-\frac{\sqrt{f_0} \omega_p^{\prime 2}}{\omega(\omega -
\mi\Gamma_0^\prime)}
\label{eqn:drude}
\end{align}
and $\epsilon_L$ is the Lorentz contribution, representing the bound
electron effects
\begin{align}
\epsilon_L(\omega) =\sum_{j=0}^{k}\frac{f_j\omega_p^{\prime 2}}{\omega_j^{\prime
2}-\omega^2+\mi\omega\Gamma_j^\prime}
\label{eqn:lorentzdrude}
\end{align}
Here, $\omega'_p$ is the plasma frequency of an electron gas.
The third and perhaps more accurate is the Brendel-Bormann model, is based instead on an
\textit{infinite} superposition of oscillators
\begin{align}
\epsilon_\text{BB}(\omega) = \frac{1}{\sqrt{2 \pi} \sigma_n} \intinfty
\exp\left(-\frac{(x-\omega'_n)}{2 \sigma_n^2}\right)
\frac{f_j \omega_p^2}{(x^2-\omega^2)+\mi \omega \Gamma'_n} \md x
\label{eqn:brendelbormann}
\end{align}

Converting between \Equation{eqn:meepdispersion} and
Equations~\ref{eqn:drude}, \ref{eqn:lorentzdrude}, and
\ref{eqn:brendelbormann} is accomplished through the subsitutions $\omega_1
= \num{1e-20}$, $\sigma_n = {f_n \omega_p^{\prime 2}}/{\omega_n^2}$,
$\epsilon_\infty=1$ and $\Gamma_n^\prime$.
