In the derivation of the plane wave solutions to Maxwell's equations,
Equations~\ref{eqn:planewaves} and~\ref{eqn:planewavesh}, the
frequency dependent permittivity is a complex parameter
\begin{align}
 \epsilon(\omega) &= \epsilon^\prime + \mi \frac{\sigma}{\omega}\\
                   &= \epsilon^\prime + \mi\epsilon^{\prime\prime}.
																			\label{eqn:complexpermittivity}
\end{align}
%This allows the use of plane wave solutions which are essentially identical
%to the solutions obtained for Maxwell's equations in the absence of
%external fields: $\nabla \cdot \mathbf{D} = 0$ and $\mathbf{J} = 0$.  This
%is justified if we can make an adiabatic approximation for the exciting
%field viz. $\sigma\gg1$ and $\rho\approx 0$. 
%\todo{Come back to this later.  Invoke argument from Alfred Leitner's lecture. $\nabla \cdot \mathrm{D} = 0$}
Physically, $\epsilon$ represents both the polarizability in
Equations~\ref{eqn:pdensity} and~\ref{eqn:dfield} and the phase delay
$\delta$ between the scalar $D_0$ and
$E_0$
\begin{equation}
 \epsilon(\omega) = \frac{D_0}{E_0} \,|\epsilon| \,\me^{\mi \delta}.
	\label{eqn:phasedelay}
\end{equation}
%\todo{can't divide vector by vector}
Equations~\ref{eqn:complexpermittivity} and~\ref{eqn:phasedelay} are
consistent with a complex refractive index $\tilde{n} = n(1+\mi \kappa) =
\sqrt{\epsilon}$.

There are several models~\cite{rakic1998optical} from which
the frequency dependent complex permittivity for metals can be obtained.
As such, it is relevant to be able to convert from one model to another.
Theoretically, the most general expression simply describes
$\epsilon(\omega)$ as a superposition of Lorentzian resonances plus a DC
term,
\begin{equation}
\epsilon(\omega)= \epsilon_\infty+\sum_n \frac{\sigma_n \omega_n^2} {\omega_n^2-\omega^2-{\mathrm{i}}\omega\Gamma_n}.
\label{eqn:meepdispersion}
\end{equation}

In \Equation{eqn:meepdispersion} $\epsilon_\infty$ is the low frequency
dielectric response, describing the material's response to a static (DC)
electric field, $\sigma_\mathrm{n}$ is the electric conductivity,
$\omega_n$ and $\Gamma_n$ are material constants, and $\sigma_n$ specifies
the strength of the $n$th resonance.  \Equation{eqn:meepdispersion} is
dimensionless ($c=\hbar=1$), and is usually encountered when working with
numerical simulations of electromagnetic fields~\cite{oskooi2010meep}.  In
most other areas, the Lorentz-Drude model is used.  Formally, the
Lorentz-Drude model is given by
\begin{equation}
\epsilon_\mathrm{LD}(\omega)=\epsilon_\mathrm{D}(\omega)+\epsilon_\mathrm{L}(\omega),
\end{equation}
where $\epsilon_\mathrm{D}$ is contribution from the Drude model, representing
free electron effects
\begin{align}
\epsilon_\mathrm{D}(\omega)=1-\frac{\sqrt{f_0} \omega_p^{\prime 2}}{\omega(\omega -
\mi\Gamma_0^\prime)},
\label{eqn:drude}
\end{align}
and $\epsilon_L$ is the Lorentz contribution, representing the bound
electron effects,
\begin{align}
\epsilon_L(\omega) =\sum_{j=0}^{k}\frac{f_j\omega_p^{\prime 2}}{\omega_j^{\prime
2}-\omega^2+\mi\omega\Gamma_j^\prime}.
\label{eqn:lorentzdrude}
\end{align}
In \Equation{eqn:lorentzdrude}, $\omega'_p$ is the plasma frequency of an
electron gas.  The third and perhaps more
accurate~\cite{jahanshahi2014study} model is the
Brendel-Bormann model, based instead on an infinite superposition
of oscillators
\begin{align}
\epsilon_\text{BB}(\omega) = \frac{1}{\sqrt{2 \pi} \sigma_n} \intinfty
\exp\left(-\frac{(x-\omega'_n)}{2 \sigma_n^2}\right)
\frac{f_j \omega_p^2}{(x^2-\omega^2)+\mi \omega \Gamma'_n} \md x.
\label{eqn:brendelbormann}
\end{align}
Converting between \Equation{eqn:meepdispersion} and
Equations~\ref{eqn:drude},~\ref{eqn:lorentzdrude}, and~\ref{eqn:brendelbormann} is accomplished through the substitutions $\sigma_n
= {f_n \omega_p^{\prime 2}}/{\omega_n^2}$, $\epsilon_\infty=1$ and
$\Gamma_n^\prime$.  To prevent numerical instability, in computer codes
$\omega_0$, the DC term, should be to be a very small but nonzero value,
e.g.  $\omega_1 = \num{1e-20}$. 

The Drude model considers only free electrons and thus has only a single
pole in its expression, limiting its accuracy.  The agreement with
experiment is particularly bad for metals such as gold and
silver~\cite{ahmedcomputational}~\cite{jahanshahi2014study}, with
deviations of approximately \SI{50}{\percent} in the real part and
\SI{150}{\percent} in the imaginary up to \SI{650}{\micro\meter}.  At
higher frequencies the Drude model continues to diverge and is not
applicable.  Furthermore, the higher error in the imaginary part of
$\epsilon$ gives narrower SPR resonances than are experimentally
observed~\cite{jahanshahi2014study}.  The simulations in this work all use
the Lorentz-Drude model, which has an accuracy of () within the (range of
frequencies I will use and stuff).  A comparison between the models and
experiment is shown in \Figure{fig:permittivityau} of
\Section{ch:reference}.
