In the case of no exernal charge or current densities, $\rho=\mathbf{J}=0$,
the propagation of disturbences in the electromagnetic field can be derived
as plane wave solutions to Maxwell's equations.  Beginning with Faraday's law
\begin{align}
\nabla \times \mathbf{E} &= -\frac{\partial \mathbf{B}} {\partial t}
\end{align}
and taking the curl of both sides
\begin{align}
\nabla \times
\left(\nabla\times\mathbf{E}\right)&=\nabla\times\left(-\frac{\partial\mathbf{B}}{\partial
t}\right)\\
&=-\frac{\partial}{\partial t}\left(\nabla \times \mathbf{B}\right)
\end{align}
The left side can be expanded and simplified using the vector identity
\begin{align}
\nabla \times \left( \nabla \times \mathbf{A} \right) = \nabla \left(
\nabla \cdot \mathbf{A} \right) - \nabla^2 \mathbf{A}
\end{align}
%The right hand side is modified using Ampere's law with $\mathbf{B} =\mu
%\mathbf{H}$.
\begin{align}
\nabla\left(\nabla\cdot\mathbf{E}\right)-\nabla^2\mathbf{E}
&=-\mu\frac{\partial}{\partial t}\left(\nabla \times \mathbf{H}\right)\\
-\nabla^2\mathbf{E}&=-\mu\frac{\partial}{\partial t}\left( \frac{\partial \mathbf{D}}{\partial t} + \mathbf{J}\right) \\
-\nabla^2\mathbf{E}&=-\mu\frac{\partial^2 \mathbf{D}}{\partial t^2} -\mu\frac{\partial\mathbf{J}}{\partial t} \\
\nabla^2\mathbf{E} -\mu\epsilon\frac{\partial^2 \mathbf{E}}{\partial t^2} -\mu\sigma\frac{\partial\mathbf{E}}{\partial t}  &= 0 
\label{eqn:ewe}
\end{align}
This is the electromagnetic wave equation in terms of the electric field.
It is solved by separation of varaibles using a plane wave
ansatz which yields
\begin{align}
%\Aboxed{
 \mathbf{E} ( \mathbf{r}, t ) &= \mathbf{E}_0\, \me^{\mi (\mathbf{k}
 \cdot \mathbf{r} - \omega t )}
%}
\label{eqn:planewaves}
\end{align}
where
\begin{equation}
 \mathbf{k}^2 = \mu \epsilon \omega^2 + \mi \mu \epsilon \omega = \omega^2 \mu \left(\epsilon + \mi \frac{\sigma}{\omega}\right)
 \label{eqn:planewavedispersion}
\end{equation}
and $k=\omega/c=\omega\sqrt{\epsilon\mu}$.  In this notation,
$\mathbf{k}$ is the material specific vectorial wavenumber, $\mathbf{r}$ is the
spatial position, $\omega$ is angular frequency, and $t$ is
the dimension of time.
The initial value is chosen with the vectorial constant $\mathbf{E}_0$.
The magnetic field follows the same form
\begin{align}
%\Aboxed{
 \mathbf{H} ( \mathbf{r}, t ) &= \mathbf{H}_0\, \me^{\mi (\mathbf{k}
 \cdot \mathbf{r} - \omega t )}
%}
\end{align}
where $\mathbf{H}$ is it is orthogonal $\mathbf{E}$ by 
\begin{align}
\mathbf{E} \times \mathbf{H} = 0
\end{align}
and two are mutually orthogonal with respect to $\mathbf{k}$.
