\begin{figure}[ht]
\centering
\begin{tikzpicture}
\begin{axis}[
xlabel=$k_x$,
ylabel=$\omega$,
xmin=0,
xmax=0.6e8,
ymin=0.75e15,
ymax=6.5e15,
thick,
width=0.75\textwidth,
height=0.75\textwidth,
minor tick num=2,
legend pos = south east,
legend style = { cells = {anchor=west} },
clip=false,
]
\addplot[color=black] file {existence/figures/sppdielectric.dat};
\addlegendentry{SPP/dielectric}

\addplot[color=black] file {existence/figures/photondielectric.dat}
node [anchor=south west,yshift=-10pt]{$ck/\sqrt{\epsilon_1}$}; 
\addlegendentry{photon/dielectric}

\addplot[color=black] file {existence/figures/photondielectrictilted.dat}
node [anchor=south west,yshift=-10pt]{$ck \sin \theta /\sqrt{\epsilon_1}$};
\addlegendentry{photon/dielectric, angle}


%\addplot[color=colorg] file {dispersion/kx_photon_glass_tilted.dat}
%node [anchor=south west,yshift=-10pt]{$ck \sin \theta /\sqrt{\epsilon_1}$};
%\addlegendentry{photon/dielectric angle}
%
%\addplot[color=colorc] file {dispersion/kx_photon_vacuum.dat}
%node [anchor=south west,yshift=-10pt]{$ck$}; 
%\addlegendentry{photon/vacuum}
%
%\addplot[color=colord] file {dispersion/kx_sp_metal_glass_lower.dat};
%\addlegendentry{SPP/dielectric}
%
%\addplot[color=colore] file {dispersion/kx_sp_metal_vacuum_lower.dat};
%\addlegendentry{SPP/vacuum}
%
%\addplot[color=colord] file {dispersion/kx_sp_metal_glass_upper.dat};
%\addplot[color=colore] file {dispersion/kx_sp_metal_vacuum_upper.dat};
%
\addplot[color=gray,dashed] coordinates { (6e7,5.5925e+15) (0,5.5925e+15) }
node [anchor=east,xshift=-15pt]{$\omega_p$}; 
%
\addplot[color=gray,dashed] coordinates { (6e7,4936476840790959) (0,4936476840790959) }
node [anchor=east,xshift=-15pt]{$\omega_p/\sqrt{1+\epsilon_2}$}; 

\draw [decorate,decoration={brace,amplitude=8pt}] (axis cs:6e7,6.5e+15)--(axis cs:6e7,5.819396016587478e+15)
node [midway,anchor=west,xshift=10pt,align=left]{radiative modes\\ $k_x,k_z \in \mathbb{R}$};

\draw [decorate,decoration={brace,amplitude=8pt}] (axis cs:6e7,5.819396016587478e+15)--(axis cs:6e7,4959276840790959)
node [midway,anchor=west,xshift=10pt,align=left]{quasi-bound modes\\ $k_x \in \mathbb{I}$,  $k_z \in \mathbb{R}$};

\draw [decorate,decoration={brace,amplitude=8pt}] (axis cs:6e7,4959276840790959)--(axis cs:6e7,0.75e+15)
node [midway,anchor=west,xshift=10pt,align=left]{bound modes\\ $k_x \in \mathbb{R}$\\ $k_z \in \mathbb{I}$};

\end{axis}
\end{tikzpicture}
\caption{Dispersion relation for photons and plasmons.  Conditions under
which SPPs may occur at the intersection between the photon light line and
corresponding SPP light line for the medium in question.  Based on a
similar plot found in \cite{shsongspp}.}
\label{fig:dispersionrelation}
\end{figure}
