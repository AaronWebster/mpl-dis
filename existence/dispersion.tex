\input{dispersion/kretschmann}
With plane wave solutions in hand, boundary conditions consistent with a
dielectric-metal interface can be imposed.
At a planar interface, it is convenient to restrict the problem to two
dimensions; in this case the $x$-$z$ plane as shown schematically in
\Figure{fig:kretschmanngeosimplified}.  Since $\mathbf{k}=(k_x,0,k_z)$,
$\mathbf{r}\cdot\mathbf{k}=k_x x + k_z z$ and $\mathbf{E}_0 = (E_x, 0,
E_z)$, the electric field can be written 
\begin{align}
\mathbf{E} ( \mathbf{r}, t ) &= \mathbf{E}_0\, \me^{\mi (\mathbf{k}
\cdot \mathbf{r} - \omega t )}\\
\mathbf{E}(x,z,t)&=\begin{pmatrix}
E_x\\ 0\\ E_z
\end{pmatrix}
\, \me^{\mi(k_{x,i}x+k_{z,i}z-\omega t)}
\label{eqn:planewavexz}
\end{align}
The magnetic field propagates in the same direction with the
same $\mathbf{k}$-vector components, but the direction
$\mathbf{H}$ must be orthogonal to $\mathbf{E}$ by
\Equation{eqn:faradayslaw}
\begin{align}
\mathbf{H} ( \mathbf{r}, t ) &= \mathbf{H}_0\, \me^{\mi (\mathbf{k}
\cdot \mathbf{r} - \omega t )}\\
\mathbf{H}(x,z,t)&=\begin{pmatrix}
0\\ H_y\\ 0
\end{pmatrix}
\, \me^{\mi(k_{x,i}x+k_{z,i}z-\omega t)}
\end{align}
At the interface there are two values for the (complex) permittivity,
$\epsilon_1$ in the dielectric and $\epsilon_2$ in the metal.  Consequently
there are two sets of plane wave solutions
\begin{align}
\left.\begin{aligned}
\mathbf{H}_1(x,z,t) &=
\begin{pmatrix}
0\\
H_{y,1}\\
0
\end{pmatrix} \me^{\mi(k_{x,1}x+k_{z,1}z-\omega t)}\\
\mathbf{E}_1(x,z,t) &=
\begin{pmatrix}
E_{x,1}\\
0\\
E_{z,1}\\
\end{pmatrix} \me^{\mi(k_{x,1}x+k_{z,1}z-\omega t)}
\end{aligned}
\right\}& \quad \text{dielectric}, \epsilon_1\label{eqn:planewavedielectric}\\
\left.\begin{aligned}
\mathbf{H}_2(x,z,t) &=
\begin{pmatrix}
0\\
H_{y,2}\\
0
\end{pmatrix}
\me^{\mi(k_{x,2}x+k_{z,2}z-\omega t)}\\
\mathbf{E}_2(x,z,t) &=
\begin{pmatrix}
E_{x,2}\\
0\\
E_{z,2}\\
\end{pmatrix}
\me^{\mi(k_{x,2}x+k_{z,2}z-\omega t)}
\end{aligned} 
\right\}&\quad \text{metal}, \epsilon_2
\label{eqn:planewavemetal}
\end{align}
where the subscript designates which material the wave equation refers to.
(e.g. $\mathbf{E}_1$ is the electric field in the dielectric, $\mathbf{H}_2$
the magnetic field in the metal, etc.)
Continuity of \Equation{eqn:planewavedielectric} and
\ref{eqn:planewavemetal} at this interface requires that
\begin{align}
E_{x,2}&=E_{x,1}\\
H_{y,2}&=H_{y,1}\\
\epsilon_2 E_{z,2}&=\epsilon_1 E_{z,1}
\end{align}
Applying Ampere's law (\Equation{eqn:ampereslaw}) to the field on the
each boundary gives
\begin{align}
\nabla \times \mathbf{H}_i &= \epsilon_i \frac{\partial \mathbf{E}_i}{\partial t}\\
\begin{pmatrix}
\frac{\partial H_{z,i}}{\partial y} - \frac{\partial H_{y,i}}{\partial z}\\
\frac{\partial H_{x,i}}{\partial y} - \frac{\partial H_{z,i}}{\partial z}\\
\frac{\partial H_{y,i}}{\partial y} - \frac{\partial H_{x,i}}{\partial z}
\end{pmatrix}
&= \begin{pmatrix}
-\mi k_{z,i} H_{y,i}\\
0\\
\mi k_{x,i} H_{y,i}
\end{pmatrix}
\\
&= \begin{pmatrix}
-\mi \omega \epsilon_i E_{x,i}\\
0\\
-\mi \omega \epsilon_i E_{z,i}
\end{pmatrix}
\label{eqn:vectordisp}
\end{align}
where $\mathbf{E}_i$, $\mathbf{H}_i$, $i=1,2$ represent the field in either the
dielectric or the metal.  The vector components of
\Equation{eqn:vectordisp} are therefore related 
\begin{align}
-\mi k_{z,i} H_{y,i} &= -\mi \omega \epsilon_i E_{x,i}\\
k_{z,1} H_{y,1} &= \omega \epsilon_1 E_{x,1}\\
k_{z,2} H_{y,2} &= \omega \epsilon_2 E_{x,2}
\label{eqn:spderivsteptwo}
\end{align}
Since the components of both $\mathbf{E}_i$ and $\mathbf{H}_i$ are
parallel to the interface, $E_{x,i}$ and $H_{y,i}$ are also
continuous. By substitution of $E_{x,i}$, \Equation{eqn:spderivsteptwo} becomes
\begin{align}
\frac{k_{z,1}}{\epsilon_1}H_{y,1}&=\frac{k_{z,2}}{\epsilon_2}H_{y,2}\\ 
\Aboxed{
\frac{k_{z,1}}{\epsilon_1}&=\frac{k_{z,2}}{\epsilon_2} 
}
\label{eqn:sprcondition}
\end{align}
This is the surface plasmon resonance condition.  In terms of
its vector components, the following holds in general for all electromagnetic
waves
\begin{align}
\mathbf{k}^2=\epsilon_i \left(\frac{\omega}{c}\right)^2=k_x^2 + k_{z,i}^2\\
\epsilon_i k_0^2=k_x^2 + k_{z,i}^2
\label{eqn:dispersion1}
\end{align}
Substitution of \Equation{eqn:sprcondition} with the relation 
$k_{x,1}=k_{x,2}$ into \Equation{eqn:dispersion1} allows 
$k_x$ and $k_{z,i}$ to be rewritten in the form of a dispersion relation
\begin{align}
k_x &= k_0\sqrt{\frac{\epsilon_1 \epsilon_2}{\epsilon_1+\epsilon_2}} 
= \frac{\omega}{c}\sqrt{\frac{\epsilon_1 \epsilon_2}{\epsilon_1+\epsilon_2}}\\
k_{z,i} &= k_0\frac{\epsilon_i}{\sqrt{\epsilon_1+\epsilon_2}}
= \frac{\omega}{c}\frac{\epsilon_i}{\sqrt{\epsilon_1+\epsilon_2}}
\end{align}

This relation, shown in \Figure{fig:dispersionrelation}, is useful because
it shows graphically the condition under which SPPs may exist: at the
intersection between the photon light line and corresponding SPP light line
for the medium in question.  Note that for a photon and a plasmon in both a
vacuum and a dielectric, this never happens; the photon assumes $\omega = c
k /\sqrt{\epsilon_i}$ for $i=1,2$, diverging to infinity while the SPP
asymptotically approaches $\omega_p/\sqrt{1+\epsilon_i}$ as $k_x\to\infty$.
However, if light is incident from a dielectric $\epsilon_1$ at an angle
$\theta$, the slope of $\omega(k_x)$ is modified to $c k \sin
\theta/\sqrt{\epsilon_1}$ and the dispersion relations can be matched.
This is the principle exploited by Kretschmann \cite{kretschmann1968} to
excite SPPs with prisms.

There are several regions of interest in \Figure{fig:dispersionrelation},
depending on the relative value of $\epsilon_1$ and $\epsilon_2$.  We first
assume that $\epsilon_1$ is a dielectric with $\epsilon_1\in\mathbb{R}$ and
$\epsilon_1 > 0$ (true for most if not all glass).  In this case, for
$\Re(\epsilon_2)>0$, both $k_x$ and $k_z$ are real.  These are known as
``radiative modes'' -- an SPP mode is supported normal to the interface but
it is not bound there.  For $-\epsilon_1<\Re(\epsilon_2)<0$, $k_z$ is real
and $k_x$ imaginary.  The SPP is not confined to the interface and decays
evanescently there; this is known as a ``quasi-bound mode''  However, for
$\Re(\epsilon_2)<-\epsilon_1$, $k_x$ is real and $k_z$ is imaginary.  This
indicates the SPP is localized at the interface in $z$ while having a
propagating solution in $x$, and is exactly the condition we wish to match.
\begin{figure}[ht]
\centering
\begin{tikzpicture}
\begin{axis}[
xlabel=$k_x$,
ylabel=$\omega$,
xmin=0,
xmax=0.6e8,
ymin=0.75e15,
ymax=6.5e15,
thick,
width=0.75\textwidth,
height=0.75\textwidth,
minor tick num=2,
legend pos = south east,
legend style = { cells = {anchor=west} },
clip=false,
]
\addplot[color=colora] file {dispersion/kx_photon_glass.dat}
node [anchor=south west,yshift=-10pt]{$ck/\sqrt{\epsilon_1}$}; 
\addlegendentry{photon/dielectric}

\addplot[color=colorg] file {dispersion/kx_photon_glass_tilted.dat}
node [anchor=south west,yshift=-10pt]{$ck \sin \theta /\sqrt{\epsilon_1}$};
\addlegendentry{photon/dielectric angle}

\addplot[color=colorc] file {dispersion/kx_photon_vacuum.dat}
node [anchor=south west,yshift=-10pt]{$ck$}; 
\addlegendentry{photon/vacuum}

\addplot[color=colord] file {dispersion/kx_sp_metal_glass_lower.dat};
\addlegendentry{SPP/dielectric}

\addplot[color=colore] file {dispersion/kx_sp_metal_vacuum_lower.dat};
\addlegendentry{SPP/vacuum}

\addplot[color=colord] file {dispersion/kx_sp_metal_glass_upper.dat};
\addplot[color=colore] file {dispersion/kx_sp_metal_vacuum_upper.dat};

\addplot[color=colore,dashed] coordinates { (6e7,5.819396016587478e+15) (0,5.819396016587478e+15) }
node [anchor=east,xshift=-15pt]{$\omega_p$}; 

\addplot[color=colore,dashed] coordinates { (6e7,5.461458222933e+15) (0,5.461458222933e+15) }
node [anchor=east,xshift=-15pt]{$\omega_p/\sqrt{2}$}; 

\addplot[color=colord,dashed] coordinates { (6e7,4959276840790959) (0,4959276840790959) }
node [anchor=east,xshift=-15pt]{$\omega_p/\sqrt{1+\epsilon_2}$}; 


\draw [decorate,decoration={brace,amplitude=8pt}] (axis cs:6e7,6.5e+15)--(axis cs:6e7,5.819396016587478e+15)
node [midway,anchor=west,xshift=10pt,align=left]{radiative modes\\ $k_x,k_z \in \mathbb{R}$};

\draw [decorate,decoration={brace,amplitude=8pt}] (axis cs:6e7,5.819396016587478e+15)--(axis cs:6e7,4959276840790959)
node [midway,anchor=west,xshift=10pt,align=left]{quasi-bound modes\\ $k_x \in \mathbb{I}$,  $k_z \in \mathbb{R}$};

\draw [decorate,decoration={brace,amplitude=8pt}] (axis cs:6e7,4959276840790959)--(axis cs:6e7,0.75e+15)
node [midway,anchor=west,xshift=10pt,align=left]{bound modes\\ $k_x \in \mathbb{R}$\\ $k_z \in \mathbb{I}$};

\end{axis}
\end{tikzpicture}
\caption{Dispersion relation for photons and plasmons.  Conditions under
which SPPs may occur at the intersection between the photon light line and
corresponding SPP light line for the medium in question.  Based on a
similar plot found in \cite{shsongspp}.}
\label{fig:dispersionrelation}
\end{figure}
%The dispersion diagram relates the time-variation of the wave (given by its
%frequency &omega) to the spatial variation of the wave (given by its
%wave-vector kx)
