\documentclass[a4paper]{article}
\usepackage{amsmath}
\usepackage{graphicx} % required for pdf transparency
\usepackage{xcolor} % required for pdf transparency
\usepackage{amsthm}
\usepackage{amssymb}
\usepackage{booktabs}
\usepackage{tabularx}
\usepackage{subfigure}
\usepackage{mathrsfs}
\usepackage{textcomp}
\usepackage{siunitx}
\usepackage{fullpage}
\usepackage[multiple]{footmisc}
\usepackage{hyperref}
\usepackage{multirow}
\usepackage{empheq}
\usepackage{calc}
\usepackage[version=3]{mhchem}
\usepackage{tikz}
\usepackage{pgfplots}
\usepackage{xcolor}
\usepackage[section]{placeins}
\pgfplotsset{compat=newest}


%\hypersetup{
% colorlinks=false,
% linkbordercolor={1 1 1},
% citebordercolor={1 1 1},
% urlbordercolor={1 1 1} 
%}

\sisetup{ 
% load-configurations=abbreviations
% round-mode = places,
}%

% New definition of square root:
% it renames \sqrt as \oldsqrt
% This definition puts a little vertical guy at the end so it's more
% obvious where the square root actually ends.
\let\oldsqrt\sqrt
% it defines the new \sqrt in terms of the old one
\def\sqrt{\mathpalette\DHLhksqrt}
\def\DHLhksqrt#1#2{%
\setbox0=\hbox{$#1\oldsqrt{#2\,}$}\dimen0=\ht0
\advance\dimen0-0.2\ht0
\setbox2=\hbox{\vrule height\ht0 depth -\dimen0}%
{\box0\lower0.4pt\box2}}

% integrals with infinity bounds
\newcommand{\intinfty}{\int_{-\infty}^{\infty}}

% consistent formatting of object labels
\newcommand{\Figure}[1]{Figure \ref{#1}}
\newcommand{\Equation}[1]{Equation \ref{#1}}
\newcommand{\Table}[1]{Table \ref{#1}}
\newcommand{\Section}[1]{Section \ref{#1}}
\newcommand{\Chapter}[1]{Chapter \ref{#1}}
\newcommand{\Appendix}[1]{Appendix \ref{#1}}

% missing mathematical operators
\DeclareMathOperator{\sinc}{sinc}
\DeclareMathOperator{\sech}{sech}
\DeclareMathOperator{\sgn}{sgn}
\DeclareMathOperator{\erf}{erf}
\DeclareMathOperator{\inverf}{inverf}
\DeclareMathOperator{\arcsinh}{arcsinh}
\DeclareMathOperator{\arccosh}{arccosh}
\DeclareMathOperator{\arctanh}{arctanh}
%\DeclareMathOperator{\Re}{Re}
%\DeclareMathOperator{\Im}{Im}

% use roman type for natural base e and sqrt(-1)
\newcommand{\me}{{\mathrm{e}}}
\newcommand{\mi}{{\mathrm{i}}}

% roman type for the derivative, plus a space
\newcommand{\md}{\,\mathrm{d}}

% fourier transform and the reverse
\newcommand{\ff}[1]{{\mathscr{F}^{+}\bigl(#1\bigr)}}
\newcommand{\fr}[1]{{\mathscr{F}^{-}\bigl(#1\bigr)}}

% hilbert transform and the reverse
\newcommand{\hf}[1]{{\mathscr{H}^{+}\bigl(#1\bigr)}}
\newcommand{\hr}[1]{{\mathscr{H}^{-}\bigl(#1\bigr)}}

% custom lengths for figures

% width for side by side figures
\newlength{\twoupwidth}
\setlength{\twoupwidth}{7.5cm}

% width and height for default single figure
\newlength{\oneupwidth}
\setlength{\oneupwidth}{0.90\textwidth}
\newlength{\oneupheight}
\setlength{\oneupheight}{0.55623059\textwidth}

% custom colors for 2D plots - these are the same ones mathematica uses by
% default
\definecolor{colora}{RGB}{63,61,153}
\definecolor{colorb}{RGB}{153,61,113}
\definecolor{colorc}{RGB}{153,139,61}
\definecolor{colord}{RGB}{61,153,86}
\definecolor{colore}{RGB}{61,90,153}
\definecolor{colorf}{RGB}{153,61,144}
\definecolor{colorg}{RGB}{153,109,61}
\definecolor{colorh}{RGB}{67,153,61}
\definecolor{colori}{RGB}{61,121,153}
\definecolor{colorj}{RGB}{132,61,153}

% hilight text... a "work in progress" type thing
\tikzstyle{todobox} = [draw=red, fill=blue!10, very thick,
rectangle, rounded corners, inner sep=10pt, inner ysep=20pt]
\tikzstyle{todotitle} =[fill=red, text=white]

\newcommand{\todo}[1]{%
\begin{center}
\begin{tikzpicture}
\node [todobox] (box){%
 \begin{minipage}{0.9\textwidth}
 #1
 \end{minipage}
};
\node[todotitle, right=10pt] at (box.north west) {To Do:};
\end{tikzpicture}
\end{center}

}

\pgfplotsset{
 /pgfplots/colormap={jet}{rgb255(0cm)=(0,0,128) rgb255(1cm)=(0,0,255)
 rgb255(3cm)=(0,255,255) rgb255(5cm)=(255,255,0) rgb255(7cm)=(255,0,0)
 rgb255(8cm)=(128,0,0)}
}
\usepgfplotslibrary{units}
\usetikzlibrary{pgfplots.units} 


\begin{document}
\title{What We Know About QCM Centrifuge Measurements}
\date{Last Update: \today}
\maketitle
\tableofcontents

\section{Equivalent Circuit}
The typical circuit used to analyze QCM behavior is called the
\textit{Butterworth van Dyke} (BvD) circuit.  It consists of a capacitor
$C_s$, an inductor $L_s$, and a resistor $R_s$ in series with a parallel
capacitance $C_0$.
\begin{center}
 \begin{circuitikz}[scale=0.75]
 \draw (1,0) node[anchor=east] {}
  to[short, o-*] (2,0)
  to[short] (2,1)
  to[R, l^=$R_s$] (4,1)
  to[C, l^=$C_s$] (6,1)
  to[L, l^=$L_s$] (8,1)
  to[short] (8,-1)
  to[short] (6,-1)
  to[C, l^=$C_0$] (4,-1)
  to[short] (2,-1)
  to[short] (2,0);
 \draw (9,0) node[anchor=west] {}
  to[short, o-*] (8,0);
\end{circuitikz}
\end{center}

The top branch is the \textit{motional branch}, and relates to the crysal
and its interaction with the environment.  The bottom is the \textit{static
branch}, representing the paracitic capitances of the quartz and its driver.
In the SRS QCM200\cite{srsqcmmanual}, and probably any other similar
compensated phase locked oscillator circuit, $C_0$ is nulled with
additional circuitry.  This is absolutely crutial, as the parallel $C_0$ 
pertdubes the resonance frequency of the circuit by about
\SI{0.825}{\hertz\per\pico\farad}.  The SRS manual gives a higher value
of \SI{2}{\hertz\per\pico\farad}.

Typical values for the \SI{1}{in} \SI{5}{\mega\hertz} AT cut quartz
crystal used in with the QCM200\cite{srsqcmmanual} are

\begin{table}[h]
\begin{tabular}{ll}
 $R_s$ & \SI{400}{\ohm} (water), \SI{10}{\ohm} (air) \\
 $C_s$ & \SI{33}{\femto\farad} (SRS manual)\\
 $L_s$ & \SI{30}{\milli\henry} (SRS manual), \SI{40}{\milli\henry} (my prediction) \\
 $C_0$ & \SI{20}{\pico\farad} (SRS manual)
\end{tabular}
\end{table}

The circuit may be also be solved using the following second order linear
differential equation for charge
\begin{equation}
 L\ddot{q}+R\dot{q}+q/C = V(t)
\end{equation}
The natural frequency is 
\begin{equation}
 f_0 = \frac{1}{2 \pi L C}
\end{equation}
Where $C$ in the above equation takes into account both $C_0$ and $C_s$
\begin{equation}
 C = \frac{C_s C_0}{C_s + C_0}
\end{equation}

This, as well as more complicated equivalent circuit models can also be
computed directly using SPICE, as per the following code snippet
\begin{minted}{text}
* Butterworth van Dyke equivalent circuit
V1 0 1 ac 1 dc 0
Rs 1 2 100
R1 2 3 375
C0 2 0 20e-12
C1 3 4 33.0e-15
L1 4 0 30e-3
.control
ac lin 100000 5.052e6 5.063e6
write bvd.raw all
\end{minted}

Liquid sensing under a rigid coating is modeled with a modified Butterworth
van Dyke circuit as follows 

\begin{center}
 \begin{circuitikz}[scale=0.75]
 \draw (1,0) node[anchor=east] {}
  to[short, o-*] (2,0)
  to[short] (2,1)
  to[R, l^=$R_\text{s}$] (4,1)
  to[C, l^=$C_\text{s}$] (6,1)
  to[L, l^=$L_\text{s}$] (8,1)
  to[L, l^=$L_\text{c}$] (10,1)
  to[L, l^=$L_\text{l}$] (12,1)
  to[R, l^=$R_\text{l}$] (14,1)
  to[short] (14,0);
  \draw (14,0) node[anchor=west] {}
  to[short] (14,-1)
  to[C, l^=$C_0$] (8,-1)
  to[short] (2,-1);
 \draw (14,0) node[anchor=west] {}
  to[short] (14,-3)
  to[C, l^=$C_\text{l}$] (6,-3)
  to[short] (2,-3)
  to[short] (2,0);

 %\draw (9,0) node[anchor=west] {}
 % to[short, o-*] (14,0);
\end{circuitikz}
\end{center}


\subsection{Effect of Resonance on Changing the Calues of the Elements}
From \Equation{eqn:bvdresponse} the following relationships between $f$ and
the individual circuit elements are clear
\begin{itemize}
 \item As $R_s$ goes up, $f$ is unchanged, but the linewidth broadens.
 \item As $C_s$ goes up, $f$ \textit{decreases} and vice versa.
 \item As $L_s$ goes up, $f$ \textit{decreases} and vice versa. 
 \item As $C_0$ goes up, $f$ \textit{increases}.
\end{itemize}

\section{Measurements of Frequency and Resistance}
The QCM200 measures both absolute and relative frequency, $f$, and
resistance, $R$.  Unfortunately, we only have absolute resistance
measurements so the frequencies are relative through each run.

\section{Cause For Different Types of Signals}
A great review of the response of a QCM to different phenomena is found in \cite{walls1995fundamental}

Quantity[-2970, (("Centimeters")^2 "Hertz")/("Kilograms")]*
 Quantity[0.124, "Grams"]/(1 Quantity[1.885`*^8, ("Microns")^2])*100
%%%%%%%%%%%%%%%%%%
Essentialy you should just make a table: name, pred. shift, our shift
%%%%%%%%%%%%%%%%%%
\subsection{Electric Fields}
\begin{quote}
For a resonator to exhibit a sensitivity, the dc electric field must have a
component along the axis of the crystal. True AT-cuts with regular electrodes
therefore exhibit little, if any, sensitivity to electric fields. \cite{walls1995fundamental}
\end{quote}
\subsection{Magnetic Fields}
\begin{quote}
Quartz resonators’ inherent magnetic field sensitivity is
probably smaller than \SI{10}{\per\tesla} for fields smaller than \SI{10}{\tesla}. \cite{walls1995fundamental}
\end{quote}
 

\subsection{Temperature}
I did a quick qualitiative test to assertain the sign of the change in both
$f$ and $R$ by placing the crystal next to either 
\begin{inparaenum}[\itshape a\upshape)]
 \item a hot source, consisting of a beaker filled with hot water and 
 \item a cold source, consisting of a beaker filled with ice
\end{inparaenum}

Heat made $f$ go up and $R$ go down.  Cold made $f$ go down and $R$ go
up.  $R$ goes back to its original value much more slowly than $f$.

all my data is between 20 and 25c

The SRS manual quotes a first order temperature dependence of
\SI{8}{\hertz\per\degree\celsius} and \SI{4}{\ohm\per\degree\celsius} in
water based on its temperature coefficent of viscosity.

THIS IS NOT TRUE!  The temp coefficent cannot explain.

Temperature dependence is a second major issue. It is small for AT-cut
crystals; however, temperature fluctuations cause fluctuations in Rq
inversely
proportional to Q [7]. This effect is small compared to temperature ef-
fects having their origin in properties of the measurand. In liquid appli-
cations, the most temperature-sensitive value is the liquid viscosity.
 Here,
 √
 temperature-induced variations in frequency increase with ω, whereas
 mass sensitivity increases with ω. Therefore, an elevated resonance
 frequency
 is helpful.



\subsection{Pressure}
\subsection{Mechanical Stress}
When I push on the holder from the top, $f$ goes up and $R$ goes down, but
not by much.  The same is true for flipping the holder and pressing from
the other side.

\cite{fletcher1979comparison}

\begin{quote}
Our static results for symmetrical bending should give an indication of the
frequency changes to be expected from such accelerations and also indicate
optimum mounting angles. With an applied load of 5 grammes force the force
applied to the centre of the crystal 200 is times its mass.  We therefore
expect similar behaviour to that due to an acceleration of 200 'G'.\cite{fletcher1979comparison}
\begin{equation}
\Delta f = 1.62 g^{1.72} 
\end{equation}
\end{quote}
 



\subsubsection{Orientation and Gravity}
\subsection{Sedimentation Potential}
A secondary effect due to the crystal ionic
mean energy of the lattice is
conductivity can be observed. The initial frequen-
cy shift due to the previous effects
 is followed
by a relaxation phenomenon as shown on fig. 12.
This relaxation corresponds to the migration
 of
ionic impurities (Na', Li',
 ' K
 .. .
 ) , under the DC
 where the summation is performed over all normal
field. These alkalin ions are trapped by A1++'
 modes of wave number k and frequency
 W and
 over
atoms and occupy an interstitial position.



\subsection{Viscous Penetration Depth}
\subsection{Hydrostatic Pressure}

\subsection{Sauerbrey Mass Loading}
The Sauerbrey equation has been modified to predict the frequency and
resistance changes.

\begin{align}
 \Delta f = -f_u^{3/2} \left(\frac{\rho_l \eta_l}{\pi \rho_q \mu_q}\right)^{1/2}
\end{align}
where

\begin{tabular}{ll}
$f_u$     & frequency of unloaded crystal \\
$\mu_q$   & shear modulus of quartz \\
$\rho_q$  & density of quartz \\
$\rho_l$  & density of contact liquid \\
$\eta_l$   & viscosity of contact liquid
\end{tabular}

\begin{align}
 \Delta R = 2 n f_s L_u \left(\frac{4 \pi f_s \rho_l \eta_l}{\rho_q \mu_q}\right)^{1/2}
\end{align}

where
\begin{tabular}{ll}
$\Delta R$ & change in series resistance \\
$n$        & number of sides in contact with liquid \\
$f_s$      & oscillation frequency \\
$L_u$      & intrinsic inductance of unloaded crystal \\
\end{tabular}

From these equations, classic liquid loading in this thin film Sauerbrey
approximation causes an decrease in $\Delta f$ and an increase in $\delta
R$.

\subsection{Dybwad Model}

\begin{tikzpicture}[every node/.style={draw,outer sep=0pt,thick}]
\tikzstyle{spring}=[thick,decorate,decoration={zigzag,pre length=0.3cm,post length=0.3cm,segment length=6}]
\tikzstyle{damper}=[thick,decoration={markings,  
  mark connection node=dmp,
  mark=at position 0.5 with 
  {
    \node (dmp) [thick,inner sep=0pt,transform shape,rotate=-90,minimum width=15pt,minimum height=3pt,draw=none] {};
    \draw [thick] ($(dmp.north east)+(2pt,0)$) -- (dmp.south east) -- (dmp.south west) -- ($(dmp.north west)+(2pt,0)$);
    \draw [thick] ($(dmp.north)+(0,-5pt)$) -- ($(dmp.north)+(0,5pt)$);
  }
}, decorate]
\tikzstyle{ground}=[fill,pattern=north east lines,draw=none,minimum width=0.75cm,minimum height=0.3cm]
\node[circle] (M1) [minimum width=0.5cm] {$m_\text{p}$};
\node[circle] (M2) [below=1.5cm of M1.south,minimum width=0.5cm] {$m_\text{q}$};
\node (ground1) [ground,below=1.5cm of M2.south,anchor=north] {};
\draw [spring] (M1.south) -- (M2.north) node
[draw=none,fill=none,midway,right=0.5cm] {$k_\text{p}$};
\draw [spring] (M2.south) -- (ground1.north) node
[draw=none,fill=none,midway,right=0.5cm] {$k_\text{q}$};
\draw [thick,decorate,decoration={brace,raise=0.75cm}] (ground1.north) -- (M2.north)
node [draw=none,fill=none,midway,left=1.0cm] {QCM};
\draw [thick,decorate,decoration={brace,raise=0.75cm}] (M2.north) -- (M1.north)
node [draw=none,fill=none,midway,left=1.0cm] {particle};

%\node (N) at (M.north) [minimum width=3.5cm,minimum height=2cm] {$m$};
%
%\draw (ground1.north west) -- (ground1.north east);
\end{tikzpicture}


\section{Buffer Shifts}

\section{Pressing Down}

\section{Oligos}

\section{Lambda DNA}

\section{Mechanical Assembly}
\subsection{PDMS Cell}

\section{Shift Directions for Unloading}
\begin{tabular}{lll}
 \toprule
 sample & $\Delta f$ & $\Delta R$ \\
 \midrule
 Sauerbrey & $\uparrow$ &   $\downarrow$ \\
 air       & $\uparrow$ & $\uparrow$ \\
 water     & $\uparrow$ &   $\downarrow$ \\
 HBFR      & $\downarrow$ & $\uparrow$ \\
 \bottomrule
\end{tabular}

Plot Shapes?
\section{Important Numerical Quantities}
\begin{table}
 \centering
 \begin{tabularx}{\textwidth}{ l l }
  \toprule
  temperature dependent viscosity of water & $\mu(T)=\num{2.414e-5}\times 10^{247.8/(T-140)}$\\
  \bottomrule
 \end{tabularx}
\end{table}

\end{document}
