The most dominant feature of weirdospace is known as {\it primary stripes}.
Primary stripes are thought to arise from single scattering off the tip
(events of type \Rmnum{1}).  Consider the phase accumulation in the case of
single scattering off the tip.  It consists of only two terms, the local
phase $\varphi_\mathrm{l}$ sampled by the tip position
$(x_\mathrm{tip},y_\mathrm{tip})$, and the phase to the far field
$\varphi_\mathrm{ff}$.  
The intensity in wierdospace for primary stripes given by Bert is 
\begin{align}
I(\phi,x_\mathrm{tip},y_\mathrm{tip})=
\left|
a e^{i \Phi} + b e^{i \varphi_\mathrm{l}+\varphi_\mathrm{ff}}
\right|^2
\end{align}
where the argument $a e^{i \Phi}$ is the overall phase and intensity of the
``background'' signal.  From this he derives the expression
\begin{align}
\label{eqn:primarystripes}
I(\phi,x_\mathrm{tip},y_\mathrm{tip}) \propto 1 + \cos\left(k_\mathrm{sp}x_\mathrm{tip} + k_0
\sin\theta_\mathrm{sp}\left(x_\mathrm{tip}\cos\phi+y_\mathrm{tip}\sin\phi\right)
+ \varphi_a\right)
\end{align}
which is a little off because
\begin{align}
\left|
a e^{i\alpha}+be^{i\beta}
\right|^2 = a^2 + b^2 + 2ab\cos(\alpha - \beta)
\end{align}
says that the argument in the second exponential should have a minus sign,
but I suppose if you just set $a=b=1$, then the arbitrary phase term
$\varphi_a$ could make up for any of that.

Bert also gives the expected wavelength of the primary stripes as
\begin{align}
\lambda_\mathrm{I}=
\frac{2 \pi}{\sqrt{\left(k_\mathrm{sp} + k_0
\sin{\theta_\mathrm{sp}}
\cos{\phi}\right)^2 + \left(k_0 \sin{\theta_\mathrm{sp}} \sin
\phi\right)^2}}
\end{align}

This agrees very well with experimental data as was shown in Bert's thesis
(Figure \ref{fig:bertconeangle}).
\begin{figure}
\label{fig:bertconeangle}
\begin{center}
\psset{xunit=1.6711cm,yunit=4cm}
\readdata{\dataa}{primarystripes/cone5.txt}
\readdata{\datab}{primarystripes/acone5.dat}
\begin{pspicture}(0,0)(6.2832,1.5)
 \psaxes[trigLabelBase=2,dx=\psPiH,trigLabels,Dy=0.2]{-}(0,0)(6.2832,1.5)
\listplot[linecolor=red,plotstyle=dots,dotstyle=o]{\dataa}
\psclip{\psframe[linestyle=none](0,0)(6,1.4)}
\listplot[linecolor=blue]{\datab}
\endpsclip
\end{pspicture}
\end{center}
\caption{Primary stripe angle as a function position around the ring:
experiment and theory.}
\end{figure}

If we analytically evaluate Equation \ref{eqn:primarystripes} to produce 
weirdospace images to compare to those produced through the Monte Carlo
simulation with only single scattering events included, an almost perfect
agreement is achieved.

\begin{figure}
\label{fig:basescatt}
\begin{center}
\begin{overpic}[width=416pt]{primarystripes/gbr.eps}
\psset{unit=52pt}
\begin{pspicture}(0,0)(1,8)
\psaxes[trigLabelBase=8,dx=1,trigLabels,Dx=1,ticksize=0,linewidth=0pt](0.5,0)(0,0)(8,0)[$\phi$,0][$ $,0]
\end{pspicture}
\end{overpic}
\end{center}
\end{figure}

\begin{figure}
\label{fig:singlescatcompare}
\begin{center}
\begin{overpic}[width=416pt]{primarystripes/singlescatsbs1.eps}
\psset{unit=52pt}
\begin{pspicture}(0,0)(1,8)
\psaxes[trigLabelBase=8,dx=1,trigLabels,Dx=1,ticksize=0,linewidth=0pt](0.5,0)(0,0)(8,0)[$\phi$,0][$ $,0]
\end{pspicture}
\end{overpic}
\end{center}
\caption{Comparision of analytic evaluation of Equation
\ref{eqn:primarystripes} (bottom) against a Monte Carlo simulation with
only single scattering off the tip included.}
\end{figure}

\subsection{Evidence of Origin}
Analysis of the clever grid also provides good evidence for the origin of
primary stripes due to single scattering, but in a more obvious way.  For
every simulation where single scattering off the tip is included, primary
stripes appear.  The inverse is also true: for all simulations where single
scattering has been removed, primary stripes disappear.

In the limit of many ($N$) scatterers, the contribution from single
scattering is reduced to $1/N$.  In this situation the primary stripes are
supressed.

