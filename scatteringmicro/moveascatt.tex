The use of nanoparticle adsorption as a scattering mechanism was inspired
by an experimental technique known as scanning plasmon near field optical
microscopy (\gls{spom}).  This technique makes use of a sharp tungsten tip from a
scanning tunneling microscope (STM) to interact with the local \gls{spp} field.
Such instruments have found great utility in X, Y, and Z.

(the above sentences need a WHY)

In collaboration with \name{Gregory} of the University of Oregon, we have
analyzed a series of experimental datasets collected by
\name{Schumann}~\cite{schumann2009surface}.  The experimental protocol is
described in detail in~\cite{schumann2009surface}; here we briefly relay
the important points.  An N-BK7 hemisphere, diameter
\SI{24.5}{\milli\meter} was used as the supporting structure.  On the
prism's hypotenuse was deposited a thin layer of \ce{CaF2} to enhance the
surface roughness of the \SI{50}{\nano\meter} silver layer deposited above.
The prism was mounted in a vacuum housing and a \SI{632.8}{\nano\meter}
helium-neon laser was used at $\thetasp\approx\SI{44}{\degree}$ with a
$\mathrm{NA}=0.2$ Gaussian focal spot incident on the central point of the
hypotenuse, exciting \glspl{spp} on the metal-vacuum interface.  The cone was
projected onto a sheet of paper and recorded from above with an imaging
sensor.  In the vacuum housing, a sharp tungsten tip mounted on a
feedback-controlled piezo system is brought within the evanescent field of
the focused beam.  The $z$ height of the tip is controlled my monitoring
the tunneling current between the tip and the sample.  The tip is then
rastered within a square region of the focal spot, and an image of the cone
is taken for every tip position in the raster.  Thus, the tip is a moving
scatterer amongst a fixed background.

