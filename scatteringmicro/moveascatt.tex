The use of nanoparticle adsorption as a scattering mechanism was inspired
by an experimental technique known as scanning plasmon near field optical
microscopy (SPOM).  This technique makes use of a sharp tungsten tip from a
scanning tunneling microscope (STM) to interact with the SPP field.  

In collaboration with \name{Stephen Gregory} of the University of Oregon,
we have analyzed a series of experimental datasets collected by
\name{Schumann}~\cite{schumann2009surface}, which we now describe.

(the above two sentences need a WHY)

The experimental protocol is described in detail
in~\cite{schumann2009surface}, something relay the important points.

An N-BK7 hemisphere, diameter \SI{24.5}{\milli\meter} was used as the
supporting structure.  On the prism's hypotenuse was deposited a thin layer
of \ce{CaF2} to enhance the surface roughness of the \SI{50}{\nano\meter}
silver layer deposited above.  The prism was mounted in a vacuum housing
and a \SI{632.8}{\nano\meter} helium-neon laser was used at
$\thetasp\approx\SI{44}{\degree}$ with its $\mathrm{NA}=0.2$ Gaussian focal
spot on the central point of the hypotenuse excite SPPs on the metal vacuum
surface.  The cone was projected onto a sheet of paper and recorded from
above with an imaging sensor.  In the vacuum housing, a sharp tungsten tip
mounted on a feedback-controlled piezo system is brought within the
evanescent field of the focussed beam.

A sharp tungsten tip was brought into the evanescent field of the
$\mathrm{NA}=0.2$ Gaussian beam

