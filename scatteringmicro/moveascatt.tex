The use of nanoparticle adsorption as a scattering mechanism was inspired
by an experimental technique known as scanning plasmon nearfield optical
microscopy (SPOM).  This technique makes use of a sharp tungsten tip from a
scanning tunneling microscope (STM) to interact with the local SPP field.
Such instruments have found great utility in X, Y, and Z.

(the above sentences need a WHY)

In collaboration with \name{Stephen Gregory} of the University of Oregon,
we have analyzed a series of experimental datasets collected by
\name{Schumann}~\cite{schumann2009surface}.  The experimental protocol is
described in detail in~\cite{schumann2009surface}; here we breifly relay
the important points.  An N-BK7 hemisphere, diameter
\SI{24.5}{\milli\meter} was used as the supporting structure.  On the
prism's hypotenuse was deposited a thin layer of \ce{CaF2} to enhance the
surface roughness of the \SI{50}{\nano\meter} silver layer deposited above.
The prism was mounted in a vacuum housing and a \SI{632.8}{\nano\meter}
helium-neon laser was used at $\thetasp\approx\SI{44}{\degree}$ with a
$\mathrm{NA}=0.2$ Gaussian focal spot incident on the central point of the
hypotenuse, exciting SPPs on the metal-vacuum interface.  The cone was
projected onto a sheet of paper and recorded from above with an imaging
sensor.  In the vacuum housing, a sharp tungsten tip mounted on a
feedback-controlled piezo system is brought within the evanescent field of
the focussed beam.  The $z$ height of the tip is controlled my monitoring
the tunneling current between the tip and the sample.  The tip is then
rastered through a square region in the focal spot, and an image of the
cone is taken for every tip position in the raster.  Thus, the tip is a
moving scatterer amongst a fixed background and a tool for exploring
surface plasmon random scattering.

Central to these experiments is the generation of a topology map known
colloqually as ``weirdospace''.  Weirdospace is, quite simply, a two
dimensional map of the intensity of a particular point on the cone as a
function of tip position.  We designate the tip position on the prism
surface as $(x,y)$

how weirdospace works
introduction, math of of primary stripes

we have found that blah primary stripes let you reconstruct the phase of
the speckle field, the vectorial field 

how your reconstruction algorithm works

say that you know that it works because of the presence of optical
vorticies, and that the vorticies in the reconstructed guy are always
coincident with the zeros of the pullback frame

statistics of vorticies?
