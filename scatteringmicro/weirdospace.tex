Central to these experiments is the generation of a topology map known
colloqually as ``weirdospace''.  Weirdospace is, quite simply, a two
dimensional map of the intensity in the far field as a function of tip position.  

We begin by assigning the tip position on the
prism surface with two dimensional cartesian coordinates $(x,y,z=0)$, and
the far field intensity on the detector as $(x^\prime,y^\prime,z^\prime)$.
The far field has two phase contributions: the first from the local SPP field
$\ksp x$ and the second from propagation into the far field,
$k_0\left((x-x^\prime)^2+ (y-y^\prime)^2+
(z-z^\prime)^2\right)^{1/2}$.  Together, this yields
\begin{equation}
E(x^\prime,y^\prime,z^\prime) = \exp\!\left( \mi\ksp x + %
\mi k_0\sqrt{ (x-x^\prime)^2+ (y-y^\prime)^2+ (z-z^\prime)^2 } \right)
\end{equation}
or, in spherical coordinates,
\begin{equation}
E(\rho,\theta,\phi) = \exp\!\left( \mi\ksp x + %
\mi k_0\sqrt{
(\rho\sin\theta\cos\phix+x)^2+(\rho\sin\theta\sin\phi+y)^2+(\rho\cos\theta+z)^2 } \right)
\end{equation}
If we assume the detector is in the far field $\rho\gg\lambda_0$, we can
make a couple of approximations.  First, we neglect the additional optical
path length accumulated propagating through the prism.  Second, if we
consider the path length difference between the origin and the far field,
$\rho$, we can consider only the path length difference
\begin{align}
E(\rho,\theta,\phi) &= \exp\!\left( \mi\ksp x + %
\mi k_0(\rho-\sqrt{ (\rho\sin\theta\cos\phix+x)^2+(\rho\sin\theta\sin\phi+y)^2+(\rho\cos\theta+z)^2
}) \right)\\
E(\rho\gg\lambda,\theta,\phi) &= \exp\!\big( \mi \ksp x
 + \mi k_0 \sin\theta \left(x\cos\phi+y\sin\phi\right)
 + \varphi\big)
	\label{eqn:primarystripes}
\end{align}
where we have assigned the variable $\varphi$ to represent the complex phase term
from non-tip-scattering events (a coherent background).  

\Equation{eqn:primarystripes} predicts an important feature of weirdospace.
For sufficiently strong tip scattering a set of angled interference stripes
will be present.  We call this feature \textit{primary stripes}.  The
frequency and orientation of primary stripes is a function of $\phi$ as per
\Equation{eqn:primarystripes}.  \Figure{fig:primarystripes} shows this
dependence graphically for a simulated set of stripes.  Note that in
\Equation{eqn:primarystripes} we have not specified $\varphi$.  This term
could lead to higher order features in wierdospace, which we will discuss
shortly.
\begin{figure}
\centering
\caption{Primary stripes around the ring.}
\label{fig:primarystripes}
\end{figure}

We will now show that primary stripes allow enable reconstruction of the
phase of the far field speckle pattern.  
