Central to these experiments is the generation of a topology map known
colloquially as ``weirdospace''.  Weirdospace is, quite simply, a two
dimensional map of the intensity in the far field as a function of tip position.

We begin by assigning the tip position on the prism surface the coordinates
$(x,y,z=0)$, and the position on the detector as
$(x^\prime,y^\prime,z^\prime)$.  The detector field has two phase
contributions: the first from the local SPP field, $\ksp x$, and the second
from propagation into the far field, Together,
\begin{equation}
\mathbf{E}(x^\prime,y^\prime,z^\prime) = \exp\!\left( \mi\ksp x + %
\mi k_0\sqrt{ (x-x^\prime)^2+ (y-y^\prime)^2+ (z-z^\prime)^2 } \right)
\end{equation}
or, in spherical coordinates,
\begin{equation}
\mathbf{E}(\rho,\theta,\phi) = \exp\!\left( \mi\ksp x + %
\mi k_0\sqrt{
(\rho\sin\theta\cos\phi x+x)^2+(\rho\sin\theta\sin\phi+y)^2+(\rho\cos\theta+z)^2 } \right)
\label{eqn:dingusthusly}
\end{equation}
If we assume the detector is in the far field, $\rho\gg\lambda_0$, we can
make a couple of approximations.  First, we neglect the additional optical
path length accumulated propagating through the prism.  Second, if we
consider the path length \textit{difference} between the origin and the far field,
$\rho$, we can simplify \Equation{eqn:dingusthusly} thusly
\begin{align}
\mathbf{E}(\rho,\theta,\phi) &= \exp\!\left( \mi\ksp x + %
\mi k_0\left(\rho-\sqrt{ (\rho\sin\theta\cos\phi x+x)^2+(\rho\sin\theta\sin\phi+y)^2+(\rho\cos\theta+z)^2
}\right) \right)\\
\mathbf{E}(\rho\gg\lambda,\theta,\phi) &= \exp\!\big( \mi \ksp x
 + \mi k_0 \sin\theta \left(x\cos\phi+y\sin\phi\right)
 + \varphi\big)
	\label{eqn:primarystripes}
\end{align}
where we have included the variable $\varphi$ to represent the complex phase term
from non-tip-scattering events (a coherent background).

\Equation{eqn:primarystripes} shows a fundamental feature of weirdospace.
For sufficiently strong tip scattering a set of angled interference stripes
will be present.  We call this feature \textit{primary stripes},
frequency and orientation of which is a function of $\phi$.
\Figure{fig:primarystripes} shows this
dependence graphically for a simulated set of stripes.  Note that in
\Equation{eqn:primarystripes} we have not specified $\varphi$.  This term
could lead to higher order features in weirdospace, which we will discuss
in turn.
\begin{figure}
\centering
\caption{Primary stripes around the ring.}
\label{fig:primarystripes}
\end{figure}
