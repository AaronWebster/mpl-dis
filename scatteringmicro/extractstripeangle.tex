It turns out that the phase of the primary stripes in a weirdospace image
can be used to determine the relative phase of the far field speckle
pattern from which the primary stripes are derived.  To see how this is
done, first observe that the frequency of the primary stripes are set in
\Equation{eqn:primarystripes} only as a function of tip position.  The tip
is always in phase with the SPP field.  However, the phase of all
non-tip-scattering paths interfere with the this path and determine the
phase of the primary stripes.  Thus, if we make a raster scan of the
surface and take a pullback image, i.e. an image with no tip present, the
phase of the primary stripes will represent the phase of this field (again,
we already know the phase of the tip).

For each photometry map, the wavevectors $k_x$ and $k_y$ describing the
spatial frequency and orientation of the primary stripes was determined
from the peaks in its Fourier spectrum~\cite{huntley1986speckle}.  The
phase, $\varphi=\mathbf{k}\cdot\mathbf{r}_0$, was similarly determined
by using the shift property of the Fourier transform
\begin{align}
\ff{\mathbf{E}(\mathbf{r}-\mathbf{r}_0)}(\mathbf{k}) &= 
\me^{\mi \mathbf{k}\cdot \mathbf{r}_0}\,
\mathbf{E}\left(\mathbf{k}\right)\\
&= \me^{\mi \varphi}\, \mathbf{E}\left(\mathbf{k}\right)
\end{align}
with respect to a set of reference stripes
\begin{equation}
I_\mathrm{ref} = \cos(k_x x + k_y y)
\end{equation}
Again, $k_x$, $k_y$, and $\varphi$ are parameters for weirdospace.

We can verify experimentally that the phase retrieval algorithm works as
advertised by considering the presence of optical vortices.  As described
in \Section{sec:x}, the locations occur at the intersections of the zeros
of the real and imaginary components of the complex field; at these
locations the intensity is zero and the phase is undefined.  Therefore, at
these locations the pullback intensity should also be zero.  Furthermore, a
trajectory passing through (or near) a vortex should see a sudden jump of
phase (a $\pi$ shift, I think).

\begin{figure}
\centering
\caption{Phase extraction technique.}
\label{fig:phaseextractiontechnique}
\end{figure}

We show an example dataset in \Figure{fig:}
