\subsection{Motivation}
SPR based biosensors have all but reached
their fundamental limits in terms of bulk refractive index
sensitivity~\cite{piliarik2009surface}.  But as biodetection goals have
moved from bulk sensing into the single molecule regime, new techniques for
the detection of discrete events are desired.  To this accord,
nanoparticle-based amplification
strategies~\cite{wang2005nanomaterial}~\cite{jain2007review} have seen
increasing popularity for enhancing SPR measurements, and reports of
attomolar resolution are becoming
mainstream~\cite{fang2006attomole}~\cite{krishnan2011attomolar}~\cite{kwon2012nanoparticle}~\cite{sim2010attomolar}.
Impressive as this is, such measurements still consitiute bulk sensing;
the presence of a specific nanoparticle is not in and of itself a
specific feature of the SPR sensogram.

In this section we describe a new method for the detection of single
nanoparticle adsorption events using changes in the cone speckle caused by
a change in the scattering microstructure.  To wit, the addition of a
single scatterer.

As outlined in \Section{sec:bulkri}, a traditional SPR biosensor looks at
the intensity minimum, or notch, in the system's angular spectrum causd by
interference between the specularly reflected light and the antiphase
re-radiated SPP field.  The location of this notch is of course
extraordinary sensitive to local refractive index perturbations of the
sample medium, facilitating detection of the adsorption of analytes to the
sensor surface~\cite{homola1999surface}.  

Light is also scattered into the cone.  Though the cone has nearly the same
angular bulk response as the notch notch (\Section{sec:bulkri}), its
azimuthal component exhibits speckle, which is already known to be
extraordinary sensitive to both the configuration and
motion~\cite{berkovits1994correlations} of its scatterers, even down to
sub-wavelength displacements~\cite{berkovits1991sensitivity} or
inclusions~\cite{berkovits1990theory} of single scatterers.

This is in essence the principle we seek to exploit.  If a nanoparticle
adsorbs to the sensor surface, the scattering microstructure will be
altered and the speckle pattern will change.  Detecting this change will
correspond to a nanoparticle binding event.

\subsection{Theoretical Sensitivity}
The theoretical sensitivity of a speckle pattern to the addition of a
single scatterer depends on whether the underlying scattering process is in
the single or multiple scattering regime.

scattering regime, single or multiple, and
the degree to which each process takes place in the speckle pattern.  To
begin, we treat them separately.


To quantify the difference between two specklegrams, we use the Pearson's
correlation coefficent, $C_I$, defined as 
\begin{equation}
C_I = \frac{1}{N-1} \frac{1}{M-1} 
\sum_{i=1}^N \sum_{j=1}^M 
\left(\frac{X_{i,j} - \bar{X}}{\sigma_X}\right)
\left(\frac{Y_{i,j} - \bar{Y}}{\sigma_Y}\right)
\end{equation}
for two $N \times M$ specklegrams $X$ and $Y$ with indicies $i$ and $j$.
For all $X$ and $Y$, $C_I \in (-1,1)$.
The advantage of this formalism is that dividing subtracting the mean and
dividing by the standard deviation effectively normalizes intensity
fluctuations which occur in a real experiment.

Consider first a single scattering speckle pattern which is the result of a
random phasor sum, similar to \Equation{eqn:phasorsum}
\begin{equation}
\mathbf{E}(\mathbf{r}) = \frac{1}{\sqrt{N}} \sum_{n=1}^{N} a_n \me^{\mi \phi_n}
\end{equation}
This is equivalent to a 2D random walk in the complex plane.  Because each
phasor is randomly distributed on the interval $[0,2\pi)$
\begin{equation}
\left<|\mathbf{E}(\mathbf{r})|^2\right> = \frac{N}{\sqrt{N}^2} = 1
\end{equation}
where $\left<\dots\right>$ denotes ensemble averaging.  We thus expect the
addition of a single scatterer to decorrelate the speckle pattern by $N$
(or something like that.  Why not derive this thing again?

\subsection{Experiment}
To demonstrate the principle of our method, we introduce
\SI{50}{\nano\meter} citrate-capped spherical gold nanoparticles in water
into the microfluidic cell and monitor the system as the particles adsorb
to the gold sensor surface.  These particles scatter light strongly enough
that we can observe them with the inverted microscope (similar to a TIRF
setup) while simultaneously recording the speckle.  With regards to the
speckle we quantify changes using the Pearson product-moment
correlation coefficient $C_I$, defined as 
\begin{equation}
C_I = \frac{\mathrm{cov}(a,b)}{\sigma_a \sigma_b}
\end{equation}
where $\mathrm{cov}(a,b)$ is the covariance between signals $a$ and $b$,
and $\sigma_a$ and $\sigma_b$ are the respective standard deviations.  This
allows us to measure changes in the speckle structure while removing the
effect of global intensity fluctuations which we identify as noise.

A sample of the sensor output is shown in \Figure{fig:expdata}

Consider a speckle pattern which is the consequence of $N$ fixed
scatterers.  How does the speckle pattern change upon the addition of a
single ($N+1$) scatterer?  


