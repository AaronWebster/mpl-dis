\subsection{Motivation}
In this section a new method for the detection of single nanoparticle
adsorption events is described which uses changes in the cone speckle
caused by a changes in the underlying scattering microstructure.  To wit,
the addition of a single scatterer.

As outlined in \Section{sec:bulkri}, a traditional SPR biosensor looks at
the intensity minimum, or notch, in the system's angular spectrum caused by
interference between the specularly reflected light and the antiphase
re-radiated SPP field.  The location of the notch is of course
extraordinary sensitive to local refractive index perturbations of the
sample medium, facilitating detection of the adsorption of analytes to the
sensor surface~\cite{homola1999surface}.  

Light is also scattered into the cone.  Though the cone has nearly the same
angular bulk response as the notch notch (\Section{sec:bulkri}), its
azimuthal component exhibits speckle, which is already known to be
extraordinary sensitive to both the configuration and
motion~\cite{berkovits1994correlations} of its scatterers, even down to
sub-wavelength displacements~\cite{berkovits1991sensitivity} or
inclusions~\cite{berkovits1990theory} of single scatterers.

This is in essence the principle which will be exploit exploited.  If a
nanoparticle adsorbs to the sensor surface, the scattering microstructure
will be altered and the speckle pattern will change.  Detecting this change
will correspond to a nanoparticle binding event.

\subsection{Theoretical Sensitivity}
The theoretical sensitivity of a speckle pattern to the addition of a
single scatterer depends on strongly the specifics of the underlying
scattering process.  The dimensionality, degree of single and multiple
scattering, and geometry from which light is both incident and collected
must all be taken into account.  Such theoretical analysis has been solved
for specific cases such as in the transmission geometries of dynamic light
scattering or diffusive wave spectroscopy, but in general the problem is
difficult.  Determination of the relevant analytic expressions for the
sensitivity of a speckle pattern created by SPP single and multiple
scattering are, quite frankly, beyond our abilities.  We will therefore
emphasize our experimental findings, comparing them with numerical
simulations where we have found them descriptive.

With regards to the speckle we quantify changes using the Pearson
product-moment correlation coefficient $C_I$ of the speckle intensity,
defined as 
\begin{equation}
C_I = \frac{\mathrm{cov}(a,b)}{\sigma_a \sigma_b}
\label{eqn:pearsonproductmoment}
\end{equation}
where $\mathrm{cov}(a,b)$ is the covariance between signals $a$ and $b$,
and $\sigma_a$ and $\sigma_b$ are the respective standard deviations.  This
allows us to measure changes in the speckle structure while removing the
effect of experimental system noise.

\begin{figure}
\centering
\caption{Monte Carlo simulation comparing the Pearson product-moment
				correlation $C_I$ for the addition of a single scatterer in both
				single and multiple scattering systems.}
\label{fig:scatteringpearson}
\end{figure}


\begin{equation}
\mathbf{E}(\mathbf{r}) = \frac{1}{\sqrt{N}} \sum_{n=1}^{N} a_n \me^{\mi \phi_n}.
\label{eqn:phasorsumdingus}
\end{equation}

\Equation{eqn:phasorsumdingus} is equivalent to a 2D random walk in the
complex plane.  Because each phasor is randomly distributed on the interval
$[0,2\pi)$
\begin{equation}
\left<|\mathbf{E}(\mathbf{r})|^2\right> = \frac{N}{\sqrt{N}^2} = 1
\end{equation}
where $\left<\dots\right>$ denotes ensemble averaging.  

The influence on the speckle field in a multiple scattering regime is much
more difficult to 

(single scattering result)

(segway to multiple scattering result being a bit more tricky)


The results from (single and multiple scattering, show theoretical plots)

\Figure{fig:scatteringpearson}


\subsection{Experiment}
To demonstrate the principle of our method, wee introduce
\SI{50}{\nano\meter} citrate-capped spherical gold nanoparticles in water
into the microfluidic cell and monitor the system as the particles adsorb
to the gold sensor surface.  These particles scatter light strongly enough
that we can observe them with the inverted microscope (similar to a TIRF
setup) while simultaneously recording the speckle. 

The experimental setup is as depicted in \Figure{fig:expsetup}.  Light from
a \SI{50}{\milli\watt} \SI{660}{\nano\meter} diode laser is coupled into a
single mode fiber and is re-collimated into a cage system mounted on
rotation and translation stages.  The light is incident on a polarizing
beamsplitter passing $p$-polarized light, and continues through a 10x
objective which is focussed on the hypotenuse of a hemispherical prism.
Upon the prism's hypotenuse is a structure containing a gold surface
whereupon SPPs are excited.  The specularly directed and scattered light is
collected directly by two imaging sensors, one in the specular direction
and one in the cone \SI{90}{deg} out of this direction.  The entire setup
is mounted on an inverted microscope stage.  A 10x objective is focussed on
the underside of the prism and simultaneously records the focal spot
through an additional image sensor.  The sensing area of the prism is
enclosed in a transparent microfluidic cell, permitting the controlled
introduction of a sample.

%A sample of the sensor output is shown in \Figure{fig:expdata}

Consider a speckle pattern which is the consequence of $N$ fixed
scatterers.  How does the speckle pattern change upon the addition of a
single ($N+1$) scatterer?  

\begin{figure}
\centering
\import{includes/}{setpgfinc}
\import{scatteringmicro/figures/beforeafter/}{beforeafter}
\caption{haldo}
\label{fig:scattbeforeafter}
\end{figure}

\begin{figure}
\centering
\import{includes/}{setpgfinc}
\import{scatteringmicro/figures/spkcompare/}{spkcompare}
\caption{spkcompare}
\label{fig:scattspkcompare}
\end{figure}

