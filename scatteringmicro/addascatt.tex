\subsection{Motivation}
SPR based biosensors have all but reached
their fundamental limits in terms of bulk refractive index
sensitivity~\cite{piliarik2009surface}.  But as biodetection goals have
moved from bulk sensing into the single molecule regime, new techniques for
the detection of discrete events are desired.  To this accord,
nanoparticle-based amplification
strategies~\cite{wang2005nanomaterial}~\cite{jain2007review} have seen
increasing popularity for enhancing SPR measurements, and reports of
attomolar resolution are becoming
mainstream~\cite{fang2006attomole}~\cite{krishnan2011attomolar}~\cite{kwon2012nanoparticle}~\cite{sim2010attomolar}.
Impressive as this is, such measurements still constitute bulk sensing;
the presence of a specific nanoparticle is not in and of itself a
specific feature of the SPR sensorgram.

In this section a new method for the detection of single nanoparticle
adsorption events is described which uses changes in the cone speckle
caused by a changes in the underlying scattering microstructure.  To wit,
the addition of a single scatterer.

As outlined in \Section{sec:bulkri}, a traditional SPR biosensor looks at
the intensity minimum, or notch, in the system's angular spectrum caused by
interference between the specularly reflected light and the antiphase
re-radiated SPP field.  The location of the notch is of course
extraordinary sensitive to local refractive index perturbations of the
sample medium, facilitating detection of the adsorption of analytes to the
sensor surface~\cite{homola1999surface}.  

Light is also scattered into the cone.  Though the cone has nearly the same
angular bulk response as the notch notch (\Section{sec:bulkri}), its
azimuthal component exhibits speckle, which is already known to be
extraordinary sensitive to both the configuration and
motion~\cite{berkovits1994correlations} of its scatterers, even down to
sub-wavelength displacements~\cite{berkovits1991sensitivity} or
inclusions~\cite{berkovits1990theory} of single scatterers.

This is in essence the principle which will be exploit exploited.  If a
nanoparticle adsorbs to the sensor surface, the scattering microstructure
will be altered and the speckle pattern will change.  Detecting this change
will correspond to a nanoparticle binding event.

\subsection{Theoretical Sensitivity}
The theoretical sensitivity of a speckle pattern to the addition of a
single scatterer depends on strongly the specifics of the underlying
scattering process.  The dimensionality, degree of single and multiple
scattering, and geometry from which light is both incident and collected
must all be taken into account.  Such theoretical analysis has been solved
for specific cases such as in the transmission geometries of dynamic light
scattering or diffusive wave spectroscopy, but in general the problem is
difficult.  Determination of the relevant analytic expressions for the
sensitivity of a speckle pattern created by SPP single and multiple
scattering are, quite frankly, beyond our abilities.  We will therefore
emphasize our experimental findings, comparing them with numerical
simulations where we have found them descriptive.

With regards to the speckle we quantify changes using the Pearson
product-moment correlation coefficient $C_I$ of the speckle intensity,
defined as 
\begin{equation}
C_I = \frac{\mathrm{cov}(a,b)}{\sigma_a \sigma_b}
\end{equation}
where $\mathrm{cov}(a,b)$ is the covariance between signals $a$ and $b$,
and $\sigma_a$ and $\sigma_b$ are the respective standard deviations.  This
allows us to measure changes in the speckle structure while removing the
effect of experimental system noise.

\begin{figure}
\centering
\caption{Monte Carlo simulation comparing the Pearson product-moment
				correlation $C_I$ for the addition of a single scatterer in both
				single and multiple scattering systems.}
\label{fig:scatteringpearson}
\end{figure}


\begin{equation}
\mathbf{E}(\mathbf{r}) = \frac{1}{\sqrt{N}} \sum_{n=1}^{N} a_n \me^{\mi \phi_n}.
\label{eqn:phasorsumdingus}
\end{equation}

\Equation{eqn:phasorsumdingus} is equivalent to a 2D random walk in the
complex plane.  Because each phasor is randomly distributed on the interval
$[0,2\pi)$
\begin{equation}
\left<|\mathbf{E}(\mathbf{r})|^2\right> = \frac{N}{\sqrt{N}^2} = 1
\end{equation}
where $\left<\dots\right>$ denotes ensemble averaging.  

The influence on the speckle field in a multiple scattering regime is much
more difficult to 

(single scattering result)

(segway to multiple scattering result being a bit more tricky)


The results from (single and multiple scattering, show theoretical plots)

\Figure{fig:scatteringpearson}




\subsection{Experiment}
To demonstrate the principle of our method, we introduce
\SI{50}{\nano\meter} citrate-capped spherical gold nanoparticles in water
into the microfluidic cell and monitor the system as the particles adsorb
to the gold sensor surface.  These particles scatter light strongly enough
that we can observe them with the inverted microscope (similar to a TIRF
setup) while simultaneously recording the speckle. 
A sample of the sensor output is shown in \Figure{fig:expdata}

Consider a speckle pattern which is the consequence of $N$ fixed
scatterers.  How does the speckle pattern change upon the addition of a
single ($N+1$) scatterer?  


