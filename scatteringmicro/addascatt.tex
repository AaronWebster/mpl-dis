Consider a speckle pattern which is the consequence of $N$ fixed
scatterers.  How does the speckle pattern change upon the addition of a
single ($N+1$) scatterer?  



To quantify the difference between two specklegrams, we use the Pearson's
correlation coefficent, $C_I$, defined as 
\begin{equation}
C_I = \frac{1}{N-1} \frac{1}{M-1} 
\sum_{i=1}^N \sum_{j=1}^M 
\left(\frac{X_{i,j} - \bar{X}}{\sigma_X}\right)
\left(\frac{Y_{i,j} - \bar{Y}}{\sigma_Y}\right)
\end{equation}
for two $N \times M$ specklegrams $X$ and $Y$ with indicies $i$ and $j$.
For all $X$ and $Y$, $C_I \in (-1,1)$.
The advantage of this formalism is that dividing subtracting the mean and
dividing by the standard deviation effectively normalizes intensity
fluctuations which occur in a real experiment.

Consider first a single scattering speckle pattern which is the result of a
random phasor sum, similar to \Equation{eqn:phasorsum}
\begin{equation}
\mathbf{E}(\mathbf{r}) = \frac{1}{\sqrt{N}} \sum_{n=1}^{N} a_n \me^{\mi \phi_n}
\end{equation}
This is equivalent to a 2D random walk in the complex plane.  Because each
phasor is randomly distributed on the interval $[0,2\pi)$
\begin{equation}
\left<|\mathbf{E}(\mathbf{r})|^2\right> = \frac{N}{\sqrt{N}^2} = 1
\end{equation}
where $\left<\dots\right>$ denotes ensemble averaging.  We thus expect the
addition of a single scatterer to decorrelate the speckle pattern by $N$
(or something like that.  Why not derive this thing again?

We have performed extensive simulations which show that this is indeed the
case.

plot the addascatt stuffs
