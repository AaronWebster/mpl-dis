\subsection{Motivation}
Consider cone speckle generated by either single or multiple SPP scattering
for $N$ fixed scatterers.  How does the speckle pattern change upon the
addition of a single ($N+1$) scatterer?

As outlined in \Section{sec:bulkri}, a traditional SPR biosensor looks at
the intensity minimum, or notch, in the system's angular spectrum caused by
interference between the specularly reflected light and the antiphase
re-radiated SPP field.  The location of the notch is of course
extraordinarily sensitive to local refractive index perturbations of the
sample medium, facilitating detection of the adsorption of analytes to the
sensor surface~\cite{homola1999surface}.

Light is also scattered into the cone.  Though the cone has nearly the same
angular bulk response as the notch (\Section{sec:bulkri}), its azimuthal
component exhibits speckle, which has been shown to be extraordinarily
sensitive to both the configuration and motion~\cite{berkovits1994correlations}
of its scatterers, even down to sub-wavelength
displacements~\cite{berkovits1991sensitivity} or
inclusions~\cite{berkovits1990theory} of single scatterers.

This is in essence the principle which will be exploited.  If a single gold
nanoparticle adsorbs to the sensor surface, the scattering microstructure will
be altered and the cone speckle pattern will be altered.  The ability to
detect such a change will correspond to the detection of a nanoparticle
binding event.

\subsection{Quantifying Changes in Speckle}
The Pearson product-moment correlation coefficient (PPMCC), $C_I$, was chosen
to measure changes in the structure of cone speckle upon the addition of
scatterers to the underlying scattering microstructure.  The PPMCC is defined
as
\begin{equation}
C_I = \frac{\mathrm{cov}(a,b)}{\sigma_a \sigma_b},
\label{eqn:pearsonproductmoment}
\end{equation}
where $\mathrm{cov}(a,b)$ is the covariance between signals $a$ and $b$,
and $\sigma_a$ and $\sigma_b$ are the respective standard deviations.  $C_I$
assumes a value between $1$ (total correlation) and $-1$ (total
anti-correlation), with $0$ being no correlation.
The PPMCC is especially suited to analyzing speckle as it is relatively
insensitive to experimental artifacts such as system noise and global
intensity fluctuations.

\subsection{Theoretical Sensitivity}
The theoretical sensitivity of a speckle pattern to the addition of a single
scatterer depends strongly on the specifics of the underlying scattering
process.  The dimensionality, degree of single and multiple scattering, and
geometry from which light is both incident and collected must all be taken
into account.  Such theoretical analysis has been solved for specific cases
such as in the transmission geometries of dynamic light scattering or
diffusive wave spectroscopy, but in general the problem is difficult.  As
mentioned in \Section{sec:senspaths}, determination of the relevant analytic
expressions for the sensitivity of a speckle pattern created by SPP single and
multiple scattering are beyond the scope of this work.  Experimental findings
will therefore be emphasized and compared with numerical simulations where
they have been found descriptive.

Though speckle from a single and multiple scattering process share the same
first order statistics given in \Chapter{ch:speckle}, significant differences
emerge in \textit{correlations} between speckle patterns when the system is
perturbed.  Such correlations are amazingly difficult to compute analytically
and have up to now only been determined for a limited number of simple
transmission and reflection geometries~\cite{berkovits1994correlations}.
Additionally, the mathematical methods used in such computations (Feynman
diagrams, ladder propagators, vertex operators, etc.) are beyond the scope of
this work.  Numerical and statistical methods based on the simplified
scattering model of \Figure{fig:plasmongeosimple} are employed
to encompass the observed phenomena, drawing comparisons to established theory
when available.

\begin{figure}
\centering
\import{includes/}{setpgfinc}
\import{scatteringmicro/figures/}{pearsonssinglescatt20}
\caption{Ensemble-averaged Monte Carlo simulation comparing the Pearson
product-moment correlation $C_I$ for the addition of a single scatterer in
both single and multiple scattering systems.}
\label{fig:scatteringpearson}
\end{figure}

%\begin{equation}
%\mathbf{E}(\mathbf{r}) = \frac{1}{\sqrt{N}} \sum_{n=1}^{N} a_n \me^{\mi \phi_n}.
%\label{eqn:phasorsumdingus}
%\end{equation}
%
%\Equation{eqn:phasorsumdingus} is equivalent to a 2D random walk in the
%complex plane.  Because each phasor is randomly distributed on the interval
%$[0,2\pi)$
%\begin{equation}
%\left<|\mathbf{E}(\mathbf{r})|^2\right> = \frac{N}{\sqrt{N}^2} = 1
%\end{equation}
%where $\left<\dots\right>$ denotes ensemble averaging.
%
%The influence on the speckle field in a multiple scattering regime is much
%more difficult to
%
%(single scattering result)
%
%(segway to multiple scattering result being a bit more tricky)
%
%
%The results from (single and multiple scattering, show theoretical plots)
%
%\Figure{fig:scatteringpearson}


\subsection{Experiment}
To demonstrate the principle of the method, \SI{50}{\nano\meter}
citrate-capped spherical gold nanoparticles in water were introduced into the
microfluidic cell and monitor the system as the particles adsorb to the gold
sensor surface.  These particles scatter light strongly enough that they were
able to be observed with the inverted microscope (similar to a TIRF setup)
while simultaneously recording the speckle.

The experimental setup is as depicted in \Figure{fig:expsetup}.  Light from
a \SI{50}{\milli\watt} \SI{660}{\nano\meter} diode laser is coupled into a
single mode fiber and is re-collimated into a cage system mounted on
rotation and translation stages.  The light is incident on a polarizing
beamsplitter passing $p$-polarized light, and continues through a 10x
objective which is focused on the hypotenuse of a hemispherical prism.
Upon the prism's hypotenuse is a structure containing a gold surface
whereupon SPPs are excited.  The specularly directed and scattered light is
collected directly by two imaging sensors, one in the specular direction
and one in the cone \SI{90}{deg} out of this direction.  The entire setup
is mounted on an inverted microscope stage.  A 10x objective is focused on
the underside of the prism and simultaneously records the focal spot
through an additional image sensor.  The sensing area of the prism is
enclosed in a transparent microfluidic cell, permitting the controlled
introduction of a sample.

\begin{figure}
\centering
\import{includes/}{setpgfinc}
\import{scatteringmicro/figures/}{zoomout}
\caption{Change in speckle measured through the PPMCC (top) before (bottom
left) and after (bottom right) the adsorption of a nanoparticle.  }
\label{fig:zoomout}
\end{figure}

\begin{figure}
\centering
\import{includes/}{setpgfinc}
\import{scatteringmicro/figures/beforeafter/}{beforeafter}
\caption{Change in speckle measured through the PPMCC (top) before (bottom
left) and after (bottom right) the adsorption of a nanoparticle.  }
\label{fig:scattbeforeafterone}
\end{figure}

\begin{figure}
\centering
\import{includes/}{setpgfinc}
\import{scatteringmicro/figures/spkcompare/}{spkcompare}
\caption{Visual change in the specklegram upon the adsorption of a
				nanoparticle.  This figure is taken from the same data and timeframe
				as a compliment to \Figure{fig:scattbeforeafterone}.}
\label{fig:scattspkcomparetwo}
\end{figure}

\begin{figure}
\centering
\import{includes/}{setpgfinc}
\import{scatteringmicro/figures/}{events}
\caption{Statistics of nanoparticle adsorption.  Right: event frequency.
Left: inter-event time.  Adsorption statistics match well that of theoretical
Poisson statistics.}
\label{fig:scattbeforeafterthree}
\end{figure}

