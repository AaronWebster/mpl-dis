\section{Introduction}
SPR-based biosensors have all but reached their fundamental limits in terms of
bulk refractive index sensitivity~\cite{piliarik2009surface}, but as
biodetection goals have moved from bulk sensing into the single molecule
regime, new techniques for the detection of discrete events are desired.  To
this end, nanoparticle-based amplification
strategies~\cite{wang2005nanomaterial}~\cite{jain2007review} have seen
increasing popularity for enhancing SPR measurements, and reports of attomolar
resolution are becoming
mainstream~\cite{fang2006attomole}~\cite{krishnan2011attomolar}~\cite{kwon2012nanoparticle}~\cite{sim2010attomolar}.
Impressive as this is, such measurements still constitute bulk sensing; the
presence of a specific nanoparticle is not in and of itself a specific feature
of the SPR sensorgram.

As mentioned in the beginning of \Chapter{ch:speckle}, speckle can be viewed as
a unique fingerprint~\cite{ravikanth2001physical} of the underlying scattering
microstructure.  Indeed, even a small change in the number or configuration of
a system's scatterers can have a profound influence on the resultant
speckle~\cite{berkovits1994correlations}~\cite{feng1986sensitivity}.  The
effect is particularly significant in the multiple scattering regime, where
speckle has been shown to be sensitive to sub-wavelength
motions~\cite{berkovits1991sensitivity} or
inclusions~\cite{berkovits1990theory} of even single scatterers.  Such topics
belong to a more general class of problems with important implications in
diverse fields of study: diffusive wave spectroscopy~\cite{pine1988diffusing},
dynamic light scattering~\cite{berne2000dynamic}, tracking and identification
of targets with radio waves (the ``cruise missile''
problem~\cite{atkins1991neural}), fluctuations in signal power in cellular
telephone networks~\cite{abdi2001estimation}, and very recently in detecting
stress fractures in aggregates such as concrete.

This chapter details investigations regarding the interplay between the
underlying scattering microstructure, the in-plane scattering of SPPs, and
the optical speckle in the cone.  More fundamentally, this chapter is
centered around the behavior of speckle in systems where a single scatterer
is added to a fixed background of scatterers.
