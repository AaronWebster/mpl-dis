We now arrive on what is perhaps the most interesting and useful aspect of
our studies: the influence of the scattering microstructure on the
conically scattered light.  Consider a speckle pattern which is the
consequence of $N$ fixed scatterers.  

Specifically, we will look at two important and
closely related questions.  

\begin{itemize}
\item How does the speckle pattern change upon the addition of a single
				($N+1$) scatterer?  
\item How does the speckle pattern change 
\end{itemize}


They are%
\begin{inparaenum}[(a)]
\item the addition, and
\item the motion of,
\end{inparaenum} %
a single scatterer amongst a fixed background of scatterers.  Such topics
belong to a more general class of problems with important implications in
many diverse fields of study: diffusive wave
spectroscopy~\cite{pine1988diffusing}, tracking and identification of
targets with radio waves (the ``cruise missile''
problem~\cite{atkins1991neural}), fluctuations in signal power in cellular
telephone networks~\cite{abdi2001estimation}, and very recently in
detecting stress fractures in aggregates such as
concrete~\cite{larose2010locating}.
