\section{Introduction}
We now arrive on what is perhaps the most interesting and useful aspect of
our SPP scattering studies: the influence of the underlying scattering
microstructure on the conically scattered light.  Specifically we will look
at two important and closely related systems regarding (a) the addition and
(b) the motion of a single scatterer amongst a fixed background.  Such
topics belong to a more general class of problems with important
implications in diverse fields of study: diffusive wave
spectroscopy~\cite{pine1988diffusing}, tracking and identification of
targets with radio waves (the ``cruise missile''
problem~\cite{atkins1991neural}), fluctuations in signal power in cellular
telephone networks~\cite{abdi2001estimation}, and very recently in
detecting stress fractures in aggregates such as
concrete~\cite{larose2010locating}.  Indeed, we will see that a speckle
pattern is akin to a fingerprint~\cite{ravikanth2001physical} of the
underlying scattering microstructure, and thus even small changes will
significantly affect the speckle.
