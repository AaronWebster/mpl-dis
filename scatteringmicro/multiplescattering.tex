Thus far we have described speckle mathematically as a sum of random,
statistically independent, phasors.  This is in essence what is known as
\textit{single scattering}.  However, a much more interesting and
phenomenologically complex situation arises when we allow higher order
scattering processes to take over.  This is, unsurprisingly, known as
\textit{multiple scattering}, a theme which will arise repeatedly in our
investigations of surface plasmon random scattering.  

The transition from the single to the multiple scattering occurs roughly upon
fulfillment of the Ioffe-Regel criterion~\cite{offe1960non}, when the
transport mean free path $l^*$ is much larger than the spatial frequency of
the optical wave, or
\begin{equation}
k l^* \approx 1
\end{equation}
where $k=\omega/c$ as usual.  The transport mean path is defined as the
characteristic distance over which the incident wave is scattered out of
its incoming direction~\cite{berkovits1994correlations}.  This is the
optical analog of electron transport in condensed matter physics.  There,
one would use the Fermi wavenumber $k_F$ and take $l^*$ to be the mean free
path.  Materials for which $k_F l^* \gg 1$ are conductors, and materials
for which $k_F l^* \ll 1$ are insulators.  Furthermore, we can define the
multiple scattering regime in terms of the sample length $L$, such that if
$l^* \ll L$ and $1/(k l^*) \ll 1$, the system can be thought to be the
multiple scattering regime.  

(somethign about how if you do one, you see a certain correlation, but
another, another correlation, etc.)
