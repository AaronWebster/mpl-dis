In \Chapter{ch:speckle} the statistics of cone speckle, the intensity PDF and
contrast, were shown to be in agreement with the random phasor sum model
(\Equation{eqn:phasorsum}).  Unfortunately, this model only serves to describe
the observation; it does not describe the underlying process.

Physically, there are two possible scattering processes which can occur.  The
first is \textit{single scattering}, whereby only one scattering event occurs
in each path from source to observation.  Since scattering is a general
phenomena, paths may be traversed by any phase recording object: a photon,
SPP, electron, etc.; at the moment the object is not important.  In addition
to single scattering, there exists a much more interesting and
phenomenologically complex process known as \textit{multiple scattering}.  In
multiple scattering, each path from source to observation contains multiple
scattering events.  Ultimately, both single and multiple scattering produce
speckle described by the random phasor sum model.

%\begin{figure}[ht]
%\centering
%\import{includes/}{setpgfinc}
%\includegraphics[keepaspectratio,width=10cm]{scatteringmicro/figures/singlevmultiplescattering}
%\label{fig:scattsinglevmultiple}
%\end{figure}

The transition from the single to the multiple scattering occurs roughly upon
fulfillment of the Ioffe-Regel criterion~\cite{ioffe1960non}, when the
transport mean free path $l^*$ is much larger than the spatial frequency of
the optical wave, or
\begin{equation}
k l^* \approx 1
\end{equation}
where $k=\omega/c$ as usual.  The transport mean path is defined as the
characteristic distance over which the incident wave is scattered out of
its incoming direction~\cite{berkovits1994correlations}.  This is the
optical analog of electron transport in condensed matter physics.  There,
one would use the Fermi wavenumber $k_F$ and take $l^*$ to be the mean free
path.  Materials for which $k_F l^* \gg 1$ are conductors, and materials
for which $k_F l^* \ll 1$ are insulators.  The multiple scattering regime can
also be defined in terms of the sample length $L$, such that if
$l^* \ll L$ and $1/(k l^*) \ll 1$, the system can be thought to be the
multiple scattering regime.  

How does one determine whether a system is in the single or multiple
scattering regime?
