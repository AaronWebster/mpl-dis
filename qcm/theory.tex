\subsection{The Small Load Approximation}
A rigerous outline of QCM theory is well beyond the scope of this work.  Rather,
we begin much closer to the physics relevant to our new results with what
is known as the \textit{small load approximation}.  The small load
approximation is a statement that the for a QCM under load, the
\textit{change} in frequency, $\df$ and bandwidth (half-width at half
maximum of the resonance), $\dg$, of a QCM under load are computed by evaluating the
stress-speed ratio on the oscillating boundary, according to the
relationship
\begin{equation}
 \frac{\df+\mi\Delta\Gamma}{f_{\mathrm{F}}}=\frac{\mi}{\pi Z_{\mathrm{q}}}Z_\mathrm{L} =\frac{\mi}{\pi Z_{\mathrm{q}}}\left<\frac{\sigma}{\dot{u}}\right>
\label{eqn:comsolextract}
\end{equation}
where $Z_\mathbf{q}$ is the acoustic impedance of AT cut quartz, $\sigma$ is the stress, $\dot{u}$ is the
velocity, $Z_\mathrm{L}$ is the load impedance, and $\left<\enspace\right>$
denotes a line average along the boundary.  Note that the stress speed
ratio $\left<\sigma/\dot{u}\right>$ is dimensionally equivalent to specific
acoustic impedance, also called ``shock impedance'', sound pressure over
velocity.  In other words, the stress-speed ratio is the impedance of the
film.  Furthermore, in \Equation{eqn:comsolextract} we have used the
common convention of a complex frequency, where $\df$ represents the
maximum of the resonance and $\dg$ the half-width at half maximum of the
same resonance distribution.

The small load approximation is a perturbation of the main resonance, and
thus is only applicable when the impedance of the load is much smaller than
the impedance of the crystal, $Z_\mathrm{L} \ll Z_\mathrm{q}$, or
equivalently $\df \ll f_\mathrm{F}$.

\subsection{Mechanical Model}
\begin{figure}
\centering
\import{qcm/figures/}{qcmresponse.pdf_tex}
\caption{ Mass-spring-dashpot mechanical model for a QCM under load. In the
weak coupling regime the system experiences a positive frequency shift
identified with inertial loading, while in the strong coupling regime the
system experiences a negative frequency shift identified with mass loading.
The fixed relationship between the shift in frequency $\df$ and bandwidth
$\dg$ trace out a circle like path as a parametric function of $\kl$ .  }
\label{fig:mechanicalmodel}
\end{figure}

With the small load approximation in mind, we invoke a simple mechanical
model based on coupled oscillators.  The arrangement of the mechanical
model and the frequency response is shown in
\Figure{fig:mechanicalmodel}(a).
Here, the resonance of the quartz crystal
$\omega_\mathrm{q}^2=k_\mathrm{q}/m_\mathrm{q}$ is coupled to a sample load
with mass $m_\mathrm{L}$ though a parallel spring $k_\mathrm{L}$ and
dashpot $\xi_\mathrm{L}$ representing a Kelvin-Voigt viscoelastic material.
The spring and dashpot are not \textit{actual} springs and dashpots, rather
they are analogies for the coupling of two systems with different
resonances in the small load approximation analogous with $\ml\ll\mq$.  The
QCM itself has a small damping term, but it is small enough that we neglect
it.  The spring $\kl$ represents the ``stiffness'' of the contact -- how
strongly the particle is coupled to the QCM surface and the dashpot $\xil$
represents energy lost through the coupling.  

The coupled system in \Figure{fig:mechanicalmodel}, for displacements of
$\mq$ by $x_\mathrm{q}$ and $\ml$ by $x_\mathrm{L}$, is described by the
set of differential equations
\begin{align}
 \mq \ddot{x}_\mathrm{q} &= -\kq x_\mathrm{q} +\kl (x_\mathrm{q}-x_\mathrm{L}) + \xil (\dot{x}_\mathrm{q}-\dot{x}_\mathrm{L})\\
 \ml \ddot{x}_\mathrm{L} &= -\kl (x_\mathrm{q}-x_\mathrm{L}) - \xil (\dot{x}_\mathrm{q}-\dot{x}_\mathrm{L})
\end{align}

which, using an ansatz of
$x_\mathrm{q}A_\mathrm{q}$ and $x_\mathrm{L}=A_\mathrm{L}\me^{\mi \omega t}$
eigenvalues $\omega$ of
$N_\mathrm{L}$ is a number surface density (number per unit area) for discrete
loads.

Using the small load approximation, the response of the system
as a function of its coupling $\kl$ is
\begin{equation}
\frac{\Delta\!f + \mi \Delta \Gamma}{f_\mathrm{F}} = \frac{N_\mathrm{L}}{\pi
\mathrm{Z}_\mathrm{q}}
\frac{\ml \omega_\mathrm{q} \left( \kl + \mi
\omega_\mathrm{q} \xil\right) }
{\ml \omega_\mathrm{q}^2 - \left(\kl + \mi
\omega_\mathrm{q} \xil\right)}
\label{eqn:suppmastereq}
\end{equation}

The mechanical model is plotted in \Figure{fig:mechanical}(b), and 
has two important limits as a function of the contact
stiffness, $\kl$, known as \textit{strong} and \textit{weak} coupling.
These limits occur to the left and right of a zero crossing in $\df$ at
$k_\mathrm{zc}=\omegaq^2 \ml$.
%\vspace{-\baselineskip}
%\vspace{-\parskip}
\begin{align}
\frac{\df}{f_\mathrm{F}}&= 
\frac{N_\mathrm{L} \kl}
{\omegaq\pi \mathrm{Z}_\mathrm{q}}
&\,&\left(\text{weak,}\quad \kl\ll \ml
\omegaq^2\right)
\label{eqn:couplinguy1}
\\
\frac{\df}{f_\mathrm{F}}&=  -\frac{N_\mathrm{L}
\ml \omegaq}{\pi Z_\mathrm{q}}
&\,&\left(\text{strong,}\quad \kl\gg \ml
\omegaq^2\right)
\label{eqn:couplinguy2}
\end{align}
Note that $\dg$ goes to zero in either limit.

Strong coupling is identified with mass loading
(Sauerbrey~\cite{sauerbrey1959verwendung} behavior) and a \textit{negative}
frequency shift linearly proportional $\ml$.  This behavior is the one
which is most commonly associated with QCM measurements.  Physically this
situation is identified with a coupling rigid enough such that the particle
takes part in the oscillation of the QCM.  In the opposite limit is weak
coupling, also called inertial loading~\cite{dybwad1985sensitive}, and is
identified by a \textit{positive} frequency shift independent of the mass
and linearly proportional to $\kl$.  Here, the coupling is sufficiently
weak such that the particle remains at rest in the labratory frame.  It is
``clamped'' by its own inertia.~\cite{du2008role}

%% small load approximation -> other paper -> equivalent mechanical model

%As sensors, QCMs found their initial applications monitoring thin film
%deposition in vacuums.  Here it was found that a QCM would exhibit a
%frequency shift $\df$ proportional to the adsorbed mass $m$
%\begin{equation}
% \df=-\frac{2f_{F}^{2}}{A\sqrt{\rho_{q}\mu_{q}}}\Delta m
% \label{eqn:sauerbrey}
%\end{equation}
%where $A$ is the active area of the crystal, $\Delta m$ is the adsorbed
%mass, $\rho_{q}$ is the density of quartz and $\mu_{q}$ is the shear
%modulus of quartz. Usually \Equation{eqn:sauerbrey} is condenced such that
%$\Delta m$ is a linear function of a single ``sensitivity factor''
%$C_\mathrm{f}$
%\begin{equation}
% \Delta f=-C_{\text{f}}\Delta m
%\end{equation}
%where $C_{\text{f}}=\SI{56.6}{\hertz\per\micro\gram\centi\meter\squared}$.
%\Equation{eqn:sauerbrey} is known as the
%\textit{Sauerbrey relation}.  A typical QCM might have a
%$f_F=\SI{5}{\mega\hertz}$ and a frequency resolution of \SI{0.1}{\hertz};
%\Equation{eqn:sauerbrey} predicts sub-monolayer (and is only valid in this
%range) resolution of the thickness
%metal and dielectric layers.  This is the reason the name QCM includes the
%word ``microbalance''.
%
%Initially it was thought that the low $Q$ mechanical resonance of the QCM
%precluded its use in the liquid phase.  This was incorrect, as subsequent
%work published in 1985 by \name{Gordon} and \name{Kanazawa}~\cite{guys} extended the
%treatment of \name{Sauerbrey} to the liquid phase.  These relations,
%derived from a Butterworth van Dyke equivalent circuit, predicted the
%resonant frequency of the QCM
%to be sensitive to the density-viscosity product of the liquid in contact
%with the crystal, 
%\begin{align}
%\df=&-f_{\text{F}}^{3/2}\left(\frac{\rho_{\text{L}}\eta_{\text{L}}}{\pi\rho_{q}\mu_{q}}\right)^{1/2}\\
%\Delta R=&2f_{\text{F}}L\left(\frac{4\pi
% f_{\text{F}}\rho_{\text{L}}\eta_{\text{L}}}{\rho_{q}\mu_{q}}\right)^{1/2}
%\end{align}
%where $\rho_{\text{L}}$ and $\eta_{\text{L}}$ are the unknown density and
%viscosity of the liquid.  Note that in the treatment by both Sauerbrey and
%Gordon and Kanazawa, the frequency shifts are always \textit{negative} as a
%function of increasing mass or density-viscosity.
%
%Here, the quartz crystal with mass $\mq$ and stiffness
%$\kq$ resonates at
%$\omega_\mathrm{q}^2=\kq/\mq$ and is coupled by a spring
%$\kl$ to a load mass $\ml$.  The load mass is the actual mass of
%the particle, $4/3 \pi r^3 \rho$.  The spring $\kl$ represents the
%``stiffness'' of the contact -- how strongly the particle is coupled to the
%QCM surface.  The coupled system in \Figure{fig:dybwadschema} is described
%by the differential equations
%\begin{align}
% \mq \ddot{x}_\mathrm{q} &= -\kq x_\mathrm{q}\\
% \ml \ddot{x}_\mathrm{L} &= -\kl (x_\mathrm{q}-x_\mathrm{L})
%\end{align}
%which, using an ansatz of
%$x_{\mathrm{q},\mathrm{L}}=A_{\mathrm{q},\mathrm{L}}\me^{\mi \omega t}$ has
%eigenvalues $\omega$ of
%\begin{equation}
% 2\omega^{2}=\left(\frac{\kq}{\mq}+\frac{\kl}{\mq}+\frac{\kl}{m}\right)\pm\left(\left(\frac{\kq}{\mq}+\frac{\kl}{\mq}+\frac{\kl}{\ml}\right)^{2}-4\frac{\kq}{\mq}\frac{\kl}{\ml}\right)^{1/2}
% \label{eqn:dybwadresult}
%\end{equation}
