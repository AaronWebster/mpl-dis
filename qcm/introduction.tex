There are few experimental techniques that allow the application of force
on biological molecules. Among them, optical or magnetic tweezers and
atomic force microscopes (AFMs) have provided much insight into the
mechanics of DNA, RNA and
chromatin~\cite{felsenfeld1992chromatin}~\cite{cui2000pulling}~\cite{larson2012trigger}~\cite{marko1995stretching},
friction and wear in
proteins~\cite{suda2001origin}~\cite{bormuth2009protein}, and stepwise
motion of motor proteins~\cite{asbury2003kinesin}, all of which are
important for understanding disease.

However powerful, these methods are not able to be deployed as integrated
devices to probe the mechanical properties of heterogeneous samples in a
wide range of applications; tweezer and AFM experiments rely on highly
trained experimentalists, are not widely applicable as analytical tools,
and are often constrained to the analysis of well prepared homogeneous
samples not amenable to multiplexing.

Among direct mechanical transduction methods, the sensitivity, low cost,
and ease of use make the quartz crystal microbalance (QCM) ideal for real
time monitoring of biomechanical properties such as viscoelasticity,
conformal changes~\cite{fant2000adsorption}, and contact
rigidity~\cite{cooper2007survey}.  Naturally, these benefits
do not come without disadvantages. The underlying mechanical properties of
the sample are often not revealed by the stepwise changes in the QCM
sensorgram, an issue complicated by the choice of theoretical model.
Operation in the liquid phase is also associated with a rather low-Q
resonance QCM precluding their use for single molecule detection; up to now
it has not been possible to integrate force on biomolecules in QCM
measurements.

Among tools suitable for direct mechanical transduction, the quartz crystal
microbalance (QCM) has seen increasing real-world utility as simple, cost
effective, and highly versatile mechanical biosensing platforms. A QCM
typically presents itself as thin disk-shaped piece of strategically cut
piezoelectric quartz with electrodes on either side. When part of an
electronic oscillator circuit, the quartz can form a mechanical resonator
which vibrates at its fundamental frequencies. Changes in these frequencies
and their associated bandwidths upon sample adsorption or desorption are
related to the properties of the sample and the strength of its coupling to
the QCM. Since its introduction by Sauerbrey~\cite{sauerbrey1959verwendung}
in 1959 as sub-monolayer thin-film mass sensors in the gas phase, the
understanding of these piezoelectric devices has been repeatedly enhanced
to study phenomena such as viscoelastic films in the liquid
phase~\cite{kanazawa1985frequency}, non-destructive contact
mechanics~\cite{johannsman2007contacts}, and complex topologies of
biopolymers and biomacromolecules~\cite{marx2003quartz}. 

In light of these issues and the analytical power of force based
techniques, we introduce a novel type of instrument that uses a QCM as a
direct mechanical transducer for the response of biomolecules placed in a
variable force field provided by a standard commercial centrifuge. This
\textit{centrifugal force quartz crystal microbalance} (CF-QCM) concept is
about direct introduction of pico to nanoscale forces in the liquid phase
for analyzing the mechanical properties of biomaterials.
