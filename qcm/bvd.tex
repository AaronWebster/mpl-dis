\section{Driving Circuit}
The QCM driving circuit employed in our experiments was an SRS 
\subsection{Butterworth van Dyke Equivalent Circuit}
The typical circuit used to analyze QCM behavior is called the
\textit{Butterworth van Dyke} (BvD) circuit.  It consists of a capacitor
$C_s$, an inductor $L_s$, and a resistor $R_s$ in series with a parallel
capacitance $C_0$.
\begin{center}
 \begin{tikzpicture}[scale=0.75]
 \draw (1,0) node[anchor=east] {}
  to[short, o-*] (2,0)
  to[short] (2,1)
  to[R, l^=$R_s$] (4,1)
  to[C, l^=$C_s$] (6,1)
  to[L, l^=$L_s$] (8,1)
  to[short] (8,-1)
  to[short] (6,-1)
  to[C, l^=$C_0$] (4,-1)
  to[short] (2,-1)
  to[short] (2,0);
 \draw (9,0) node[anchor=west] {}
  to[short, o-*] (8,0);
\end{tikzpicture}
\end{center}

\begin{center}
 \begin{tikzpicture}[scale=0.75]
 \draw (0,1) node[anchor=east] {}
 to[short] (1,1)
 to[short] ++(1,1)
 to[L] (1,0)
 to[short] (0,-1);
\end{tikzpicture}
\end{center}



The top branch is the \textit{motional branch}, and relates to the crysal
and its interaction with the environment.  The bottom is the \textit{static
branch}, representing the paracitic capitances of the quartz and its driver.
In the SRS QCM200\cite{srsqcmmanual}, and probably any other similar
compensated phase locked oscillator circuit, $C_0$ is nulled with
additional circuitry.  This is absolutely crutial, as the parallel $C_0$ 
pertdubes the resonance frequency of the circuit by about
\SI{0.825}{\hertz\per\pico\farad}.  The SRS manual gives a higher value
of \SI{2}{\hertz\per\pico\farad}.

Typical values for the \SI{1}{in} \SI{5}{\mega\hertz} AT cut quartz
crystal used in with the QCM200\cite{srsqcmmanual} are

\begin{table}[h]
\begin{tabular}{ll}
 $R_s$ & \SI{400}{\ohm} (water), \SI{10}{\ohm} (air) \\
 $C_s$ & \SI{33}{\femto\farad} (SRS manual)\\
 $L_s$ & \SI{30}{\milli\henry} (SRS manual), \SI{40}{\milli\henry} (my prediction) \\
 $C_0$ & \SI{20}{\pico\farad} (SRS manual)
\end{tabular}
\end{table}

The circuit may be also be solved using the following second order linear
differential equation for charge
\begin{equation}
 L\ddot{q}+R\dot{q}+q/C = V(t)
\end{equation}
The natural frequency is 
\begin{equation}
 f_0 = \frac{1}{2 \pi L C}
\end{equation}
Where $C$ in the above equation takes into account both $C_0$ and $C_s$
\begin{equation}
 C = \frac{C_s C_0}{C_s + C_0}
\end{equation}

This, as well as more complicated equivalent circuit models can also be
computed directly using SPICE, as per the following code snippet
\begin{minted}{text}
* Butterworth van Dyke equivalent circuit
V1 0 1 ac 1 dc 0
Rs 1 2 100
R1 2 3 375
C0 2 0 20e-12
C1 3 4 33.0e-15
L1 4 0 30e-3
.control
ac lin 100000 5.052e6 5.063e6
write bvd.raw all
\end{minted}

Liquid sensing under a rigid coating is modeled with a modified Butterworth
van Dyke circuit as follows 

\begin{center}
 \begin{tikzpicture}[scale=0.75]
 \draw (1,0) node[anchor=east] {}
  to[short, o-*] (2,0)
  to[short] (2,1)
  to[R, l^=$R_\text{s}$] (4,1)
  to[C, l^=$C_\text{s}$] (6,1)
  to[L, l^=$L_\text{s}$] (8,1)
  to[L, l^=$L_\text{c}$] (10,1)
  to[L, l^=$L_\text{l}$] (12,1)
  to[R, l^=$R_\text{l}$] (14,1)
  to[short] (14,0);
  \draw (14,0) node[anchor=west] {}
  to[short] (14,-1)
  to[C, l^=$C_0$] (8,-1)
  to[short] (2,-1);
 \draw (14,0) node[anchor=west] {}
  to[short] (14,-3)
  to[C, l^=$C_\text{l}$] (6,-3)
  to[short] (2,-3)
  to[short] (2,0);

 %\draw (9,0) node[anchor=west] {}
 % to[short, o-*] (14,0);
\end{tikzpicture}
\end{center}



