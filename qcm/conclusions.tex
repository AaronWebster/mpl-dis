We have observed the QCM sensorgram under the influence of centrifugal force
for samples such as DNA monolayers, free discrete polystyrene particles,
and particles tethered to the QCM electrode with lambda DNA\@.  We present
simulations and a theoretical framework to interpret the QCM signals in the
context of the sample's properties.

The data presented thus far points to a potentially interesting avenue for
the investigation of force on biomolecules using a quartz crystal
microbalance.  In addition to the data discussed, we have also observed
other types of signals in some of our datasets within the overall trend
shown here, which we suspect may be related to ionic
transport~\cite{tolman1911electromotive}~\cite{des1893unpolarisirbare}, the
conformal state of DNA, and nonlinear viscoelastic behavior.  Objects such
as microparticles attached or tethered to a biopolymer on the QCM surface
become inertial transducers through which one can extract mechanical and
thermodynamic properties of the macromolecules. Furthermore, the technique
is applicable to microscopic biological objects such as viruses, bacteria,
and cells where measurements of mechanical properties and their changes
have been directly linked to
disease~\cite{merkel1989molecular}~\cite{yeri2009mutation}~\cite{tevet2011friction}.

The enhanced signal for most samples under centrifugal load points to a
interesting avenue of increasing the sensitivity of a state of the art QCM
biosensor.  This is true even with the present state of our instrument,
which is limited to low-g regimes when compared to other commercial
centrifuges. With operation below \SI{90}{g}, we have observed sensitivity
increases corresponding to changes of \SI{10}{\percent} in the
density-viscosity product for viscoelastic loads, and up to a factor of 10
increase in sensitivity for discrete particles.  However, there is no
technical reason why future incarnations could not spin much faster and
considerably clarify the CF-QCM sensorgram.  There are also possibilities in
using this platform in other related modalities such as the
nanotribological effects of sliding friction~\cite{krim1991nanotribology}
caused by orienting the crystal at an angle to the applied centrifugal
force, propelling biomolecules across the surface.

