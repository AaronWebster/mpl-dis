\section{What is New Here}
QCMs have been studied in the context of almost every type of sample
imaginable, but up to now the introduction and study of the behavior of
different load situations under uniform force gradients has not been
carried out.
The novel aspects of this work is the direct introduction of force
into\dots

\begin{description}
\item[{Acceleration Boundary Effects}] While the $kT$ energy of a
small biomolecule are orders of magnitude greater than gravity (energy,
gravity, huh?)
, acceleration effects are readily observable for hybridized
oligos and bufers stuff.  We confirm the 2g effect for oligos and provide
data into the 100g range, as well as confirm the effect for random buffer
solutions.
\item[{Contact Mechanics}] Pressing beads and looking at the response,
 maybe the equivalent model for contact stiffness in the liquid phase,
 which is just about shear stuffs.
\item[{one more thing so this list isn't empty}] That would be nice.
\end{description}

QCMs have been the subject of extensive study, especially in the context of
biosensing, and have for some time been implemented in commercial sensing
devices.  I will only cover areas related to new results here.  For a
comprehensive overview to the principles and applications of these
mechanical resonators, the reader is directed to \cite{steinemreview}.

