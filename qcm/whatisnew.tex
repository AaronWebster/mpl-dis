QCMs have been explored in an amazingly wide variety of situations.  Our
main contribution is the study of the response of a QCM for samples as a
function of varying centrifugal force fields.  For a sample in contact with
the crystal, this force may be in to or away from the face of the
crystal.  We call the specific combination of force orientation and sample
material a \textit{load situation}.  Thus, our main contribution is the
study of the following load situations:

\begin{description}
\item[{Acceleration Boundary Effects}] While the $kT$ energy of a
small biomolecule are orders of magnitude greater than gravity (energy,
gravity, huh?)
, acceleration effects are readily observable for hybridized
oligos and buffers stuff.  We confirm the 2g effect for oligos and provide
data into the 100g range, as well as confirm the effect for random buffer
solutions.
\item[{Contact Mechanics}] Pressing beads and looking at the response,
 maybe the equivalent model for contact stiffness in the liquid phase,
 which is just about shear stuffs.
\item[{Particle Sizing}]
\item[{Influence of Electrolytes}]
\item[{one more thing so this list isn't empty}] That would be nice.
\end{description}

For a comprehensive overview to the principles and applications of these
mechanical resonators, the reader is directed to \cite{steinemreview}.

