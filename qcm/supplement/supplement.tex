\documentclass[floatfix,superscriptaddress,a4paper,notitlepage]{revtex4-1}
\usepackage{t1enc}
\usepackage[T1]{fontenc}
\usepackage{textcomp}
\usepackage{lmodern}
\usepackage[utf8]{inputenc}
\usepackage{ulem}
\usepackage{import}
\usepackage{amsmath}
\usepackage{amssymb}
\usepackage{paralist}
\usepackage{booktabs}
\usepackage[version=3]{mhchem}
\usepackage{wrapfig}
\usepackage{pifont}
\usepackage{mathrsfs}
\usepackage{xcolor}
\usepackage{color}
\usepackage{enumitem}
\usepackage{datetime}
\usepackage{relsize}
\usepackage{tabularx}

% external references
\usepackage{xr}
\externaldocument{../manuscript-naturecomm}

\usepackage{pgfplots}
\usepackage{pgfplotstable}
\usetikzlibrary{pgfplots.groupplots}
\pgfplotsset{compat=newest}
\usepgfplotslibrary{units}
\usepgfplotslibrary{external}
\pgfplotsset{filter discard warning=false}

\usepackage{tikz}
\usetikzlibrary{pgfplots.external}
\usetikzlibrary{pgfplots.units}
\usetikzlibrary{calc}
%\tikzexternalize[prefix=external/]% externalize!

\usepackage{siunitx}
%\DeclareSIUnit\molar{\mole\per\cubic\deci\metre}
%\DeclareSIUnit\textsc{M}{\textsc{M}}

\usepackage{hyperref}
\hypersetup{
 colorlinks=false,
 hidelinks=true,
}

% New definition of square root:
% it renames \sqrt as \oldsqrt
% This definition puts a little vertical guy at the end so it's more
% obvious where the square root actually ends.
\let\oldsqrt\sqrt
% it defines the new \sqrt in terms of the old one
\def\sqrt{\mathpalette\DHLhksqrt}
\def\DHLhksqrt#1#2{%
\setbox0=\hbox{$#1\oldsqrt{#2\,}$}\dimen0=\ht0
\advance\dimen0-0.2\ht0
\setbox2=\hbox{\vrule height\ht0 depth -\dimen0}%
{\box0\lower0.4pt\box2}}

\newcommand{\Figure}[1]{FIG.~\ref{#1}}
\newcommand{\Equation}[1]{EQN.~\ref{#1}}
\newcommand{\Table}[1]{TBL.~\ref{#1}}
\newcommand{\Section}[1]{SEC.~\ref{#1}}
\newcommand{\Chapter}[1]{CH.~\ref{#1}}
\newcommand{\Appendix}[1]{APPENDIX~\ref{#1}}
\newcommand{\Ref}[1]{REF.~\cite{#1}}

% use roman type for natural base e and sqrt(-1)
\newcommand{\me}{{\mathrm{e}}}
\newcommand{\mi}{{\mathrm{i}}}
\DeclareMathOperator{\sgn}{sgn}

\definecolor{colora}{RGB}{24,90,169}
\definecolor{colorb}{RGB}{238,46,47}
\definecolor{colorc}{RGB}{0,140,72}
\definecolor{colord}{RGB}{244,125,35}
\definecolor{colore}{RGB}{61,90,153}
\definecolor{colorf}{RGB}{102,44,145}

\definecolor{tangoorange}{RGB}{245,121,0}

% shortcut for comsol
\newcommand{\comsol}{\texttt{COMSOL}}


% shortcut for a column vector
\newcount\colveccount
\newcommand*\colvec[1]{
        \global\colveccount#1
        \begin{pmatrix}
        \colvecnext
}
\def\colvecnext#1{
        #1
        \global\advance\colveccount-1
        \ifnum\colveccount>0
                \\
                \expandafter\colvecnext
        \else
                \end{pmatrix}
        \fi
}

\newcommand{\df}{\Delta\!f}
\newcommand{\dg}{\Delta\Gamma}
\newcommand{\xil}{\xi_\mathrm{L}}
\newcommand{\kl}{k_\mathrm{L}}
\newcommand{\ml}{m_\mathrm{L}}
\newcommand{\kq}{k_\mathrm{q}}
\newcommand{\mq}{m_\mathrm{q}}
\newcommand{\Rm}{R_\mathrm{m}}
\newcommand{\Ac}{A_\mathrm{c}}
\newcommand{\Lm}{L_\mathrm{m}}
\newcommand{\omegaq}{\omega_\mathrm{q}}

% import colorbrewer2 color library
\import{colors/}{colors}

% stops errors when making table of contents
\usepackage{etoolbox}
\makeatletter
\let\ams@starttoc\@starttoc
\makeatother
\usepackage[parfill]{parskip}
\makeatletter
\let\@starttoc\ams@starttoc
\patchcmd{\@starttoc}{\makeatletter}{\makeatletter\parskip\z@}{}{}
\makeatother

\renewcommand\thesection{S\arabic{section}}
\renewcommand\thesubsection{S\arabic{section}.\arabic{subsection}}
\renewcommand\thesubsubsection{S\arabic{section}.\arabic{subsection}.\arabic{subsubsection}}

\renewcommand\thefigure{S\arabic{figure}}
\renewcommand\theequation{S\arabic{equation}}
\renewcommand\thetable{S\arabic{table}}


\newcommand{\todo}[1]{%
 \textcolor{tangoorange}{#1}
}

\begin{document}

\title{Probing biomechanical properties with a centrifugal force quartz
 crystal microbalance\\ Supplementary Information}
\author{Aaron~Webster}
\affiliation{Max Planck Institute for the Science of Light, Erlangen D-91058, Germany}
\author{Frank~Vollmer}
\affiliation{Max Planck Institute for the Science of Light, Erlangen D-91058, Germany}
\author{Yuki~Sato}
\affiliation{The Rowland Institute at Harvard, Harvard University, Cambridge, Massachusetts 02142, USA}
\date{\today}

%\tikzexternaldisable
%\tikz[overlay,remember picture] {
% \node at ($(current page.west)+(1,0)$) [rotate=90]
% {\large\textcolor{gray}{\input{commithash} \pdfdate}};
%}
%\tikzexternalenable

\maketitle

\tableofcontents

%\todo{Hey, in the supplement, should we have some stuff on calculating the
%contact surface density $A_c$ that we use in the manuscript?
%Also if you already have any derivation-related things for Equations
%(1)-(3), those should also be added to the supplement.}
\section{Introduction}
This document is supplementary information to
\textit{Probing biomechanical properties with a centrifugal force quartz
crystal microbalance}.  Herein we give details on numerical simulations and
further derivations which are referenced in the main text.  Unless
specified otherwise, all section, figure, table, and equation references
pertaining to the supplement are prefixed by an ``S'', while references to
the corresponding manuscript are not prefixed.

\section{Simulation Details}
Finite element simulations were carried out using the software
\texttt{COMSOL Multiphysics 4.3 (4.3.0.233)}~\cite{multiphysics1994comsol}.
Though this software's source code is not available for
scrutinous review, our implementation is generic and may be carried out
using other software (e.g.\ \texttt{OpenFOAM}~\cite{jasak2007openfoam},
\texttt{SU2}~\cite{palacios2013stanford}).  Unless otherwise stated,
implementation specific information is applicable to \comsol.

The simulation is done by solving the steady state incompressible Navier
Stokes equations, neglecting turbulence, using finite element analysis in
two dimensions
\begin{align}
 \rho\left(\mathbf{\dot{u}}\cdot \nabla\right)\mathbf{\dot{u}}
 &=\nabla \cdot \left( -\rho \mathbf{I} + \eta \left(\nabla \mathbf{\dot{u}} +
 \left( \nabla \mathbf{\dot{u}}\right)^\mathrm{T}\right)\right) + \mathbf{F}\\
 \rho \nabla \cdot \mathbf{\dot{u}} &= 0
\end{align}
where $\mathbf{\dot{u}}$ is flow velocity field, $\rho$ is fluid density,
$\eta$ is dynamic viscosity, and $\mathbf{F}$ is the body force per unit
volume.  The computational domain is set up as shown in
\Figure{fig:compgeometry}.
\begin{figure}[h]
 \centering
 \import{figures/}{geometry.pdf_tex}
 \caption{Computational geometry for the simulation.}
\label{fig:compgeometry}
\end{figure}

The left hand side (
\tikz[baseline=-0.75ex]{
 \draw [dash pattern=on 3pt off 2pt on 1pt off 2pt, thick] (0,0) -- (0.5,0);
}) %
is a sliding wall and is given a tangential velocity of $\dot{u}_\perp = \mi
\omega u_0$, where $\omega=2\pi f$ is the angular frequency of oscillation
and $u_0=\SI{1e-2}{\nano\meter}$ is its amplitude.  The top and bottom
(%
\tikz[baseline=-0.75ex]{
 \draw [dash pattern=on 3pt off 1pt, thick] (0,0) -- (0.5,0);
}%
) are given a periodic flow condition such that their pressure difference
is zero.  The right hand side (
\tikz[baseline=-0.75ex]{
 \draw [dash pattern=on 1pt off 1pt, thick] (0,0) -- (0.5,0);
}) %
has a zero-slip condition.  The two materials 1, the particle, and 2, the
medium in domains $D_1$ and $D_2$ are assigned a volume force $\mathbf{F}=\mi \omega \rho
\mathbf{\dot{u}}$,
where $\rho$ is the density of the material, (e.g.\ $\rho_1 =
\SI{1.06}{\gram\per\centi\meter\cubed}$ for polystyrene and $\rho_2 =
\SI{1}{\gram\per\centi\meter\cubed}$ for water).  Finally. an initial pressure point
constraint of $p_0=0$ is assigned to the point in the bottom right of the
domain.

The sphere (or, more appropriately, cylinder in 2D) comprising $D_1$ with
diameter $d$ and radius $r$ is located a distance $s>-r$, measured from the
bottom of the sphere, from the oscillating
boundary.  If $s>0$, the sphere does not make contact with the boundary.
If $s\leq0$, the sphere is truncated at the boundary resulting in a finite
contact radius $r_\mathrm{c}$.  It is useful to sample in either domain, so
we either sweep $r_\mathrm{c}$ or $s$, converting between them with
\begin{align}
 r_\mathrm{c}\left(s\right) &= \sqrt{2r-s}\sqrt{s}\\
 s\left(r_\mathrm{c}\right) &= \left(r-\sqrt{r^2-r_\mathrm{c}^2}\right)\sgn\left(r_\mathrm{c}\right)
\end{align}
The number density $N_\mathrm{L}$ was controlled by increasing or
decreasing the height of the domain proportional to the size of the sphere.

The materials in the simulation are assigned a complex dynamic viscosity
$\eta = \eta'-\mi\eta''$.  This is related to the complex bulk modulus
$G=G'+\mi G''$ by
\begin{align}
 \eta = \frac{G}{\mi \omega}
\end{align}
and the loss tangent, the angle between the real and imaginary components
is
\begin{equation}
 \tan \delta = \frac{G''}{G'}
\end{equation}

Specific to the stationary solver in \comsol, in the
\texttt{Study$\rightarrow$Stationary Solver$\rightarrow$Advanced} window,
``Allow complex-valued output from functions with real input'' was checked.
This, coupled with the complex valued input for the body forces, will
produce shear waves in the simulation.

The mesh settings were calibrated in \comsol~for fluid dynamics with a
maximum mesh size of \SI{1e-9}{\meter} along the oscillating boundary.  All
other meshes were generated automatically.  As for the size of the
computational domain, we find that a height (parallel to the oscillating
boundary) of $h=2d$ and width (perpendicular to the oscillating boundary)
$w=2d$, minimum of \SI{500}{\nano\meter}, produces consistent results with a
minimum of error and computational resources.

It is important to note that, because the simulation is two-dimensional,
what is actually simulated are infinite cylinders rather than spheres.  For
Sauerbrey type viscoelastic films, the influence of this is negligible.
However, for asperity contacts such as spheres, we find that the results,
however qualitatively correct, do not always result in exact numerical
agreement with experiment~\cite{Vittorias2010489}.  An extended simulation
in three dimensions at some point is warranted.

\subsection{Contact Surface Density}

In the main manuscript we plot shifts in $\df$ and $\dg$ as a function of a
parameter called \textit{contact surface density}, $\Ac$.  This allows one
to compare shifts between particles with different radii.  To be clear, we
define $\Ac$ as the ratio of the area of the sphere in contact with the
oscillating boundary (twice the contact radius in 2D) to the area (length
in 2D) of the domain.  This means, for example, that for a domain with
$h=2d$, the maximum contact area will be $d$ and the maximum value of
$\Ac=0.5$

\subsection{Extracting Shifts}
Shifts in frequency, $\df$, and bandwidth (half-width at half maximum),
$\dg$, are computed by evaluating the stress-speed ratio of the oscillating
boundary according to the relationship
\begin{equation}
 \frac{\df+\mi\Delta\Gamma}{f_{\mathrm{F}}}=\frac{\mi}{\pi Z_{\mathrm{q}}}Z_\mathrm{L} =\frac{\mi}{\pi Z_{\mathrm{q}}}\left<\frac{\sigma}{\dot{u}}\right>
\label{eqn:comsolextract}
\end{equation}
where $\sigma$ is the (complex) stress, $\dot{u}$ is the (complex)
velocity, $Z_\mathrm{L}$ is the load impedance, and $\left<\enspace\right>$
denotes a line average along the boundary.  In \comsol, and in the
coordinates of \Figure{fig:compgeometry}, $\sigma$ is
\texttt{Total stress, y component} (\texttt{v}) and $\dot{u}$ is
\texttt{Velocity field, y component}, (\texttt{spf.T\_stressy}).  Note that
the stress speed ratio $\left<\sigma/\dot{u}\right>$ is dimensionally equivalent to
specific acoustic impedance, also called ``shock impedance'', sound
pressure over velocity.  In other words, the stress-speed ratio is the
impedance of the film.

\subsection{Verification Examples}
Here we check the validity of the numerical simulation with examples for
which the QCM's response is well known.

\subsubsection{Evanescent Shear Wave}
Setting up the model as described produces a shear wave which closely
matches theory, as shown in \Figure{fig:suppshearwave}.  An analytic expression
of the evanescent shear wave in a liquid has been reported~\cite{steinem2007piezoelectric} to be
\begin{equation}
 \frac{u(z)}{u_0} = \exp\left(-\sqrt{\frac{\mi \rho \omega}{\eta}} z\right)
\end{equation}
where $\eta=\eta'-\mi\eta''$ is the complex viscosity, $\rho$ is the density
of the liquid, $\omega$ is the angular frequency of the QCM oscillation,
and $z$ is the spatial extension.  The $1/\me$ penetration depth $\delta$ is
\begin{equation}
 \delta =
 -\left(\Im\left(\sqrt{\frac{\rho\omega}{\mi\eta}}\right)\right)^{-1}
\label{eqn:suppshearwavedelta}
\end{equation}

With $\rho=\SI{1}{\gram\per\centi\meter\cubed}$ and
$|\eta|^2=\SI{1}{\milli\pascal\second}$, \Equation{eqn:suppshearwavedelta}
predicts
$\delta\approx\SI{252}{\nano\meter}$.  This is in good agreement with the
simulation data, shown in \Figure{fig:suppshearwave}.

\begin{figure}[h]
 \centering
\pgfplotsset{
 minor tick num=3,
 footnotesize,
 every
 axis/.style={
  xmin=0,xmax=1e-6,ymin=-0.1,ymax=1.1,
  width=0.71\textwidth,height=0.45\textwidth,
  xticklabel style={/pgf/number format/fixed,
                     /pgf/number format/precision=1},
 },
 xlabel = $z$ distance,
 ylabel = $\|\mathbf{\dot{u}}\|$,
 x unit = \si{\micro\meter},
 y unit = a.u.,
 max space between ticks=1000pt,
}
\begin{tikzpicture}[baseline]
\begin{axis}
 \addplot [color=colora, mark=o, each nth point={5}, only marks ] file {data/evanescentwave_water_comsol.dat};
 \addlegendentry{simulation}
 \addplot [color=colora,] file {data/evanescentwave_water_theory.dat};
 \addlegendentry{theory}
\end{axis}
\end{tikzpicture}
\caption{QCM shear wave decay at \SI{5}{\mega\hertz}, comparison between
simulation and theory.}
\label{fig:suppshearwave}
\end{figure}

\subsubsection{Semi-Infinite Viscoelastic Medium}
A semi-infinite medium will produce a complex response described
by~\cite{kanazawa1985frequency}~\cite{martin1991characterization}
\begin{align}
 \frac{\df+\mi\dg}{f_\mathrm{F}}&=\frac{\mi}{\pi Z_\mathrm{q}}\sqrt{\rho G}\\
                                &= \frac{1}{\pi Z_\mathrm{q}}\frac{\left(-1+\mi\right)}{\sqrt{2}}\sqrt{\omega \rho \eta}
\label{eqn:impeq}
\end{align}

%\Equation{eqn:comsolextract} is equivalent to
$f=\omega/(2\pi)$ is the frequency of the crystal, $\rho$ and $\eta$ are the density and viscosity
of the medium in contact with the crystal, and $\rho_\mathrm{q}$ and
$\mu_\mathrm{q}$ are the density and shear modulus of quartz.  This model
applies for crystals with one side in contact with the viscoelastic
material.  This is related to the dynamic viscosity and shear
modulus by
\begin{align}
\frac{\df+\mi\Delta\Gamma}{f_{\mathrm{F}}}&=\frac{\mi}{\pi Z_{\mathrm{q}}}
\frac{\left(-1+\mi\right)}{\sqrt{2}}\sqrt{\rho\omega\left(\eta'-\mi\eta''\right)}\\
&=\frac{\mi}{\pi Z_{\mathrm{q}}}\sqrt{\rho\left(G'+\mi G''\right)}
\label{eqn:viscoshear}
\end{align}
The simulation geometry was set as described in \Figure{fig:compgeometry}
but without $D_1$ (no sphere).  Extracted values of $\df$ and $\dg$ are
presented in \Figure{fig:viscosweep} as a function of the viscosity $\eta$
of medium 1.  In \Figure{fig:viscosweep}(a) we sweep $\eta'$ for a
Newtonian fluid, $\eta''=0$.  In \Figure{fig:viscosweep}(b) we model a
non-Newtonian sample; $\eta'=\SI{1}{\milli\pascal\second}$ and $\eta''$ is
swept.  The excellent agreement with theory demonstrates that our
simulation is applicable for a wide range of materials.
\begin{figure}[h]
\centering
 \pgfplotsset{
  minor tick num=3,
  footnotesize,
  legend style={font=\footnotesize},
  every axis/.style={
   height=0.50\textwidth,
   width=0.50\textwidth,
   thick,
   %ymin = -0.0003,
   %ymax =  0.0003,
   %xmin = -0.0001,
   %xmax = 0.0038,
  },
  max space between ticks=50pt,
 }
 \begin{tabular}{cc}
 \begin{tikzpicture}[baseline]
  \pgfplotstableread{data/gksimulation.dat}{\datatablea}
  \pgfplotstableread{data/gktheory.dat}{\datatableb}
  \begin{axis}[
    xlabel = real viscosity $\eta'$,
    x unit = \si{\pascal\second},
    scaled y ticks=base 10:-3,
    ylabel = {$\df$, $\dg$},
    y unit = \si{hertz},
    ylabel absolute,
    restrict x to domain={-0.01:0.6e-2},
    ylabel style={
      %at={(yticklabel* cs:0.25)},
      %xshift=36pt,
      anchor=center,
     },
   ]
   %\addplot table [ y expr=\thisrowno{1} ] {\datatablea};
   \addplot [color=colora, mark=,] table [ y expr=\thisrowno{1} ] {\datatableb};
   \addplot [color=colorb, mark=, densely dashed] table [ y expr=\thisrowno{2} ] {\datatableb};

   \addplot [color=colora, mark=o, only marks] table [ y expr=\thisrowno{1} ] {\datatablea};
   \addplot [color=colorb, mark=o, only marks] table [ y expr=\thisrowno{2} ] {\datatablea};

   \draw [color=gray,dashed,semithick] (axis cs:-0.0001,0) -- (axis cs:6e-2,0);
   \node[anchor=north west] at (yticklabel* cs:1) {(a)};

  \end{axis}
 \end{tikzpicture}
 &
 \begin{tikzpicture}[baseline]
  \pgfplotstableread{data/gksimulation2.dat}{\datatablea}
  \pgfplotstableread{data/gktheory2.dat}{\datatableb}
  \begin{axis}[
    xlabel = imaginary viscosity $\eta''$,
    x unit = \si{\pascal\second},
    scaled y ticks=base 10:-3,
    ylabel = ,
    y unit = ,
    ylabel absolute,
    %restrict x to domain={0:0.6e-2},
    legend to name=named,
    legend columns=-1,
%    legend entries={%
%     $\df$ theory,
%     $\dg$ theory,
%     $\df$ simulation,
%     $\dg$ simulation,
%    },
    ylabel style={
      %at={(yticklabel* cs:0.25)},
      %xshift=36pt,
      anchor=center,
     },
   ]
   \addplot [color=colora, mark=,] table [ y expr=\thisrowno{1} ] {\datatableb};
   \addlegendentry{$\df$ theory~~}
   \addplot [color=colorb, mark=, densely dashed] table [ y expr=\thisrowno{2} ] {\datatableb};
   \addlegendentry{$\dg$ theory~~}

   \addplot [color=colora, mark=o, only marks] table [ y expr=\thisrowno{1} ] {\datatablea};
   \addlegendentry{$\df$ simulation~~}
   \addplot [color=colorb, mark=o, only marks] table [ y expr=\thisrowno{2} ] {\datatablea};
   \addlegendentry{$\dg$ simulation~~}

   \draw [color=gray,dashed,semithick] (axis cs:-0.001,0) -- (axis cs:6e-2,0);
   \node[anchor=north west] at (yticklabel* cs:1) {(b)};

  \end{axis}
 \end{tikzpicture}
 \\[1.5cm]
 \multicolumn{2}{c}{ \ref{named} }
\end{tabular}
\caption{Comparison of $\df$ and $\dg$ versus viscosity $\eta$ for both the
 finite element simulation and \Equation{eqn:impeq}.  (a) $\eta''=0$ and
 $\eta'$ is swept (a Newtonian liquid).  (b)
$\eta'=\SI{1}{\milli\pascal\second}$ and $\eta''$ is swept.}
\label{fig:viscosweep}
\end{figure}

It is of note that our Navier-Stokes approach (which solves for
$\dot{\mathbf{u}}$) does not converge in the limit of a perfectly elastic
material, e.g.\ $G=G'$ or $\eta=\eta''$; these systems are typically
solved for $\mathbf{u}$.  It is perhaps possible to couple these two
domains, but we have not attempted to do so.

\section{Mechanical Model}
We have employed a mechanical model based on coupled oscillators shown in
\Figure{fig:mechanicalmodel}.  Here the resonance of the QCM at
$\omegaq^2=\kq/\mq$ is coupled to a mass $\ml$ through a spring $\kl$ and
dashpot $\xil$.  The spring and dashpot are not \textit{actual} springs and
dashpots, rather they are analogies for the coupling of two systems with
different resonances in the small load approximation.  The QCM itself has a
small damping term, but it is small enough that we neglect it.
\begin{figure}[ht]
 \centering
 \import{figures/}{dybwadmodel.pdf_tex}
 \caption{Coupled oscillator mechanical model for QCM behavior.}
\label{fig:mechanicalmodel}
\end{figure}

The derivation of this equation is described in detail in
\Ref{olsson2012probing}.  Here we use the same equation but as a function
of $\kl$, rather than that of the load resonance
$\omega_\mathrm{L}^2=\kl/\ml$; the former being more appropriate to our
analysis.  Using the small load approximation, the response of the system
as a function of its coupling $\kl$ is
\begin{equation}
\frac{\Delta\!f + \mi \Delta \Gamma}{f_\mathrm{F}} = \frac{N_\mathrm{L}}{\pi
\mathrm{Z}_\mathrm{q}}
\frac{\ml \omega_\mathrm{q} \left( \kl + \mi
\omega_\mathrm{q} \xil\right) }
{\ml \omega_\mathrm{q}^2 - \left(\kl + \mi
\omega_\mathrm{q} \xil\right)}
\label{eqn:suppmastereq}
\end{equation}
where $\mathrm{Z}_\mathrm{q}$ is the acoustic impedance of AT cut quartz,
$f_\mathrm{F}$ is the fundamental frequency of the resonator, and
$N_\mathrm{L}$ is a number surface density (number per unit area) for discrete
loads.

\Figure{fig:supplowersphere} shows a comparison between the finite element
simulation and a best fit of \Equation{eqn:suppmastereq}.  Here the only
free system parameter is $\xil$, which we find to be approximately
\SI{7.5e-7}{\newton\second\per\meter}.  The domain of the fit is set by
$A_\mathrm{c}=0$ and the frequency zero crossing at
$k_\mathrm{zc}=\ml\omegaq^2$.

Fitting is simplified with a parametric plot of $\dg$ versus $\df$. For hard
spheres this will form a circle with radius $r_\mathrm{L}$,
\begin{align}
\frac{r_\mathrm{L}}{f_\mathrm{F}} &=
\frac{N_\mathrm{L}}{\pi \mathrm{Z}_\mathrm{q}}
\frac{\left(m_\mathrm{L}\omega_\mathrm{q}\right)^2}{2 \xi_\mathrm{L}}
\sqrt{1+\left(\frac{\xi_\mathrm{L}}{m_\mathrm{L}\omega_\mathrm{q}}\right)^2}\\
&\approx \frac{N_\mathrm{L}}{\pi \mathrm{Z}_\mathrm{q}}
\frac{\left(m_\mathrm{L}\omega_\mathrm{q}\right)^2}{2 \xi_\mathrm{L}}, \quad \mathrm{for}\; \xi_\mathrm{L} \ll 1
\label{eqn:suppradius}
\end{align}

\begin{figure}[h]
 \centering
\pgfplotsset{
 minor tick num=3,
 footnotesize,
 custom/.style={thick},
 every axis/.style={  height=0.45\textwidth,width=0.71\textwidth, },
}
\begin{tikzpicture}[ baseline ]
 \pgfplotstableread{../figures/data/lowersphere10umnew.dat}{\datatablea}
 \pgfplotstableread{../figures/data/lowersphere10umnewfit.dat}{\datatableb}
 \begin{axis}[%
   %7.5000e-07
   xmin=-576.2676,xmax=1.9237e+03,
   xticklabels={,,0,500,1000,1500},
   ymax=1,ymin=-0.6,
   axis x line*=top,
   axis y line*=right,
   ytick=\empty,
   yticklabels={,,},
   max space between ticks=40pt,
   x unit=\si{\newton\per\meter},
   xlabel=coupling $k_\mathrm{L}$,
  ]
  \draw [color=gray,dashed,semithick] (axis cs:0,1) -- (axis cs:0,-0.6);
 \end{axis}
 \begin{axis}[
 %height=0.45\textwidth,width=0.71\textwidth,
 %enlargelimits=false,
 xmin=-0.12,xmax=0.40,
 ymax=1,ymin=-0.6,
 %ymin=-3.25e5,ymax=8.5e5,
 xlabel=contact surface density $A_\mathrm{c}$,
 axis x line*=bottom,
 x unit=-,
 ylabel={$\Delta\!f/N_\mathrm{L}$, $\Delta\Gamma/N_\mathrm{L}$},
 y unit = \si{\hertz\meter\squared},
 %restrict x to domain = {-1e-6:2e-6},
 max space between ticks=50pt,
 %scaled x ticks = {real:1e-6},
 %xticklabel style={/pgf/number format/fixed},
 %every x tick scale label/.style={color=white},
 legend pos = north east,
 %ylabel style={yshift=-10pt},
 thick,
 legend style={font=\footnotesize},
 ]
 % \newcommand\YA{5e5}
 % \newcommand\YB{-3e5}
  %\addplot [smooth,color=colora] table [ y expr=\thisrowno{1} ] {\datatablea};
  %\addlegendentry{$\Delta\!f/N_\mathrm{L}$, \SI{1}{\micro\meter}}
  %\addplot [smooth,color=colorb,
  % dash pattern={on 3pt off 0.5pt on 1pt off 0.5pt},
  %] table [ y expr=\thisrowno{2} ] {\datatablea};
  %\addlegendentry{$\Delta\Gamma/N_\mathrm{L}$, \SI{1}{\micro\meter}}

  \addplot [custom,smooth,color=colora] table [ y expr=\thisrowno{1} ] {\datatablea};
  \label{p1}
  \addlegendentry{$\Delta\!f/N_\mathrm{L}$, sim}
  \addplot [custom,smooth,color=colorb, dash pattern={on 3pt off 0.5pt}, ] table [ y expr=\thisrowno{2} ] {\datatablea};
  \label{p2}
  \addlegendentry{$\Delta\Gamma/N_\mathrm{L}$, sim}
  \addplot [custom,smooth,color=colora,only marks,mark=o,mark size=1pt] table [ y expr=\thisrowno{1} ] {\datatableb};
  \label{p3}
  \addlegendentry{$\Delta\!f/N_\mathrm{L}$, theory}
  \addplot [custom,smooth,color=colorb,only marks,mark=square,mark size=1pt] table [ y expr=\thisrowno{2} ] {\datatableb};
  \label{p4}
  \addlegendentry{$\Delta\!f/N_\mathrm{L}$, theory}

  %\node [draw,fill=white,anchor=south west] at (rel axis cs: 0.01,{0.01*1.25}) {\shortstack[l]{
  %  \raisebox{-1.5pt}{\ref{p1}} \tiny $\Delta\!f/N_\mathrm{L}$, sim\\
  %  \raisebox{-1.5pt}{\ref{p2}} \tiny $\Delta\Gamma/N_\mathrm{L}$, sim }};
%
%    \node [draw,fill=white,anchor=south east] at (rel axis cs: 0.99,{0.01*1.25}) {\shortstack[l]{
%    \raisebox{-1.5pt}{\ref{p3}} \tiny $\Delta\!f/N_\mathrm{L}$, theory\\
%    \raisebox{-1.5pt}{\ref{p4}} \tiny $\Delta\Gamma/N_\mathrm{L}$, theory}};

 \end{axis}
\end{tikzpicture}
\caption{Comparison between mechanical model and finite element simulation
for a \SI{10}{\micro\meter} polystyrene particle.}
\label{fig:supplowersphere}
\end{figure}

\section{Noise and Comparison to QCM-D}
\label{sec:suppqcmdcomp}
Our experimental setup used a \SI{25}{\milli\meter} diameter
\SI{5}{\mega\hertz} gold coated crystal in combination with an SRS QCM200
PLL based driver circuit and an external rubidium frequency standard.
Typical of most QCM circuits, the QCM200 provides an output proportional to
$\df$, which we use directly in all discussions of $\df$.  However, unlike
a QCM-D device which gives a ``dissipation factor'', $D$, defined in
terms of the bandwidth $\dg$ as $D=2\dg/f_\mathrm{F}$, the QCM200
outputs the motional resistance $\Rm$ of the Butterworth van Dyke
equivalent circuit.  $\Rm$ is, related to
the bandwidth $\Gamma$ and the QCM-D dissipation $D$ by
\begin{align}
 \Rm&=\left(4 \pi \Lm\right) \Gamma\\
 &=\left(2 \pi \Lm f_\mathrm{F}\right) D
\end{align}
where $\Lm$ is the Butterworth van Dyke equivalent motional inductance.
Because of the small load approximation, $\df/f_\mathrm{F} \ll 1$ and
likewise $\Delta L/\Lm \ll 1$, we can effectively treat $\Lm$ as a
constant.~\cite{geelhood2002transient}
This value is
typically in the range of
\SI{30}{\milli\henry}~\cite{srsqcm200manual}~\cite{hussain2005ots} to
\SI{40}{\milli\henry}~\cite{gottschling2000detection}~\cite{arnau2002circuit}~\cite{snellings2001response},
with \SI{40}{\milli\henry} being more common and the value we use in our
analysis.  At $\Lm=\SI{40}{\milli\henry}$, we find in water
$\Rm=\SI{359}{\ohm}$, which is within \SI{1}{\percent} of the predicted
value~\cite{kanazawa1985frequency} of \SI{357}{\ohm}.  In this sense,
dissipation $D$ and motional resistance $\Rm$ are independent but
equivalent measures of the QCM bandwidth.

We have calculated the noise in $\df$ and $\Rm$ for the SRS QCM200 in our
experiment and find it to be about \SI{0.4}{\hertz} (\SI{0.008}{ppm}) for
$\df$ and \SI{0.006}{\ohm} (\SI{13}{ppm}) for $\Rm$, corresponding to a
signal to noise ratio of \SI{110}{\decibel}.  This is close to the
manufacturer's specification~\cite{srsqcm200manual} of \SI{0.1}{\hertz} for
$\df$ and $\pm\SI{28}{ppm}$ for $\Rm$.  We have analyzed the noise
separately for both the loading and unloading orientations of the crystal,
as well as for different centrifuge spin speeds.  We find no discernible
difference in the noise between any of these cases.  We did not at any
point modify the centrifuge or bucket assembly in an attempt to try and
reduce system noise.

In comparison, a typical QCM-D such as those sold by
Q-Sense~\footnote{BiolinScientific / Q-Sense, Hängpilsgatan 7, SE-426 77
Västra Frölunda, Sweden,  \url{http://www.q-sense.com/}} will have noise of
about \SI{0.3}{\hertz} in $\df$ and \num{0.2e-6} in $D$ at
\SI{5}{\mega\hertz}~\cite{su2005comparison}~\cite{peh2007understanding}.
Converting from $\Rm$ to $D$ and vice-versa, we find the SRS QCM200 has an
equivalent noise in $D$ of \num{0.005e-6} and the Q-Sense QCM-D an
equivalent noise in motional resistance of \SI{0.25}{\ohm}.  In terms of
$\dg$, the Q-Sense QCM-D has a noise of \SI{0.5}{\hertz} and the SRS QCM200
\SI{0.01}{\hertz}.  Even though we stated that our chosen value for $\Lm$
gives $\Rm$ within \SI{1}{\percent} of the predicted value for water,
uncertainties in the value of $\Lm$ used for the $\Rm$-$\dg$ conversion do
not significantly affect this analysis for the range of $\Lm$ values quoted
in the literature.

It is clear with this comparison that the SRS QCM200 PLL based driver and a
QCM-D device are both measures of the same underlying physical phenomena
taking place in a resonating quartz crystal.~\cite{geelhood2002transient}
Moreover, as exemplified by the discrete particle data in the main text,
the QCM signal amplitudes increase further with the application of
centrifugal force for the device reported here.  It is important to note
that the important aspect of the CF-QCM, the centrifugal force, is
independent of the type of technique used to drive and monitor the quartz
crystal.  We therefore see no reason why our technique would not apply to
all QCM based measurement techniques.

\section{Parametric Representation}
The different load situations depicted in the main manuscript are shown
parametrically in \Figure{fig:suppparametricplot}.

\newcommand{\myplot}[7]{
\nextgroupplot[title={\textbf{#1}}]
\addplot [
 color=colorc,
 mark=halfcircle*,
 mark size=1.5pt,
] table [x index=1, y index=5] {data/#2-both.txt};
% unloading
\addplot [
 color=colorc,
 mark=halfcircle*,
 mark size=1.5pt,
 mark options={rotate=180},
] table [x index=1, y index=5] {data/#3-both.txt};
\addplot [
 color=colorc,
 mark=halfcircle*,
 mark size=1.5pt,
] table [x index=1, y index=5] {data/#2-both.txt};
% unloading
\addplot [
 color=colorc,
 mark=halfcircle*,
 mark size=1.5pt,
 mark options={rotate=180},
] table [x index=1, y index=5] {data/#3-both.txt};
\node[anchor=north west] at (yticklabel* cs:1) {(#6)};
}


\begin{figure*}
 \centering
\pgfplotsset{
 minor tick num=3,
 footnotesize,
 legend style={font=\footnotesize},
}
\begin{tikzpicture}
 \begin{groupplot}[
   ylabel=$\Delta\Gamma$,
   xlabel=$\Delta\!f$,
    group style={
        group name=loadplot,
        group size=3 by 3,
        horizontal sep=1.5cm,
        vertical sep=1.5cm,
        %xlabels at=edge bottom,
        %ylabels at=edge left,
    },
    height=163.2pt,width=163.2pt,
   ]
\myplot{Air} {QCM00431b} {QCM00434} {-9} {5} {a} {100}
\myplot{Water} {QCM00450} {QCM00170} {-9} {5} {b} {90}
\myplot{Free Particles} {QCM00306c} {QCM00306a} {-30} {45} {c} {100}
\myplot{Attached Particles} {QCM00328b} {QCM00321b} {-35} {9} {d} {100}
\myplot{Lambda DNA} {QCM00461} {QCM00458} {-21} {8} {e} {75}
\myplot{Tethered Particles} {QCM00510a} {QCM00511c} {-65} {25} {f} {70}
  \end{groupplot}
 \end{tikzpicture}
 \caption{Parametric representations of the load situations from the
 manuscript.}
\label{fig:suppparametricplot}
\end{figure*}


\subsection{Asperity Contacts}
The response of a QCM to micron-sized particles was first examined by
Dybwad~\cite{dybwad1985sensitive} in 1985.  Dybwad used a coupled
oscillator approach similar to what we have done, but neglected the damping term
$\xil$.  It is of note that our Navier-Stokes approach (which solves for
$\dot{\mathbf{u}}$) does not converge in the limit of a perfectly elastic
material, e.g.\ $G=G'$ or $\eta=\eta''$; these systems are typically
solved for $\mathbf{u}$.  It is perhaps possible to couple these two
domains, but we have not attempted to do so.  In our model we assume
polystyrene $\left|G\right|=\SI{2}{\giga\pascal}$ at a loss tangent
of $\delta=0.16$~\cite{}.

\subsubsection{Contact Radius}
We have asserted in our manuscript that centrifugal force has the effect of
increasing the lateral contact stiffness $k_L$ of a microsphere on the QCM
surface.  To explore this assertion, changes in $\df$ and $\dg$ were
calculated for a different diameter polystyrene microsphere in water
($\eta'=\SI{1e-3}{\pascal\second}$).  The computational geometry is as per
\Figure{fig:compgeometry}.  The results are shown in
\Figure{fig:contactradius} for 2, 10, and \SI{25}{\micro\meter} diameter
particles as a function of the distance $s$ between the
potentially truncated particle and the oscillating boundary.
\begin{figure}[h]
\centering
 \pgfplotsset{
  minor tick num=3,
  footnotesize,
  legend style={font=\footnotesize},
  every axis/.style={
   height=0.30\textwidth,
   width=0.75\textwidth,
  },
  cycle list name=cbDark27qual,
  max space between ticks=25pt,
 }
 \begin{tabular}{c}
 \begin{tikzpicture}[baseline]
  \pgfplotstableread{data/out-0189.tsv}{\datatable}
  \begin{axis}[
    xlabel = $s$,
    x unit = \si{\meter},
    ylabel = $\df$,
    y unit = \si{hertz},
    ]
    \addplot table [ y expr=\thisrowno{1} ] {\datatable};
    \addplot table [ y expr=\thisrowno{2} ] {\datatable};
    \draw [dashed, semithick] (axis cs:0,-1.8e4) -- (axis cs:0,0.4e4);
    \node [anchor=west] at (rel axis cs:0.01,0.9) {\SI{2}{\micro\meter} particles};
    \legend{$\df$,$\dg$}
  \end{axis}
 \end{tikzpicture}
 \\
 \begin{tikzpicture}[baseline]
  \pgfplotstableread{data/out-1000.tsv}{\datatable}
  \begin{axis}[
    xlabel = $s$,
    x unit = \si{\meter},
    ylabel = $\df$,
    y unit = \si{hertz},
    ]
    \addplot table [ y expr=\thisrowno{1} ] {\datatable};
    \addplot table [ y expr=\thisrowno{2} ] {\datatable};
    \draw [dashed, semithick] (axis cs:0,-1.25e5) -- (axis cs:0,0.9e5);
    \node [anchor=west] at (rel axis cs:0.01,0.9) {\SI{10}{\micro\meter} particles};
    \legend{$\df$,$\dg$}
  \end{axis}
 \end{tikzpicture}
 \\
 \begin{tikzpicture}[baseline]
  \pgfplotstableread{data/out-25um.tsv}{\datatable}
  \begin{axis}[
    xlabel = $s$,
    x unit = \si{\meter},
    ylabel = $\df$,
    y unit = \si{hertz},
    ]
    \addplot table [ y expr=\thisrowno{1} ] {\datatable};
    \addplot table [ y expr=\thisrowno{2} ] {\datatable};
    \draw [dashed, semithick] (axis cs:0,-3e5) -- (axis cs:0,5e5);
    \node [anchor=west] at (rel axis cs:0.01,0.9) {\SI{25}{\micro\meter} particles};
    \legend{$\df$,$\dg$}
  \end{axis}
 \end{tikzpicture}
\end{tabular}
\caption{Changes in frequency $\df$ and bandwidth $\dg$ for 2, 10, and
 \SI{25}{\micro\meter} diameter particles as a function of separation $s$
between the (potentially truncated) particle and the oscillating boundary.}
\label{fig:contactradius}
\end{figure}



For a Hertzian sphere-plate contact, the contact radius $r_c$ between two
materials with indicies 1 and 2 is a function of the materials Young's
moduli $E_{1,2}$, Poisson's ratio $\nu_{1,2}$, particle radius $r$, and
appled force~\cite{nalwa1999handbook}
$F_\perp$
\begin{equation}
 r_c = \left(\frac{3}{4} F_\perp  r
 \left(\frac{1-\nu_1^2}{E_1}+\frac{1-\nu_2^2}{E_2}\right)\right)^{1/3}
\label{eqn:contactradius}
\end{equation}
The laterial Hertzian spring constant $k_\parallel$ is a function of the
material's shear moduli $G_{1,2}$
\begin{equation}
 k_\parallel=8 G_r r_c
\end{equation}
where $G$ is the reduced shear modulus
\begin{equation}
 G_r=\left(\frac{2-\nu_1}{G_1}+\frac{2-\nu_2}{G_2}\right)^{-1}
\end{equation}

The force on the particle $F_\perp$ will result from a combination of
gravitational and Van der Waals forces.  The gravitational component $F_g$ is
straightforward
\begin{equation}
 F_g = \frac{4}{3}\pi r^3 g \rho
\end{equation}
The Van der Waals force between a sphere and a plate with seperation $d$ is
approximately
\begin{equation}
 F_\mathrm{VdW} = \frac{A_H r}{6 d}
\end{equation}
where $A_H$ is the effective Hamaker constant for the three material system
(estimated here to be approximately \SI{10e-20}{\joule}).


%Material constants used in this simulation are found in \Table{tbl:materialvalues}.

%\begin{table}[h]
%\begin{tabular}{llll}
% \toprule
% material & $E$ (\si{\giga\pascal}) & $G$ (\si{\giga\pascal}) & $\nu$ \\
% \midrule
% Au & 78 & 27 & 0.43 \tabularnewline
% PS & 3.25 & 2 & 0.35 \tabularnewline
% \bottomrule
%\end{tabular}
%\caption{Material constants used in the simulation. PS stands for
%polystyrene.}
%\label{tbl:materialvalues}
%\end{table}

\subsubsection{Rigid Attachment}
When the contact stiffness is sufficiently increased, the system moves from
the weak coupling regime to the strong coupling regime.  This


\subsubsection{Particle Mass and Density}
According to \Equation{eqn:dingus}, a linear increase in mass will cause a
the zero crossing frequency at
$f_\mathrm{zc} = \ml \omegaq^2$ to increase linearly while the radius of
the circle-like plots will increase as $\ml^2$.

To confirm this relationship, the density of a \SI{25}{\micro\meter}
diameter sphere was changed as a function of contact radius.

For a gold electrode and a gold sphere, assuming a separation distance of
\SI{0.17}{\nano\meter}, the equations reproduce pretty well Dybwad's
original results.  Unfortunatly we can't do this because we're solving the
Navier-Stokes which doesn't work for really hard contacts.

Under normal gravity, $F_\mathrm{VdW}$ dominates $F_\mathrm{g}$ by over
four orders of magnitude.  So why is spinning doing anything in the first
place?  That's a good question!

\subsubsection{Particle Size}
In the simulation, the particle size is related explicitly to the limiting
value of the coupling.
(put particle size stuff here to show that you can independently extract
the particle size).

\subsubsection{Overtone Dependence and the Voigt Model}

The Voigt model predicts an overtone dependence on frequency shift of
$n^{-1}$.  Our model assumes a \SI{25}{\micro\meter} diameter polystyrene
sphere on the electrode surface.  Assuming a seperation of
\SI{0.17}{\nano\meter} the Van der Waals attraction is
\SI{7.2e6}{\pico\newton} and the corresponding contact radius and lateral
spring constants are $r_c=\SI{266.6}{\nano\meter}$ and $k_\parallel =
\SI{2415}{\newton\meter}$, respectively.  Assuming a loss tangent of
0.157~\cite{}, the corresponding viscosity is $\nu =
\SI{10-100i}{\pascal\second}$.  The results of this

\subsubsection{Shear Modulus and Viscosity}
\begin{figure}[h]
\centering
 \pgfplotsset{
  minor tick num=3,
  footnotesize,
  legend style={font=\footnotesize},
  every axis/.style={
   height=0.4\textwidth,
   width=0.4\textwidth,
  },
  cycle list name=cbDark27qual,
  max space between ticks=25pt,
 }
 \begin{tabular}{cc}
 \begin{tikzpicture}[baseline]
  \tikzset{ every pin/.style={pin distance=0.0cm} }
  \pgfplotstableread{data/out_modulusa2.tsv}{\datatablea}
  \pgfplotstableread{data/out_modulusa3.tsv}{\datatableb}
  \pgfplotstableread{data/out_modulusa4.tsv}{\datatablec}
  \pgfplotstableread{data/out_modulusa5.tsv}{\datatabled}
  \begin{axis}[
    xlabel = $r_c$,
    x unit = \si{\meter},
    ylabel = $\df$,
    y unit = \si{hertz},
    restrict x to domain =0:0.2e6,
    ]

    \addplot [color=Dark27qual1,mark=x ] table [ y expr=\thisrowno{1} ] {\datatablea};
    \addplot [color=Dark27qual2,mark=x ] table [ y expr=\thisrowno{2} ] {\datatablea};
    \node[coordinate,pin=above:{25}] at (axis cs:4e4,9.07e5) {};

    \addplot [color=Dark27qual1,mark=x ] table [ y expr=\thisrowno{1} ] {\datatableb};
    \addplot [color=Dark27qual2,mark=x ] table [ y expr=\thisrowno{2} ] {\datatableb};
    \node[coordinate,pin=above:{50}] at (axis cs:5.5001e+04,1.0064e+06) {};

    \addplot [color=Dark27qual1,mark=x ] table [ y expr=\thisrowno{1} ] {\datatablec};
    \addplot [color=Dark27qual2,mark=x ] table [ y expr=\thisrowno{2} ] {\datatablec};
    \node[coordinate,pin=above:{75}] at (axis cs:6.7500e+04,1.1882e+06) {};

    \addplot [color=Dark27qual1,mark=x ] table [ y expr=\thisrowno{1} ] {\datatabled};
    \addplot [color=Dark27qual2,mark=x ] table [ y expr=\thisrowno{2} ] {\datatabled};
    \node[coordinate,pin=above:{100}] at (axis cs:80000,1.4695e+06) {};

  \end{axis}
 \end{tikzpicture}
&
\begin{tikzpicture}[baseline]
  \tikzset{ every pin/.style={pin distance=0.0cm} }
  \pgfplotstableread{data/out_modulusb2.tsv}{\datatablea}
  \pgfplotstableread{data/out_modulusb3.tsv}{\datatableb}
  \pgfplotstableread{data/out_modulusb4.tsv}{\datatablec}
  \pgfplotstableread{data/out_modulusb5.tsv}{\datatabled}
  \begin{axis}[
    xlabel = $\df$,
    x unit = \si{\hertz},
    ylabel = $\dg$,
    y unit = \si{hertz},
    ]
    \node[coordinate,pin=above:{25}] at (axis cs:4e4,9.07e5) {};
    \node[coordinate,pin=above:{50}] at (axis cs:5.5001e+04,1.0064e+06) {};
    \node[coordinate,pin=above:{75}] at (axis cs:6.7500e+04,1.1882e+06) {};
    \node[coordinate,pin=above:{100}] at (axis cs:80000,1.4695e+06) {};

    \addplot [color=Dark27qual3,mark=x ] table [ y expr=\thisrowno{1} ] {\datatablea};

    \addplot [color=Dark27qual3,mark=x ] table [ y expr=\thisrowno{1} ] {\datatableb};

    \addplot [color=Dark27qual3,mark=x ] table [ y expr=\thisrowno{1} ] {\datatablec};

    \addplot [color=Dark27qual3,mark=x ] table [ y expr=\thisrowno{1} ] {\datatabled};

  \end{axis}
 \end{tikzpicture}
\end{tabular}
\caption{Sweeping shear modulus.  Units are in \si{\giga\pascal}.}
\end{figure}




% for increasing the contact area of 10, 25,
% and \SI{50}{\micro\meter} gold microspheres
%In the experiment, 10, 25, and \SI{50}{\micro\meter} diameter gold
%spheres were deposited on the surface of the QCM in air and the frequency
%shift $\df$ was recorded.  From $\df$ and its zero crossing at $\kl = \ml
%\omegaq^2$, the bonding force $k$ and the particle mass $\ml$ was obtained.

%The simulation was initiated assuming a gold particle, shear modulus
%$G'=\SI{29}{\giga\pascal}$~\cite{borovsky2001measuring} and
%$\rho_\mathrm{Au}=\SI{19.3}{\gram\per\centi\meter\cubed}$.  The surrounding
%medium was air, $\eta=\SI{1.9e-5}{\pascal\second}$, and
%$\rho_\mathrm{air}=\SI{1.2922}{\kilo\gram\per\meter\cubed}$.  The simulated
%response for the different particle sizes is shown in \Figure{fig:dybwad}.
%
%\begin{figure}[h]
% \pgfplotsset{
%  minor tick num=3,
%  footnotesize,
%  legend style={font=\footnotesize},
%  every axis/.style={
%   height=0.50\textwidth,
%   width=0.75\textwidth,
%   %ymin = -0.0003,
%   %ymax =  0.0003,
%   %xmin = -0.0001,
%   %xmax = 0.0038,
%  },
%  cycle list name=cbDark23qual,
% }
% \begin{tikzpicture}[baseline]
%  % data generated by Mathematica notebook fitting-gk.nb
%  \pgfplotstableread{data/out-smallbeads.tsv}{\datatable}
%  \begin{axis}[
%    axis y line*=left,
%    xlabel = $s$,
%    x unit = \si{\meter},
%    ylabel = $\df$,
%    y unit = \si{hertz},
%    ]
%    \addplot [color=Dark23qual1,mark=x] table [ y expr=\thisrowno{1} ] {\datatable};
%   %\addlegendentry{simulation}
%   %\addplot table [ y expr=\thisrowno{3} ] {\datatable};
%   %\addlegendentry{GK}
%   %\addplot table [ y expr=\thisrowno{5} ] {\datatable};
%   %\addlegendentry{\Equation{eqn:viscoshear}}
% \end{axis}
%  \begin{axis}[
%    hide x axis,
%    xlabel = viscosity $\eta$,
%    x unit = \si{\pascal\second},
%    ylabel = $\dg$,
%    y unit = \si{hertz},
%    axis y line*=right,
%    ylabel near ticks,
%    legend pos={south west},
%   ]
%   \addplot [color=Dark23qual2,mark=+] table [ y expr=\thisrowno{2} ] {\datatable};
%   %\draw [dashed, thick] (axis cs:-0.0001,0) -- (axis cs:0.0038,0);
%   \legend{simulation,\Equation{gkrelations},\Equation{eqn:viscoshear}}
% \end{axis}
%
% \end{tikzpicture}
%\caption{Comparison of $\df$ and $\dg$ as the viscosity $\eta$ is swept as
% obtained with the simulation compared with the Gordon-Kanazawa relations
% and \Equation{eqn:viscoshear}.  The frequency shifts $\df$ are all negative
% and appear below the bandwidth shifts $\dg$, which are all positive.  The simulation can be seen to be in
%excellent agreement with theory.}
%\label{fig:viscosweep}
%\end{figure}
%
%
%
%\begin{figure}[h]
% \pgfplotsset{
%  minor tick num=3,
%  footnotesize,
%  legend style={font=\footnotesize},
%  every axis/.style={
%   height=0.50\textwidth,
%   width=0.75\textwidth,
%   %ymin = -0.0003,
%   %ymax =  0.0003,
%   %xmin = -0.0001,
%   %xmax = 0.0038,
%  },
%  cycle list name=cbDark23qual,
% }
% \begin{tikzpicture}[baseline]
%  % data generated by Mathematica notebook fitting-gk.nb
%  \pgfplotstableread{data/out_modulus.tsv}{\datatable}
%  \begin{axis}[
%    axis y line*=left,
%    xlabel = $s$,
%    x unit = \si{\meter},
%    ylabel = $\df$,
%    y unit = \si{hertz},
%    ]
%    \addplot [color=Dark23qual1,mark=x] table [ y expr=\thisrowno{1} ] {\datatable};
%   %\addlegendentry{simulation}
%   %\addplot table [ y expr=\thisrowno{3} ] {\datatable};
%   %\addlegendentry{GK}
%   %\addplot table [ y expr=\thisrowno{5} ] {\datatable};
%   %\addlegendentry{\Equation{eqn:viscoshear}}
% \end{axis}
%  \begin{axis}[
%    hide x axis,
%    xlabel = viscosity $\eta$,
%    x unit = \si{\pascal\second},
%    ylabel = $\dg$,
%    y unit = \si{hertz},
%    axis y line*=right,
%    ylabel near ticks,
%    legend pos={south west},
%   ]
%   \addplot [color=Dark23qual2,mark=+] table [ y expr=\thisrowno{2} ] {\datatable};
%   %\draw [dashed, thick] (axis cs:-0.0001,0) -- (axis cs:0.0038,0);
%   \legend{simulation,\Equation{gkrelations},\Equation{eqn:viscoshear}}
% \end{axis}
%
% \end{tikzpicture}
%\caption{Comparison of $\df$ and $\dg$ as the viscosity $\eta$ is swept as
% obtained with the simulation compared with the Gordon-Kanazawa relations
% and \Equation{eqn:viscoshear}.  The frequency shifts $\df$ are all negative
% and appear below the bandwidth shifts $\dg$, which are all positive.  The simulation can be seen to be in
%excellent agreement with theory.}
%\label{fig:viscosweep}
%\end{figure}

\begin{figure}[h]
\centering
 \pgfplotsset{
  minor tick num=3,
  footnotesize,
  legend style={font=\footnotesize},
  every axis/.style={
   height=0.30\textwidth,
   width=0.75\textwidth,
  },
  cycle list name=cbDark27qual,
  max space between ticks=25pt,
 }
 \begin{tikzpicture}[baseline]
  \pgfplotstableread{data/out-0189.tsv}{\datatablea}
  \pgfplotstableread{data/out-0189-rigid.tsv}{\datatableb}
  %\pgfplotstableread{data/out-1000.tsv}{\datatableb}
  %\pgfplotstableread{data/out-25um.tsv}{\datatablec}
  \begin{axis}[
    xlabel = $s$,
    x unit = \si{\meter},
    ylabel = $\df$,
    y unit = \si{hertz},
    ymin=-2e4,ymax=0.6e4,
    ]
    \addplot table [ y expr=\thisrowno{1} ] {\datatablea};
    \addplot table [ y expr=\thisrowno{2} ] {\datatablea};
    \addplot table [ y expr=\thisrowno{1} ] {\datatableb};
    \addplot table [ y expr=\thisrowno{2} ] {\datatableb};

    \draw [dashed, semithick] (axis cs:0,-2.8e4) -- (axis cs:0,0.5e4);
    \legend{$\df$ free,$\dg$ free,$\df$ attached, $\dg$ attached}
  \end{axis}
 \end{tikzpicture}
 \caption{Rigidly attaching a \SI{1.89}{\micro\meter} PS bead.}
\end{figure}

\section{Mechanical Model}
We employ a mechanical model based on coupled oscillators shown in
\Figure{fig:mechanicalmodel}.  Here the resonance of the QCM at
$\omegaq^2=\kq/\mq$ is coupled to a mass $\ml$ through a spring $\kl$ and
dashpot $\xil$.  The spring and dashpot are not \textit{actual} springs and
dashpots, rather they are analogies for the coupling of two systems with
different resonances in the small load approximation.  The QCM itself has a
small damping term, but it is small enough that we neglect it.
\begin{figure}[ht]
 \centering
 \import{figures/}{dybwadmodel.pdf_tex}
 \caption{Coupled oscillator mechanical model for QCM behavior.}
\label{fig:mechanicalmodel}
\end{figure}

The derivation of this equation is described in detail in
\Ref{olsson2012probing}.  Here we use the same equation but as a function
of $\kl$ instead of the sphere resonance $\omega_\mathrm{L}^2=\kl/\ml$; the
former being more approprate to our analysis.
Using the small load approximation, the response
of the system as a function of its coupling $\kl$ is then
\begin{equation}
\frac{\Delta\!f + \mi \Delta \Gamma}{f_\mathrm{F}} = \frac{N_\mathrm{L}}{\pi
\mathrm{Z}_\mathrm{q}}
\frac{\ml \omega_\mathrm{q} \left( \kl + \mi
\omega_\mathrm{q} \xil\right) }
{\ml \omega_\mathrm{q}^2 - \left(\kl + \mi
\omega_\mathrm{q} \xil\right)}
\label{eqn:mastereq}
\end{equation}
where $\mathrm{Z}_\mathrm{q}$ is the acoustic impedance of AT cut quartz,
$f_\mathrm{F}$ is the fundamental frequency of the resonator, and
$N_\mathrm{L}$ is a number surface density (number per unit area) for discrete
loads.

\bibliographystyle{unsrt}
\bibliography{../bibliography}
\end{document}
