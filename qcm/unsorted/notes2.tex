\documentclass[a4paper]{article}
\usepackage{amsmath}
\usepackage{graphicx} % required for pdf transparency
\usepackage{xcolor} % required for pdf transparency
\usepackage{amsthm}
\usepackage{amssymb}
\usepackage{booktabs}
\usepackage{tabularx}
\usepackage{subfigure}
\usepackage{mathrsfs}
\usepackage{textcomp}
\usepackage{siunitx}
\usepackage{fullpage}
\usepackage[multiple]{footmisc}
\usepackage{hyperref}
\usepackage{multirow}
\usepackage{empheq}
\usepackage{calc}
\usepackage[version=3]{mhchem}
\usepackage{tikz}
\usepackage{pgfplots}
\usepackage{xcolor}
\usepackage[section]{placeins}
\pgfplotsset{compat=newest}


%\hypersetup{
% colorlinks=false,
% linkbordercolor={1 1 1},
% citebordercolor={1 1 1},
% urlbordercolor={1 1 1} 
%}

\sisetup{ 
% load-configurations=abbreviations
% round-mode = places,
}%

% New definition of square root:
% it renames \sqrt as \oldsqrt
% This definition puts a little vertical guy at the end so it's more
% obvious where the square root actually ends.
\let\oldsqrt\sqrt
% it defines the new \sqrt in terms of the old one
\def\sqrt{\mathpalette\DHLhksqrt}
\def\DHLhksqrt#1#2{%
\setbox0=\hbox{$#1\oldsqrt{#2\,}$}\dimen0=\ht0
\advance\dimen0-0.2\ht0
\setbox2=\hbox{\vrule height\ht0 depth -\dimen0}%
{\box0\lower0.4pt\box2}}

% integrals with infinity bounds
\newcommand{\intinfty}{\int_{-\infty}^{\infty}}

% consistent formatting of object labels
\newcommand{\Figure}[1]{Figure \ref{#1}}
\newcommand{\Equation}[1]{Equation \ref{#1}}
\newcommand{\Table}[1]{Table \ref{#1}}
\newcommand{\Section}[1]{Section \ref{#1}}
\newcommand{\Chapter}[1]{Chapter \ref{#1}}
\newcommand{\Appendix}[1]{Appendix \ref{#1}}

% missing mathematical operators
\DeclareMathOperator{\sinc}{sinc}
\DeclareMathOperator{\sech}{sech}
\DeclareMathOperator{\sgn}{sgn}
\DeclareMathOperator{\erf}{erf}
\DeclareMathOperator{\inverf}{inverf}
\DeclareMathOperator{\arcsinh}{arcsinh}
\DeclareMathOperator{\arccosh}{arccosh}
\DeclareMathOperator{\arctanh}{arctanh}
%\DeclareMathOperator{\Re}{Re}
%\DeclareMathOperator{\Im}{Im}

% use roman type for natural base e and sqrt(-1)
\newcommand{\me}{{\mathrm{e}}}
\newcommand{\mi}{{\mathrm{i}}}

% roman type for the derivative, plus a space
\newcommand{\md}{\,\mathrm{d}}

% fourier transform and the reverse
\newcommand{\ff}[1]{{\mathscr{F}^{+}\bigl(#1\bigr)}}
\newcommand{\fr}[1]{{\mathscr{F}^{-}\bigl(#1\bigr)}}

% hilbert transform and the reverse
\newcommand{\hf}[1]{{\mathscr{H}^{+}\bigl(#1\bigr)}}
\newcommand{\hr}[1]{{\mathscr{H}^{-}\bigl(#1\bigr)}}

% custom lengths for figures

% width for side by side figures
\newlength{\twoupwidth}
\setlength{\twoupwidth}{7.5cm}

% width and height for default single figure
\newlength{\oneupwidth}
\setlength{\oneupwidth}{0.90\textwidth}
\newlength{\oneupheight}
\setlength{\oneupheight}{0.55623059\textwidth}

% custom colors for 2D plots - these are the same ones mathematica uses by
% default
\definecolor{colora}{RGB}{63,61,153}
\definecolor{colorb}{RGB}{153,61,113}
\definecolor{colorc}{RGB}{153,139,61}
\definecolor{colord}{RGB}{61,153,86}
\definecolor{colore}{RGB}{61,90,153}
\definecolor{colorf}{RGB}{153,61,144}
\definecolor{colorg}{RGB}{153,109,61}
\definecolor{colorh}{RGB}{67,153,61}
\definecolor{colori}{RGB}{61,121,153}
\definecolor{colorj}{RGB}{132,61,153}

% hilight text... a "work in progress" type thing
\tikzstyle{todobox} = [draw=red, fill=blue!10, very thick,
rectangle, rounded corners, inner sep=10pt, inner ysep=20pt]
\tikzstyle{todotitle} =[fill=red, text=white]

\newcommand{\todo}[1]{%
\begin{center}
\begin{tikzpicture}
\node [todobox] (box){%
 \begin{minipage}{0.9\textwidth}
 #1
 \end{minipage}
};
\node[todotitle, right=10pt] at (box.north west) {To Do:};
\end{tikzpicture}
\end{center}

}

\pgfplotsset{
 /pgfplots/colormap={jet}{rgb255(0cm)=(0,0,128) rgb255(1cm)=(0,0,255)
 rgb255(3cm)=(0,255,255) rgb255(5cm)=(255,255,0) rgb255(7cm)=(255,0,0)
 rgb255(8cm)=(128,0,0)}
}
\usepgfplotslibrary{units}
\usetikzlibrary{pgfplots.units} 


\usepgfplotslibrary{groupplots}
\begin{document}
\title{QCM Centrifuge Notes}
\author{Aaron Webster}
\date{Last Update: \today}
\maketitle
\tableofcontents
\newpage

Typical values for a 1” diameter, 5 MHz crystal used in the QCM200 System are
Cm = 33 fF, Lm = 30 mH, and R m = 10 Ω (for a dry crystal), Rm = 400 Ω (for a crystal
with one face in water), or Rm = 3500 Ω (for a crystal with one face in 85% glycerol).
The motional arm is shunted by the parasitic capacitance, Co , which represents the sum
of the static capacitances of the crystal’s electrodes, holder, and connector capacitance. In
the QCM200 System 4, Co is about 20 pF, a value which has been kept small by placing
the electronics directly on the Crystal Holder, thereby eliminating any cable capacitance.


\section{Stress on the Crystal}
\cite{filler1988acceleration}


When I push on the holder from the top, $f$ goes up and $R$ goes down, but
not by much.  The same is true for flipping the holder and pressing from
the other side.

\section{Sauerbrey Equation}
The Sauerbrey equation has been modified to predict the frequency and
resistance changes.

\begin{align}
 \Delta f = -f_u^{3/2} \left(\frac{\rho_l \eta_l}{\pi \rho_q \mu_q}\right)^{1/2}
\end{align}
where

\begin{tabular}{ll}
$f_u$     & frequency of unloaded crystal \\
$\mu_q$   & shear modulus of quartz \\
$\rho_q$  & density of quartz \\
$\rho_l$  & density of contact liquid \\
$\eta_l$   & viscosity of contact liquid
\end{tabular}

\begin{align}
 \Delta R = 2 n f_s L_u \left(\frac{4 \pi f_s \rho_l \eta_l}{\rho_q \mu_q}\right)^{1/2}
\end{align}

where
\begin{tabular}{ll}
$\Delta R$ & change in series resistance \\
$n$        & number of sides in contact with liquid \\
$f_s$      & oscillation frequency \\
$L_u$      & intrinsic inductance of unloaded crystal \\
\end{tabular}

From these equations, classic liquid loading in this thin film Sauerbrey
approximation causes an decrease in $\Delta f$ and an increase in $\delta
R$.

\section{Air}
R 7-12 Ohm
shifts on unloading <0.1 Ohm

370 ohm water

\section{Temperature}
Heat makes $f$ go up and $R$ go down.  Cold makes $f$ go down and $R$ go
up.  $R$ goes back to its original value much more slowly than $f$.

\section{Mumon's Commentary}

\section{Force-Parameter Plots}
Force-parameter plots plot one of the QCM parameters ($f$, $R$, etc.) as a
function of force.  There are several commonly occuring features in these
plots that are worth noting.  To explore these I look at

\begin{center}
\begin{tikzpicture}
\begin{axis}
\end{axis}
\end{tikzpicture}
\end{center}

%\section{Direction of Shift in $f$ and $R$}
%
%\begin{tabular}{llllll}
%\toprule
%sample & func. & $f$ & $R$ & comment \\
%\midrule
%\SI{1.07}{\micro\meter} PS       & none & $\uparrow$ & $\uparrow$ &  \\
%\SI{15}{\micro\meter}   PS       & none & $\uparrow$ & $\uparrow$ &  \\
%\SI{24.8}{\micro\meter} PS       & none & $\uparrow$ & $\uparrow$ &  \\
%\SI{1.89}{\micro\meter} paramag. & none & $\uparrow$ & $\uparrow$ &  \\
%\SI{5.86}{\micro\meter} paramag. & none & $\uparrow$ & $\uparrow$ &  \\
%\SI{9.10}{\micro\meter} paramag. & none & $\uparrow$ & $\uparrow$ &  \\
%\SI{1.07}{\micro\meter} PS       & oligo & $\uparrow$ & $\uparrow$ &  \\
%\SI{15}{\micro\meter}   PS       & oligo & $\uparrow$ & $\uparrow$ &  \\
%\SI{24.8}{\micro\meter} PS       & oligo & $\uparrow$ & $\uparrow$ &  \\
%\SI{1.89}{\micro\meter} paramag. & oligo & $\uparrow$ & $\uparrow$ &  \\
%\SI{5.86}{\micro\meter} paramag. & oligo & $\uparrow$ & $\uparrow$ &  \\
%\SI{9.10}{\micro\meter} paramag. & oligo & $\uparrow$ & $\uparrow$ &  \\
%\SI{1.07}{\micro\meter} PS       & lambda & $\uparrow$ & $\uparrow$ &  \\
%\SI{15}{\micro\meter}   PS       & lambda & $\uparrow$ & $\uparrow$ &  \\
%\SI{24.8}{\micro\meter} PS       & lambda & $\uparrow$ & $\uparrow$ &  \\
%\SI{1.89}{\micro\meter} paramag. & lambda & $\uparrow$ & $\uparrow$ &  \\
%\SI{5.86}{\micro\meter} paramag. & lambda & $\uparrow$ & $\uparrow$ &  \\
%\SI{9.10}{\micro\meter} paramag. & lambda & $\uparrow$ & $\uparrow$ &  \\
%HBFR                             & none   & $\uparrow$ & $\uparrow$ &  \\
%water                            & none   & $\uparrow$ & $\uparrow$ &
%\bottomrule
%\end{tabular}
%
%\section{Big Loop Stuff}
%Big loop stuff corresponds to a delay.
%

\section{Shift Directions for Unloading}
\begin{tabular}{lll}
 \toprule
 sample & $\Delta f$ & $\Delta R$ \\
 \midrule
 Sauerbrey & $\uparrow$ &   $\downarrow$ \\
 air       & $\uparrow$ & $\uparrow$ \\
 water     & $\uparrow$ &   $\downarrow$ \\
 HBFR      & $\downarrow$ & $\uparrow$ \\
 $\lambda$ only &
 \bottomrule
\end{tabular}
\end{document}



