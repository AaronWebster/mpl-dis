\documentclass[a4paper]{article}
\usepackage{amsmath}
\usepackage{graphicx} % required for pdf transparency
\usepackage{xcolor} % required for pdf transparency
\usepackage{amsthm}
\usepackage{amssymb}
\usepackage{booktabs}
\usepackage{tabularx}
\usepackage{subfigure}
\usepackage{mathrsfs}
\usepackage{textcomp}
\usepackage{siunitx}
\usepackage{fullpage}
\usepackage[multiple]{footmisc}
\usepackage{hyperref}
\usepackage{multirow}
\usepackage{empheq}
\usepackage{calc}
\usepackage[version=3]{mhchem}
\usepackage{tikz}
\usepackage{pgfplots}
\usepackage{xcolor}
\usepackage[section]{placeins}
\pgfplotsset{compat=newest}


%\hypersetup{
% colorlinks=false,
% linkbordercolor={1 1 1},
% citebordercolor={1 1 1},
% urlbordercolor={1 1 1} 
%}

\sisetup{ 
% load-configurations=abbreviations
% round-mode = places,
}%

% New definition of square root:
% it renames \sqrt as \oldsqrt
% This definition puts a little vertical guy at the end so it's more
% obvious where the square root actually ends.
\let\oldsqrt\sqrt
% it defines the new \sqrt in terms of the old one
\def\sqrt{\mathpalette\DHLhksqrt}
\def\DHLhksqrt#1#2{%
\setbox0=\hbox{$#1\oldsqrt{#2\,}$}\dimen0=\ht0
\advance\dimen0-0.2\ht0
\setbox2=\hbox{\vrule height\ht0 depth -\dimen0}%
{\box0\lower0.4pt\box2}}

% integrals with infinity bounds
\newcommand{\intinfty}{\int_{-\infty}^{\infty}}

% consistent formatting of object labels
\newcommand{\Figure}[1]{Figure \ref{#1}}
\newcommand{\Equation}[1]{Equation \ref{#1}}
\newcommand{\Table}[1]{Table \ref{#1}}
\newcommand{\Section}[1]{Section \ref{#1}}
\newcommand{\Chapter}[1]{Chapter \ref{#1}}
\newcommand{\Appendix}[1]{Appendix \ref{#1}}

% missing mathematical operators
\DeclareMathOperator{\sinc}{sinc}
\DeclareMathOperator{\sech}{sech}
\DeclareMathOperator{\sgn}{sgn}
\DeclareMathOperator{\erf}{erf}
\DeclareMathOperator{\inverf}{inverf}
\DeclareMathOperator{\arcsinh}{arcsinh}
\DeclareMathOperator{\arccosh}{arccosh}
\DeclareMathOperator{\arctanh}{arctanh}
%\DeclareMathOperator{\Re}{Re}
%\DeclareMathOperator{\Im}{Im}

% use roman type for natural base e and sqrt(-1)
\newcommand{\me}{{\mathrm{e}}}
\newcommand{\mi}{{\mathrm{i}}}

% roman type for the derivative, plus a space
\newcommand{\md}{\,\mathrm{d}}

% fourier transform and the reverse
\newcommand{\ff}[1]{{\mathscr{F}^{+}\bigl(#1\bigr)}}
\newcommand{\fr}[1]{{\mathscr{F}^{-}\bigl(#1\bigr)}}

% hilbert transform and the reverse
\newcommand{\hf}[1]{{\mathscr{H}^{+}\bigl(#1\bigr)}}
\newcommand{\hr}[1]{{\mathscr{H}^{-}\bigl(#1\bigr)}}

% custom lengths for figures

% width for side by side figures
\newlength{\twoupwidth}
\setlength{\twoupwidth}{7.5cm}

% width and height for default single figure
\newlength{\oneupwidth}
\setlength{\oneupwidth}{0.90\textwidth}
\newlength{\oneupheight}
\setlength{\oneupheight}{0.55623059\textwidth}

% custom colors for 2D plots - these are the same ones mathematica uses by
% default
\definecolor{colora}{RGB}{63,61,153}
\definecolor{colorb}{RGB}{153,61,113}
\definecolor{colorc}{RGB}{153,139,61}
\definecolor{colord}{RGB}{61,153,86}
\definecolor{colore}{RGB}{61,90,153}
\definecolor{colorf}{RGB}{153,61,144}
\definecolor{colorg}{RGB}{153,109,61}
\definecolor{colorh}{RGB}{67,153,61}
\definecolor{colori}{RGB}{61,121,153}
\definecolor{colorj}{RGB}{132,61,153}

% hilight text... a "work in progress" type thing
\tikzstyle{todobox} = [draw=red, fill=blue!10, very thick,
rectangle, rounded corners, inner sep=10pt, inner ysep=20pt]
\tikzstyle{todotitle} =[fill=red, text=white]

\newcommand{\todo}[1]{%
\begin{center}
\begin{tikzpicture}
\node [todobox] (box){%
 \begin{minipage}{0.9\textwidth}
 #1
 \end{minipage}
};
\node[todotitle, right=10pt] at (box.north west) {To Do:};
\end{tikzpicture}
\end{center}

}

\pgfplotsset{
 /pgfplots/colormap={jet}{rgb255(0cm)=(0,0,128) rgb255(1cm)=(0,0,255)
 rgb255(3cm)=(0,255,255) rgb255(5cm)=(255,255,0) rgb255(7cm)=(255,0,0)
 rgb255(8cm)=(128,0,0)}
}
\usepgfplotslibrary{units}
\usetikzlibrary{pgfplots.units} 


\DeclareSIUnit\molar{\mole\per\cubic\deci\metre}
\DeclareSIUnit\Molar{\textsc{M}}
\DeclareSIUnit\inch{in}
\usepgfplotslibrary{groupplots}
\usepackage{longtable}
\begin{document}
\title{Protocols}
\author{Aaron Webster\\ Rowland Institute at Harvard}
\date{Last Update: \today}
\maketitle
\tableofcontents

\section{PDMS Microfluidic Cells}
The quartz crystal is 

prepared using Sylgard 184 two part elastomer.
A thin layer of the uncured is poured on to a silicon 
poured 


\begin{enumerate}

 \item Put a clean, smooth surfaced silicon wafer on a hotplate at
  \SI{200}{\celsius} for at least \SI{20}{\minute}.  This will remove water
  molecules from the wafer's surface and make PDMS removal simpler.
  Usually it is clean such that dusting the wafer with nitrogen is
  sufficent to clean the wafer after removal from its packaging.
 \item 
  
  Mix the PDMS base to curing agent at a ratio of 10:1 for at least 10
  minutes 

\end{enumerate}

\section{Cleaning and Preparing the Crystals}
The quartz crystals must be cleaned prior to use to remove surface
contaminants and allow the chemistry to function.  There are two methods
which seem to work more or less equivalently.
\subsection{Method 1: Piranha Solution}
A solution of fresh piranha will both remove organic contaminates and
hydroxylate the surface, rendering it hydrophobic and immediately suitable
for surface chemistry.
\begin{enumerate}
 \item Immerse crystals in a fresh piranha (3:1 mixture of
  \SI{97}{\percent} \ce{H2SO4} and \SI{30}{\percent} \ce{H2O2}) for
  \SI{5}{\minute}.  
 \item Remove and rinse liberally with pure water.  The crystals are clean
  and ready for chemistry.  
 \item If the crystals are to be stored and not used immediately, subject
  them to either oxygen/plasma cleaning for \SI{5}{\minute} or to UV/ozone
  cleaning for \SI{20}{\minute} prior to chemistry.  This will remove
  organic contaminants and make the surface hydrophilic.
\end{enumerate}

\subsection{Method 3: Dilute \ce{H2SO4} and UV/Ozone}
This method seems to work the best when the surfaces are not functionalized
with oligos.  It has the advantage of being quick and simple to prepare.
\begin{enumerate}
 \item Immerse crystals in a 1:1 solution of \ce{H2SO4} and \ce{H2O} for 30
  minutes.
 \item Remove and rinse liberally with water.
 \item Dry under a stream of \ce{N2}.
 \item UV/Ozone clean for 10 minutes.  The surfaces are now ready.
\end{enumerate}

\section{Cleaning PDMS}

\section{Preparation of Buffers}
The following buffers are used in oligo attaching procedure.
\begin{description}
 \item[DBFR] \SI{1}{\Molar} potassium phosphate (PBS) buffer,  pH 3.8.
  This buffer has a high salt concentration which accelerates the binding
  of thiol to gold.
 \item[MCH] \SI{1}{\milli\Molar} 6-Mercapto-1-hexanol.  Dilute with water.
  this blocks residual reactive sites.
 \item[HBFR] \SI{1}{\Molar} \ce{NaCl} with \SI{10}{\milli\Molar} Tris
  buffer, pH 7.4 and \SI{1}{\milli\Molar} EDTA.  This is the hybridization
  buffer, not sure what it does.
\end{description}
The pH of both DBFR and HBFR must be corrected after preparation.

\section{Attaching Oligos to a Gold Surface}
The oligo used to attach to the gold surface was 5'-ThioMC6-TTT TTT TTT CAC
TAA AGT TCT TAC CCA TCG CCC-3' from IDT.  

\begin{enumerate}
 \item Using a crystal whose surface is immediately hydrophobic from one of
  the two cleaning procedures above, immerse each surface in a
  \SI{1}{\micro\Molar} solution of oligos in DBFR and wait for
  \SI{1}{\hour}.
 \item Gently rinse the surface of unattached oligos with water and immerse
  in \SI{1}{\milli\Molar} MCH for \SI{1}{\hour}.
 \item Rinse surfaces again and hybridize in HBFR.
\end{enumerate}

\section{Attaching Oligos to Streptavidin Coated Particles}
Complementary to the oligo which attaches to the gold,
5'-biotin-CT CAC TAT AGG GCG ATG GGT AAG AAC TTT AGT-3' was attached to
\SI{20}{\micro\meter} streptavidin coated polystyrene particles using the
following procedure:
\begin{enumerate}
 \item Aliquot \SI{100}{\micro\liter} of streptavidin coated particles into
  a microcentrifuge tube.
 \item Wash two times in a \SI{100}{\micro\liter} solution of HBFR by
  centrifuging at \SI{5000}{RPM} for \SI{3}{\minute} and decanting the
  supernatant.
 \item Resuspend particles in \SI{20}{\micro\liter} of HBFR and add
  \SI{10}{\micro\gram} oligo.
 \item Incubate for \SI{15}{\minute} at room temperature on a vortexer.
 \item Wash two times again in a \SI{100}{\micro\liter} solution of HBFR by
  centrifuging at \SI{5000}{RPM} for \SI{3}{\minute} and decanting the
  supernatant.
 \item Resuspend in \SI{100}{\micro\liter} HBFR.
\end{enumerate}

\section{Attaching the Modified Particles to the Modified Surface}
Attaching the modified particles to the surface is straightforward.
Carefully inject whatever amount of particles you desire on the surface of
the crystal in HBFR and wait about \SI{15}{\minute} for the oligos to bind
to each other.  The longer one waits the more oligos attach and the
stronger the particles are bound to the surface.

\section{Releasing the Beads}
Inject the release strand and away we go.

\section{List of Oligos}
\begin{table}[h]
 \centering
 \begin{tabularx}{\textwidth}{l l X}
  \toprule
  name & description & sequence \\
  \midrule
  Goldlink & attach to gold surface &5'-ThioMC6-TTT TTT TTT CAC TAA AGT TCT
  TAC CCA TCG CCC-3'\\
  Goldlink & attach to gold surface &5'-ThioMC6-TTT TTT TTT CAC TAA AGT TCT
  TAC CCA TCG CCC-3'-Alexa FluorTM 488 dyes (Ex. 492 and Em. 517)\\
  Beadlink & attach to streptavidin coated PS beads &5'-biotin-CT CAC TAT AGG
  GCG ATG GGT AAG AAC TTT AGT-3'\\
  Release  & release beads by toehold exchange & 5'-CACTAAAGTTCTTACC CATCGC
  CCTATAGTGAGTCGTATTAAT-3'\\
 \bottomrule
 \end{tabularx}
\end{table}

\section{Equipment}
\begin{table}[h]
 \centering
 \begin{tabularx}{\textwidth}{X X X}
  \toprule
  item & manufacturer & part number \\
  \midrule
  UV/ozone cleaner & Bioforce Nanosciences & UV/Ozone ProCleaner \\
  \SI{5}{\mega\hertz} QCM crystal, \SI{1}{\inch}, Gold, \ce{Cr} polished.  &
  Stanford Research Systems &  \\
  \bottomrule
 \end{tabularx}
\end{table}

\end{document}
