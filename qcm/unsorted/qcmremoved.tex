\section{Overview}
Mechanical properties such as
viscoelasticity~\cite{steinem2007piezoelectric}, contact
stiffness~\cite{johannsman2007contacts} and adhesion, and force-dependent
conformational changes~\cite{fant2000adsorption} have been shown to be
instrumental in understanding biofunctional behavior, from single molecules
to complex collections of cells~\cite{li2008thickness}. These properties
are often descriptive for specific molecules and cell types. For example,
force extension curves of DNA vary with assembly of histone proteins into
chromatin~\cite{cui2000pulling}~\cite{larson2012trigger}. In cells,
viscoelasticity alone can discriminate between healthy and cancerous
tissue~\cite{rebelo2013comparison}.  Biosensors that can probe mechanical
properties are therefore ideal for both fundamental studies in life
sciences as well as diagnostic assays improving public health.

Perhaps the simplest way to probe a biomechanical property is by monitoring
its response to direct application of force. Current force-based
approaches, for example those based on atomic force microscopy or optical
or magnetic tweezers, are powerful but limited in their applicability as
biosensors. In particular, their operation requires significant expertise
on the part of the investigator, and is often constrained to well prepared
samples not amenable to multiplexing.

Among tools suitable for direct mechanical transduction, the quartz crystal
microbalance (QCM) has seen increasing real-world utility as simple, cost
effective, and highly versatile mechanical biosensing platforms.  A QCM
typically manifests itself as a thin disk of piezoelectric quartz with
electrodes on either side.  The quartz is cut such that, when driven by a
potential, the crystal produces acoustic shear waves on either of its two
faces.  When a material is in contact with the crystal, its resonance
condition will change.  These changes can be used to interrogate both the
viscoelastic properties of the material and the way in which it is coupled
to the QCM.  This mechanical turns out to be extrorinary sensitive, even in
liquid environments, with sensitivities to mass on the order of X.  The
function of a QCM is depicted schematically in \Figure{fig:qcmschema}.

%\begin{figure}
% \caption{Schematic representation of the operation of a QCM.  The
%  quartz crystal is driven by a pair of electrodes on either sides to
%  excite \si{\mega\hertz} acoustic shear waves.  A sample in contact with
%  the crystal will typically change its resonant frequency $\df$ and
%  bandwidth $\dg$, which are connected to the viscoelastic properites of
% the sample and the way in which it is coupled to the device.}
%\end{figure}

Since its introduction by Sauerbrey~\cite{sauerbrey1959verwendung} in 1959
as sub-monolayer thin-film mass sensors in the gas phase, the understanding
of these piezoelectric devices has been repeatedly enhanced to study
phenomena such as viscoelastic films in the liquid
phase~\cite{kanazawa1985frequency} , non-destructive contact
mechanics~\cite{borovsky2001measuring}~\cite{johannsman2007contacts}, and
complex topologies of biopolymers and
biomacromolecules~\cite{marx2003quartz}. However, despite their popularity
QCMs suffer from low-Q resonances which negatively impact their
sensitivity. In addition, extracting quantitative mechanical information
from biomolecules is often confounded by non-trivial interpretation of the
discrete shifts in the system's frequency and bandwidth.

The centrifugal force quartz crystal microbalance (CF-QCM), shown in
\Figure{fig:cfqcm} is our tool to do something.  The CF-QCM is a new type
of instrument which places a quartz crystal microbalance in a standard
commercial centrifuge.  When spinning, controllable centrifugal force is
applied to a biomaterial under assay, and the QCM signal as a function of
applied force is monitored in situ and in real time.

\chapter{Future Work}
