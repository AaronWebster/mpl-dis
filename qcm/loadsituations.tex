\begin{figure}[ht]
\centering
\includegraphics[width=16cm,keepaspectratio]{qcm/figures/figure1.pdf}
\caption{Overview of the CF-QCM and its sensogram.  (a) Schematic of the
experimental setup.  (b) Example CF-QCM sensogram
for a sample consisting of \SI{1}{\micro\meter} particles in water,
$N_\mathrm{L}=\SI{1.58e11}{\particle\per\meter\squared}$, in the
``loading'' configuration (inset).  The horizontal arrows indicate the
motion of the QCM's transverse shear mode.  }
\label{fig:expsetup}
\end{figure}
The central theme of the CF-QCM work is the study of its sensogram for
different load situations.  An example of such a load situation is shown in
\Figure{fig:expsetup}(b).  Here, free \SI{1}{\micro\meter} streptavidin
coated polystyrene particles in water are introduced into the sample cell.
When the cell is rotated to the loading configuration under the influence
of gravity alone, the particles fall toward the sensing surface and a
positive shift in the QCM's frequency signal is observed.  When the cell is
then rotated 180 degrees by hand to the unloading configuration, the
particles fall off and the frequency response returns to its original
state.  Again the cell is rotated 180 degrees to the loading configuration
and the positive frequency shift is observed.  As the centrifuge spins up
towards \SI{90}{g}, the particles are ``pressed'' towards the QCM surface
and a more than four fold increase in the frequency shift is observed.  The
centrifuge then spins down and the baseline frequency shift under gravity
alone is recovered.

Traditional QCM experiments assume that the inertial properties and
rigidity of the sample's coupling are taken as a fixed parameter (or
statistical distribution) under assay.  With this approach however, one is
only able to obtain discrete values in an otherwise continuous parameter
space.  Hybrid-QCM experiments involving
nanoindenters~\cite{borovsky2001measuring} or AFM probe
tips~\cite{richter2003pathways} have shown intriguing behavior when force
is applied to a sample in a QCM measurement.  There have also been reports
that accelerations as small as \SI{1}{g} have a measurable effect on a
QCM's response for viscoelastic monolayers such as
DNA~\cite{fawcett2004evidence}, and even for pure Newtonian
liquids~\cite{yoshimoto2002effect}.  All of these responses have been found
to be significant compared to the baseline acceleration sensitivity of the
QCM itself~\cite{filler1988acceleration}.  With the integration of a
centrifuge to a standard QCM, one can observe these effects under enhanced
g-forces and make endpoint measurements (measurements taken after the
addition of a sample) in the sample's parameter space continuously and
repeatedly.

To demonstrate the CF-QCM concept, six different samples were examined
under variable accelerations from approximately \SIrange{1}{90}{g}.  The
samples were chosen to be examples of the breadth of load situations
accessible with the CF-QCM technique.  They are:
\begin{enumerate}
\item air,
\item deionized water,
\item free particles in water,
\item paramagnetic particles attached to the sensor via short oligonucleotides,
\item \SI{48}{kbp} lambda phage DNAs attached to the gold electrode, and
\item polystyrene particles tethered to the sensor via \SI{48}{kbp} lambda phage DNAs.
\end{enumerate}

\begin{figure}[ht]
\centering
%\hspace{-1.0cm}
%\includegraphics{Figure-3_webster.pdf}
\includegraphics[width=16cm,keepaspectratio]{qcm/figures/figure2.pdf}
\caption{Load situations.
Change in frequency $\df$ and bandwidth $\dg$
(in hertz, inferred from motional resistance) of
the CF-QCM under different load situations as the centripetal acceleration
is directed in to (\textit{loading}, represented by circles with the top half
colored) and out of (\textit{unloading}, represented by circles with the
bottom half colored) the plane of the crystal.
The situations are 
(a) unloaded crystal in air, 
(b) deionized water, 
(c) free \SI{1}{\micro\meter} diameter streptavidin coated polystyrene
particles, $N_\mathrm{L}=\SI{1.58e11}{\particle\per\meter\squared}$,
(d) \SI{2}{\micro\meter} diameter streptavidin coated paramagnetic
particles, $N_\mathrm{L}=\SI{1.65e10}{\particle\per\meter\squared}$,
attached with \SI{25}{mer} oligonucleotides, 
(e) lambda DNA only attached to the gold electrode, and
(f) \SI{25}{\micro\meter} diameter streptavidin
coated polystyrene particles,
$N_\mathrm{L}=\SI{3.25e7}{\particle\per\meter\squared}$, tethered to the sensor surface with
\SI{48}{kbp} lambda DNAs.  Error bars are derived from uncertainties
(standard deviation) in
the centrifuge both spinning up and spinning down in a single experimental
run.
} 
\label{fig:loadplot}
\end{figure}
The quartz crystals are always cleaned before use by immersion
in fresh piranha solution (3:1 mixture of \SI{97}{\percent} \ce{H2SO4} and
\SI{30}{\percent} \ce{H2O2}) for \SI{5}{\minute} and rinsing liberally with
pure water.

\section{Air}
As a first step, the instrument's response was recorded in air, shown
parametrically in \Figure{fig:loadplot}(a).  The base acceleration
sensitivity (change in frequency versus change in g-force) of AT cut quartz
normal to the plane of the crystal has a reported
value~\cite{valdois2794influence} of $\df/\Delta g
=\SI{2.188+-0.006e-2}{\hertz\per\g}$ (see
\Table{tbl:environmentaleffects}).  The CF-QCM data shows similar
behavior: $\df/\Delta g =\SI{2.682+-0.023e-2}{\hertz\per\g}$ in the loading
configuration.  The signs of $\df/\Delta g$ are found to be opposite in the
loading and unloading configurations.  Though no references for
the bandwidth or motional resistance dependence of a QCM under
acceleration are known, the data predicts a linear relationship of $\dg/\Delta g =
\SI{9.203+-0.171e-4}{\hertz\per\g}$.  The slight increase in $\df/\Delta g$
in the unloading situation is attributed to the asymmetric stress caused by
the QCM holder under different orientations.

\section{Deionized Water}
Next, deionized water was used as a control sample for measurement in the
liquid phase, as shown in \Figure{fig:loadplot}(b).  The initial shift in
frequency and bandwidth is in agreement with what is obtained the
Kanazawa-Gordon relations~\cite{kanazawa1985frequency} 
\begin{align}
\df=&-f_{\text{F}}^{3/2}\left(\frac{\rho_{\text{L}}\eta_{\text{L}}}{\pi\rho_{q}\mu_{q}}\right)^{1/2}\\
\Delta R=&2f_{\text{F}}L\left(\frac{4\pi
 f_{\text{F}}\rho_{\text{L}}\eta_{\text{L}}}{\rho_{q}\mu_{q}}\right)^{1/2}
\end{align}
where $\rho_{\text{L}}$ and $\eta_{\text{L}}$ are the unknown density and
viscosity of the liquid.   Substituting in the values for water,
$\rho=\SI{1}{\gram\per\centi\meter\cubed}$ and
$\eta=\SI{1}{\milli\pascal\second}$, $\df = \SI{-714}{\hertz}$ and
$R=\SI{359}{\ohm}$, which are close to the measured values of $\df =
\SI{-716}{\hertz}$ and $R=\SI{357}{\ohm}$.  As is evidenced in
\Figure{fig:loadplot}(b), the response under centrifugal
load is linear and smaller than that of air: $\df/\Delta g=
\SI{1.357+-0.024e-2}{\hertz \per\g}$ and $\dg/\Delta g=
\SI{2.865+-0.073e-3}{\hertz\per\g}$ as found by linear least squares fitting.  

It is obvious that acceleration dependent forces in the liquid
phase are not necessarily commensurate with those in the gas phase.
However, because the inherent acceleration dependent signal is small, it
does not contribute significantly to the dominant features of the CF-QCM
signal for actual loads.

\section{Free Particles}
\begin{figure}
\centering
\includegraphics[keepaspectratio,width=12cm]{qcm/figures/qcm_beadssurface.jpg}
\caption{Free particles on the surface of the QCM, observed with a
microscope to obtain the number density.}
\label{fig:countparticles}
\end{figure}
Utilizing the flexibility that the instrument provides in modifying the
coupling between the load and the sensor surface, he technique is applied 
to the study of discrete micron sized particles.  
Free particles (Spherotech SVP-10-5, SVM-15-10, and SVP-200-4), of
different diameters were prepared by diluting a solution of
\SI{30}{\micro\liter} particles in \SI{300}{\micro\liter} \ce{H2O}.  A
\SI{125}{\micro\liter} aliquot of the \SI{300}{\micro\liter} volume was
then placed in the PDMS cell in contact with the sensing side of the
crystal.  The sensing area was calculated to be
\SI{1.195}{\centi\meter\squared}.  The particles in solution experience a
buoyancy, reducing their apparent mass. The surface density
$N_\mathrm{L}$ was determined as in \Figure{fig:countparticles} by counting the average number of particles
per unit area with a microscope and was found to be within
\SI{20}{\percent} of the value predicted by the volume concentration.

As first referenced in \Figure{fig:expsetup}(b), the frequency and
bandwidth shifts of free particles in the liquid phase as a function of
g-force is shown in \Figure{fig:loadplot}(c).  Free streptavidin coated
polystyrene particles with mean diameter $\bar{d}=\SI{1.07}{\micro\meter}$
are placed in the sample volume with a surface density of
$N_\mathrm{L}=\SI{1.58e11}{\particle\per\meter\squared}$ and the signal is
observed in both the loading and unloading configurations.  The particles
did not exhibit adhesion to either the unmodified gold electrode or the
glass/PDMS cell surrounding it; in the unloading configuration, the
particles quickly drifted away from the sensing area and a signal identical
to water was observed.  In the loading configuration, a large positive
shift in $\df$ and $\dg$ was observed, consistent with previously observed
responses for weakly coupled particles in the micron size
range~\cite{johannsman2007contacts}.  

% This is shown in ... settling velocity, flip cell upside/rightside, etc.

The initial shift under \SI{1}{g} was found to be $\df= \SI{2.2}{\hertz}$
and $\dg=\SI{7.5}{\hertz}$.  At the maximum acceleration of \SI{90}{g} the
signal increases to $\df = \SI{16.5}{\hertz}$ and $\dg=\SI{37}{\hertz}$.
This also represents a sensitivity enhancement in the minimum resolvable
surface density of the particles.  The scaling of $\df$ and $\dg$ with
increasing centrifugal load is nonlinear in the applied load, implying
non-Hertzian behavior~\cite{borovsky2001measuring}.

The same experiment was also carried out with \num{2}, \num{6}, \num{15},
and \SI{25}{\micro\meter} polystyrene particles.
%(actual mean diameters $d=\bar{d}=1.89, 5.86,,
%15\,\mathrm{and}\;\SI{24.8}{\micro\meter}$).
The loading curves all followed the same trend, but the relative shifts in
$\df$ and $\dg$ differed based on particle size.  The results from these
loads are summarized in \Table{tbl:particlesize}.
\begin{table}[h]
 \centering
	Frequency and Bandwidth Shifts\\
 \begin{tabularx}{240pt}{XXXXX}
 \toprule
 $\bar{d}$ [\si{\micro\meter}] & $\df_1/N_\mathrm{L}$ & $\dg_1/N_\mathrm{L}$ & $\df_{90}/N_\mathrm{L}$ & $\dg_{90}/N_\mathrm{L}$ \\
 \midrule
 %cat table.txt | sed -e s/E-/\\\\text{\\\\sc{e}-}/g | sed -e ``s/\t/ \& /g''
$1.07^p$ & 1.61\text{\sc{e}-}11 & 3.85\text{\sc{e}-}11 & 6.98\text{\sc{e}-}11 & 1.43\text{\sc{e}-}10\\
$1.89^m$ & 5.58\text{\sc{e}-}11 & 6.55\text{\sc{e}-}11 & 7.90\text{\sc{e}-}10 & 2.43\text{\sc{e}-}10\\
$5.86^m$ & 4.00\text{\sc{e}-}09 & 3.09\text{\sc{e}-}09 & 3.43\text{\sc{e}-}10 & 3.41\text{\sc{e}-}10\\
$15.0^p$ & 1.32\text{\sc{e}-}07 & 6.39\text{\sc{e}-}08 & 3.99\text{\sc{e}-}08 & 9.50\text{\sc{e}-}09\\
$24.80^p$ & 5.01\text{\sc{e}-}07 & 1.53\text{\sc{e}-}07 & 3.26\text{\sc{e}-}07 & 1.65\text{\sc{e}-}07\\
 \bottomrule
\end{tabularx}
\caption{Normalized frequency and bandwidth shifts (in
 \si{\hertz\meter\squared}) at \num{1} and
\SI{90}{g} for various particle sizes in water. The quoted diameter
$\bar{d}$ is
their mean diameter. $p$: polystyrene particles, $m$:
magnetite coated polystyrene.}
\label{tbl:particlesize}
\end{table}

\section{Oligo Attached Particles}
In contrast to the situation of free particles, the CF-QCM behavior was
also studied in a regime where particles are rigidly coupled to the sensor.
Rigid coupling was accomplished by attaching \SI{2}{\micro\meter} (mean
diameter $\bar{d}=\SI{1.89}{\micro\meter}$) streptavidin coated
paramagnetic particles modified with biotinylated \SI{25}{mer} oligos to
complimentary strands conjugated to the QCM gold surface via thiol bonds 

To attach oligos to the QCM, they were first immersed in a
\SI{1}{\micro\textsc{M}} solution of thiolated oligos (5'-ThioMC6-TTT TTT
TTT CAC TAA AGT TCT TAC CCA TCG CCC-3') in a \SI{1}{\textsc{M}} potassium
phosphate buffer, \SI{0.5}{\textsc{M}} \ce{KH2PO4}, pH 3.8 for
\SI{1}{\hour}.  Following, immersion in \SI{1}{\milli\textsc{M}}
6-Mercapto-1-hexanol (MCH) was used to block residual reactive sites on the
gold electrode.  After rinsing, attachment to the prepared particles was
done in STE buffer: \SI{1}{\textsc{M}} \ce{NaCl} with
\SI{10}{\milli\textsc{M}} Tris buffer, pH 7.4 and \SI{1}{\milli\textsc{M}}
EDTA\@.  A complimentary strand (5'-biotin-CT CAC TAT AGG GCG ATG GGT AAG
AAC TTT AGT-3') was attached to the streptavidin coated particles.  The
particles were first washed two times by alliquoting a
\SI{100}{\micro\liter} base solution of particles in \SI{100}{\micro\liter}
STE buffer, \SI{5000}{RPM} for \SI{3}{\minute} and decanting the
supernatant.  The particles were resuspended in \SI{20}{\micro\liter} of
STE buffer and \SI{10}{\micro\gram} of oligos were added.  The mixture was
incubated for \SI{15}{\minute} at room temperature under slow vortexing,
then washed again and resuspended in \SI{100}{\micro\liter} STE buffer.
The oligo attached particle suspension was allowed to attach to the gold
surface for \SI{15}{\minute} before spinning.

The response for oligo attached particles is shown in
\Figure{fig:loadplot}(d).  Note that $\df$ and $\dg$ are both negative and
decrease with centrifugal force in the loading orientation.  When spinning
with the oligo attached particles, it is suspected that the presence of the
particle is not sensed directly but rather the conformational state of the
oligonucleotide layer.  Such an acceleration effect has been observed
before~\cite{yoshimoto2002effect}~\cite{fawcett2004evidence}, but only
within the \SI{2}{g} orientation difference of gravity.  When the oligo
layer is under centrifugal load, it compresses, causing the
density-viscosity product to increase.  This behavior is consistent with
the behavior of DNA observed on QCMs under the influence of gravity
alone~\cite{fawcett2004evidence}.

\subsection{Verification of the Oligo Attachment}
The experimental setup provided two separate mechanisms for confirming both
the presence of thiolated oligos attached to the QCM gold surface and the
rigid attachment of particles to the surface by oligos.

First, with the attachment of oligos the motional resistance of the QCM
increased from $R=\SI{357}{\ohm}$ (pure water) to a typical value of about
\SI{550}{\ohm} for oligos with a standard deviation of \SI{4.3}{\ohm} over
five different runs.  The increase of motional resistance indicates the
adsorption of a viscoelastic layer.  Second, while observing the sample by
eye under a microscope, a handheld neodymnium magnet was brought in
proximity to the sample.  In the absence of the oligo attachment, the
particles readily moved away from QCM surface towards the magnet.  For
samples for which particles were believed to be attached, no such motion
was observed.  

%The initial trend in
%frequency follows Sauerbrey (negative frequency shift proportional to mass
%adsorption) like behavior.  For mass loading at this number density, the
%predicted frequency shift is  $\SI{-0.49}{\hertz}$.  At \SI{95}{g}
%$\df = \SI{-30}{\hertz}$, which approximately follows a linear
%increase of inertial mass.  
%We observe a \SI{5}{\percent} change in this product at \SI{90}{g}.  

\section{Lambda DNA}
Moving from particles to viscoelastic monolayers, in
\Figure{fig:loadplot}(e), \SI{48}{kbp} lambda phage DNA in STE buffer were
attached to the gold sensor electrode via a complimentary thiolated oligo.  
Lambda DNAs were prepared by combining \SI{50}{\micro\liter} of lambda DNA
at \SI{500}{\micro\gram\per\milli\liter}, \SI{5.5}{\micro\liter} of 10x T4
ligase, and \SI{0.5}{\micro\liter} of diluted \SI{10}{\micro\textsc{M}}
thiolated linker oligonucleotide and heating to \SI{70}{\celsius} for
\SI{5}{\minute}.  The suspension was left to cool to room temperature as
the litigation of the oligos to the DNA COS ends occured. Once the mixture
was at room temperature, \SI{15}{\micro\liter} 10x ligase buffer,
\SI{127}{\micro\liter} \ce{H2O}, and \SI{2}{\micro\liter} T4 DNA ligase was
added to the annealed linker.  The reaction was allowed to proceed at room
temperature for \SI{3}{\hour}.  Upon the attachment of the lambda DNA, the
motional resistance of the QCM was observed to have increased from its base
value of $R=\SI{357}{\ohm}$ for pure water to a typical value of
\SI{575}{\ohm}.  Unfortunately, the surface functionalization of the bare
crystal could not be performed \textit{in situ}, and therefore values for
$\df$ are unknown.  The QCM200 PLL based driver reports absolute values for
$R$, but relative values for $\df$.  The instrument has a limited range for
$\df$, which is set to zero before each experimental run.

Previous studies have shown that, through the use of dissipation
monitoring, QCMs are sensitive to not only the adsorbed mass and viscosity,
but the physical conformal state (``shape'') of DNAs hybridized to the
sensor surface~\cite{tsortos2008shear}.  In the experiment, even though the
force on the lambda DNAs is on the order of femtonewtons, a strong linear
decrease ($\dg=\SI{-2.912+-0.0095}{\hertz\per\g}$) in the bandwidth as
function of g-force was observed, indicating an increase in viscoelastic
loss.  However, under larger g-forces the sign of $\df$ reverses.  The
origin of the $\df$ reversal is not understood, but could indicate a
nonlinear viscoelastic compliance under load.  The unloading configuration
sees a smaller negative response in $\dg$ with little effect on $\df$.

At this point it is important to make an observation about the influence of
salt buffer, which was used in experiments involving DNA.  There are
several studies~\cite{encarnaccao2007influence}~\cite{lin1995role}
regarding the effects of various electrolytic buffer solutions and their
concentrations on QCM measurements, including reports of an immersion angle
(and therefore gravity) dependence~\cite{yoshimoto2006characteristics}.
These reports suggest the immersion angle depencence may be related to the
behavior of the interfacial layer and ion transport in monovalent
electrolytic solutions in accelerating
frames~\cite{tolman1911electromotive}~\cite{des1893unpolarisirbare}.  In
the CF-QCM experiment, a significant contribution in the unloading
configuration was observed for STE buffer alone
($\df=\SI{-0.326+-0.0029}{\hertz\per\g}$, and $\dg$ nonlinear), which was
subsequently ``screened''~\cite{zhang2002insulating} by the presence of
both oligos and lambda DNAs, making the effect negligible in the current
set of experiments.  Further investigation is required to explain this
precisely.

\section{Tethered Particles}
With the sensitivity to both particles and monolayers, the instrument
raises a possibility for using beads tethered by lambda DNA as a
transduction mechanism to investigate its kinetics.  One such example is
shown in \Figure{fig:loadplot}(f).  Streptavidin coated polystyrene
particles with a mean diameter of \SI{24.8}{\micro\meter} were tethered to
the CF-QCM by means of a \SI{48}{kbp} lambda phage DNA.  Experiments were
done in STE buffer whose density reduced the maximum force the bead could
exert to about \SI{40}{\pico\newton} which, according to the worm-like
chain model~\cite{marko1995stretching}, should almost fully extend the
lambda DNA to a length of \SI{16}{\micro\meter}.

In addition to the streptavidin coated polystyrene particles, visual
confirmation of attachment was made by eye with a microscope, magnet, and
paramagnetic paricles in the same way as was done for the free particles.

Though the instrument has not yet been developed enough to make accurate
quantitative measurements of teathered particles, the behavior of the data
is a clear indication of its potential.  As the tethered bead extends the
DNA under centrifugal force, $\df$ increases and $\dg$ decreases.  In the
case where the DNAs are trapped and pushed between the bead and the
surface, both $\df$ and $\dg$ increase.  The signs of the shifts were
confirmed with \SI{10}{\micro\meter} and \SI{6}{\micro\meter} paramagnetic
particles, using a magnet to either pull or push the particles toward or
away from the sensor surface.  The observed behavior  was distinct from
either the case of lambda DNA or free particles alone.

At $F_\mathrm{c}=\SI{40}{\pico\newton}$, the frequency shift indicates an
effective decrease in the density-viscosity product of \SI{10}{\percent} or
about \SI{1.5}{\pico\gram}.  For the surface densities involved
($N_\mathrm{L}=\SI{3.25e7}{\particle\per\meter\squared}$), the equivalent
interfacial mass lost for a fully extended lambda DNA predicted by the
worm-like chain model are in the picogram range and cannot account for the
more than $10^6$ signal difference shown.  If indeed the response is due to
lambda DNA extension, future experiments involving high frequency, large
centrifugal force CF-QCMs could easily detect the kinetics of a single
tether.

\section{Particle Sizing}
\begin{figure}[ht]
\centering
\includegraphics{qcm/figures/circlefit.pdf}
\caption{ Particle sizing.  Method for sizing micron-sized particles using the CF-QCM. 
$\df$ versus $\dg$ is plotted parametrically as a function of g-force, and
the data is fit to a circle.  The point on the circle for which $\dg=0$ and
$\df<0$ provides an estimate of mass adsorption, and thus particle size.
Results for particles with diameters $\bar{d}_\mathrm{actual}=1, 2,
15,\,\mathrm{and}\;\SI{25}{\micro\meter}$ are shown in
\Table{tbl:particlesize}.  Fit circle has a radius of \SI{55.77}{\hertz} and
a center of $(-37.83,41.05) \si{\hertz}$. }
\label{fig:circlefit}
\end{figure}

The coupled oscillator model (\Equation{eqn:mastereq}), when analyzed for
samples of free particles (\Figure{fig:expsetup}(b),
\Figure{fig:loadplot}(c), and \Table{tbl:particlesize}), suggests an avenue
to allow QCMs to determine the size of large micron-sized particles in the
liquid phase.  Thus far this has only been possible with nanometer-sized
particles which lie within the QCM's shear acoustic
wave~\cite{olsson2013using}.  If one plots $\df$ verses $\dg$ in
\Equation{eqn:mastereq} as a parametric function of $\kl$, the points are
found to lie on a circle with radius $r_\mathrm{L}$ (see
\Figure{fig:mechanicalmodel}).  

Making the approximation $\xil\ll \kl$, and that the damping
is small, $\xil\ll1$, the radius is found to be 
\begin{align}
\frac{r_\mathrm{L}}{f_\mathrm{F}} &=
\frac{N_\mathrm{L}}{\pi \mathrm{Z}_\mathrm{q}}
\frac{\left(m_\mathrm{L}\omega_\mathrm{q}\right)^2}{2 \xi_\mathrm{L}}
\sqrt{1+\left(\frac{\xi_\mathrm{L}}{m_\mathrm{L}\omega_\mathrm{q}}\right)^2}\\
&\approx \frac{N_\mathrm{L}}{\pi \mathrm{Z}_\mathrm{q}}
\frac{\left(m_\mathrm{L}\omega_\mathrm{q}\right)^2}{2 \xi_\mathrm{L}}, \quad \mathrm{for}\; \xi_\mathrm{L} \ll 1.
\label{eqn:radius}
\end{align}
In considering the parametric representation of $\df$ and $\dg$, the
physical mechanism modifying $\kl$ is removed from the problem.  If a
circle is fitted to the experimentally observed $\df$-$\dg$ data (again,
plotted parametrically as a function of g-force), its behavior can be
extrapolated to the strong coupling regime by finding the point at which
$\dg=0$ and $\df<0$.  Knowing $\df$, \Equation{eqn:couplinguy2} can then be
inverted to solve for either number density or particle size/mass.  An
example of the fitting procedure is shown in \Figure{fig:circlefit} using
the same data for \SI{1}{\micro\meter} particles shown in
\Figure{fig:expsetup} and \Figure{fig:loadplot}.  Inset is a table for the
same predictions done for particles with known diameter
$\bar{d}_\mathrm{actual}=1, 2, 15,\,\mathrm{and}\;\SI{25}{\micro\meter}$.
In all cases the surface density was known and the diameter
$\bar{d}_\text{predicted}$ was derived from the mass $\ml$, found by
inverting \Equation{eqn:couplinguy2}.  The results are surprisingly
accurate despite the exploratory nature of the instrument's construction.
It should also be mentioned that with knowledge of the way in which the
g-force modifies $\kl$, the frequency zero crossing at
$k_\mathrm{zc}=\omegaq^2\ml$ can be used to determine the mass $\ml$
without knowledge of the number density $N_\mathrm{L}$. 

%\section{Strange Buffer Effects}
%\begin{itemize}
%\item Temperature effects as described in the manual are most accurate,
%but are too slow compared to the force-frequency curves we see.
%\item Pressure is also ruled out. The mass difference between water and
%\SI{1}{M} \ce{NaCl} is about \SI{7.5}{\milli\gram}, while the signal
%difference is \SI{-1}{\hertz} for water and \SI{-30}{\hertz} for
%\SI{1}{M} \ce{NaCl}.
%\item Mechanical stress also ruled out for the same reason.
%\end{itemize}
