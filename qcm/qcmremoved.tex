\chapter{Foundations}

\section{Overview}
Mechanical properties such as
viscoelasticity~\cite{steinem2007piezoelectric}, contact
stiffness~\cite{johannsman2007contacts} and adhesion, and force-dependent
conformational changes~\cite{fant2000adsorption} have been shown to be
instrumental in understanding biofunctional behavior, from single molecules
to complex collections of cells~\cite{li2008thickness}. These properties
are often descriptive for specific molecules and cell types. For example,
force extension curves of DNA vary with assembly of histone proteins into
chromatin~\cite{cui2000pulling}~\cite{larson2012trigger}. In cells,
viscoelasticity alone can discriminate between healthy and cancerous
tissue~\cite{rebelo2013comparison}.  Biosensors that can probe mechanical
properties are therefore ideal for both fundamental studies in life
sciences as well as diagnostic assays improving public health.

Perhaps the simplest way to probe a biomechanical property is by monitoring
its response to direct application of force. Current force based
approaches, for example those based on atomic force microscopy or optical
or magnetic tweezers, are powerful but limited in their applicability as
biosensors. In particular, their operation requires significant expertise
on the part of the investigator, and is often constrained to well prepared
samples not amenable to multiplexing.

Among tools suitable for direct mechanical transduction, the quartz crystal
microbalance (QCM) has seen increasing real-world utility as simple, cost
effective, and highly versatile mechanical biosensing platforms.  A QCM
typically manifests itself as a thin disk of piezoelectric quartz with
electrodes on either side.  The quartz is cut such that, when driven by a
potential, the crystal produces acoustic shear waves on either of its two
faces.  When a material is in contact with the crystal, its resonance
condition will change.  These changes can be used to interrogate both the
viscoelastic properties of the material and the way in which it is coupled
to the QCM.  This mechanical turns out to be extrorinary sensitive, even in
liquid environments, with sensitivities to mass on the order of X.  The
function of a QCM is depicted schematically in \Figure{fig:qcmschema}.

%\begin{figure}
% \caption{Schematic representation of the operation of a QCM.  The
%  quartz crystal is driven by a pair of electrodes on either sides to
%  excite \si{\mega\hertz} acoustic shear waves.  A sample in contact with
%  the crystal will typically change its resonant frequency $\df$ and
%  bandwidth $\dg$, which are connected to the viscoelastic properites of
% the sample and the way in which it is coupled to the device.}
%\end{figure}

Since its introduction by Sauerbrey~\cite{sauerbrey1959verwendung} in 1959
as sub-monolayer thin-film mass sensors in the gas phase, the understanding
of these piezoelectric devices has been repeatedly enhanced to study
phenomena such as viscoelastic films in the liquid
phase~\cite{kanazawa1985frequency} , non-destructive contact
mechanics~\cite{borovsky2001measuring}~\cite{johannsman2007contacts}, and
complex topologies of biopolymers and
biomacromolecules~\cite{marx2003quartz}. However, despite their popularity
QCMs suffer from low-Q resonances which negatively impact their
sensitivity. In addition, extracting quantitative mechanical information
from biomolecules is often confounded by non-trivial interpretation of the
discrete shifts in the system's frequency and bandwidth.

The centrifugal force quartz crystal microbalance (CF-QCM), shown in
\Figure{fig:cfqcm} is our tool to do something.  The CF-QCM is a new type
of instrument which places a quartz crystal microbalance in a standard
commercial centrifuge.  When spinning, controllable centrifugal force is
applied to a biomaterial under assay, and the QCM signal as a function of
applied force is monitored in situ and in real time.

Where typical mechanical biosensing produces a single, stepwise sensor
response that can be linked to, in the case of a QCM, the static adsorbed
mass, contact stiffness, or viscoelastic compliance, our proposal allows
for the direct manipulation of elective dynamic versions of these
parameters through centrifugal force gradients.
Examples of loads which could be probed and the possible action of
centrifugal force are shown in \Figure{fig:cfqcm}
\begin{figure}
 \caption{Schematic of the CF-QCM.}
\end{figure}
\begin{table}
 \centering
\begin{tabular}{lll}
\toprule
& load & action\\
\midrule
(a)& monolayer of DNA molecules& change the conformal shape of DNA\\
(b)& protein coated microparticles& affect the contact stiffness\\
(c)& microparticles tethered with lambda DNA& extend and stretch DNA\\
(d)& heterogeneous collections of cells& modify cell adhesion and contact area\\
\bottomrule
\end{tabular}
\caption{Load situations depicted in \Figure{fig:cfqcm}}
\end{table}

\section{What is New Here}
QCMs have been studied in the context of almost every type of sample
imaginable, but up to now the introduction and study of the behavior of
different load situations under uniform force gradients has not been
carried out.
The novel aspects of this work is the direct introduction of force
into\dots

\begin{description}
\item[{Acceleration Boundary Effects}] While the $kT$ energy of a
small biomolecule are orders of magnitude greater than gravity (energy,
gravity, huh?)
, acceleration effects are readily observable for hybridized
oligos and bufers stuff.  We confirm the 2g effect for oligos and provide
data into the 100g range, as well as confirm the effect for random buffer
solutions.
\item[{Contact Mechanics}] Pressing beads and looking at the response,
 maybe the equivalent model for contact stiffness in the liquid phase,
 which is just about shear stuffs.
\item[{one more thing so this list isn't empty}] That would be nice.
\end{description}

QCMs have been the subject of extensive study, especially in the context of
biosensing, and have for some time been implemented in commercial sensing
devices.  I will only cover areas related to new results here.  For a
comprehensive overview to the principles and applications of these
mechanical resonators, the reader is directed to \cite{steinemreview}.

\section{Historical Perspective}
As sensors, QCMs found their initial applications monitoring thin film
deposition in vacuums.  Here it was found that a QCM would exhibit a
frequency shift $\df$ proportional to the adsorbed mass $m$
\begin{equation}
 \df=-\frac{2f_{F}^{2}}{A\sqrt{\rho_{q}\mu_{q}}}\Delta m
 \label{eqn:sauerbrey}
\end{equation}
where $A$ is the active area of the crystal, $\Delta m$ is the adsorbed
mass, $\rho_{q}$ is the density of quartz and $\mu_{q}$ is the shear
modulus of quartz. Usually \Equation{eqn:sauerbrey} is condenced such that
$\Delta m$ is a linear function of a single ``sensitivity factor''
$C_\mathrm{f}$
\begin{equation}
 \Delta f=-C_{\text{f}}\Delta m
\end{equation}
where $C_{\text{f}}=\SI{56.6}{\hertz\per\micro\gram\centi\meter\squared}$.
\Equation{eqn:sauerbrey} is known as the
\textit{Sauerbrey relation}.  A typical QCM might have a
$f_F=\SI{5}{\mega\hertz}$ and a frequency resolution of \SI{0.1}{\hertz};
\Equation{eqn:sauerbrey} predicts sub-monolayer (and is only valid in this
range) resolution of the thickness
metal and dielectric layers.  This is the reason the name QCM includes the
word ``microbalance''.

Initially it was thought that the low $Q$ mechanical resonance of the QCM
precluded its use in the liquid phase.  This was incorrect, as subsequent
work published in 1985 by \name{Gordon} and \name{Kanazawa}~\cite{guys} extended the
treatment of \name{Sauerbrey} to the liquid phase.  These relations,
derived from a Butterworth van Dyke equivalent circuit, predicted the
resonant frequency of the QCM
to be sensitive to the density-viscosity product of the liquid in contact
with the crystal, 
\begin{align}
\df=&-f_{\text{F}}^{3/2}\left(\frac{\rho_{\text{L}}\eta_{\text{L}}}{\pi\rho_{q}\mu_{q}}\right)^{1/2}\\
\Delta R=&2f_{\text{F}}L\left(\frac{4\pi
 f_{\text{F}}\rho_{\text{L}}\eta_{\text{L}}}{\rho_{q}\mu_{q}}\right)^{1/2}
\end{align}
where $\rho_{\text{L}}$ and $\eta_{\text{L}}$ are the unknown density and
viscosity of the liquid.  Note that in the treatment by both Sauerbrey and
Gordon and Kanazawa, the frequency shifts are always \textit{negative} as a
function of increasing mass or density-viscosity.

The same year, \name{G. Dybwad} published a rather elegant
experiment~\cite{dybwad} in which he looked at the frequency shift of a QCM
in air when a single micron sized gold particle rests on the sensor
surface.  Remarkably, \name{Dybwad} reported a \textit{positive}
frequency shift.  The shift was explained with a coupled oscillator model
shown in \Figure{fig:dybwad}.
%\begin{figure}
% \caption{Coupled oscillator model used by \name{Dybwad}.}
% \label{fig:dybwadschema}
%\end{figure}
Here, the quartz crystal with mass $\mq$ and stiffness
$\kq$ resonates at
$\omega_\mathrm{q}^2=\kq/\mq$ and is coupled by a spring
$\kl$ to a load mass $\ml$.  The load mass is the actual mass of
the particle, $4/3 \pi r^3 \rho$.  The spring $\kl$ represents the
``stiffness'' of the contact -- how strongly the particle is coupled to the
QCM surface.  The coupled system in \Figure{fig:dybwadschema} is described
by the differential equations
\begin{align}
 \mq \ddot{x}_\mathrm{q} &= -\kq x_\mathrm{q}\\
 \ml \ddot{x}_\mathrm{L} &= -\kl (x_\mathrm{q}-x_\mathrm{L})
\end{align}
which, using an ansatz of
$x_{\mathrm{q},\mathrm{L}}=A_{\mathrm{q},\mathrm{L}}\me^{\mi \omega t}$ has
eigenvalues $\omega$ of
\begin{equation}
 2\omega^{2}=\left(\frac{\kq}{\mq}+\frac{\kl}{\mq}+\frac{\kl}{m}\right)\pm\left(\left(\frac{\kq}{\mq}+\frac{\kl}{\mq}+\frac{\kl}{\ml}\right)^{2}-4\frac{\kq}{\mq}\frac{\kl}{\ml}\right)^{1/2}
 \label{eqn:dybwadresult}
\end{equation}

%\begin{figure}
% \caption{Resonant frequency of the Dybwad model as a function of coupling
%  strength, $\kl$.}
% \label{fig:dybwadplot}
%\end{figure}
\Equation{eqn:dybwadresult}, shown in \Figure{fig:dybwadplot} has two
important limits about $\omegaq^2 \ml$
\begin{align}
 \lim_{\kl\to\infty} \omega^2 &= \frac{\kq}{\mq+\ml}\\
 \lim_{\kl\to0} \omega^2 &= \frac{\kq}{\mq}
\end{align}
In the weak coupling regime, $\kl\ll\omegaq^2\ml$, the particle causes a
positive frequency shift proportional to $\kl$ and independent of $\ml$.
Indeeed, in \name{Dybwad}'s model, in the weak coupling regime the
particle is seen to be at rest in the labratory frame. In the limit of strong
coupling, $\kl\gg\omegaq^2\ml$, the mass adsorption predicted by
\name{Sauerbrey} takes over.  Dybwad's work is important because it
establishes a continuum model for QCM behavior, encompassing both positive
and negative frequency shifts.

Some dingus guys use a nanoindenter probe combined with a QCM, this is
where we jump off.

Last, Yuki got the idea from other applicaions of force on stuff done by
Ken and Wesley.  So plan to jump into the main text.


\section{DF Ratio}
\chapter{Experimental Setup}
The experimental setup consists of a 
%\begin{figure}[h]
%\centering
%\import{figures/}{qcm_annotated_picture.pdf_tex}
%\end{figure}

\section{Mechanical}
\subsection{Centrifuge}
\subsection{Microfluidic Cell}
\subsection{Mounting and Stress}

\section{Driving Circuit}
The QCM driving circuit employed in our experiments was an SRS 
\subsection{Butterworth van Dyke Equivalent Circuit}
The typical circuit used to analyze QCM behavior is called the
\textit{Butterworth van Dyke} (BvD) circuit.  It consists of a capacitor
$C_s$, an inductor $L_s$, and a resistor $R_s$ in series with a parallel
capacitance $C_0$.
\begin{center}
 \begin{tikzpicture}[scale=0.75]
 \draw (1,0) node[anchor=east] {}
  to[short, o-*] (2,0)
  to[short] (2,1)
  to[R, l^=$R_s$] (4,1)
  to[C, l^=$C_s$] (6,1)
  to[L, l^=$L_s$] (8,1)
  to[short] (8,-1)
  to[short] (6,-1)
  to[C, l^=$C_0$] (4,-1)
  to[short] (2,-1)
  to[short] (2,0);
 \draw (9,0) node[anchor=west] {}
  to[short, o-*] (8,0);
\end{tikzpicture}
\end{center}

The top branch is the \textit{motional branch}, and relates to the crysal
and its interaction with the environment.  The bottom is the \textit{static
branch}, representing the paracitic capitances of the quartz and its driver.
In the SRS QCM200\cite{srsqcmmanual}, and probably any other similar
compensated phase locked oscillator circuit, $C_0$ is nulled with
additional circuitry.  This is absolutely crutial, as the parallel $C_0$ 
pertdubes the resonance frequency of the circuit by about
\SI{0.825}{\hertz\per\pico\farad}.  The SRS manual gives a higher value
of \SI{2}{\hertz\per\pico\farad}.

Typical values for the \SI{1}{in} \SI{5}{\mega\hertz} AT cut quartz
crystal used in with the QCM200\cite{srsqcmmanual} are

\begin{table}[h]
\begin{tabular}{ll}
 $R_s$ & \SI{400}{\ohm} (water), \SI{10}{\ohm} (air) \\
 $C_s$ & \SI{33}{\femto\farad} (SRS manual)\\
 $L_s$ & \SI{30}{\milli\henry} (SRS manual), \SI{40}{\milli\henry} (my prediction) \\
 $C_0$ & \SI{20}{\pico\farad} (SRS manual)
\end{tabular}
\end{table}

The circuit may be also be solved using the following second order linear
differential equation for charge
\begin{equation}
 L\ddot{q}+R\dot{q}+q/C = V(t)
\end{equation}
The natural frequency is 
\begin{equation}
 f_0 = \frac{1}{2 \pi L C}
\end{equation}
Where $C$ in the above equation takes into account both $C_0$ and $C_s$
\begin{equation}
 C = \frac{C_s C_0}{C_s + C_0}
\end{equation}

This, as well as more complicated equivalent circuit models can also be
computed directly using SPICE, as per the following code snippet
\begin{minted}{text}
* Butterworth van Dyke equivalent circuit
V1 0 1 ac 1 dc 0
Rs 1 2 100
R1 2 3 375
C0 2 0 20e-12
C1 3 4 33.0e-15
L1 4 0 30e-3
.control
ac lin 100000 5.052e6 5.063e6
write bvd.raw all
\end{minted}

Liquid sensing under a rigid coating is modeled with a modified Butterworth
van Dyke circuit as follows 

\begin{center}
 \begin{tikzpicture}[scale=0.75]
 \draw (1,0) node[anchor=east] {}
  to[short, o-*] (2,0)
  to[short] (2,1)
  to[R, l^=$R_\text{s}$] (4,1)
  to[C, l^=$C_\text{s}$] (6,1)
  to[L, l^=$L_\text{s}$] (8,1)
  to[L, l^=$L_\text{c}$] (10,1)
  to[L, l^=$L_\text{l}$] (12,1)
  to[R, l^=$R_\text{l}$] (14,1)
  to[short] (14,0);
  \draw (14,0) node[anchor=west] {}
  to[short] (14,-1)
  to[C, l^=$C_0$] (8,-1)
  to[short] (2,-1);
 \draw (14,0) node[anchor=west] {}
  to[short] (14,-3)
  to[C, l^=$C_\text{l}$] (6,-3)
  to[short] (2,-3)
  to[short] (2,0);

 %\draw (9,0) node[anchor=west] {}
 % to[short, o-*] (14,0);
\end{tikzpicture}
\end{center}



\section{Chemistry and Surface Functionalization}
\subsection{Cleaning the Crystals}

\chapter{Load Situations}
\section{Environmental Effects and Sources of Noise}
\subsection{Temperature Drift}
\section{Air}
\section{Deionized Water}
\section{Salt Buffers}
\section{Microparticles}
magnetic lift off
different radii
\section{Lambda DNA}
\section{Oligo Attached Particles}
\section{Particles Tethered to the Surface with Lambda DNA}

\chapter{Strange Effects}
\section{Bubble Resonance}

\chapter{Future Work}
