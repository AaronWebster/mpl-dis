\begin{figure}[ht]
\centering
\import{qcm/figures/}{qcm_annotated_picture.pdf_tex}
\caption{Annotated picture of the prototype centrifugal force quartz
crystal microbalance.}
\label{fig:cfqcmexpsetup}
\end{figure}
The CF-QCM experimental setup is shown in \Figure{fig:cfqcmexpsetup}.  It
consists of a QCM integrated into the arm of a commercial swinging bucket
centrifuge.  The QCM is connected in proximity to a remote driver which is
tethered via a slip-ring connector to external data acquisition
electronics.  The crystal itself is mounted in a holder radially by its
edges such that the centrifugal force $F_\mathrm{c}$ is always normal to
the surface of the crystal.  On the sensing side of the crystal is a
$\SI{125}{\micro\liter}$ volume PDMS/glass cell containing the sample.  The
non-sensing side of the crystal remains in air.  When in operation, the
crystal and cell are mounted in either the \textit{loading} configuration,
where the centrifugal force is \textit{in to} the sensing side or, by
mounting it upside down, in the \textit{unloading} configuration, where the
force is \textit{away from} the sensing side.

The experimental setup used a \SI{25}{\milli\meter} diameter
\SI{5}{\mega\hertz} gold coated quartz crystal in combination with an SRS
QCM200 phase-locked loop based driver circuit.  


On the sensing side of the crystal is a
\SI{125}{\micro\liter} PDMS/glass cell containing the specimen under
investigation.  The cell is made of a thin o-ring of PDMS (Sylgard 184,
10:1 ratio, cured \SI{20}{\minute} at \SI{120}{\celsius})
$\text{OD}=\SI{25}{\milli\meter}$, $\text{ID}=\SI{15.5}{\milli\meter}$ in
contact with the sensing side of the crystal and covered with
\SI{25}{\milli\meter} round N\raisebox{0.25em}{\relsize{-2}\b{o}}~1
coverglass, nominal thickness \SI{0.15}{\milli\meter}.  The non-sensing
side of the crystal remains in air and is isolated from the body of the
centrifuge.  
