As sensors, QCMs found their initial applications monitoring thin film
deposition in the vacuum phase, where it was found that a QCM would exhibit
a frequency shift $\df$ linearly proportional to the adsorbed mass.  Such
experiments~\cite{sauerbrey1959verwendung} were first carried out by
\name{Sauerbrey} in the vacuum phase.  Known as the \textit{Sauerbrey
relation}, the negative proportionality of $\df$ to adsorbed mass was found
to be valid in the regime of sub-monolayer thicknesses of rigidly attached
(metal, dielectric) layers.

Initially it was thought that the low Q mechanical resonance of the QCM
precluded its use in the liquid phase, however subsequent work published in
1985 by \name{Gordon} and \name{Kanazawa} showed liquid phase measurements
were indeed possible, extending the treatment of \name{Sauerbrey} to
encompass viscoelasticity.  The relations of \name{Gordon} and
\name{Kanazawa}, derived from a Butterworth van Dyke equivalent
circuit~\cite{kanazawa1985frequency}, predicted the resonant frequency of
the QCM to be sensitive to the density-viscosity product of the liquid in
contact with the crystal.  In both the gas phase and for viscoelastic
materials, the frequency shifts were again always found to be
\textit{negative} as a function of increasing mass or density-viscosity.

In the same year as \name{Gordon} and \name{Kanazawa}, \name{Dybwad}
published a rather elegant experiment~\cite{dybwad1985sensitive} in which
he looked at the frequency shift of a QCM in air when a single micron sized
gold particle was placed on the sensor surface.  Remarkably, \name{Dybwad}
reported a \textit{positive} frequency shift.  Dybwad explained the
positive frequency shift result with a coupled oscillator model.  The
coupling between the two systems was mediated by the strength of the
contact between the particle and the QCM, which determined the sign of the
frequency shift.  In the weak coupling regime, the particle was at rest in
the laboratory frame while the QCM moved beneath it, and a positive
frequency shift was observed.  In the strong coupling regime, the QCM and
the particle moved together, and the negative frequency shift of
\name{Sauerbrey} was observed.  The coupled oscillator model was more
recently developed by \name{Johannsmann}, extending the mathematics to
encompass heterogenous collections particles in the liquid
phase~\cite{johannsman2007contacts}.

Further work has been done with nano-indentation probes in the gas phase,
particularly by \name{Borovsky}~\cite{borovsky2001measuring}, looking at
the influence of micron-sized spherical tips pressed against the sensor
surface.  When a micron-sized tip is pressed against the QCM surface, the
same positive frequency shift as \name{Dybwad} is observed as a function of
applied force, which is identified with a lateral (sphere-plate) Hertzian
spring constant.

The present CF-QCM concept~\cite{webster2013qcm} was first proposed by
\name{Sato}, inspired by a multiplexed optical centrifugal force microscopy
technique~\cite{halvorsen2010massively} developed by \name{Halvorsen} and
\name{Wong}.  The present work lies in between that of \name{Johannsmann}
and \name{Borovsky}; it is conjectured that the application of force for
discrete objects on a QCM will modify the mechanism of coupling which will
be observed as a specific feature of the QCM sensorgram.
