As sensors, QCMs found their initial applications monitoring thin film
deposition in vacuums.  Here it was found that a QCM would exhibit a
frequency shift $\df$ linearly proportional to the adsorbed mass.  This
work was first carried out by \name{Sauerbrey} in the vacuum phase.  This
\textit{Sauerbrey relation} predicted (and was only valid for) sub-monolayer
thicknesses of rigidly attached (metal, dielectric) layers.  This is the
reason the name QCM includes the word ``microbalance''.

Initially it was thought that the low $Q$ mechanical resonance of the QCM
precluded its use in the liquid phase.  This was incorrect, as subsequent
work published in 1985 by \name{Gordon} and
\name{Kanazawa}~\cite{kanazawa1985frequency} extended the treatment of
\name{Sauerbrey} to the liquid phase.  These relations, derived from a
Butterworth van Dyke equivalent circuit, predicted the resonant frequency
of the QCM to be sensitive to the density-viscosity product of the liquid
in contact with the crystal.  In each of these situations, the frequency
shifts were always found to be \textit{negative} as a function of
increasing mass or density-viscosity.

The same year, \name{Dybwad} published a rather elegant
experiment~\cite{dybwad1985sensitive} in which he looked at the frequency
shift of a QCM in air when a single micron sized gold particle was placed
on the sensor surface.  Remarkably, \name{Dybwad} reported a
\textit{positive} frequency shift.  Dybwad explained this result with a
coupled oscillator model, asserting that the gold particle remained at rest
in the laboratory frame while the QCM oscillated beneath it.  The coupling
between the two systems was mediated by the strength of the contact between
the particle and the QCM, which determined the sign of the frequency shift.
In the weak coupling regime, the particle was at rest and the negative
frequency shift predicted by \name{Sauerbrey} was observed.  In the strong
coupling regime, the particle took part in the motion of the QCM and a
positive frequency shift was observed.

We will partake of the details of this model shortly, for now it is enough
to note the importance of \name{Dybwad}'s work as a first attempt at a
continuum model for QCM behavior, encompassing both positive and negative
frequency shifts.  This formalism was more recently developed by 
\name{Johannsmann}, exdending the mathematics to encompass heterogenous
collections particles in the liquid phase~\cite{johannsman2007contacts}.

The present concept for applying centrifugal force~\cite{webster2013qcm}
was first proposed by our collaborator \name{Sato}, who was inspired by a
multiplexed optical centrifugal force microscopy
technique~\cite{halvorsen2010massively} developed by \name{Halvorsen} and
\name{Wong}.  
