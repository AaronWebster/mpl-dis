\section{Sensitivity Simulations}
Though there are several reports of the occurance of the SPR wiggles, the
sensitivity of these wiggles to changes in the refractive index of the
analyte layer have up to this point not been established.  

The wiggles in the specular and conical directions were calculated by
numerically evaluating their respective integrals
\begin{align}
 E_\text{spec}(x,z) &=  \intinfty \tilde{g}(k_x)\, \tilde{r}_{123}(k_x)\,
 \me^{\mi k_{z,1} z}\, \me^{\mi k_x x} \md k_x\\
 E_\text{cone}(x,z) &= \intinfty \tilde{g}(k_x)\,
 \tilde{r}_{321}(k_x)\,\me^{\mi k_{z,1} z}\, \me^{\mi k_x x} \md k_x
\end{align}
where 
\begin{align}
 \tilde{g}(k_x) = \frac{\omega_x}{\sqrt{2}}\,\exp\left({-(k_x-k_\text{SP})^2 \omega_x^2}\right)
\end{align}
is a Gaussian beam about the surface plasmon resonance condition
$k_\text{SP}$ and $\tilde{r}_{123}(k_x)$ and $\tilde{r}_{321}(k_x)$ are the
three layer Fresnel reflectivities for the system.  

The numerical calculation was facilitated through use of fast Fourier
transform (FFT) as detailed in REF.

\subsection{Angular Interogation Simulations}

\subsubsection{Minimum Resolvable Shift Without Noise}


\subsubsection{Minimum Resolvable Shift With Noise}
show the minimum RIU that gives you a reliable one pixel shift - shott
noise

\subsection{Intensity Interrogation Simulations}


\subsection{Explaination of Behavior}
