The theoretical groundwork for the existence of surface plasmon polaritons
(SPPs) was first introduced by Richie in his seminal 1957 paper
\textit{Plasma Losses by Fast Electrons in Thin Films}
\cite{ritchie1957plasma}.  Like any scientific work, Richie's was
incremental and has its roots in earlier theoretical proposals by Pines and
Bohm~\cite{bohm1951collective}~\cite{pines1952collective}.  Ultimately this
research functioned to explain the phenomena of sharp and spectrally narrow
energy losses observed in diffraction gratings by Wood in 1902, known as
``Wood's anomaly''.

Optical excitation of surface plasmons was made accessible through
pioneering work in the late 1960's by Kretschmann~\cite{kretschmann1968},
Raether~\cite{raether1965springer} and Otto~\cite{otto1968excitation}.
These experiments used the principle of attenuated total reflection (ATR)
to excite surface plasmons evanescently, using a prism to match their
resonance condition.  A great deal of understanding on the topic of surface
plasmons took place in the subsequent decade, such as an improved
theoretical understanding based on Fresnel relations
\cite{chen1976excitation} and descriptions of of conically scattered light
in the presence of surface roughness~\cite{simon1976directional}.  A
concise overview of this research can be found in~\cite{raether1997surface}.

The introduction of SPR as a biosensing platform began in the early 1980's
with work by Liedberg, Nylander and Lundstrom~\cite{liedberg1983surface}
who described the extraordinary sensitivity of the surface plasmon
resonance condition to perturbations in the refractive index of the medium
on one side of the film.  The subsequent commercialization of SPR
biosensors has largely been influenced by these authors and Pharmacia
Biosensor AB (now Biacore)~\cite{liedberg1995biosensing}.

The commercial success of biosensors based on surface plasmon resonance
seems to have brought about a knowledge gap between the biosensing
community and their more theoretical predecessors from whom the field owes
its genesis.  This is to say that the scope of SPR biosensing experiments
is disproportionately narrower than the breadth of phenomena discovered
since Richie.  As an example particular to this dissertation, in 2005 and
2007, two papers~\cite{andaloro2005optical}~\cite{simon2007observation}
based on theoretical work by Chuang~\cite{chuang1986lateral} and Chen
\cite{chen1976excitation} reported a curious interference pattern occurring
in the specularly reflected light for certain (among them,
Kretschmann-Raether type) systems illuminated with focused Gaussian beams.
This was also independently reported a year later in
\cite{schumann2008near}.  Interestingly, observation of this interference
required nothing more than the addition of a lens pair to a fairly
ubiquitous optical setup, but it somehow escaped attention during earlier
research.  

Many of the topics here are inspired by work done at the University of
Oregon in the labratory of Stephen Gregory, summarized in a 2009 thesis
{\it Surface Plasmon Random Scattering and Related Phenomena}
\cite{schumann2009surface} by R.P. Schumann.  Here is described experiments
performed on thin metal films in a Kretschmann-Raether configuration with
the addition of a scanning apertureless near-field probe.  This probe (a
sharp tungsten tip) is able to elastically scatter SPPs in a way analogous
to surface roughness, but in this case its location and interaction can be
precisely controlled.  


